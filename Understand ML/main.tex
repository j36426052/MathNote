\documentclass[12pt,reqno]{amsart}
%\usepackage[margin=1in]{geometry}
\usepackage{tcolorbox}
\usepackage{amssymb}
\usepackage{amsthm}
\usepackage{amsmath}
\usepackage{amssymb}
\usepackage{mathrsfs}
\usepackage{centernot}
\usepackage{lastpage}
\usepackage{fancyhdr}
\usepackage{accents}
\usepackage{tasks}
\usepackage{graphicx}
\usepackage{natbib}
\usepackage{tabularx}
\usepackage{multirow}
\usepackage{booktabs}
\usepackage{hyperref}
\usepackage{bm}
\usepackage{float}
\theoremstyle{plain}
\usepackage{multicol}
\usepackage{enumitem,kantlipsum}

% 加上浮水印
%\usepackage{wallpaper}
%\CenterWallPaper{.180}{../qsnake-logo.jpg}


\linespread{1.2}
\parindent = 0pt
\pagestyle{fancy}
\setlength{\parindent}{0pt}
%\everymath{\displaystyle}
%new area
%\usepackage[utf8]{inputenc}
%\usepackage{CJKutf8}
%\xeCJKsetup{AutoFakeBold=true, AutoFakeSlant=true}

% 設定頭部
\fancyhead[L]{Midterm} % 左邊頭部清空
\fancyhead[C]{} % 中間頭部清空
\fancyhead[R]{} % 右邊頭部顯示頁碼

% Adjust the footer as desired:
\fancyfoot[L]{} % Left footer: Empty.
\fancyfoot[C]{\thepage} % Center footer: Empty.
\fancyfoot[R]{} % Right footer: Empty.



% we will modify sections, subsections and sub subsections
\RequirePackage{titlesec}
% Modification of section 
\titleformat{\section}[block]{\normalsize\bfseries\filcenter}{\thesection.}{.3cm}{} 


% modification of subsection and sub sub section
\titleformat{\subsection}[runin]{\bfseries}{ \thesubsection.}
{1mm}{}[.\quad]
\titleformat{\subsubsection}[runin]{\bfseries\itshape}{ \thesubsubsection.}
{1mm}{}[.\quad]

\newenvironment{solution}
  {\renewcommand\qedsymbol{$\blacksquare$}
  \begin{proof}[Solution]}
  {\end{proof}}
\renewcommand\qedsymbol{$\blacksquare$}

\newcommand{\ubar}[1]{\underaccent{\bar}{#1}}

%%%%%%%%%%%%%%%%%%%%%%%%%%%%%% Textclass specific LaTeX commands.
%\theoremstyle{plain}
%\newtheorem{thm}{\protect\theoremname}[section]
\newtheorem{thm}{\textbf{Theroem}}[section]
\newtheorem{cor}[thm]{Corollary}
\newtheorem{lmma}[thm]{Lemma}
\newtheorem*{defn}{\underline{Definition}}
\newtheorem*{prop*}{Proposition}
\newtheorem*{ex*}{Example}
\newtheorem*{sol*}{Solution}
\newtheorem*{cor*}{Corollary}
\newtheorem*{thm*}{Theorem}
\newtheorem*{lmma*}{Lemma}
\newtheorem*{rmk*}{Remark}
\newtheorem*{pf*}{\underline{\textbf{Proof\ }}}

%%%%%%%%%%%%%%%%%%%%%%%%%%%%%% User specified LaTeX commands.
\renewcommand{\P}{\mathscr{P}}
\newcommand{\B}{\mathscr{B}}
\newcommand{\A}{\mathscr{A}}
\newcommand{\C}{\mathbb{C}}
\newcommand{\CC}{\mathscr{C}}
\newcommand{\R}{\mathbb{R}}
\newcommand{\Q}{\mathbb{Q}}
\newcommand{\Z}{\mathbb{Z}}
\newcommand{\N}{\mathbb{N}}
\newcommand{\X}{\mathcal{X}}
\newcommand{\T}{\mathscr{T}}
\newcommand{\arbuni}{\bigcup_{\alpha\in I}}
\newcommand{\finint}{\bigcap_{i=1}^n}
\newcommand{\Ua}{{\textsc{U}_\alpha}}
\newcommand{\Ui}{\textsc{U}_i}
\newcommand{\pair}[2]{\left( \,#1\,,\,#2\,\right) }
\newcommand{\dint}[2]{\int_{#1}^{#2}}
\newcommand{\sett}[1]{\left\{ \,#1 \,\right\}}
\newcommand{\linearcombination}[2]{#1_1#2_1+\cdots+#1_n#2_n}
\newcommand{\slinearcombination}[1]{#1_1+\cdots+#1_n}
\newcommand{\spann}[1]{\text{span($#1$)}}
\newcommand{\sub}[1]{\text{sup}}
\newcommand{\inn}[1]{\left< #1 \right>}
\newcommand{\kernal}[1]{Ker(#1)}
\newcommand{\image}[1]{Im(#1)}
\newcommand{\norm}[1]{\parallel #1 \parallel}
\newcommand{\dia}[0]{\text{dia}}
\newcommand{\marking}[1]{\text{\color{red} #1}}
%%%%%%%%%%%%%%%%%%%%%%%%%%%%%%

\begin{document}

\lhead{Understand Machine Learing} 
\rhead{QSnake Edition} 
\cfoot{\thepage} %\ of \pageref{LastPage}}

\section*{Chapter2: A Gentle Start}
This chapter is talking about a general model of machine learning and commin error.

\subsection*{2.1 Formal Model} $ $\\

\textbf{$\bullet$ The learner's input}
	\begin{enumerate}
		\item[$\cdot$] \textbf{Domain set:} An arbitrary set, $\chi$. This is the set of objects that we may wish to label.
		\item[$\cdot$] \textbf{Label set:} The Answer of the Domain set, usually $\sett{0,1}$ or $\sett{-1,+1}$
		\item[$\cdot$] \textbf{Training data:} $S = ((x_1,y_1),\cdots,(x_m,y_m))$ is a sequence of labeled domain points.
	\end{enumerate}


\textbf{$\bullet$ The learner's output}
\begin{enumerate}
	\item[$\cdot$] $h:\chi \rightarrow y,$ a prediction function, also called a predictor, hypothesis, classifier.
	
	Formally, the learner should choose tin advance a set of predictors. This set is called a hypothesis class and is denoted by $H$. Each $h \in H$ is a function mapping from $\chi$ to $y$.
\end{enumerate}


\textbf{$\bullet$ Other assumption for ML}

\begin{enumerate}
	\item[$\cdot$] \textbf{A data-generation model:} We now explain how the training data is generated by som probability distribution. Let us denote that probability distribution over $\chi$ by $D$.
	\item[$\cdot$] \textbf{Measure of Success:} To know is the output is good or not, we define the loss function to check it
	\begin{enumerate}
		\item \textbf{True error:} $L_{D,f}(h) = \mathbb{P}_{x\sim D}[h(x) \neq f(x)] = D(\sett{x ~|~ h(x) \neq f(x)})$
		\item \textbf{Training error:} $L_S(h) = \dfrac{|\sett{i \in m ~|~ h(x_i) \neq y_i} |}{m}$ where $[m] = \sett{1,\cdots,m}$
	\end{enumerate}
	
	\textbf{\color{red} picture here \color{black} (same training error but different true error)}
	
	\item[$\cdot$] Usually, we denote the probability of getting a non-representative sample by $\delta$, and call ($1 - \delta$) the
\textbf{confidence parameter} of our prediction
\end{enumerate}


\subsection*{2.2 Improve Model} $ $

\textbf{Empirical Risk Minimization}

The method to proof the model is to minimize the loss function by using training data, i.e. $L(S)$

\textbf{Overfitting}

cause by ERM, the data is too fit the training set

example: $h_s(x) = \begin{cases}
	y_i~\text{if}~\exists~i \in [m]~\text{s.t. } x_i = x\\
	0~\text{otherwise}
\end{cases}$

\subsection*{2.3 The upper bound of $L_{(D,f)}(h_s)$ in finite hypothsis} $ $\\

\textbf{Inductive bias \& finite hypothsis}

a method to avoid overfitting, limit the predictor from predictor set $H$

the following article is too hard, it is proofing the upper bound of error.

\textbf{\color{red} a little hard here :)}

\newpage


\section*{Chapter5: The Bias-Complexity Tradeoff}

First, we can decomposition the error to following terms:

$$L_D(h_s) = \epsilon_{app} + \epsilon_{est} ~\text{ where: } \epsilon_{app} = \min_{h \in H}L_D(h),~\epsilon_{est} = L_D(h_s) - \epsilon_{app}$$


\textbf{Error type}
\begin{enumerate}
	\item[$\cdot$] \textbf{The Approximation Error($\epsilon_{app}$)}
	\item[$\cdot$] \textbf{The Estimation Error($\epsilon_{est}$)}
\end{enumerate}

\textbf{Bias-complexity Tradeoff}













\end{document}