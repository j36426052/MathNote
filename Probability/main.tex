\documentclass[12pt,reqno]{amsart}
%\usepackage[margin=1in]{geometry}
\usepackage{tcolorbox}
\usepackage{amssymb}
\usepackage{amsthm}
\usepackage{amsmath}
\usepackage{amssymb}
\usepackage{mathrsfs}
\usepackage{centernot}
\usepackage{lastpage}
\usepackage{fancyhdr}
\usepackage{accents}
\usepackage{tasks}
\usepackage{graphicx}
\usepackage{natbib}
\usepackage{tabularx}
\usepackage{multirow}
\usepackage{booktabs}
\usepackage{hyperref}
\usepackage{bm}
\usepackage{float}
\theoremstyle{plain}
\usepackage{multicol}
\usepackage{enumitem,kantlipsum}

% 加上浮水印
%\usepackage{wallpaper}
%\CenterWallPaper{.180}{../qsnake-logo.jpg}


\linespread{1.2}
\parindent = 0pt
\pagestyle{fancy}
\setlength{\parindent}{0pt}
%\everymath{\displaystyle}
%new area
%\usepackage[utf8]{inputenc}
%\usepackage{CJKutf8}
%\xeCJKsetup{AutoFakeBold=true, AutoFakeSlant=true}

% 設定頭部
\fancyhead[L]{Midterm} % 左邊頭部清空
\fancyhead[C]{} % 中間頭部清空
\fancyhead[R]{} % 右邊頭部顯示頁碼

% Adjust the footer as desired:
\fancyfoot[L]{} % Left footer: Empty.
\fancyfoot[C]{\thepage} % Center footer: Empty.
\fancyfoot[R]{} % Right footer: Empty.



% we will modify sections, subsections and sub subsections
\RequirePackage{titlesec}
% Modification of section 
\titleformat{\section}[block]{\normalsize\bfseries\filcenter}{\thesection.}{.3cm}{} 


% modification of subsection and sub sub section
\titleformat{\subsection}[runin]{\bfseries}{ \thesubsection.}
{1mm}{}[.\quad]
\titleformat{\subsubsection}[runin]{\bfseries\itshape}{ \thesubsubsection.}
{1mm}{}[.\quad]

\newenvironment{solution}
  {\renewcommand\qedsymbol{$\blacksquare$}
  \begin{proof}[Solution]}
  {\end{proof}}
\renewcommand\qedsymbol{$\blacksquare$}

\newcommand{\ubar}[1]{\underaccent{\bar}{#1}}

%%%%%%%%%%%%%%%%%%%%%%%%%%%%%% Textclass specific LaTeX commands.
%\theoremstyle{plain}
%\newtheorem{thm}{\protect\theoremname}[section]
\newtheorem{thm}{\textbf{Theroem}}[section]
\newtheorem{cor}[thm]{Corollary}
\newtheorem{lmma}[thm]{Lemma}
\newtheorem*{defn}{\underline{Definition}}
\newtheorem*{prop*}{Proposition}
\newtheorem*{ex*}{Example}
\newtheorem*{sol*}{Solution}
\newtheorem*{cor*}{Corollary}
\newtheorem*{thm*}{Theorem}
\newtheorem*{lmma*}{Lemma}
\newtheorem*{rmk*}{Remark}
\newtheorem*{pf*}{\underline{\textbf{Proof\ }}}

%%%%%%%%%%%%%%%%%%%%%%%%%%%%%% User specified LaTeX commands.
\renewcommand{\P}{\mathscr{P}}
\newcommand{\B}{\mathscr{B}}
\newcommand{\A}{\mathscr{A}}
\newcommand{\C}{\mathbb{C}}
\newcommand{\CC}{\mathscr{C}}
\newcommand{\R}{\mathbb{R}}
\newcommand{\Q}{\mathbb{Q}}
\newcommand{\Z}{\mathbb{Z}}
\newcommand{\N}{\mathbb{N}}
\newcommand{\X}{\mathcal{X}}
\newcommand{\T}{\mathscr{T}}
\newcommand{\arbuni}{\bigcup_{\alpha\in I}}
\newcommand{\finint}{\bigcap_{i=1}^n}
\newcommand{\Ua}{{\textsc{U}_\alpha}}
\newcommand{\Ui}{\textsc{U}_i}
\newcommand{\pair}[2]{\left( \,#1\,,\,#2\,\right) }
\newcommand{\dint}[2]{\int_{#1}^{#2}}
\newcommand{\sett}[1]{\left\{ \,#1 \,\right\}}
\newcommand{\linearcombination}[2]{#1_1#2_1+\cdots+#1_n#2_n}
\newcommand{\slinearcombination}[1]{#1_1+\cdots+#1_n}
\newcommand{\spann}[1]{\text{span($#1$)}}
\newcommand{\sub}[1]{\text{sup}}
\newcommand{\inn}[1]{\left< #1 \right>}
\newcommand{\kernal}[1]{Ker(#1)}
\newcommand{\image}[1]{Im(#1)}
\newcommand{\norm}[1]{\parallel #1 \parallel}
\newcommand{\dia}[0]{\text{dia}}
\newcommand{\marking}[1]{\text{\color{red} #1}}
%%%%%%%%%%%%%%%%%%%%%%%%%%%%%%

\begin{document}

%\lhead{Linear Algebra} 
%\rhead{Sabrina Edition} 
\cfoot{\thepage} %\ of \pageref{LastPage}}

\textbf{Note}

\begin{enumerate}[wide, label = $\bullet$]
	\item experiment : the process of obtaining an observed result of some phenomenon.
	\item trial : a performance of an experiment
	\item outcome : observed result
\end{enumerate}

\begin{defn}
	The set of all possible outcomes of an experiment is called the sample space, denoted by $S$.
\end{defn}

\begin{defn}
	If a sample space $S$ is either finite or countably, then it is called a discrete sample space. Otherwise, is called a continuous sample space.
\end{defn}

\begin{defn}
	An event is a subset of the sample space $S$. If $A$ is an event, then "$A$ occurred" if "$A$ contains the out come that occurred"
\end{defn}

\begin{defn}
	An event is called and elementary event if it contains exactly one outcome of the experiment.
\end{defn}

\begin{defn}$ $
	\begin{enumerate}[wide, label = $\bullet$]
		\item Two events $A$ and $B$ are called mutually exclusive if $A \cap B = \emptyset$
		\item Events $A_1,A_2,A_3,\cdots$ are said to be mutually exclusive if they are pairwise mutually exclusive. That is, if $A_i \cap A_j = \emptyset$ whenever $i \neq j$.
	\end{enumerate}
\end{defn}

\begin{defn}
	For a given experiment, $S$ denotes the sample space and $A_1,\cdots$ represent possible events. A set function that associates a real value $P(A)$ with each event $A$ is called a probability set function, and $P(A)$ is called the probability of $A$, if the following properties are satisfied:
	
	\begin{enumerate}[wide, label = $\roman*)$]
		\item $0 \leq p(A)$ for every $A$
		\item $P(S) = 1$
		\item $P(\bigcup_{i=1}^{\infty} A_i) = \sum^{\infty}_{i = 1}P(A_i)$ if $A_1,\cdots$ are pairwise mutually exclusive events.
	\end{enumerate}
\end{defn}

%\begin{defn}
	%If an object is chosen from a finite collection of distinct objects in such a manner that each object has the same probability of begin chosen, then we say that the object was chosen at random.
%\end{defn}

\begin{thm*}$ $
	\begin{enumerate}[wide,label = $\bullet$]
		\item $P(A) = 1 - P(A')$
		\item $P(A) \leq 1$, for any event $A$
		\item For any two event $A$ and $B$, $P(A \cup B) = P(A) + P(B) - P(A \cap B)$
		\item If $A \subset B,$ then $P(A) \leq P(B)$
		\item $P\left( \bigcup^{\infty}_{i = 1} A_i \right) \leq \sum^{\infty}_{i = 1}P(A_i)$ If $A_1,\cdots$ is a sequence of events
		\item If $A_1,A_2,\cdots,A_k$ are events, then $P(\bigcap_{i=1}^{k}A_i) \geq 1 - \sum_{i=1}^kP(A_i')$ 
	\end{enumerate}
\end{thm*}

\newpage

\begin{defn}
	The conditional probability of an event $A$, given the event $B$, is defined by 
	
	$$P(A~|~B) = \frac{P(A \cap B)}{P(B)}$$
	
	if $P(B) \neq 0$
\end{defn}

\begin{thm*}
	For any events $A$ and $B$,
	
	$$P(A \cap B) = P(B)P(A~|~B) = P(A)P(B~|~A)$$
\end{thm*}

\begin{thm*}
	If $B_1,\cdots,B_k$ is a collection of mutually exclusive and exhaustive events, then for any event $A$
	
	$$P(A) = \sum^{k}_{i = 1}P(B_i)P(A~|~B_i)$$
\end{thm*}

\textbf{Note.} exhaustive events:  A collection of event which union is sample space.

\begin{thm*}
	If $B_1,\cdots,B_k$ is mutually exclusive and exhaustive events, then for any event $A$ and each $j = 1,\cdots,k$
	
	$$P(B_j ~|~ A) = \dfrac{P(B_j)P(A~|~B_j)}{\sum^{k}_{i=1}P(B_i)P(A~|~B_i)} \left(  = \dfrac{P(A \cap B_j)}{P(A)}\right)$$
\end{thm*}

\begin{defn}
	Two events $A$ and $B$ are called independent events if
	
	$$P(A \cap B) = P(A)P(B)$$
	
	Otherwise, $A$ and $B$ are called dependent event
\end{defn}

\begin{thm*}
	If $A$ and $B$ are events such that $P(A) > 0$ and $P(B) > 0$, and $A$ and $B$ are independent, we get
	
	$$P(A \cap B) = P(A)P(B) \Leftrightarrow P(A~|~B) = P(A) \Leftrightarrow P(B~|~A) = P(B)$$
\end{thm*}

\begin{thm*}
	\begin{eqnarray*}
		P(A \cap B) &=& P(A)P(B)\\
		\Leftrightarrow P(A' \cap B) &=& P(A')P(B)\\
		\Leftrightarrow P(A \cap B') &=& P(A)P(B')\\ 
		\Leftrightarrow P(A' \cap B') &=& P(A')P(B')
	\end{eqnarray*}
\end{thm*}


\begin{defn}
	The $k$ events $A_1,\cdots,A_k$ are said to b independent or mutually independent if for every $j = 2,3,\cdots ,k$ and every subset of distinct indices $i_1,i_2,\cdots,i_j$
	
	$$P(A_{i1} \cap A_{i2} \cap \cdots \cap A_{ij}) = P(A_{i1})P(A_{i2})\cdots P(A_{ij})$$
\end{defn}

\begin{defn}
	A random variable, say $X$, is a function defined over a sample space $S$, that associated a real number with each possible outcome in $S$
	
	$$X(e) = x, \text{ where } e \in S$$
\end{defn}

\begin{defn}
	If the set of all possible values of a random variable, $X$, is a countable set, $x_1,x_2,\cdots,x_n,$ or $x_1,\cdots,$ then $X$ is called a discrete random variable.
	
	The function
	
	$$f(x) = P[X = x]~ x = x_1,x_2,\cdots$$
	
	that assigns the probability to each possible value $x$ will be called the discrete probability density function (discrete pdf).
\end{defn}

\textbf{Property.} $f(x_i) \geq 0,~\sum_{\text{all }x_i}f(x_i) = 1$
\begin{defn}
	The cumulative distribution function(CDF) of a random variable $X$ is defined for any real $x$ by
	
	$$F(x) = P[X \leq x]$$
\end{defn}


\begin{thm*}
	Let $X$ be a discrete random variable with pdf $f(x)$ and CDF $F(x)$. If the possible values of $X$ are indexed in increasing order, $x_1 < x_2 < x_3 < \cdots ,$ then:
	
	\begin{enumerate}[wide,label = ($\roman*$)]
		\item $f(x_1) = F(x_i)$
		\item for any $i>1,~f(x_i) = F(x_i) - F(x_{i-1})$
		\item if $x < x_1$ then $F(x) = 0$
		\item $F(x) = \sum_{x_i \leq x}f(x_i)$
	\end{enumerate}
\end{thm*}

\begin{thm*}
	A function $F(x)$ is a CDF for some random variable $X$ if and only if it satisfies: 
	
	\begin{enumerate}[wide, label = ($\roman*$)]
		\item $\lim_{x \rightarrow -\infty}F(x) = 0$
		\item $\lim_{x \rightarrow \infty}F(x) = 1$
		\item $\lim_{h \rightarrow 0^+} F(x+h) = F(x)$
		\item $a < b$ implies $F(a) \leq F(b)$
	\end{enumerate}
\end{thm*}

\begin{defn}
	If $X$ is a discrete random variable with pdf $f(x)$, then the expected value of $X$ is 
	
	$$E(X) = \sum_{x}xf(x)$$
\end{defn}

\begin{defn}
	A random variable variable $X$ is called a continuous random variable if there is a function $f(x)$, called the probability density function of $X$, such that the CDF can be represented as
	
	$$F(x) = \int^{x}_{-\infty}f(t)dt$$
\end{defn}

\textbf{Properties} A function $f(x)$ is a pdf for some continuous random variable $X$ if and only if it satisfies:

\begin{enumerate}[label = $(\roman*$)]
	\item $f(x) \geq 0 ~\forall x$
	\item $\int^{\infty}_{-\infty}f(x)dx = 1$  
\end{enumerate}

\begin{defn}
	If $X$ is a continuous random variable with pdf $f(x)$, then the expected value of $X$ is defined by
	
	$$E(X) = \int^{\infty}_{-\infty}xf(x)dx$$
	
	if it's absolutely convergent. Otherwise we say $E(X)$ does not exist.
\end{defn}

\textbf{Properties.}

\begin{enumerate}[wide,label = $\bullet$]
	\item If $X$ is a random variable with pdf $f(x)$ and $u(x)$ is a real-valued function whose domain includes the possible values of $X$, then 
	
	$$E[u(X)] = \sum_{x}u(x)f(x) \text{ if $X$ is discrete}$$
	
	$$E[u(X)] = \int^{\infty}_{-\infty}u(x)f(x)dx \text{ if $X$ is continuous }$$
	
	Note : we can consider $u(X)$ is a new random variable, if $u(x)$ is not one-to-one, $P[u(X) = u(x_1)] \neq P[X - x_1]$, but $P[u(X) = u(x')] = \sum P[X = x_i]$ where $u(x_i) = u(x')$
	\item If $X$ is a random variable with pdf $f(x)$, $a$ and $b$ are constants,  $g(x)$ and $h(x)$ are real-valued functions whose domains include the possible values of $X$, then 
	
	$$E[ag(X) + bh(X)] = aE[g(X)]+bE[h(X)]$$
	
	Note: regard $ag(X) + bh(X)$ as $u(X)$, we can use the above properties to proof it.
\end{enumerate}

\newpage

\begin{defn}
	The variance of a random variable $X$ is given by 
	
	$$\text{Var}(X) = E[(X-\mu)]$$
	
	where $\mu = E(X)$
\end{defn}

\text{Note.}

\begin{enumerate}[wide,label = $\bullet$]
	\item $k$th moment about the origin of a random variable $X$ is 
	
	$$\mu_k' = E(X^k)$$
	\item and $k$th moment about the mean is
	
	$$\mu_k = E[X - E(X)]^k = E(X - \mu)^k$$
\end{enumerate}

\begin{thm*}$ $
	\begin{enumerate}[wide,label = $\bullet$]
		\item $\text{Var}(X) = E(X^2) - E(X)^2$
		
		Note: Consider $X^2,1$ as random variable of $S$ and $E(X)$ as constant
		
		\item Var$(aX + b) = a^2$Var$(X)$
	\end{enumerate}
\end{thm*}

\begin{thm*}
	$P[u(X) \geq c] \leq \dfrac{E[u(X)]}{c}$
\end{thm*}

\begin{thm*}
	$P[|X - \mu| \geq k\sigma] \leq \dfrac{1}{k^2}$
\end{thm*}






















\end{document}