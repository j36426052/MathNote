\section*{2. Topological Space and Continuous functions}

We will introduce some basic topological space.

e.g. Order topology, Product topology, Subspace topology,

Metric topology, (Quotient topology)

\subsection*{$\S$ 12 Topological Spaces} $ $

\begin{defn}
	Let $X$ be a nonempty set $\P(X) = 2^X$ power set of $X$. We say that $\T \subseteq \P(X)$ is a topology on $X$ if
	
	\begin{enumerate}
		\item $\emptyset,X \in \T$
		\item $U_{\alpha} \in \T,~\alpha \in I \implies \bigcup_{\alpha \in I}U_{\alpha} \in \T$
		\item $U_1 , \cdots,U_n \in \T \implies U_1 \cap \cdots \cap U_n \in \T$
	\end{enumerate}
	
	If $\T$ is a topology on $X$, then the pair $(X,\T)$ or simply $X$ is called a topological space and members in $\T$ are called open sets in $X$
\end{defn}

\textbf{Example.}

\begin{enumerate}
\item $X = \{a,b,c\}$
		\begin{enumerate}
		\item The following are topological space on $X$, $\T_1 = \{\emptyset,X\}$,
			
			$\T_2 = \{\emptyset,\{a\},\{a,b\},X\}$, $\T_3 = \P(X)$
		\item The following are not topology on $X$
		
		$\A = \{\emptyset,\{a\},\{b\},X\}$ ($\because \{a\} \cup \{b\} = \{a,b\} \notin \A$)
		
		$\B = \{\emptyset,\{a,b\},\{b,c\},X\}$ ($\because \{a,b\} \cap \{b,c\} = \{b\} \notin \B$)
		\end{enumerate}
		\item Any set with more than $1$ element has at least two topology $\{\emptyset,X\}$(in discrete topology) and $\P(X)$(discrete)\\
		and former is smallest one, another is the largest one.
\end{enumerate}

\begin{defn}
	$\T_{op} = \{\T ~|~ \T \text{ is a topology on }X\}~ \T_1 \leq \T_2 \Leftrightarrow \T_1 \subseteq \T_2$
	
	Claim $"\leq"$ is a partial ordering on $\T_{op}$
	
	\begin{enumerate}[label = $\star$]
		\item Reflexive: $\forall \T \in \T_{op},~\T \leq \T$
		\item Anti-symmetry: $\forall \T_1 , \T_2 \in \T_{op},~\T_1 \leq \T_2$ and $\T_2 \leq \T_1 \implies \T_1 = \T_2$
		\item Transitive: $\forall \T_1,\T_2,\T_3 \in \T_{op}$, $\T_1 \leq \T_2$ and $\T_2 \leq \T_3 \implies \T_1 \leq \T_3$
	\end{enumerate}
\end{defn}

\textbf{Example.} Let $X$ be a set, $\T_f = \{U \subseteq X,~U = \emptyset \text{ or } X-U \text{ is finite }\}$

Then $\T_f$ is a topology on $X$, called the "finite complement topology" on $X$

\begin{proof}$ $
	\begin{enumerate}
		\item $\emptyset,X \in \T_f$ ($\because X - X = \emptyset$)
		\item $U_{\alpha} \in \T_f,~\alpha \in I$\\
		If $\bigcup_{\alpha \in I}U_{\alpha} = \emptyset,$ then $\bigcup_{\alpha \in I}U_{\alpha} \in \T_f$.\\
		If $U_{\alpha \in I}U_{\alpha} \neq \emptyset$, then $\exists \alpha_0 \in I \ni U_{\alpha_0} \neq \emptyset$ and $X-U_{\alpha}$ is finite\\
		$X - \bigcup_{\alpha \in I}U_{\alpha} = \bigcap_{\alpha \in I}(X - U_{\alpha}) \subseteq X - U_{\alpha_0} \implies X - (\bigcup_{\alpha \in I}U_{\alpha})$ is finte $\implies \bigcup_{\alpha \in I}U_{\alpha} \in \T_f$
		\item $U_1,\cdots,U_n \in \T_f$\\
		If $U_1 \cap \cdots \cap U_n = \emptyset$, then $U_1 \cap \cdots \cap U_n \in \T_f$\\
		If $U_1 \cap \cdots \cap U_n \neq \emptyset$, then $X - (U_1 \cap \cdots \cap U_n) = (X - U_1) \cap \cdots \cup (X - U_n)$ is finite since each $X - U_i$ is finite. Thus $U_1 \cap \cdots \cap U_n \in \T_f$
	\end{enumerate} 
	From (1)(2)(3), $\T_f$ is a topology on $X$.
\end{proof}

\begin{rmk*}
	If $X$ is a finite set, then $\T_f$ is the discrete topology on $X$
\end{rmk*}

\textbf{Example.} Let $X$ be a set and $\T_c = \{U \subseteq X~|~ U = \emptyset \text{ or } X - U \text{ is countable }\}$. Then as in example above, $\T_c$ is a topology on $X$, called the countable complement topology on $X$. Moreover, if $X$ is countable, then $\T_c$ is just a discrete topology on $X$

\begin{defn}
	Let $\T$ and $\T'$ be two topologies on $X$. We say that $\T'$ is (strictly) finer then $\T$ or $\T$ is (strictly) coaser that $\T'$ if \\$\T \leq \T'(\T < \T')$, i.e. $\T \subseteq \T'(\T \subsetneqq \T')$
\end{defn}

\begin{rmk*}$ $
	\begin{enumerate}
		\item Two topologies on $X$ need not be comparable
		\item Other terminology, if $\T' \supset T$, $\T'$ is larger(stronger) than $\T$ and $\T$ is smaller(weaker) than $\T$
	\end{enumerate}
\end{rmk*}

\subsection*{$\S$ 13 Bases for a topology}

\begin{defn}
	Let $X$ be a set. A base for a topology on $X$ is a collection $\B \subseteq \P(X)$ satisfying
	
	\begin{enumerate}
		\item $U\B = X$ $(\bigcup \B = \bigcup_{B \in \B}B)$
		\item Given $B_1,B_2 \in \B$ and $x \in B_1 \cap B_2 ~ \exists B_3 \in \B \ni x \in B_3 \subseteq B_1 \cap B_2$
	\end{enumerate}
	Members in $\B$ are called basic open sets in $X$
\end{defn}

Given a base $\B$ for a topology on $X$, we can define the smallest topology $\T$ on $X$ containing $\B$ called the topology on $X$ generated by $\B$.

Usually, there are two ways to describe it

\begin{enumerate}[wide]
	\item[(I)] $\T = \{U \subseteq X , \forall x \in U \exists B \in \B \ni x \in B \subseteq U\}$. Clearly, $\B \subseteq \T$
	
	\begin{enumerate}
		\item $\emptyset , X \in \T$(by the definition of bases (1))
		\item $U_{\alpha} \in \T,~\alpha \in I \implies \bigcup_{\alpha \in I}U_{\alpha} \in \T$. Given $x \in \bigcup_{\alpha \in I}U_{\alpha},x \in U_{\alpha_0}$ for some $\alpha_0 \in I,~\exists B \in \B \ni x \in B \subseteq U_{\alpha_0} \subseteq \bigcup_{\alpha \in I}U_{\alpha}$
		\item $U_1,\cdots,U_n \in \T \implies U_1 \cap \cdots \cap U_n \in \T$. By induction on $n$, we only prove $n=2$. Given $x \in U_1 \cap U_2,~x \in U_1$ and $x \in U_2$\\
		$\implies \exists B_1,B_2 \in \B \ni x \in B_1 \subseteq U_1$ and $X$ in $ \B_2 \subseteq U_2 \implies$\\
		$x \in B_1 \cap B_2 \subseteq U_1 \cap U_2 \implies \exists B_3 \in \B \ni x \in B_3 \subseteq B_1 \cap B_2 \subseteq U_1 \cap U_2 \implies U_1 \cap U_2 \in \T$
	\end{enumerate}
	
	\item[(II)]  $\T' = \{\bigcup\A ~|~ \A \subseteq \B\} = \{\bigcup_{\alpha \in I}A_{\alpha} ~|~ A_{\alpha} \in \B\}$
	
	Clearly, $\B \subseteq \T'$(only choose one element in $\B$)
	
	\begin{enumerate}
		\item $\emptyset,X \in \T'$(trivial)
		\item $U_{\alpha} \in \T',~\alpha \in I \implies \bigcup_{\alpha \in I}U_{\alpha} \in \T'$\\
		$\forall \alpha \in I,U_{\alpha} = \bigcup_{\beta \in I_{\alpha}}A_{\beta}$. Then $\bigcup_{\alpha \in I}U_{\alpha} = \bigcup_{\alpha \in I}\bigcup_{\beta \in I_{\alpha}}A_{\beta} \implies \bigcup_{\alpha \in I}U_{\alpha} \in \T'$
		\item $U_1,\cdots,U_n \in \T' \implies U_1 \cap \cdots \cap U_n \in \T'$. By induction on $n$, we only to prove that $n = 2$. For $i=1,2,\cdots$, $U_i = \bigcup_{\alpha \in I_j}A_{\alpha}$.\\
		$U_1 \cap U_2 = \bigcup_{\alpha \in I_2}(A_{\beta}^1 \cap A_{\alpha}^2) $. $\forall x \in U_1 \cap U_2,~ x \in A'_{\beta} \cap A_{\alpha}^2 \implies U_1 \cap U_2 = \bigcup_{x \in U_1 \cap U_2}B_X \in \T'$
	\end{enumerate}
	
	\item[(III)] $\T = \T'$
		\begin{enumerate}
		\item[($\subseteq$)] Given $U \in \T,~\forall x \in U,~\exists B_x \in \B \ni x \in B_x \subseteq U \implies U = \bigcup_{x \in U}B_x \in \T'$
		\item[($\supseteq$)] Given $U \in \T' ~ U = \bigcup_{\alpha \in I}A_{\alpha},A_{\alpha} \in \B$\\
		$\forall x \in U,x \in A_{\alpha}$ for some $\alpha \in I$ and $A_{\alpha} \in \B$, i.e. $X \in A_{\alpha} \in U$ and $A_{\alpha} \in \B \implies U \in \T$. Hence $\T = \T'$
		\end{enumerate}
\end{enumerate}

\textbf{Example.}

\begin{enumerate}
	\item Let $\B$ be the collection of all open balls in $\R^n$. Then $\B$ is a base for a topology on $\R^n$, namely, then Euclidean topology on $\R^n$
	\item Let $\B'$ be the collection of all $n-$dimentional open intervals in $\R$. Then $\B'$ is a base for a topology on $\R^n$. In fact, $\beta$ and $\beta'$ generate the same topology on $\R^n$
\end{enumerate}

\begin{lmma*}
	Let $X$ be a set, let $\B$ be a basis for a topology $\T$ on $X$. $\T$ equals the collection if all unions of elements of $\B$.
\end{lmma*}

\begin{lmma*}
	Let $X$ be a topological space and $\mathscr{C}$ be a collection of open sets of $X \ni \forall$ open set $U$ in $X$ and $\forall x \in U ~\exists C \in \mathscr{C} \ni x \in C \subseteq U$. Then $\mathscr{C}$ is a base for the topology of $X$.
\end{lmma*}

\begin{proof}
	\begin{enumerate}
		\item $\bigcup \mathscr{C} = X$\\
		Since $X$ is open $\forall x \in X,~\exists C_x \in \mathscr{C} \ni x \in C_x \subseteq X \implies x \in \bigcup \mathscr{C} \implies X = \bigcup \mathscr{C}$
		\item Given $C_1,C_2 \in \mathscr{C}$ and $x \in C_1 \cap C_2$. Since $C_1 \cap C_2$ is open, $\exists C \in \mathscr{C} \ni x \in C \subseteq C_1 \cap C_2$, $\therefore \mathscr{C}$ is a base for a topology of $X$
	\end{enumerate}
\end{proof}

\begin{rmk*}
	Let $\T$ be the original topology on $X$ and $\T'$ be the topology generated by $\mathscr{C}$. Then $\T = \T'$
\end{rmk*}

\begin{proof}$ $

	\begin{enumerate}
		\item[($\subseteq$)] Given $U \in \T$, $\forall x \in U \exists C \in \mathscr{C} \ni x \in C \subseteq U \implies U \in \T'$
		\item[($\supseteq$)] Given $v \in \T'$, by lemma, $V = \bigcup \A$ for some $\
		A \subseteq \mathscr{C}$. Since $\mathscr{C} \subseteq \T$, $\A \subseteq \T$, $\therefore V = \bigcup \A \in \T$
	\end{enumerate}
\end{proof}

\begin{lmma*}
	Let $\B$ and $\B'$ be bases for the topology $\T$ and $\T'$ on $X$ respective TFAE
	
	\begin{enumerate}
		\item $\T$ is finer that $\T$ i.e. $\T \subseteq \T'$
		\item $\forall x \in X$ and $B \in \B$ with $x \in B$, $\exists B' \in \B \ni x \in B' \subseteq B$
	\end{enumerate}
\end{lmma*}

\begin{proof}$ $
	
	\begin{enumerate}
		\item[$(a)\Rightarrow(b)$]
		Suppose $\T \subseteq \T'$. Given $x \in X$ and $B \in \B$ with $x \in B$. Since $\T \subseteq \T',~B \in \T,\exists B' \in \B \ni x \in B' \subseteq B$ 
		\item[$(b)\Rightarrow(a)$] 
		Suppose $(b)$ holds. Given $U \in \T,~\forall x \in U,~\exists B_x \in \B \ni x\in B_x \subseteq U.$ By $(b),~\exists B'_x \in \B \ni x\in B_x' \subseteq B_x \subseteq U \implies U \in \T'$
	\end{enumerate}
	
\end{proof}

\textbf{Example. } In $\S 13$, example 1,2

$\B:$ all open balls in $\R^n$ for a topology on $\R^n$

$\B':$all open intervals in $\R^n$ for a topology on $\R^n$

By lemma above, they generate the same Euclidean topology on $\R^n$

We now define $3$ topologies on the real line $\R$
\begin{defn}$ $
	\begin{enumerate}
		\item $\B = \{(a,b)~|~-\infty < a < b < \infty\}$: the collection of all open intervals in $\R$ which is the base for the usual topology on $\R$
		\item $\B' = \{[a,b)~|~-\infty < a < b < \infty\}$ the collection of all closed-open interval in $\R$, which is also a base for a topology of $\R$ called the lower limit topology on $\R$. We denote it by $\R_l$
		\item Let $K = \{\frac{1}{n}~|~n \in \N\}$ and $\B'' = \{B \subseteq \R ~|~ B = (a,b) \text{ or }B=(a,b) - K \text{ for } -\infty < a < b < \infty\}$. Claim: $\B'$ is a base for a topology on $\T$
			\begin{enumerate}[label = $\star$]
			\item Clearly, $U \B'' = \R$
			\item Given $B_1,B_2 \in \B''$ and $x \in B_1 \cap B_2$. We have $4$ cases:
				\begin{enumerate}
				\item $B_1$ and $B_2$ are open intervals which is clearly.
				\item $B_1 = (a,b)$ and $B_2 = (c,d) - K$. Let $\alpha = \max\{a,c\}$ and $\beta = \min\{b,c\}$. $x \in (\alpha,\beta) - K \subseteq B_1 \cap B_2$ and $(\alpha , \beta) - K \in \B''$
				\item (3)(4) similarly
				\end{enumerate} 
			\end{enumerate} 
			The topology on $\R$ generated by $B'$ is called the $K-$topology on $\R$ and denoted $\R_k$
	\end{enumerate}
\end{defn}

\begin{lmma*}
	The topologies of $\R_l$ and $\R_k$ are strictly finer than the Euclidean topology of $\R$ but are not comparable with one another
\end{lmma*}

\begin{proof}
	Let $\T,\T'$ and $\T''$ be the topologies of $\R,\R_l,\R_k$ generated by $\B,\B',\B''$ respectly. We use lemma above to prove it.
	\begin{enumerate}[label = $\star$]
		\item $\T \subsetneqq \T'$
		Given $(a,b) \in \B$ and $x \in (a,b)$. We have $[x,b) \in \B'$ with $x \in [x,b) \subseteq (a,b)$. By lemma, $\T \subseteq \T'$, $\forall a<b,~[a,b) \in \B'$ so $[a,b) \in \T',$ but $[a,b)' \notin \T$
		\item Clearly, $\T \subseteq \T''$ by $\B \subseteq \B''$. Moreover $B'' = (-1,1) - K \in \B''$, so $B'' \in \T''$ but $B'' \notin \T$.
		\item $\T'$ and $\T''$ are not comparable\\
		$(-1,1) - K \in \T''$, but $(-1,1) - K \notin \T'(\because$ not $[0,c) \in \B' \ni 0 \in [0,c) \subseteq (-1,1) - K$). $[0,1) \in \T$ but no $\B'' \in \B'' \ni 0 \in B'' \subseteq [0,1)$
	\end{enumerate} 
\end{proof}

\begin{defn}
	A subbase $\mathscr{S}$ for a topology on $X$ is a collection of subsets of $X$ with $\bigcup \mathscr{S} = X$ and elements in $\mathscr{S}$ are calle subbasic open sets in $X$
\end{defn}

Given subbase on $X$

$$\B = \{S_1 \cap \cdots \cap S_k,~k \in \N,S_1,\cdots,S_k \in S\}$$

Claim $\B$ is a base for a topology on $X$

\begin{defn}
	The topology on $X$ generated by a subbase $\mathscr{S}$ is defined to be the topology generated by the base $\B$.
\end{defn}

\subsection*{$\S $ 14 The Order Topology}

(which provides many counterexample in topology)

\begin{defn}
	A relation $C$ on a set is called an "order relation"  (or a simple order) if it satisfies
	\begin{enumerate}
		\item Comparable: $\forall x \neq y$ in $X$ either $xCy$ or $yCx$
		\item Non-reflexivity: no $xCx$
		\item Transitivity: $xCy$ and $yCz \implies xCz$
	\end{enumerate}
\end{defn}

Given a simple order set $(X,<)$ and $a,b \in X$ with $a<b$( Note: $a \leq b$ means $a < b$ or $a = b$). We can define:

$(a,b) = \{x \in X ~|~ a < x < b\}$ open interval

$(a,b] = \{x \in X ~|~ a < x \leq b\}$ open interval

$[a,b) = \{x \in X ~|~ a \leq x < b\}$ open interval

$[a,b] = \{x \in X ~|~ a \leq x \leq b\}$ open interval


We assume that $|X| \geq 2$. Let $\B$ be the collection of all subsets of the following types

\begin{enumerate}
	\item All open intervals $(a,b)$ in $X$
	\item All intervals of the forms $[a_0,b)$ where $a_0$ is the smallest elements of $X$
	\item All intervals of the forms $(a,b_0]$ where $b_0$ is the largest elements of $X$
\end{enumerate}

\begin{defn}
	The topology generated by $\B$ is called the order topology on $X$
\end{defn}

\textbf{Example.}

\begin{enumerate}
	\item If $X$ is an order set and $T \subseteq X,$ then so is $Y$
	\item In $\R$ we give the usually ordering and the order topology on $\R$ is the usual topology on $\R$
	\item In $\R^{\star} = \R \cup \{-\infty , \infty\}$ with the usual ordering is an order set.
	\item In $\R \times \R$ with the dictionary order is an order set whose basis for the order topology is of the form
	\item $\N$ with the usual ordering is an order set with the smallest element $1$. What is the order topology?
		\begin{enumerate}
		\item[$\star$] $[1,b):b\in \N$ and $(a,b),~a<b$. In particular, $\{1\} = [1,2)$ and $\{n\} = (n-1,n+1),n>1$ are basic open sets in $\N$
		
		$\therefore$ the order topology on $\N$ is the discrete topology on $\N$
		\end{enumerate}
	\item The set $X = \{1,2\} \times \N = \{1 \times n\}^{\infty}_{n = 1} = a_n  \cup  b_n = \{2 \times n \}^{\infty}_{n = 1}$ in the dictionary order with the smallest element $1 \times 1$. The order topology on $X$ is not discrete topology on $X$
	
	$X:a_1,a_2,\cdots,b_1,b_2,\cdots$, $a_i < a_{i+1},~b_j < b_j + 1,~a_i < b_j$
	\begin{enumerate}[label = $\star$]
		\item $\{a_1\} = [a_1,a_2)$
		\item $\{a)n\} = (a_{n-1},a_{n+1}),n \geq 2$
		\item $\{b_n\} = (b_{n-1},,b_{n+1}),n\geq 2$
	\end{enumerate}
	But $\{b_1\}$ is not open, $\because b_1$ is not the smallest elements any basic open set in the order topology containing $b_1$ must of the form $(a_l,b_j)$ for some $l \geq 1$ and $j > 1$
\end{enumerate}


\begin{defn}
	Let $X$ be an ordered set and $a \in X$. We define the rays determine by $a$
	
	\begin{enumerate}[label = $\star$]
		\item $(a,\infty) = \{x \in X ~|~ x > a\}$
		\item $(-\infty,a) = \{x \in X ~|~ x < a\}$
		\item $[a,\infty) = \{x \in X ~|~ x \geq a\}$
		\item $(\infty,a] = \{x \in X ~|~ x \leq a\}$
	\end{enumerate}
\end{defn}

Some facts:

\begin{enumerate}
	\item open rays in $X$ are open in the order topology of $X$. In fact, $(a,\infty) = (a,b_0]$ if $X$ has the largest element which is a basic open set in the order topology of $X$. If $X$ has no largest element, then $(a,\infty) = \bigcup_{a < x}(a,x)$ which is open in the order topology of $X$
	\item closed rays is close
	\item The order topology of $X$ is contained in the topology on $X$ generated by open rays in $X$. $\because (a,b) = (a,\infty) \cap (-\infty,b)$. \\
	If $X$ has the smallest element $a_0,~ [a_0,b) = (-\infty,b)$\\
	If $X$ has the largest element $b_0,~(a,b_0] = (a,\infty)$
\end{enumerate}

\subsection*{$\S$ 15 The Product Topology on $X \times Y$}

Similarly for $X_1,\cdots, X_n$

Let $X$ and $Y$ be topology spaces and 

$$\B = \{U \times V ~|~ U \text{ is open in } X,V \text{ is open in } Y\}$$

Claim $\B$ is a base for a topology on $X \times Y$

\begin{enumerate}[label = $\bullet$]
	\item $\bigcup \B = X \times Y$
	\item Given $U_i \times V_i \in \B,~i=1,2$ and $(a,b) \in (U_1 \times V_1) \cap (U_2 \times V_2)$
	
	$(a,b) \in U \times V \subseteq (U_1 \times V_1) \cap (U_2 \times V_2)$ where $U = U_1 \cap U_2,~V = V_1 \cap V_2$
\end{enumerate}

\begin{defn}
	The topology on $X \times Y$ generate by $\B$ is called the product topology on $X \times Y$
\end{defn}

\begin{rmk*}
	If $X_1,\cdots,X_n$ are topological space, them
	
	\begin{enumerate}
		\item $\B = \{U_1 \times \cdots \times U_n ~|~ U_i \text{ is open in }X_i,~1\leq i \leq n\}$ is a base for the product topology on $X_1 \times \cdots \times X_n$
		\item The product topology on $\R^n = \R \times \cdots \times \R$ is the usual topology on $\R^n$ generate b the collection of all $n$-dimensional open intervals.
		
		$$\{I_1 \times \cdots \times I_n ~|~ I_j \text{ is an open interval in }\R,~1 \leq j \leq n\}$$
	\end{enumerate}
\end{rmk*}

\begin{thm*}
	Let $X$ and $Y$ be topological space with bases $\B_X$ and $\B_Y$ on $X$ and $Y$ respectively. Then
	
	$$\mathscr{D} = \{B \times C ~|~ B \in \B_X,~C \in \B_Y\}$$

	forms a basis for the product topology on $X \times Y$
\end{thm*}

\begin{proof}
	Let $\B = \{U \times V ~|~ U \text{ is open in }X \text{ and } V \text{ is open in }Y\}$. We know that $\B$ is a base for the product topology on $X \times Y$
	
	Given $U \times V \in \B$ with $(a,b) \in U \times V \implies a \in U,~b \in V \implies \exists B \in \B_X$ and $C \in \B_Y \ni a \in B \subseteq U,~b \in C \subseteq V$
	
	$\therefore (a,b) \in B \times C \subseteq U \times V$ and $B \times C \in \mathscr{D}$
\end{proof}

Redefine the product topology on $X_1 \times \cdots \times X_n$ by using subbase

The projection onto $X_i$

\begin{eqnarray*}
	\pi_i: X_1 \times \cdots \times X_n &\rightarrow & X_i\\
	(x_1,\cdots,x_n) &\rightarrow& x_1 ,~1 \leq i \leq n
\end{eqnarray*}

If $U_i \subseteq X_i~ \pi_i^{-1}(U_i) = X_1 \times \cdots X_{i-1} \times U_i \times X_{i+1} \times \cdots \times X_n$

Let $\delta = \{\phi_i^{-1}(U_i)~|~ U_i \subseteq X_i \text{ is open and } 1\leq i \leq n\}$

Note: $\bigcup_{i = 1}^n\pi_i^{-1}(U_i) = X_1 \times \cdots \times X_n$

$\therefore \delta$ is a subbase for a topological on $X_1 \times \cdots \times X_n$ with base

$$\{U_1 \times \cdots \times U_n~|~ U_i \text{ is open in }X_i,~1 \leq i \leq n\}$$

Hence, the product topology on $X_1 \times \cdots \times X_n$ is generated by $\delta$

\subsection*{$\S$ 16 The subspace topolgoy}

Let $X$ be a topology space with topology $\T$ and $Y \subseteq X$. Let $\T_Y = \{U \cap Y ~|~ U \in \T , \text{ i.e. } U \text{ is open in }X\}$

\begin{defn}
	The topology $\T_Y$ on $Y$ is called the subspace topology of $Y$ in $X$. With this topology, $Y$ is called a subspace of $X$
\end{defn}

\begin{lmma*}
	If $\B$ is a base for the topology $\T$ of $X$, then $\B_Y = \{B \cap Y ~|~ B \in \B\}$ is a base for the subspace topology on $Y$.
\end{lmma*}

\begin{proof}
	Given an open set $V$ in $Y$ and $y \in V$. Then $y \in V =  \cap Y$ for some open set in $X \implies y \in U \implies \exists B \in \B \ni y \in B \subseteq U \implies y \in B \cap Y \subseteq U \cap Y = V$
	
	$\therefore \B_Y$ is a base for the subspace topology of $Y$.
\end{proof}

\begin{lmma*}
	Let $Y$ be a subspace of $X$. If $Y$ is open in $X$ and $V$ is open in $Y$, then $V$ is open in $X$.
\end{lmma*}

\begin{thm*}
	If $A$ is a subspace of $X$ and $B$ is a subspace of $Y$. Then the product topology on $A \times B$ is the same as the subspace topology $A \times B$ inherits as a subspace of $X \times Y$
\end{thm*}

\begin{proof}
	Let $\B = \{U \times V ~|~ U \text{ is open in }X,~V\text{ is open in }Y\}$. Then $\B$ is a base for the product topology on $X \times Y$. By lemma above, $\B_{A \times B} = \{(U \times V) \cap (A \times B)~|~ U \times V \in \B\}$ is a base for the subspace topology on $A \times B$
	
	$\B_{A \times B} = \{(U \cap A) \times (U \cap B)~|~ U \cap A \text{ is open in }A,~ V \cap B \text{ is open in } B\}$ which is a base for the product space $A \times B$. Thus ...
\end{proof}

\textbf{Example.}

\begin{enumerate}
	\item Consider $Y = [0,1]$ in $\R$. The subspace topology of $Y$ in $\R$ has a base of the form
	
	$$\{(a,b) \cap Y~|~ -\infty < a < b < \infty\}$$
	
	Note that
	
	$$(a,b) \cap Y = \begin{cases}
		(a,b) & \text{ if } a,b \in Y\\
		[0,b) & \text{ if only} b \in Y\\
		(0,1] & \text{ if only} a \in Y\\
		\emptyset \text{ or } Y & \text{ if } a,b \notin Y 
	\end{cases}$$
	The order topology on $Y$ has a base of the form $[0,b) b\in Y,~(a,1] a \in Y,~(a,b) ~ a,b \in Y$
	
	\item Let $Y = [0,1) \cup \{2\} \subseteq \R$. In the subspace topology of $Y$ in $\R$. $\{2\} = (\dfrac{3}{2},\dfrac{5}{2}) \cap Y$ is open in $Y$. In the order topology of $Y$, $\{2\}$ is not open in $Y$
	
	\begin{proof}
		$\because$ any basic open set in the order of $Y$ containing $2$ is of the form
		
		$$(a,2] = \{y \in Y ~|~ a < y \leq 2\} \text{ where} a \in Y$$
		
		must contain points not equal $2$, $\therefore$ The two topologies are different
	\end{proof} 
	\item $I = [0,1]$. The dictionary order on $I \times I$ is just the restriction to $I \times I$ of the dictionary order on $\R^2 = \R \times \R$
	
	The set $V = \{\dfrac{1}{2}\} \times ( \dfrac{1}{2},1 ]$ is open in the subspace topology of $I \times I$
	
	V is not open in the order topology $I \times I$
	
	$\because$ any basic open set in the order topology of $I \times I$ containing $\dfrac{1}{2} \times 1$ is of the form $(a\times b,c\times d)$
	
	There is no basic open set $B$ in the order topology of $I \times I$ such that $\dfrac{1}{2} \times 1 \in B \subseteq \{\dfrac{1}{2}\} \times (\dfrac{1}{2} , 1]$
	
	$\therefore$ The two topologies on $I \times I$ are distinct.
\end{enumerate}

\begin{defn}
	Given an order set $X$. A subset $Y \subseteq X$ is convex if $\forall a < b$ in $Y$, $(a,b) \subseteq Y$
	
	In fact, $[a,b] \subseteq Y$
\end{defn}

\begin{thm*}
	Let $X$ be an order set with order topology and $Y \subseteq X$ be a convex set of $X$. Then the order topology on $Y$ and the subspace topology on $Y$ concise.
\end{thm*}

\begin{proof}
	Let $\T_O$ and $\T_Y$ be the order topology and subspace topology on $Y$, respectively.
	
	$\T_O \supseteq \T_Y$
	
	Note that the order topology on $Y$ is generate by the subbasic open sets  of all rays in $Y$ of the forms
	
	$$(a,\infty)\cap Y \text{ and } (-\infty,b) \cap Y,~ a,b \in Y$$
	
	the order topology on $X$ is generated by subbasic open sets
	
	$$(a,\infty) \text{ and } (-\infty,b)~a,b \in X$$
	
	The subbasic open sets in the subspace topology $\T_Y$
	
	$$(a , \infty) \cap Y,~ ( -\infty , b) \cap Y,~ a,b \in X$$
	
	If $a \in Y$, then $(a , \infty) \cap Y$ is an open ray in $Y$ which is a subbasic open set in the order topology $\T_O$ of $Y$, thus, $(a,\infty) \cap Y \in \T_O$
	
	If $a \notin Y$, then since $Y$ is convex, a is either a lower bound for $Y$ or an upper bound for $Y$. Therefore,
	
	$$(a,\infty) \cap Y = \begin{cases}
		Y & \text{ if } a \text{ is a lower bound of } Y\\
		X & \text{ if } a \text{ is a upper bound of } Y\\
	\end{cases}$$
	
	In any case, $(a,\infty) \cap Y \in \T_O \forall a \in X$. Similarly $(-\infty , b) \cap Y \in \T_O \forall b \in X$, $\therefore \T_Y \subseteq \T_O$
	
	and the other way don't need convex.
\end{proof}

\subsection*{$\S$17 Closed Sets and Limit Points}

\subsubsection*{$\S$17.1} Closed Sets

\begin{defn}
	Let $X$ be a topological space and $A \subseteq X$, $A$ is closed if $A^c = X-A = A$ is open in $X$
\end{defn}

\textbf{Example}:

\begin{enumerate}
	\item $\forall -\infty < a \leq b < \infty,~[a,b],[a,\infty),(-\infty,a)$ are closed in $\R$
	\item $A = \{(x,y) \in \R^2,~x\geq 0,~y \geq 0\}$ is closed in $\R^2$
	\item In the finite complement topology on a set $X$. the closed set in $X$ are $X$ and all finite subsets of $X$
	\item In a discrete topological space $X$ every subset of $X$ is closed
	\item Consider the subspace $Y = [0,1] \cup (2,3)$ of $\R$, $[0,1]$ is open in $Y$, $(2,3)$ is open in $Y$ and $\R$. Since $Y - [0,1] = (2,3)$ and $Y - (2,3) = [0,1]$, $[0,1]$ and $(2,3)$ are both open and closed in $Y$.
\end{enumerate}

\begin{thm*}[17.1]
	Let $X$ be a topological space. Then 
	\begin{enumerate}
		\item $\emptyset,X$ are closed
		\item $A_{\alpha}$ is closed in $X$, $\alpha \in I \implies \bigcap_{\alpha \in I}A_{\alpha}$ is closed
		\item $A_1,\cdots,A_n$ are closed $\implies ~ A_1 \cup \cdots \cup A_n$ is closed. 
	\end{enumerate}
\end{thm*}

\begin{rmk*}$ $
	\begin{enumerate}
		\item In definition (3) of topology is false for infinitely many open set, e.g. $\bigcap^{\infty}_{n = 1}\left( \dfrac{1}{-n},1 + \dfrac{1}{n} \right) = [0,1]$ is not open in $\R$
		\item In (3) of Thm 17.1 is false for infinitely many closed set. e.g. $\bigcup^{\infty}_{n = 1}[\dfrac{1}{n} , 1 - \dfrac{1}{n}] = (0,1)$ is not closed in $\R$
	\end{enumerate}
\end{rmk*}

\begin{thm*}[17.2]
	Let $Y$ be a subspace of a topological space $X$ and $A \subseteq X$. Then $A$ is closed in $Y$ iff $A - B \cap Y$ for some closed set $B$ in $X$.
\end{thm*}

\begin{proof}
	\begin{eqnarray*}
		A \text{ is closed in }Y &\Leftrightarrow & Y-A \text{ is open in } Y\\
		&\Leftrightarrow& Y - A = U \cap Y, ~ U \text{ is open in }X\\
		&\Leftrightarrow& A = Y-(U \cap Y) = (X - U) \cap Y \text{ is closed in } X
	\end{eqnarray*}
\end{proof}

\begin{thm*}
	Let $Y$ be a subspace of a topological space $X$. If $A$ is closed in $Y$ and $Y$ is closed in $X$, then $A$ is closed in $X$.
\end{thm*}

\begin{proof}
	By Thm 17.2, trivial
\end{proof}

\subsubsection*{$\S$ 17.2}Closure and Interior of a set

\begin{defn}
	Let $X$ be a topological space and $A \subseteq X$
	
	\begin{enumerate}
		\item The interior of $A$, $A^{\circ} = int(A) = \bigcup_{\stackrel{A \subseteq U}{U \text{ is closed}}}U$
		\item The closure of $A$, $\overline{A} = cl(A) = \bigcap_{\stackrel{A \subseteq F}{F \text{ is closed}}}F$
	\end{enumerate}
\end{defn}

\begin{rmk*}$ $
	\begin{enumerate}
		\item $A^{\circ}$ is the largest open set in $X$ contained in $A$(w.r.t $\subseteq$)
		\item $A^{\circ} \subseteq A \subseteq \overline{A}$, $A^{\circ}$ is open in $X$ and $\overline{A}$ is closed in $X$.
		\item $A$ is open iff $A^{\circ} = A$. In particular, $A^{\circ\circ} = A^{\circ}$, $A$ is closed iff $\overline{A} = A$. In particular $\overline{A} = \overline{\overline{A}}$
		\item Let $X$ be a topological space and $Y \subseteq X$ be a subspace $\forall A \subseteq X$, we have the closure of $A$ in $X: \overline{A}$, and the closure of $A$ in $Y: \overline{A}^Y$, in general, $\overline{A} \neq \overline{A}^Y$\\
		e.g. $X = \R,~Y = [0,1),~A = (\dfrac{1}{2},1) $ \\ $\implies \overline{A} = [\dfrac{1}{2},1),~\overline{A}^Y = [\dfrac{1}{2},1)$
	\end{enumerate}
\end{rmk*}

\begin{thm*}
	Let $Y$ be a subspace of $X$ and $A \subseteq  Y$. Then $\overline{A}^Y = \overline{A}\cap Y$
\end{thm*}

\begin{proof}
	By Thm 17.2 and $\overline{A}$ is closed in $X$. $\overline{A} \cap Y$ is closed in $Y$. Since $A \subseteq Y$ is closed subset in $Y$ containing $A$, $\overline{A}^Y \subseteq \overline{A} \subseteq \overline{A} \cap Y$\\
	Conversely, $\overline{A}^Y$ is closed in $Y \implies \overline{A}^Y = F \cap Y$ for some closed set $F$ in $X$. Clearly, $A \subseteq F \implies \overline{A} \subseteq \overline{F} \implies \overline{A} \subseteq F \implies \overline{A} \cap Y \subseteq F \cap Y = \overline{A}^Y \therefore \overline{A}^Y = \overline{A} \cap Y$
\end{proof}

\begin{defn}$ $
	\begin{enumerate}
		\item A set $A$ intersects a set $B$ if $A \cap B \neq \emptyset$
		\item A neighborhood of a point $x$ is an open set containing $x$
	\end{enumerate}
\end{defn}

\begin{defn}
	Let $X$ be a topological space and $A \subseteq X$. A point $x \in X$ is an adherent point of $A$ if $\forall$ nhd $U$ of $x$, $U \cap A \neq \emptyset$ 
\end{defn}

\begin{thm*}[17.5]
	Let $X$ be a topological space and $A \subseteq X$
	
	\begin{enumerate}
		\item $x \in \overline{A}$ iff $x$ is an adherent point of $A$
		\item Suppose the topological of $X$ is given by a base $\B$. Then $x \in \overline{A}$ iff $\forall$ basic nhd $B$ of $x,~B \cap A \neq \emptyset$
	\end{enumerate}
\end{thm*}

\begin{proof}$ $
	\begin{enumerate}
		\item[(a)]
		\begin{enumerate}
		\item[($\Rightarrow$)] Suppose $x \in \overline{A}$. If $x$ is not an adherent point of $A$, then $\exists$ nhd $U$ of $x \ni U \cap A = \emptyset$. Thus, $A \subseteq X - U$ which is closed $\implies \overline{A} \subseteq X - U \implies x \notin \overline{A} (\rightarrow\leftarrow)$ 
		\item[($\Leftarrow$)] Suppose $x$ is an adherent point of $A$. If $x \notin \overline{A}$, then $x \in X-\overline{A} \equiv U$ is a nhd of $x$ with $U \cap A = \emptyset (\rightarrow\leftarrow)$ to $x$ is an adherent point
		\end{enumerate}
		\item[(b)] H.W.
	\end{enumerate}
\end{proof}

\textbf{Example.} In $\R$ by Thm 17.5, we have

\begin{enumerate}[label = $\bullet$]
	\item $(0,1] = [0,1]$
	\item $\{\dfrac{1}{n}~n \in \N\} = \{0\} \cup \{\dfrac{1}{n}~|~ n \in \N\}$
	\item $\overline{\Q} = \R$, i.e. $\Q$ is dense in $\R$
	\item $\overline{\N} = \N,~\overline{\Z} = \Z$
	\item $\overline{\R^+} = \R+ \cup \{0\}$
\end{enumerate}

\textbf{Example.} $Y = (0,1] \subseteq \R,~ A = (0,\dfrac{1}{2}) \subseteq Y$

$\overline{A}^Y = \overline{A} \cap Y = [0,\dfrac{1}{2}] \cap (0,1] = (0,\dfrac{1}{2}]$

\subsubsection*{$\S$ 17.3} Limit Points(Accumulation or cluster)

\begin{defn}
	Let $X$ be a topological space. $A \subseteq X$ and $x \in X$. $x$ is a limit point of $A$ if $\forall$ nhd $U$ of $x$, $U \cap A - \{x\} \neq \emptyset$ denote by $A'$ the set of all limit points of $A$ called the derived set of $A$
\end{defn}

\begin{rmk*}
	$x \in A'$,~$x$ may not in $A$
\end{rmk*}

\textbf{Example} In $\R$, we have

\begin{enumerate}[label = $\bullet$]
	\item $[0,1]' = [0,1]$
	\item $\{\dfrac{1}{n}~|~ n \in \N\}' = \{0\}$
	\item $(\{0\} \cup(1,2))' = [1,2]$
	\item $\Q' = \R$
	\item $\N' = \Z' = \emptyset$
	\item $\R'+ = \R+ \cup \{0\} = \overline{\R+}$
\end{enumerate}

\begin{thm*}[17.6]
	Let $X$ be a topological space and $A \subseteq X$. Then $\overline{A} = A \cup A'$
\end{thm*}

\begin{proof}
	Clearly $A \subseteq \overline{A}$ and $A' \subseteq \overline{A} \implies A \cup A' \subseteq \overline{A}$. Conversely, given $x \in \overline{A}$. If $x \in A$, then $x \in A \cup A'$. If $x \notin A'$, then $\forall$ nhd $U$ of $x$, $U \cap A \neq \emptyset \implies U \cap A - \{x\} \neq \emptyset (\because x \notin A) \implies x \in A' \implies x \in A \cup A'$ 
\end{proof}

\begin{cor*}
	$A$ is closed in $X$ iff $A' \subseteq A$
\end{cor*}

\begin{proof}
	$A = \overline{A} = A \cup A'$(trivial)
\end{proof}

\subsubsection*{$\S$ 17.4} Hausdorff Spaces( or $T_2$-spaces)

\textbf{Exmpale} $X = \{a,b,c\},~\T = \{\emptyset,\{a,b\},\{b\},\{b,c\},X\}$ which is a topology on $X$, $\{b\}$ is open in $X$ but $\{b\}$ is not closed. Consider the sequence $\{x_n\}$ in $X$ with $x_n = b \forall n \geq 1$. Then $\{x_n\}$ convergences to any point in $X$.

\begin{defn}
	A topological space $X$ is called a Hausdorff space (or $T_2$ space) if every two distinct points in $X$ can be separated by open sets. i.e. $\forall x_1 \neq x_2$ in $X$, $\exists$ nhd $U_i$ of $x_i,~i = 1,2 \ni U_1 \cap U_2 = \emptyset$
\end{defn}

\begin{thm*}
	Every finite set in $T_2$-space $X$ is closed. In particular, every singleton is closed
\end{thm*}

\begin{proof}
	Given a finite set $F = \{x_1 ,\cdots,x_n\}$ Write $F = \bigcup_{i = 1}^n\{x_i\}$. It suffices to show that every singleton $\{x\}$ is closed in $X$
	
	\begin{eqnarray*}
		\forall y \in X-\{x\} &\implies& y \neq x\\
		&\implies & \exists \text{ nhd } U \text{ of } x \text{ and } V \text{ of } y \ni U \cap V = \emptyset\\
		&\implies & y \in V \subseteq X - \{x\}\\
		& \implies &X - \{x\} \text{ is open in } X\\
		& \implies &\{ x\} \text{ is closed in }X
	\end{eqnarray*}
\end{proof}

\begin{rmk*}
	The converse fails, e.g. In a finite complement topological space $X$, where $X$ is an infinite set, every singleton is closed in $X$, but $X$ is not $T_2$
	
	$\because \forall x \neq y$ and $U$ of $x$ and $V$ of $y$. If $U \cap V~X-(U \cap V) = (X - U) \cup (X - V) \implies X$ is finite$(\rightarrow\leftarrow)$.
\end{rmk*}

\begin{defn}
	A topological space $X$ is said to be $T_1$ if every singleton is closed in $X$.
	
	Equivalently, $\forall x \neq y$ in $X,~ \exists$ a neighborhood $U$ of $x$ such that $y \notin U$ and $\exists$ a neighborhood of $y$ such that $a \notin V$
\end{defn}

\begin{thm*}
	Let $X$ be a $T_1$ space and $A \subseteq X$. Then $x \in A$ iff $\forall$ neighborhood $U$ of $x$, $U \cap A$ is an infinite set
\end{thm*}

\begin{proof}$ $
	\begin{enumerate}
		\item[($\Leftarrow$)] trivial (By definition)
		\item[($\Rightarrow$)] Suppose $x \in A$. If $\exists$ a neighborhood $U$ of $x \ni ~ U\cap A$ is a finite set, so is $U \cap A - \{x\}$, say, $U \cap A - \{x\} = \{x_1,\cdots,x_n\}$.\\
		Since $X$ is $T_1 \{x_1,\cdots,x_n\}$ is closed in $X$\\
		$\implies X - \{x_1,\cdots,x_n\}$ is open\\
		$\implies V \equiv U \cap (X - \{x_1,\cdots,x_n\})$ is a neighborhood of $x$ with $V \cap A - \{x\} = \emptyset (\rightarrow\leftarrow)$ to $x \in A'$
	\end{enumerate}
\end{proof}

\begin{defn}
	A topological space $X$ is said to be $T_0$ if every two distinct points $x$ and $y$ in $X$ one of them has a neighborhood not containing the other one.
\end{defn}

\begin{rmk*}
	$T_1 \implies T_1 \implies T_0$, but the converse fails, e.g. $T_0 \rightarrow T_1$, $X = a,b$ with topology $\T = \{\emptyset,\{a\},X\}$ which is $T_0$ but not $T_1$
\end{rmk*}

\begin{thm*}
	If $X$ is Hausdorff space, then a sequence $\{x_n\}$ in $X$ can converge to at most on point in $X$.
\end{thm*}

\begin{proof}
	If $\{x_n\}$ converges to $x$ and $x'$, $x \neq x'$ choose neighborhood $U$ of $x$ and $U'$ of $x' \ni U \cap U' = \emptyset$. Choose $N >> 0 \ni \forall n \geq N,~x_n \in U \cap U' (\rightarrow \leftarrow)$ to $U \cap U' = \emptyset$
\end{proof}

\begin{thm*}$ $
	\begin{enumerate}
		\item Every order set $X$ with order topology is $T_2$
		\item If $X$ and $Y$ are $T_2$, so is $X \times Y$
		\item If $X$ is $T_2$, then so is it's subspace.
	\end{enumerate}
\end{thm*}

\begin{proof}
	skip in latex :).
\end{proof}

\newpage

\subsection*{$\S$ 18 Continuous Mapping}
















