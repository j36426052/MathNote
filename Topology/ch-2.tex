\section*{2. Topological Space and Continuous functions}

We will introduce some basic topological space.

e.g. Order topology, Product topology, Subspace topology,

Metric topology, (Quotient topology)

\subsection*{$\S$ 12 Topological Spaces} $ $

\begin{defn}
	Let $X$ be a nonempty set $\P(X) = 2^X$ power set of $X$. We say that $\T \subseteq \P(X)$ is a topology on $X$ if
	
	\begin{enumerate}
		\item $\emptyset,X \in \T$
		\item $U_{\alpha} \in \T,~\alpha \in I \implies \bigcup_{\alpha \in I}U_{\alpha} \in \T$
		\item $U_1 , \cdots,U_n \in \T \implies U_1 \cap \cdots \cap U_n \in \T$
	\end{enumerate}
	
	If $\T$ is a topology on $X$, then the pair $(X,\T)$ or simply $X$ is called a topological space and members in $\T$ are called open sets in $X$
\end{defn}

\textbf{Example.}

\begin{enumerate}
\item $X = \{a,b,c\}$
		\begin{enumerate}
		\item The following are topological space on $X$, $\T_1 = \{\emptyset,X\}$,
			
			$\T_2 = \{\emptyset,\{a\},\{a,b\},X\}$, $\T_3 = \P(X)$
		\item The following are not topology on $X$
		
		$\A = \{\emptyset,\{a\},\{b\},X\}$ ($\because \{a\} \cup \{b\} = \{a,b\} \notin \A$)
		
		$\B = \{\emptyset,\{a,b\},\{b,c\},X\}$ ($\because \{a,b\} \cap \{b,c\} = \{b\} \notin \B$)
		\end{enumerate}
		\item Any set with more than $1$ element has at least two topology $\{\emptyset,X\}$(in discrete topology) and $\P(X)$(discrete)\\
		and former is smallest one, another is the largest one.
\end{enumerate}

\begin{defn}
	$\T_{op} = \{\T ~|~ \T \text{ is a topology on }X\}~ \T_1 \leq \T_2 \Leftrightarrow \T_1 \subseteq \T_2$
	
	Claim $"\leq"$ is a partial ordering on $\T_{op}$
	
	\begin{enumerate}[label = $\star$]
		\item Reflexive: $\forall \T \in \T_{op},~\T \leq \T$
		\item Anti-symmetry: $\forall \T_1 , \T_2 \in \T_{op},~\T_1 \leq \T_2$ and $\T_2 \leq \T_1 \implies \T_1 = \T_2$
		\item Transitive: $\forall \T_1,\T_2,\T_3 \in \T_{op}$, $\T_1 \leq \T_2$ and $\T_2 \leq \T_3 \implies \T_1 \leq \T_3$
	\end{enumerate}
\end{defn}

\textbf{Example.} Let $X$ be a set, $\T_f = \{U \subseteq X,~U = \emptyset \text{ or } X-U \text{ is finite }\}$

Then $\T_f$ is a topology on $X$, called the "finite complement topology" on $X$

\begin{proof}$ $
	\begin{enumerate}
		\item $\emptyset,X \in \T_f$ ($\because X - X = \emptyset$)
		\item $U_{\alpha} \in \T_f,~\alpha \in I$\\
		If $\bigcup_{\alpha \in I}U_{\alpha} = \emptyset,$ then $\bigcup_{\alpha \in I}U_{\alpha} \in \T_f$.\\
		If $U_{\alpha \in I}U_{\alpha} \neq \emptyset$, then $\exists \alpha_0 \in I \ni U_{\alpha_0} \neq \emptyset$ and $X-U_{\alpha}$ is finite\\
		$X - \bigcup_{\alpha \in I}U_{\alpha} = \bigcap_{\alpha \in I}(X - U_{\alpha}) \subseteq X - U_{\alpha_0} \implies X - (\bigcup_{\alpha \in I}U_{\alpha})$ is finte $\implies \bigcup_{\alpha \in I}U_{\alpha} \in \T_f$
		\item $U_1,\cdots,U_n \in \T_f$\\
		If $U_1 \cap \cdots \cap U_n = \emptyset$, then $U_1 \cap \cdots \cap U_n \in \T_f$\\
		If $U_1 \cap \cdots \cap U_n \neq \emptyset$, then $X - (U_1 \cap \cdots \cap U_n) = (X - U_1) \cap \cdots \cup (X - U_n)$ is finite since each $X - U_i$ is finite. Thus $U_1 \cap \cdots \cap U_n \in \T_f$
	\end{enumerate} 
	From (1)(2)(3), $\T_f$ is a topology on $X$.
\end{proof}

\begin{rmk*}
	If $X$ is a finite set, then $\T_f$ is the discrete topology on $X$
\end{rmk*}

\textbf{Example.} Let $X$ be a set and $\T_c = \{U \subseteq X~|~ U = \emptyset \text{ or } X - U \text{ is countable }\}$. Then as in example above, $\T_c$ is a topology on $X$, called the countable complement topology on $X$. Moreover, if $X$ is countable, then $\T_c$ is just a discrete topology on $X$

\begin{defn}
	Let $\T$ and $\T'$ be two topologies on $X$. We say that $\T'$ is (strictly) finer then $\T$ or $\T$ is (strictly) coaser that $\T'$ if \\$\T \leq \T'(\T < \T')$, i.e. $\T \subseteq \T'(\T \subsetneqq \T')$
\end{defn}

\begin{rmk*}$ $
	\begin{enumerate}
		\item Two topologies on $X$ need not be comparable
		\item Other terminology, if $\T' \supset T$, $\T'$ is larger(stronger) than $\T$ and $\T$ is smaller(weaker) than $\T$
	\end{enumerate}
\end{rmk*}

\subsection*{$\S$ 13 Bases for a topology}

\begin{defn}
	Let $X$ be a set. A base for a topology on $X$ is a collection $\B \subseteq \P(X)$ satisfying
	
	\begin{enumerate}
		\item $U\B = X$ $(\bigcup \B = \bigcup_{B \in \B}B)$
		\item Given $B_1,B_2 \in \B$ and $x \in B_1 \cap B_2 ~ \exists B_3 \in \B \ni x \in B_3 \subseteq B_1 \cap B_2$
	\end{enumerate}
	Members in $\B$ are called basic open sets in $X$
\end{defn}

Given a base $\B$ for a topology on $X$, we can define the smallest topology $\T$ on $X$ containing $\B$ called the topology on $X$ generated by $\B$.

Usually, there are two ways to describe it

\begin{enumerate}[wide]
	\item[(I)] $\T = \{U \subseteq X , \forall x \in U \exists B \in \B \ni x \in B \subseteq U\}$. Clearly, $\B \subseteq \T$
	
	\begin{enumerate}
		\item $\emptyset , X \in \T$(by the definition of bases (1))
		\item $U_{\alpha} \in \T,~\alpha \in I \implies \bigcup_{\alpha \in I}U_{\alpha} \in \T$. Given $x \in \bigcup_{\alpha \in I}U_{\alpha},x \in U_{\alpha_0}$ for some $\alpha_0 \in I,~\exists B \in \B \ni x \in B \subseteq U_{\alpha_0} \subseteq \bigcup_{\alpha \in I}U_{\alpha}$
		\item $U_1,\cdots,U_n \in \T \implies U_1 \cap \cdots \cap U_n \in \T$. By induction on $n$, we only prove $n=2$. Given $x \in U_1 \cap U_2,~x \in U_1$ and $x \in U_2$\\
		$\implies \exists B_1,B_2 \in \B \ni x \in B_1 \subseteq U_1$ and $X$ in $ \B_2 \subseteq U_2 \implies$\\
		$x \in B_1 \cap B_2 \subseteq U_1 \cap U_2 \implies \exists B_3 \in \B \ni x \in B_3 \subseteq B_1 \cap B_2 \subseteq U_1 \cap U_2 \implies U_1 \cap U_2 \in \T$
	\end{enumerate}
	
	\item[(II)]  $\T' = \{\bigcup\A ~|~ \A \subseteq \B\} = \{\bigcup_{\alpha \in I}A_{\alpha} ~|~ A_{\alpha} \in \B\}$
	
	Clearly, $\B \subseteq \T'$(only choose one element in $\B$)
	
	\begin{enumerate}
		\item $\emptyset,X \in \T'$(trivial)
		\item $U_{\alpha} \in \T',~\alpha \in I \implies \bigcup_{\alpha \in I}U_{\alpha} \in \T'$\\
		$\forall \alpha \in I,U_{\alpha} = \bigcup_{\beta \in I_{\alpha}}A_{\beta}$. Then $\bigcup_{\alpha \in I}U_{\alpha} = \bigcup_{\alpha \in I}\bigcup_{\beta \in I_{\alpha}}A_{\beta} \implies \bigcup_{\alpha \in I}U_{\alpha} \in \T'$
		\item $U_1,\cdots,U_n \in \T' \implies U_1 \cap \cdots \cap U_n \in \T'$. By induction on $n$, we only to prove that $n = 2$. For $i=1,2,\cdots$, $U_i = \bigcup_{\alpha \in I_j}A_{\alpha}$.\\
		$U_1 \cap U_2 = \bigcup_{\alpha \in I_2}(A_{\beta}^1 \cap A_{\alpha}^2) $. $\forall x \in U_1 \cap U_2,~ x \in A'_{\beta} \cap A_{\alpha}^2 \implies U_1 \cap U_2 = \bigcup_{x \in U_1 \cap U_2}B_X \in \T'$
	\end{enumerate}
	
	\item[(III)] $\T = \T'$
		\begin{enumerate}
		\item[($\subseteq$)] Given $U \in \T,~\forall x \in U,~\exists B_x \in \B \ni x \in B_x \subseteq U \implies U = \bigcup_{x \in U}B_x \in \T'$
		\item[($\supseteq$)] Given $U \in \T' ~ U = \bigcup_{\alpha \in I}A_{\alpha},A_{\alpha} \in \B$\\
		$\forall x \in U,x \in A_{\alpha}$ for some $\alpha \in I$ and $A_{\alpha} \in \B$, i.e. $X \in A_{\alpha} \in U$ and $A_{\alpha} \in \B \implies U \in \T$. Hence $\T = \T'$
		\end{enumerate}
\end{enumerate}

\textbf{Example.}

\begin{enumerate}
	\item Let $\B$ be the collection of all open balls in $\R^n$. Then $\B$ is a base for a topology on $\R^n$, namely, then Euclidean topology on $\R^n$
	\item Let $\B'$ be the collection of all $n-$dimentional open intervals in $\R$. Then $\B'$ is a base for a topology on $\R^n$. In fact, $\beta$ and $\beta'$ generate the same topology on $\R^n$
\end{enumerate}

\begin{lmma*}
	Let $X$ be a set, let $\B$ be a basis for a topology $\T$ on $X$. $\T$ equals the collection if all unions of elements of $\B$.
\end{lmma*}

\begin{lmma*}
	Let $X$ be a topological space and $\mathscr{C}$ be a collection of open sets of $X \ni \forall$ open set $U$ in $X$ and $\forall x \in U ~\exists C \in \mathscr{C} \ni x \in C \subseteq U$. Then $\mathscr{C}$ is a base for the topology of $X$.
\end{lmma*}

\begin{proof}
	\begin{enumerate}
		\item $\bigcup \mathscr{C} = X$\\
		Since $X$ is open $\forall x \in X,~\exists C_x \in \mathscr{C} \ni x \in C_x \subseteq X \implies x \in \bigcup \mathscr{C} \implies X = \bigcup \mathscr{C}$
		\item Given $C_1,C_2 \in \mathscr{C}$ and $x \in C_1 \cap C_2$. Since $C_1 \cap C_2$ is open, $\exists C \in \mathscr{C} \ni x \in C \subseteq C_1 \cap C_2$, $\therefore \mathscr{C}$ is a base for a topology of $X$
	\end{enumerate}
\end{proof}

\begin{rmk*}
	Let $\T$ be the original topology on $X$ and $\T'$ be the topology generated by $\mathscr{C}$. Then $\T = \T'$
\end{rmk*}

\begin{proof}$ $

	\begin{enumerate}
		\item[($\subseteq$)] Given $U \in \T$, $\forall x \in U \exists C \in \mathscr{C} \ni x \in C \subseteq U \implies U \in \T'$
		\item[($\supseteq$)] Given $v \in \T'$, by lemma, $V = \bigcup \A$ for some $\
		A \subseteq \mathscr{C}$. Since $\mathscr{C} \subseteq \T$, $\A \subseteq \T$, $\therefore V = \bigcup \A \in \T$
	\end{enumerate}
\end{proof}

\begin{lmma*}
	Let $\B$ and $\B'$ be bases for the topology $\T$ and $\T'$ on $X$ respective TFAE
	
	\begin{enumerate}
		\item $\T$ is finer that $\T$ i.e. $\T \subseteq \T'$
		\item $\forall x \in X$ and $B \in \B$ with $x \in B$, $\exists B' \in \B \ni x \in B' \subseteq B$
	\end{enumerate}
\end{lmma*}

\begin{proof}$ $
	
	\begin{enumerate}
		\item[$(a)\Rightarrow(b)$]
		Suppose $\T \subseteq \T'$. Given $x \in X$ and $B \in \B$ with $x \in B$. Since $\T \subseteq \T',~B \in \T,\exists B' \in \B \ni x \in B' \subseteq B$ 
		\item[$(b)\Rightarrow(a)$] 
		Suppose $(b)$ holds. Given $U \in \T,~\forall x \in U,~\exists B_x \in \B \ni x\in B_x \subseteq U.$ By $(b),~\exists B'_x \in \B \ni x\in B_x' \subseteq B_x \subseteq U \implies U \in \T'$
	\end{enumerate}
	
\end{proof}

\textbf{Example. } In $\S 13$, example 1,2

$\B:$ all open balls in $\R^n$ for a topology on $\R^n$

$\B':$all open intervals in $\R^n$ for a topology on $\R^n$

By lemma above, they generate the same Euclidean topology on $\R^n$

We now define $3$ topologies on the real line $\R$
\begin{defn}$ $
	\begin{enumerate}
		\item $\B = \{(a,b)~|~-\infty < a < b < \infty\}$: the collection of all open intervals in $\R$ which is the base for the usual topology on $\R$
		\item $\B' = \{[a,b)~|~-\infty < a < b < \infty\}$ the collection of all closed-open interval in $\R$, which is also a base for a topology of $\R$ called the lower limit topology on $\R$. We denote it by $\R_l$
		\item Let $K = \{\frac{1}{n}~|~n \in \N\}$ and $\B'' = \{B \subseteq \R ~|~ B = (a,b) \text{ or }B=(a,b) - K \text{ for } -\infty < a < b < \infty\}$. Claim: $\B'$ is a base for a topology on $\T$
			\begin{enumerate}[label = $\star$]
			\item Clearly, $U \B'' = \R$
			\item Given $B_1,B_2 \in \B''$ and $x \in B_1 \cap B_2$. We have $4$ cases:
				\begin{enumerate}
				\item $B_1$ and $B_2$ are open intervals which is clearly.
				\item $B_1 = (a,b)$ and $B_2 = (c,d) - K$. Let $\alpha = \max\{a,c\}$ and $\beta = \min\{b,c\}$. $x \in (\alpha,\beta) - K \subseteq B_1 \cap B_2$ and $(\alpha , \beta) - K \in \B''$
				\item (3)(4) similarly
				\end{enumerate} 
			\end{enumerate} 
			The topology on $\R$ generated by $B'$ is called the $K-$topology on $\R$ and denoted $\R_k$
	\end{enumerate}
\end{defn}

\begin{lmma*}
	The topologies of $\R_l$ and $\R_k$ are strictly finer than the Euclidean topology of $\R$ but are not comparable with one another
\end{lmma*}

\begin{proof}
	Let $\T,\T'$ and $\T''$ be the topologies of $\R,\R_l,\R_k$ generated by $\B,\B',\B''$ respectly. We use lemma above to prove it.
	\begin{enumerate}[label = $\star$]
		\item $\T \subsetneqq \T'$
		Given $(a,b) \in \B$ and $x \in (a,b)$. We have $[x,b) \in \B'$ with $x \in [x,b) \subseteq (a,b)$. By lemma, $\T \subseteq \T'$, $\forall a<b,~[a,b) \in \B'$ so $[a,b) \in \T',$ but $[a,b)' \notin \T$
		\item Clearly, $\T \subseteq \T''$ by $\B \subseteq \B''$. Moreover $B'' = (-1,1) - K \in \B''$, so $B'' \in \T''$ but $B'' \notin \T$.
		\item $\T'$ and $\T''$ are not comparable\\
		$(-1,1) - K \in \T''$, but $(-1,1) - K \notin \T'(\because$ not $[0,c) \in \B' \ni 0 \in [0,c) \subseteq (-1,1) - K$). $[0,1) \in \T$ but no $\B'' \in \B'' \ni 0 \in B'' \subseteq [0,1)$
	\end{enumerate} 
\end{proof}

\begin{defn}
	A subbase $\mathscr{S}$ for a topology on $X$ is a collection of subsets of $X$ with $\bigcup \mathscr{S} = X$ and elements in $\mathscr{S}$ are calle subbasic open sets in $X$
\end{defn}

Given subbase on $X$

$$\B = \{S_1 \cap \cdots \cap S_k,~k \in \N,S_1,\cdots,S_k \in S\}$$

Claim $\B$ is a base for a topology on $X$

\begin{defn}
	The topology on $X$ generated by a subbase $\mathscr{S}$ is defined to be the topology generated by the base $\B$.
\end{defn}

\subsection*{The Order Topology}

(which provides many counterexample in topology)

\begin{defn}
	A relation $C$ on a set is called an "order relation"  (or a simple order) if it satisfies
	\begin{enumerate}
		\item Comparable: $\forall x \neq y$ in $X$ either $xCy$ or $yCx$
		\item Non-reflexivity: no $xCx$
		\item Transitivity: $xCy$ and $yCz \implies xCz$
	\end{enumerate}
\end{defn}

Given a simple order set $(X,<)$ and $a,b \in X$ with $a<b$( Note: $a \leq b$ means $a < b$ or $a = b$). We can define:

$(a,b) = \{x \in X ~|~ a < x < b\}$ open interval

$(a,b] = \{x \in X ~|~ a < x \leq b\}$ open interval

$[a,b) = \{x \in X ~|~ a \leq x < b\}$ open interval

$[a,b] = \{x \in X ~|~ a \leq x \leq b\}$ open interval


We assume that $|X| \geq 2$. Let $\B$ be the collection of all subsets of the following types

\begin{enumerate}
	\item All open intervals $(a,b)$ in $X$
	\item All intervals of the forms $[a_0,b)$ where $a_0$ is the smallest elements of $X$
	\item All intervals of the forms $(a,b_0]$ where $b_0$ is the largest elements of $X$
\end{enumerate}

\begin{defn}
	The topology generated by $\B$ is called the order topology on $X$
\end{defn}

\textbf{Example.}

\begin{enumerate}
	\item If $X$ is an order set and $T \subseteq X,$ then so is $Y$
	\item In $\R$ we give the usually ordering and the order topology on $\R$ is the usual topology on $\R$
	\item In $\R^{\star} = \R \cup \{-\infty , \infty\}$ with the usual ordering is an order set.
	\item In $\R \times \R$ with the dictionary order is an order set whose basis for the order topology is of the form
	\item $\N$ with the usual ordering is an order set with the smallest element $1$. What is the order topology?
		\begin{enumerate}
		\item[$\star$] $[1,b):b\in \N$ and $(a,b),~a<b$. In particular, $\{1\} = [1,2)$ and $\{n\} = (n-1,n+1),n>1$ are basic open sets in $\N$
		
		$\therefore$ the order topology on $\N$ is the discrete topology on $\N$
		\end{enumerate}
	\item The set $X = \{1,2\} \times \N = \{1 \times n\}^{\infty}_{n = 1} = a_n  \cup  b_n = \{2 \times n \}^{\infty}_{n = 1}$ in the dictionary order with the smallest element $1 \times 1$. The order topology on $X$ is not discrete topology on $X$
	
	$X:a_1,a_2,\cdots,b_1,b_2,\cdots$, $a_i < a_{i+1},~b_j < b_j + 1,~a_i < b_j$
	\begin{enumerate}[label = $\star$]
		\item $\{a_1\} = [a_1,a_2)$
		\item $\{a)n\} = (a_{n-1},a_{n+1}),n \geq 2$
		\item $\{b_n\} = (b_{n-1},,b_{n+1}),n\geq 2$
	\end{enumerate}
	But $\{b_1\}$ is not open, $\because b_1$ is not the smallest elements any basic open set in the order topology containing $b_1$ must of the form $(a_l,b_j)$ for some $l \geq 1$ and $j > 1$
\end{enumerate}


\begin{defn}
	Let $X$ be an ordered set and $a \in X$. We define the rays determine by $a$
	
	\begin{enumerate}[label = $\star$]
		\item $(a,\infty) = \{x \in X ~|~ x > a\}$
		\item $(-\infty,a) = \{x \in X ~|~ x < a\}$
		\item $[a,\infty) = \{x \in X ~|~ x \geq a\}$
		\item $(\infty,a] = \{x \in X ~|~ x \leq a\}$
	\end{enumerate}
\end{defn}

Some facts:

\begin{enumerate}
	\item open rays in $X$ are open in the order topology of $X$. In fact, $(a,\infty) = (a,b_0]$ if $X$ has the largest element which is a basic open set in the order topology of $X$. If $X$ has no largest element, then $(a,\infty) = \bigcup_{a < x}(a,x)$ which is open in the order topology of $X$
	\item closed rays is close
	\item The order topology of $X$ is contained in the topology on $X$ generated by open rays in $X$. $\because (a,b) = (a,\infty) \cap (-\infty,b)$. \\
	If $X$ has the smallest element $a_0,~ [a_0,b) = (-\infty,b)$\\
	If $X$ has the largest element $b_0,~(a,b_0] = (a,\infty)$
\end{enumerate}







