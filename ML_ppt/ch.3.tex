%%%%%%%%%%%%%%%%%%%%%%%%%%%%%%%%%%%%%%%%%
% Beamer Presentation
% LaTeX Template
% Version 2.0 (March 8, 2022)
%
% This template originates from:
% https://www.LaTeXTemplates.com
%
% Author:
% Vel (vel@latextemplates.com)
%
% License:
% CC BY-NC-SA 4.0 (https://creativecommons.org/licenses/by-nc-sa/4.0/)
%
%%%%%%%%%%%%%%%%%%%%%%%%%%%%%%%%%%%%%%%%%

%----------------------------------------------------------------------------------------
% PACKAGES AND OTHER DOCUMENT CONFIGURATIONS
%----------------------------------------------------------------------------------------

\documentclass[
 11pt, % Set the default font size, options include: 8pt, 9pt, 10pt, 11pt, 12pt, 14pt, 17pt, 20pt
 %t, % Uncomment to vertically align all slide content to the top of the slide, rather than the default centered
 %aspectratio=169, % Uncomment to set the aspect ratio to a 16:9 ratio which matches the aspect ratio of 1080p and 4K screens and projectors
]{beamer}

\graphicspath{{Images/}{./}} % Specifies where to look for included images (trailing slash required)

\usepackage{booktabs} % Allows the use of \toprule, \midrule and \bottomrule for better rules in tables

%----------------------------------------------------------------------------------------
% SELECT LAYOUT THEME
%----------------------------------------------------------------------------------------

% Beamer comes with a number of default layout themes which change the colors and layouts of slides. Below is a list of all themes available, uncomment each in turn to see what they look like.

%\usetheme{default}
%\usetheme{AnnArbor}
%\usetheme{Antibes}
%\usetheme{Bergen}
%\usetheme{Berkeley}
%\usetheme{Berlin}
%\usetheme{Boadilla}
%\usetheme{CambridgeUS}
%\usetheme{Copenhagen}
%\usetheme{Darmstadt}
%\usetheme{Dresden}
\usetheme{Frankfurt}
%\usetheme{Goettingen}
%\usetheme{Hannover}
%\usetheme{Ilmenau}
%\usetheme{JuanLesPins}
%\usetheme{Luebeck}
%\usetheme{Madrid}
%\usetheme{Malmoe}
%\usetheme{Marburg}
%\usetheme{Montpellier}
%\usetheme{PaloAlto}
%\usetheme{Pittsburgh}
%\usetheme{Rochester}
%\usetheme{Singapore}
%\usetheme{Szeged}
%\usetheme{Warsaw}

%----------------------------------------------------------------------------------------
% SELECT COLOR THEME
%----------------------------------------------------------------------------------------

% Beamer comes with a number of color themes that can be applied to any layout theme to change its colors. Uncomment each of these in turn to see how they change the colors of your selected layout theme.

%\usecolortheme{albatross}
%\usecolortheme{beaver}
%\usecolortheme{beetle}
%\usecolortheme{crane}
\usecolortheme{dolphin}
%\usecolortheme{dove}
%\usecolortheme{fly}
%\usecolortheme{lily}
%\usecolortheme{monarca}
%\usecolortheme{seagull}
%\usecolortheme{seahorse}
%\usecolortheme{spruce}
%\usecolortheme{whale}
%\usecolortheme{wolverine}

%----------------------------------------------------------------------------------------
% SELECT FONT THEME & FONTS
%----------------------------------------------------------------------------------------

% Beamer comes with several font themes to easily change the fonts used in various parts of the presentation. Review the comments beside each one to decide if you would like to use it. Note that additional options can be specified for several of these font themes, consult the beamer documentation for more information.

\usefonttheme{default} % Typeset using the default sans serif font
%\usefonttheme{serif} % Typeset using the default serif font (make sure a sans font isn't being set as the default font if you use this option!)
%\usefonttheme{structurebold} % Typeset important structure text (titles, headlines, footlines, sidebar, etc) in bold
%\usefonttheme{structureitalicserif} % Typeset important structure text (titles, headlines, footlines, sidebar, etc) in italic serif
%\usefonttheme{structuresmallcapsserif} % Typeset important structure text (titles, headlines, footlines, sidebar, etc) in small caps serif

%------------------------------------------------

%\usepackage{mathptmx} % Use the Times font for serif text
%\usepackage{palatino} % Use the Palatino font for serif text

\usepackage{helvet} % Use the Helvetica font for sans serif text
\usepackage[default]{opensans} % Use the Open Sans font for sans serif text
%\usepackage[default]{FiraSans} % Use the Fira Sans font for sans serif text
%\usepackage[default]{lato} % Use the Lato font for sans serif text

%----------------------------------------------------------------------------------------
% SELECT INNER THEME
%----------------------------------------------------------------------------------------

% Inner themes change the styling of internal slide elements, for example: bullet points, blocks, bibliography entries, title pages, theorems, etc. Uncomment each theme in turn to see what changes it makes to your presentation.

%\useinnertheme{default}
\useinnertheme{circles}
%\useinnertheme{rectangles}
%\useinnertheme{rounded}
%\useinnertheme{inmargin}

%----------------------------------------------------------------------------------------
% SELECT OUTER THEME
%----------------------------------------------------------------------------------------

% Outer themes change the overall layout of slides, such as: header and footer lines, sidebars and slide titles. Uncomment each theme in turn to see what changes it makes to your presentation.

%\useoutertheme{default}
\useoutertheme{infolines}
%\useoutertheme{miniframes}
%\useoutertheme{smoothbars}
%\useoutertheme{sidebar}
%\useoutertheme{split}
%\useoutertheme{shadow}
%\useoutertheme{tree}
%\useoutertheme{smoothtree}

%\setbeamertemplate{footline} % Uncomment this line to remove the footer line in all slides
%\setbeamertemplate{footline}[page number] % Uncomment this line to replace the footer line in all slides with a simple slide count

%\setbeamertemplate{navigation symbols}{} % Uncomment this line to remove the navigation symbols from the bottom of all slides


%%%%%%%%%%%%%%%%%%%%%%%%%%%%%%%%%%%%%%%%%%%%%%%%

\usepackage{xeCJK}   % Chinese input settings
\setCJKmainfont{標楷體} % Windows使用者請使用這行
\defaultCJKfontfeatures{AutoFakeBold=0.5,AutoFakeSlant=0} %以後不用再設定粗斜

\usepackage{amsmath, amsthm, amsfonts, amssymb}
\everymath{\displaystyle}
%%%%%%%%%%%%%%%%%%%%%%%%%%%%%% Textclass specific LaTeX commands.
\theoremstyle{plain}
\newtheorem{thm}{\protect\theoremname}[section]
\newtheorem{cor}[thm]{Corollary}
\newtheorem{lmma}[thm]{Lemma}
\newtheorem*{defn}{\underline{Definition}}
\newtheorem*{ex*}{Example}
\newtheorem*{sol*}{Solution}
\newtheorem*{thm*}{Theorem}
\newtheorem*{lmma*}{Lemma}
\newtheorem*{rmk*}{Remark}
\newtheorem*{pf*}{\underline{\textbf{Proof\ }}}

%%%%%%%%%%%%%%%%%%%%%%%%%%%%%% User specified LaTeX commands.
\renewcommand{\P}{\mathscr{P}}
\newcommand{\B}{\mathscr{B}}
\newcommand{\A}{\mathscr{A}}
\newcommand{\C}{\mathbb{C}}
\newcommand{\CC}{\mathscr{C}}
\newcommand{\R}{\mathbb{R}}
\newcommand{\Q}{\mathbb{Q}}
\newcommand{\Z}{\mathbb{Z}}
\newcommand{\N}{\mathbb{N}}
\newcommand{\X}{\mathcal{X}}
\newcommand{\T}{\mathscr{T}}
\newcommand{\arbuni}{\bigcup_{\alpha\in I}}
\newcommand{\finint}{\bigcap_{i=1}^n}
\newcommand{\Ua}{{\textsc{U}_\alpha}}
\newcommand{\Ui}{\textsc{U}_i}
\newcommand{\pair}[2]{\left( \,#1\,,\,#2\,\right) }
\newcommand{\sett}[1]{\left\{#1 \right\}}
\newcommand{\dint}[2]{\int_{#1}^{#2}}
\DeclareMathOperator*{\esssup}{ess\,sup}



\title{Ch.3 Transformation}
\institute{An Introduction to Optimization}
\author{QSnake}
\date{2/24,2023}
\begin{document}

% 標題頁面
\begin{frame}
	\titlepage
\end{frame}
% 大綱頁面
\begin{frame}
	\tableofcontents
\end{frame}

%%%%%%%%常用指令區
% \begin{frame}  \end{frame}	% 開始簡報
% \section{}					% 開始某一章節
% \includegraphics[scale=]{path}% 插入圖片
% \frametitle{}					% 左上角那個標題

%%%%%%%%%%%%%%%%%%%%%%%%%%%%%%

% Presentation structure

\begin{frame}
	\section{Little Review}
	\begin{center}
	\textbf{Little Review}
	\end{center}
\end{frame}

\begin{frame}
	\frametitle{Definition inner product}

	The inner product is a real-valued function $\left< \cdot , \cdot \right>:\R^n \times \R^n \rightarrow \R$ having the following properties:

	\begin{enumerate}
		\item Positivity: $\left<x,x\right> \geq 0,~\left<x,x\right> = 0$ if and only if $x = 0$
		\item Symmetry: $\left<x,y\right> = \left<y,x\right>$
		\item Additivity: $\left<x+y,z\right> = \left<x,z\right> + \left<y,z\right>$.
		\item Homogeneity: $\left< rx,y\right> = r\left<x,y\right>$ for every $r \in \R$
	\end{enumerate}
\end{frame}

\begin{frame}
	\frametitle{Definition of Norm}
	The Euclidean norm of a vector $||x||$ has the following properties:

	\begin{enumerate}
	\item Positivity: $||x|| \geq 0,~||x||=0$ if and only if $x = 0$
	\item Homogeneity: $||rx|| = |r|||x||,r\in \R$
	\item Triangle inequality: $||x + y|| \leq ||x|| + ||y||$
	\end{enumerate}
\end{frame}

\begin{frame}
	\frametitle{Orthogonal}

	Let $V$ be an inner product space. Vectors $x$ and $y$ in $V$ are orthogonal if $\left< x, y \right> = 0$. A subset $S$ of $V$ is orthogonal.
\end{frame}

\begin{frame}
	\frametitle{Review}
	A function $L: \R^n \rightarrow \R^m$ is called a linear transformation if:
	\begin{enumerate}
	\item $L(ax) = aL(x)$ for every $x \in \R^n$ and $a \in \R.$
	\item $L(x + y) = L(x) + L(y)$ for every $x,y \in \R^n$
	\end{enumerate}

	and the linear transformation $L$ can be represented by a matrix.(Friedberg Ch.2)
\end{frame}

%\begin{frame}
%	\frametitle{Transformation matrix}
%
%	Let $\{e_1,e_2,\cdots,e_n\}$ and $\{e_1',e_2',\cdots,e_n'\}$ be two bases for $\R^n$. Define the matrix
%
%	$$T = [e_1,e_2,\cdots,e_n]^{-1}[e_1',e_2',\cdots,e_n']$$
%
%	We call $T$ the transformation matrix from $\{e_1,e_2,\cdots,e_n\}$ and $\{e_1',e_2',\cdots,e_n'\}$
%\end{frame}

%\begin{frame}
%	\frametitle{Transformation matrix}
%	let $A$ be its representation with respect to $\{e_1,\cdots,e_n\}$ and $B$ its representation with respect to $\{e_1',\cdots,e_n'\}$. Let $y = Ax$ and $y' = Bx'$. Therefore, $y' = Ty = TAx = Bx' = BTx$, and hence $TA = BT$, or $A = T^{-1}BT$.
%\end{frame}

%\begin{frame}
%	\frametitle{Terminology}
%	Two $n \times n$ matrices $A$ and $B$ are similar if there exists a nonsingular matrix $T$ such that $A = T^{-1}BT$.
%\end{frame}

\begin{frame}
	\frametitle{Adjoint Operator}

	$\left< Ax,x \right> = \left< x, A^*x\right>$

	if A is a real matrix

	$\left< Ax,x \right> = \left< x, A^T\right>$
\end{frame}

\begin{frame}
	\section{Eigenvalues and Eigenvectors}
	\begin{center}
	\textbf{Eigenvalues and Eigenvectors}
	\end{center}
\end{frame}

\begin{frame}
	\frametitle{Definition}
	Let $A$ be an $n \times n$ real square matrix. A scalar $\lambda$ and a \alert{non-zero} vector $v$ satisfying the equation $Av = \lambda v$ are said to be, respectively, an \textbf{eigenvalue} and an \textbf{eigenvector} of $A$.
\end{frame}

\begin{frame}
	\frametitle{Property}
	For $\lambda$ to be an eigenvalue it is necessary and sufficient for the matrix $\lambda I - A$ to be singular, i.e. the \textbf{characteristic polynomial} of the matrix $A$ equal $0$.
\end{frame}

\begin{frame}
	\frametitle{Theorem}
	Suppose that the characteristic equation $\det [\lambda I - A] = 0$ has $n$ distinct roots $\lambda_1,\lambda_2,\cdots,\lambda_n$. Then, there exist $n$ \textbf{linearly independent} vectors $v_1,\cdots,v_n$ such that

	$$Av_i = \lambda_iv_i,\hspace{10pt}i=1,2,\cdots,n$$
\end{frame}

% \begin{frame}
% 	\frametitle{proof of Thm 3.1}

% 	Use the induction to proof it.(Theorem 5.5 in Fredberg)
% \end{frame}

\begin{frame}
	\frametitle{Theorem}
	All eigenvalues of a real \textbf{symmetric} matrix are \textbf{real}.
\end{frame}

% \begin{frame}
% 	\frametitle{proof of Thm 3.2}
% 	Choose an eigenvalue and non-zero eigenvector, and use the adjoint operator to check it.
% \end{frame}

\begin{frame}
	\frametitle{Theorem}
	Any real symmetric $n \times n$ matrix has a set of $n$ eigenvectors that are \textbf{mutually orthogonal}.
\end{frame}

% \begin{frame}
% 	\frametitle{proof (n distinct case)}

% 	also use the inner product and adjoint operator to check

% \end{frame}

\begin{frame}
	\section{Orthogonal Projections}
	\begin{center}
	\textbf{Orthogonal Projections}
	\end{center}
\end{frame}

\begin{frame}

	\frametitle{orthogonal complement}
	If $V$ is a subspace of $\R^n$, then the orthogonal complement of $V$, denoted $V^{\perp}$, consists of all vectors that are orthogonal to every vector in $V$. Thus,

	$$V^{\perp} = \{x:\left<v,Tx\right>=0 \forall v\in V\}$$
\end{frame}

\begin{frame}
	\frametitle{orthogonal decomposition}
\end{frame}

\begin{frame}
	\frametitle{orthogonal projector}

	We say that a linear transformation $P$ is an \textbf{orthogonal projector} onto $V$ if for all $x \in \R^n$, we have $Px \in V$ and $x - Px \in V^{\perp}$
\end{frame}

\begin{frame}
	\frametitle{Example of orthogonal projector}

	we consider $\R^2$ here
\end{frame}

\begin{frame}
	\frametitle{Theorem}

	Let $A$ be a given matrix.

	Then, $R(A)^{\perp} = N(A^{T})$ and $N(A)^{\perp} = R(A^T)$
\end{frame}

\begin{frame}
	\frametitle{Theorem}
	A matrix $P$ is an orthogonal projector [onto the subspace $V = R(P)$] if and only if $P^2 = P = P^T$
\end{frame}

\begin{frame}
	\section{Quadratic Forms}
	\begin{center}
	\textbf{Quadratic Forms}
	\end{center}
\end{frame}

\begin{frame}
	\frametitle{Quadratic Forms}
	A quadratic form $f:\R^n \rightarrow \R$ is a function

	$$f(x) = x^TQx = <x,Qx>$$

	where $Q$ is an $n \times n$ real matrix.
\end{frame}

\begin{frame}
	\frametitle{the generality of quadratic form}
	There is no loss of generality in assuming $Q$ t be symmetric: $Q = Q^T$. For if the matrix $Q$ is not symmetric, we can always replace it with the symmetric matrix

	$$Q_0 = Q_0^T = \dfrac{1}{2}\left( Q + Q^T \right)$$
\end{frame}

\begin{frame}
	\frametitle{reason}
\end{frame}

\begin{frame}
	\frametitle{terminology}
	\begin{enumerate}
	\item positive definite: $x^TQx > 0$ for all $x$ non-zero vectors $x$.
	\item positive semidefinite if $x^TQx \geq 0$ for all $x$
	\end{enumerate}

	and the negative is similar
\end{frame}

\begin{frame}
	\frametitle{principal minors and leading principal minor}
\end{frame}

\begin{frame}
	\frametitle{Sylvester's Criterion}
	A quadraic form $x^TQx,~Q = Q^T$, is positive definite if and only if the leading principal minors of $Q$ are positive.
\end{frame}

\begin{frame}
	\frametitle{Example}

	Consider $Q = \left[ \begin{matrix}
	1 & 0\\
	-4 & 1
	\end{matrix} \right]$
\end{frame}

\begin{frame}
	\frametitle{Theorem 3.7}

	A symmetric matrix $Q$ is positive definite (or positive semidefinite) if and only if all eigenvalue of $Q$ are positive (or nonnegative).
\end{frame}

\begin{frame}
	\section{Matrix Norms}
	\begin{center}
	\textbf{Matrix Norms}
	\end{center}
\end{frame}


\begin{frame}
	\frametitle{Frobenius norm}

	$$||A||_F = \left( \sum\limits_{i=1}^{m}\sum\limits_{j=1}^{n}\left(a_{ij}\right)^2\right)^{\frac{1}{2}}$$

	where $A \in \R^{m \times n}$, and clearly it's a norm.

\end{frame}

\begin{frame}
	\frametitle{matrix induced norms}

	In many problems, both matrices and vectors appear simultaneously.

\end{frame}

\begin{frame}
	\frametitle{induced norms}

	We say that the matrix norm is induced by the given vector norms if for any matrix $A \in \R^{m \times n}$ and any vector $x \in \R^n$, the following inequality is satisfied:

	$$||Ax||_{(m)} \leq ||A||||x||_{(n)}$$

	We can define an induced matrix norm as

	$$||A|| = \max\limits_{||x||_{(n)}=1}||Ax||_{(m)}$$
\end{frame}

\begin{frame}
	\frametitle{proof of matrix induced norms is norms}
\end{frame}

\begin{frame}
	\frametitle{Theorem 3.8}

	Let
	$$||x|| = \left( \sum\limits^n_{k=1}|x_k|^2\right)^{1/2} = \sqrt{\left( x,x\right>}$$

	The matrix norm induced by this vector norm is

	$$||A|| = \sqrt{\lambda_1}$$

	where $\lambda_1$ is the largest eigenvalue of the matrix $A^TA$
\end{frame}

\begin{frame}
	\frametitle{Rayleigh's Inequalities}

	If an $n \times n$ matrix $P$ is real symmetric positive definite, then

	$$\lambda_{\min}(P)||x||^2 \leq <x,Px> \leq \lambda_{\max}(P)||x||^2$$

	where $\lambda_{\min}(P)$ denotes the smallest eigenvalue of $P$, and $\lambda_{\max}(P)$ denotes the largest eigenvalue of $P$.

\end{frame}

\begin{frame}
	\begin{center}
	\textbf{Conclustion}
	\end{center}
\end{frame}
\end{document}