%%%%%%%%%%%%%%%%%%%%%%%%%%%%%%%%%%%%%%%%%
% Beamer Presentation
% LaTeX Template
% Version 2.0 (March 8, 2022)
%
% This template originates from:
% https://www.LaTeXTemplates.com
%
% Author:
% Vel (vel@latextemplates.com)
%
% License:
% CC BY-NC-SA 4.0 (https://creativecommons.org/licenses/by-nc-sa/4.0/)
%
%%%%%%%%%%%%%%%%%%%%%%%%%%%%%%%%%%%%%%%%%

%----------------------------------------------------------------------------------------
% PACKAGES AND OTHER DOCUMENT CONFIGURATIONS
%----------------------------------------------------------------------------------------

\documentclass[
 11pt, % Set the default font size, options include: 8pt, 9pt, 10pt, 11pt, 12pt, 14pt, 17pt, 20pt
 %t, % Uncomment to vertically align all slide content to the top of the slide, rather than the default centered
 %aspectratio=169, % Uncomment to set the aspect ratio to a 16:9 ratio which matches the aspect ratio of 1080p and 4K screens and projectors
]{beamer}

\graphicspath{{Images/}{./}} % Specifies where to look for included images (trailing slash required)

\usepackage{booktabs} % Allows the use of \toprule, \midrule and \bottomrule for better rules in tables

%----------------------------------------------------------------------------------------
% SELECT LAYOUT THEME
%----------------------------------------------------------------------------------------

% Beamer comes with a number of default layout themes which change the colors and layouts of slides. Below is a list of all themes available, uncomment each in turn to see what they look like.

%\usetheme{default}
%\usetheme{AnnArbor}
%\usetheme{Antibes}
%\usetheme{Bergen}
%\usetheme{Berkeley}
%\usetheme{Berlin}
%\usetheme{Boadilla}
%\usetheme{CambridgeUS}
%\usetheme{Copenhagen}
%\usetheme{Darmstadt}
%\usetheme{Dresden}
\usetheme{Frankfurt}
%\usetheme{Goettingen}
%\usetheme{Hannover}
%\usetheme{Ilmenau}
%\usetheme{JuanLesPins}
%\usetheme{Luebeck}
%\usetheme{Madrid}
%\usetheme{Malmoe}
%\usetheme{Marburg}
%\usetheme{Montpellier}
%\usetheme{PaloAlto}
%\usetheme{Pittsburgh}
%\usetheme{Rochester}
%\usetheme{Singapore}
%\usetheme{Szeged}
%\usetheme{Warsaw}

%----------------------------------------------------------------------------------------
% SELECT COLOR THEME
%----------------------------------------------------------------------------------------

% Beamer comes with a number of color themes that can be applied to any layout theme to change its colors. Uncomment each of these in turn to see how they change the colors of your selected layout theme.

%\usecolortheme{albatross}
%\usecolortheme{beaver}
%\usecolortheme{beetle}
%\usecolortheme{crane}
\usecolortheme{dolphin}
%\usecolortheme{dove}
%\usecolortheme{fly}
%\usecolortheme{lily}
%\usecolortheme{monarca}
%\usecolortheme{seagull}
%\usecolortheme{seahorse}
%\usecolortheme{spruce}
%\usecolortheme{whale}
%\usecolortheme{wolverine}

%----------------------------------------------------------------------------------------
% SELECT FONT THEME & FONTS
%----------------------------------------------------------------------------------------

% Beamer comes with several font themes to easily change the fonts used in various parts of the presentation. Review the comments beside each one to decide if you would like to use it. Note that additional options can be specified for several of these font themes, consult the beamer documentation for more information.

\usefonttheme{default} % Typeset using the default sans serif font
%\usefonttheme{serif} % Typeset using the default serif font (make sure a sans font isn't being set as the default font if you use this option!)
%\usefonttheme{structurebold} % Typeset important structure text (titles, headlines, footlines, sidebar, etc) in bold
%\usefonttheme{structureitalicserif} % Typeset important structure text (titles, headlines, footlines, sidebar, etc) in italic serif
%\usefonttheme{structuresmallcapsserif} % Typeset important structure text (titles, headlines, footlines, sidebar, etc) in small caps serif

%------------------------------------------------

%\usepackage{mathptmx} % Use the Times font for serif text
%\usepackage{palatino} % Use the Palatino font for serif text

\usepackage{helvet} % Use the Helvetica font for sans serif text
\usepackage[default]{opensans} % Use the Open Sans font for sans serif text
%\usepackage[default]{FiraSans} % Use the Fira Sans font for sans serif text
%\usepackage[default]{lato} % Use the Lato font for sans serif text

%----------------------------------------------------------------------------------------
% SELECT INNER THEME
%----------------------------------------------------------------------------------------

% Inner themes change the styling of internal slide elements, for example: bullet points, blocks, bibliography entries, title pages, theorems, etc. Uncomment each theme in turn to see what changes it makes to your presentation.

%\useinnertheme{default}
\useinnertheme{circles}
%\useinnertheme{rectangles}
%\useinnertheme{rounded}
%\useinnertheme{inmargin}

%----------------------------------------------------------------------------------------
% SELECT OUTER THEME
%----------------------------------------------------------------------------------------

% Outer themes change the overall layout of slides, such as: header and footer lines, sidebars and slide titles. Uncomment each theme in turn to see what changes it makes to your presentation.

%\useoutertheme{default}
\useoutertheme{infolines}
%\useoutertheme{miniframes}
%\useoutertheme{smoothbars}
%\useoutertheme{sidebar}
%\useoutertheme{split}
%\useoutertheme{shadow}
%\useoutertheme{tree}
%\useoutertheme{smoothtree}

%\setbeamertemplate{footline} % Uncomment this line to remove the footer line in all slides
%\setbeamertemplate{footline}[page number] % Uncomment this line to replace the footer line in all slides with a simple slide count

%\setbeamertemplate{navigation symbols}{} % Uncomment this line to remove the navigation symbols from the bottom of all slides


%%%%%%%%%%%%%%%%%%%%%%%%%%%%%%%%%%%%%%%%%%%%%%%%

\usepackage{xeCJK}   % Chinese input settings
\setCJKmainfont{標楷體} % Windows使用者請使用這行
\defaultCJKfontfeatures{AutoFakeBold=0.5,AutoFakeSlant=0} %以後不用再設定粗斜

\usepackage{amsmath, amsthm, amsfonts, amssymb}
\everymath{\displaystyle}
%%%%%%%%%%%%%%%%%%%%%%%%%%%%%% Textclass specific LaTeX commands.
\theoremstyle{plain}
\newtheorem{thm}{\protect\theoremname}[section]
\newtheorem{cor}[thm]{Corollary}
\newtheorem{lmma}[thm]{Lemma}
\newtheorem*{defn}{\underline{Definition}}
\newtheorem*{ex*}{Example}
\newtheorem*{sol*}{Solution}
\newtheorem*{thm*}{Theorem}
\newtheorem*{lmma*}{Lemma}
\newtheorem*{rmk*}{Remark}
\newtheorem*{pf*}{\underline{\textbf{Proof\ }}}

%%%%%%%%%%%%%%%%%%%%%%%%%%%%%% User specified LaTeX commands.
\renewcommand{\P}{\mathscr{P}}
\newcommand{\B}{\mathscr{B}}
\newcommand{\A}{\mathscr{A}}
\newcommand{\C}{\mathbb{C}}
\newcommand{\CC}{\mathscr{C}}
\newcommand{\R}{\mathbb{R}}
\newcommand{\Q}{\mathbb{Q}}
\newcommand{\Z}{\mathbb{Z}}
\newcommand{\N}{\mathbb{N}}
\newcommand{\X}{\mathcal{X}}
\newcommand{\T}{\mathscr{T}}
\newcommand{\arbuni}{\bigcup_{\alpha\in I}}
\newcommand{\finint}{\bigcap_{i=1}^n}
\newcommand{\Ua}{{\textsc{U}_\alpha}}
\newcommand{\Ui}{\textsc{U}_i}
\newcommand{\pair}[2]{\left( \,#1\,,\,#2\,\right) }
\newcommand{\sett}[1]{\left\{#1 \right\}}
\newcommand{\dint}[2]{\int_{#1}^{#2}}
\DeclareMathOperator*{\esssup}{ess\,sup}




%%%%%%%%%%%%%%%%%%%%%%%%%%%%%%
\begin{document}

	\begin{frame}
    	\titlepage 
	\end{frame}

\begin{frame}
	\tableofcontents	
\end{frame}

% Presentation structure

\begin{frame}
	\section{Road Map}
	Road Map
	
	\begin{center}
		\includegraphics[scale = 0.25]{./figure/0-1.png}
	\end{center}
    
\end{frame}

\begin{frame}
	\section{A Gentle Start}
	\frametitle{Definition}
	\textbf{Domain set:} An arbitrary set, $\chi$. This is the set of objects that we may wish to label.
\end{frame}

\begin{frame}
	\frametitle{Definition}
	\textbf{Label set:} The Answer of the Domain set, usually $\sett{0,1}$ or $\sett{-1,+1}$
\end{frame}

\begin{frame}
	\frametitle{Definition}
	\textbf{Training data:} $S = ((x_1,y_1),\cdots,(x_m,y_m))$ is a sequence of labeled domain points.
\end{frame}

\begin{frame}
	\frametitle{Definition}
	$h:\chi \rightarrow y,$ a prediction function, also called a predictor, hypothesis, classifier.
\end{frame}


\begin{frame}
	\frametitle{Definition}
	Formally, the learner should choose tin advance a set of predictors. This set is called a hypothesis class and is denoted by $H$. Each $h \in H$ is a function mapping from $\chi$ to $y$.
\end{frame}


\begin{frame}
	\frametitle{Definition}
	\textbf{A data-generation model:} We now explain how the training data is generated by som probability distribution. Let us denote that probability distribution over $\chi$ by $D$.
	
\end{frame}

\begin{frame}
	\frametitle{Definition}
	\textbf{Measure of Success:} To know is the output is good or not, we define the loss function to check it
	
\end{frame}

\begin{frame}
	\frametitle{Definition}
	\begin{enumerate}
		\item \textbf{True error:} $L_{D,f}(h) = \mathbb{P}_{x\sim D}[h(x) \neq f(x)] = D(\sett{x ~|~ h(x) \neq f(x)})$
		\item \textbf{Training error:} \\ $L_S(h) = \dfrac{|\sett{i \in m ~|~ h(x_i) \neq y_i} |}{m}$ where $[m] = \sett{1,\cdots,m}$
	\end{enumerate}
	
\end{frame}

\begin{frame}
	\frametitle{Definition}
	\begin{center}
		\includegraphics[scale = 0.3]{./figure/2-1.png}
		
		Figure 2-1, training error is $0$ but true error is bad
	\end{center}
	
\end{frame}


\begin{frame}
	\frametitle{Definition}
	\begin{enumerate}
		\item[$\cdot$] We denote the probability of getting a non-representative sample by $\delta$, and call ($1 - \delta$) the \textbf{confidence parameter} of our prediction
		\item[$\cdot$] The \textbf{accuracy parameter}, commonly denoted by $\epsilon$. We interpret the event $L_{(D,f)}(h_s) > \epsilon$ as a failure of the learner
	\end{enumerate}
	
\end{frame}

\begin{frame}
	\section{Improve Model}
	\frametitle{Empirical Risk Minimization}
	
	The method to proof the model is to minimize the loss function by using training data, i.e. we check $L_S(h)$
	
\end{frame}

\begin{frame}
	\frametitle{Empirical Risk Minimization}
	The $\text{ERM}_H$ learner uses the ERM rule to choose a predictor $h \in H$, with the lowest possible error over $S$. Formally

$$h_S = \text{ERM}_H(S) \in \arg\min_{h \in H}L_S(h)$$
	
\end{frame}

\begin{frame}
	\frametitle{Overfitting}
	cause by ERM, the data is too fit the training set

example: $h_s(x) = \begin{cases}
	y_i~\text{if}~\exists~i \in [m]~\text{s.t. } x_i = x\\
	0~\text{otherwise}
\end{cases}$
	
\end{frame}


\begin{frame}
	\section{The upper bound of $L_{(D,f)}(h_s)$ in finite hypothsis}
	\frametitle{Inductive bias}
	Restricting the learner to choosing a predictor from $H$ are often called an \textbf{inductive bias}. In following statement, we will proof that when we have some assumption, $H$ is a finite set and have enough quantity of data set, then we can avoid the overfitting problem.
	
	
\end{frame}


\begin{frame}
	\frametitle{Some assumption}
	\textbf{The Realizability Assumption:} There exists $h^* \in H$ s.t. $L_{(D,f)}(h^*) = 0$. Note that this assumption implies that with probability $1$ over random samples, $S$ where the instances of $S$ are sampled according to $D$ and are labeled by $f$, we have $L_S(h^*) = 0$

	
\end{frame}


\begin{frame}
	\frametitle{Some assumption}
	\textbf{The i.i.d. assumption :} The examples in the training set are independently an identically distributed (i.i.d) according to the distribution $D$. We denote this assumption by $S \sim D^m$ where $m$ is the size of $S$, and $D^m$ denotes the probability over $m$-tuples induced by applying $D$ to pick each element of the tuple independently of the other members of the tuple.
	
\end{frame}


\begin{frame}
	\frametitle{Proof Section}
	The goal is to proof when $m \geq \dfrac{\log(|H|/\delta)}{\epsilon}$, then $L_{D,f}(h_s) \leq \epsilon$
	
\end{frame}

\begin{frame}
	\frametitle{Proof Section}
	Let $H_B$ be the set of "bad" hypotheses, that is,

$$H_B = \sett{h \in H~|~L_{D,f}(h)>\epsilon}$$
	
\end{frame}


\begin{frame}
	\frametitle{Proof Section}
	In addition, let

$$M = \sett{S|_x ~|~ \exists~h \in H_B,L_S(h) = 0}$$

be the set of \textbf{misleading sample}(they are bad but $L_S(h_S) = 0$)
	
\end{frame}


\begin{frame}
	\frametitle{Proof Section}
	by definition, we can write 

$$\sett{S|_x ~|~ L_{(D,f)}(h_S) > \epsilon} \subseteq M\color{red}(\star_1) $$


We can rewrite $M$ as (thought it's intersection was not empty)

$$M = \cup_{h \in H_B}\sett{S|_x ~|~ L_S(h) = 0}\color{red}(\star_2)$$
	
\end{frame}


\begin{frame}
	\frametitle{Proof Section}
	by $(\star_1),(\star_2)$,

$$D^m(\sett{S|_x~|~L_{(D,f)}(h_s)>\epsilon}) \leq D^m(M) $$

$$= D^m(\cup_{h\in H_B}\sett{S|_x ~|~ L_S(h) = 0})\color{red}(\star_3)$$
	
\end{frame}


\begin{frame}
	\frametitle{Proof Section}
	\textbf{LEMMA}(Union Bound) For any two sets $A,B$ and a distribution $D$ we have

$$D(A\cup B) \leq D(A) + D(B)$$
	
\end{frame}


\begin{frame}
	\frametitle{Proof Section}
	and the $(\star_3)$ can be bound like this

$$D^m(\sett{S|_x ~|~ L_{(D,f)}(h_S)>\epsilon}) \leq \sum_{h \in H_B}D^m(\sett{S|_x ~|~ L_S(h) = 0})\color{red}(\star_4)$$
	
\end{frame}


\begin{frame}
	\frametitle{Proof Section}
	\textbf{Next, we fix $h_B$ on the bad hypothesis $h_B \in H_B$} $\implies L_{(D,f)}(h) > \epsilon$
	
\end{frame}


\begin{frame}
	\frametitle{Proof Section}
	because the event are i.i.d, we get that

$D^m(\sett{S|_x~|~L_S(h)=0}) = D^m(\sett{S|_x ~|~\forall i,h(x_i) = f(x_i)})$

$ = \prod^m_{i=1}D(\sett{x_i~|~h(x_i) = f(x_i)})\color{red}(\star_5)$
	
\end{frame}


\begin{frame}
	\frametitle{Proof Section}
	check the each individual sampling of an element of the training set, we have

$D(\sett{x_i~|~h_B(x_i) = y_i}) = 1-D(\sett{x ~|~ h_B(x) \neq f(x)})$

$=1 - L_{D,f}(h_B) \leq 1 - \epsilon$
	
\end{frame}

\begin{frame}
	\frametitle{Proof Section}
	put it to the $\color{red}(\star_5)$ and use the inequality $1 - \epsilon \leq e^{-\epsilon}$

$$D^m(\sett{S|_x~|~ L_S(h_B) = 0}) \leq (1 -\epsilon)^m \leq e^{-\epsilon m}\color{red}(\star_6)$$
	
\end{frame}

\begin{frame}
	\frametitle{Proof Section}
	$D^m(\sett{S|_x ~|~L_{(D,f)}(h_s)>\epsilon}) = |H_B|D^m(\sett{{S|_x ~|~L_S(h_B)=0}})$
	
	$|H_B|$ means the cardinality(element number) of $H_B$
	
\end{frame}


\begin{frame}
	\frametitle{Proof Section}
	put $\color{red}(\star_6)$ back to $\color{red}(\star_4)$ 
	
	$$D^m(\sett{S|_x ~|~ L_{(D,f)}(h_S)>\epsilon}) \leq |H_B|e^{-\epsilon m} \leq |H|e^{-\epsilon m}$$
	
\end{frame}


\begin{frame}
	\frametitle{Proof Section}
	In this equation, we can know that when $m$ increase, the  overfitting hypothesis's probability (where the hypothesis of $L_S$ is small but $L_{(D,f)}$ is big, i.e. $D^m(\sett{S|_x ~|~ L_{(D,f)}(h_S)>\epsilon})$) will decrease.
	
\end{frame}

\begin{frame}
	Since $D^m(\sett{S|_x ~|~ L_{(D,f)}(h_S)>\epsilon})$ is $\delta$, and have a nature log, we get

$$m \geq \dfrac{\log(|H|/\delta)}{\epsilon}$$
\end{frame}

 
\end{document}
