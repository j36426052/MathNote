\documentclass[12pt,reqno]{amsart}
%\usepackage[margin=1in]{geometry}
\usepackage{tcolorbox}
\usepackage{amssymb}
\usepackage{amsthm}
\usepackage{amsmath}
\usepackage{amssymb}
\usepackage{mathrsfs}
\usepackage{centernot}
\usepackage{lastpage}
\usepackage{fancyhdr}
\usepackage{accents}
\usepackage{tasks}
\usepackage{graphicx}
\usepackage{natbib}
\usepackage{tabularx}
\usepackage{multirow}
\usepackage{booktabs}
\usepackage{hyperref}
\usepackage{bm}
\usepackage{float}
\theoremstyle{plain}
\usepackage{multicol}
\usepackage{enumitem,kantlipsum}

% 加上浮水印
%\usepackage{wallpaper}
%\CenterWallPaper{.180}{../qsnake-logo.jpg}


\linespread{1.2}
\parindent = 0pt
\pagestyle{fancy}
\setlength{\parindent}{0pt}
%\everymath{\displaystyle}
%new area
%\usepackage[utf8]{inputenc}
%\usepackage{CJKutf8}
%\xeCJKsetup{AutoFakeBold=true, AutoFakeSlant=true}

% 設定頭部
\fancyhead[L]{Midterm} % 左邊頭部清空
\fancyhead[C]{} % 中間頭部清空
\fancyhead[R]{} % 右邊頭部顯示頁碼

% Adjust the footer as desired:
\fancyfoot[L]{} % Left footer: Empty.
\fancyfoot[C]{\thepage} % Center footer: Empty.
\fancyfoot[R]{} % Right footer: Empty.



% we will modify sections, subsections and sub subsections
\RequirePackage{titlesec}
% Modification of section 
\titleformat{\section}[block]{\normalsize\bfseries\filcenter}{\thesection.}{.3cm}{} 


% modification of subsection and sub sub section
\titleformat{\subsection}[runin]{\bfseries}{ \thesubsection.}
{1mm}{}[.\quad]
\titleformat{\subsubsection}[runin]{\bfseries\itshape}{ \thesubsubsection.}
{1mm}{}[.\quad]

\newenvironment{solution}
  {\renewcommand\qedsymbol{$\blacksquare$}
  \begin{proof}[Solution]}
  {\end{proof}}
\renewcommand\qedsymbol{$\blacksquare$}

\newcommand{\ubar}[1]{\underaccent{\bar}{#1}}

%%%%%%%%%%%%%%%%%%%%%%%%%%%%%% Textclass specific LaTeX commands.
%\theoremstyle{plain}
%\newtheorem{thm}{\protect\theoremname}[section]
\newtheorem{thm}{\textbf{Theroem}}[section]
\newtheorem{cor}[thm]{Corollary}
\newtheorem{lmma}[thm]{Lemma}
\newtheorem*{defn}{\underline{Definition}}
\newtheorem*{prop*}{Proposition}
\newtheorem*{ex*}{Example}
\newtheorem*{sol*}{Solution}
\newtheorem*{cor*}{Corollary}
\newtheorem*{thm*}{Theorem}
\newtheorem*{lmma*}{Lemma}
\newtheorem*{rmk*}{Remark}
\newtheorem*{pf*}{\underline{\textbf{Proof\ }}}

%%%%%%%%%%%%%%%%%%%%%%%%%%%%%% User specified LaTeX commands.
\renewcommand{\P}{\mathscr{P}}
\newcommand{\B}{\mathscr{B}}
\newcommand{\A}{\mathscr{A}}
\newcommand{\C}{\mathbb{C}}
\newcommand{\CC}{\mathscr{C}}
\newcommand{\R}{\mathbb{R}}
\newcommand{\Q}{\mathbb{Q}}
\newcommand{\Z}{\mathbb{Z}}
\newcommand{\N}{\mathbb{N}}
\newcommand{\X}{\mathcal{X}}
\newcommand{\T}{\mathscr{T}}
\newcommand{\arbuni}{\bigcup_{\alpha\in I}}
\newcommand{\finint}{\bigcap_{i=1}^n}
\newcommand{\Ua}{{\textsc{U}_\alpha}}
\newcommand{\Ui}{\textsc{U}_i}
\newcommand{\pair}[2]{\left( \,#1\,,\,#2\,\right) }
\newcommand{\dint}[2]{\int_{#1}^{#2}}
\newcommand{\sett}[1]{\left\{ \,#1 \,\right\}}
\newcommand{\linearcombination}[2]{#1_1#2_1+\cdots+#1_n#2_n}
\newcommand{\slinearcombination}[1]{#1_1+\cdots+#1_n}
\newcommand{\spann}[1]{\text{span($#1$)}}
\newcommand{\sub}[1]{\text{sup}}
\newcommand{\inn}[1]{\left< #1 \right>}
\newcommand{\kernal}[1]{Ker(#1)}
\newcommand{\image}[1]{Im(#1)}
\newcommand{\norm}[1]{\parallel #1 \parallel}
\newcommand{\dia}[0]{\text{dia}}
\newcommand{\marking}[1]{\text{\color{red} #1}}
%%%%%%%%%%%%%%%%%%%%%%%%%%%%%%

\begin{document}

%\lhead{Linear Algebra} 
%\rhead{Sabrina Edition} 
\cfoot{\thepage} %\ of \pageref{LastPage}}


\textbf{Classification}
\begin{tcolorbox}
	\textbf{order}

	$$\dfrac{dy}{dx} = y^2$$
	
	which is $1$st order,  $x$ independent variable, $y$ dependent variable
	
	$$\dfrac{d^4y}{dt^4} + 5\dfrac{d^2x}{dt^2} + 3x \sin{t}$$
	
	which is $4$th order
	
	because above equation only have $1$ independent variable, they are ordinary differential equations(ODEs).
	
	$$\dfrac{\partial v}{\partial s} + \dfrac{\partial v}{\partial t} = r$$
	
	which is $1$st order
	
	$$\dfrac{\partial^2 u}{\partial x^2} + \dfrac{\partial^2 u}{\partial y^2} + \dfrac{\partial^2 u}{\partial z^2} = 0$$
	
	which is $2$nd order
	
	above equation have more than one independent variable, they are partial differential equations(PDEs).
\end{tcolorbox}

$n$th-order ODE: $F(x,y,y',\cdots,y^{(n)}) = 0$

In certain condition on $F$, it can be written as

$$y^{(n)} = f(x,y,y',\cdots,y^{(n)}) = 0 (\star)$$

Example $(y')^2 + y' + xy = 0$

$$y' = \dfrac{-1 \pm \sqrt{1 - 4xy}}{2}$$

\begin{defn}
	a function $\phi(x)$ is called a solution of ($\star$) on $a<x<b$ if $\phi^{(n)}$ exists on $a<x<b$ and 
	
	$$\phi^{(n)}(x) = f(x,\phi (x),\phi'(x),\cdots,\phi^{(n-1)}(x) )~\forall a<x<b$$
\end{defn}

\textbf{Example.} Verify that $y = e^{2x}$ is a solution of $y'' + y' - 6y = 0$

\begin{proof}
	$y''+y' - 6y = 4e^{3x} + 2e^{2x} - 6e^{3x} = 0 ~\forall -\infty < x < \infty$
	
	$\therefore y = e^{2x}$ is a solution on $-\infty < x < \infty$
\end{proof}

\newpage

\textbf{Note.} $y' = \dfrac{xy}{x + y + 1}$ is derivative form $\Leftrightarrow dy = \dfrac{xy}{x+y+1}dx$ or $xydx - (x + y + 1)dy = 0$ is differential form.

\begin{defn}
	An ODE of order $n$ is called linear if it may be written in the form
	
	$$b_0(x)y^{(n)} + b_1(x)y^{(n-1)} + \cdots + b_{n-1}(x)y' + b_n(x)y = R(x)$$
	
	where $b_0 \neq 0$, An ODE that is not linear is called nonlinear ODE.
\end{defn}

\textbf{Example}
\begin{tcolorbox}
linear 
	\begin{eqnarray*}
		y'(x) + 5y'(x) + 6y(x) &=& 0\\
		y'''(x) + x^2 y''(x) + x^3y'(x) &=& xe^x
	\end{eqnarray*}
non linear
	\begin{eqnarray*}
		y''(x) + 5y'(x) + 6y^2(x) &=& 0\\
		y''(x) + 5(y'(x))^3 + 6y(x) &=& 0
	\end{eqnarray*}	
\end{tcolorbox}

Initial-Value Problem(IVP): same point (and $1$st order)
\begin{tcolorbox}
	$\begin{cases}
		\dfrac{d^2 y}{dx^2} + y = 0\\ y(1) = 3 \\ y(1) = 2
	\end{cases}$
\end{tcolorbox}

Boundary-Value Problem(BVP): two or more different points

\begin{tcolorbox}
	$\begin{cases}
		\dfrac{d^2 y}{dx^2} + y = 0\\ y(1) = 3\\ y(2) = 2
	\end{cases}$
\end{tcolorbox}

\newpage

\begin{thm*}[Existence and uniqueness]
	Consider
	
	$$\begin{cases}y' = f(x,y) \\ y(x_0) = y_0\end{cases}$$
	
	where $x_0,y_0 \in \R$ are given
	
	Let $T = \{(x,y)~|~ |x-x_0| \leq a,|y-y_0| \leq b\}$, where $a,b > 0$. Suppose that $f$ and $fy$ are continuous in $T$. Then (IVP) has a unique solution defined on $[x_0 - h,x_0 + h]$ for some $h > 0$
\end{thm*}

\textbf{$\S$ Separable equation} $A(x)dx = B(y)dy$

$ $

\textbf{Example.}

\begin{enumerate}[wide]
	\item $\dfrac{dy}{dx} = \dfrac{2y}{x}$
	\begin{solution}
		\begin{eqnarray*}
			\dfrac{1}{y}dy &=& \dfrac{2}{x}dx\\
			\implies \int \dfrac{1}{y}dy &=& \int \dfrac{2}{x}dx\\
			\implies \ln |y| &=& 2\ln|x| + C
		\end{eqnarray*}
	\end{solution}
	
	\item $\begin{cases}
		(1+y^2)dx + (1+x^2)dy = 0\\ y(0) = -1
	\end{cases}$

	\begin{solution}
		\begin{eqnarray*}
			(1+y^2)dx &=& -(1+x^2)dy\\
			\implies \dfrac{dx}{-(1+x^2)} &=& \dfrac{dy}{(1 + y^2)}\\
			\implies \int \dfrac{1}{1+x^2}dx &=& - \int \dfrac{1}{1+y^2}dy
		\end{eqnarray*}
		
		you can let $x = \tan\theta \implies dx = \sec^2\theta d\theta $
		
		$\therefore \int\dfrac{1}{1+x^2}dx = \int \cos^2\theta \sec^2\theta d\theta = \theta + C = \tan^{-1}x + C$
		
		$\implies \tan^{-1}x = -\tan^{-1}y + C$
		
		$y(0) = -1 \implies 0 = \dfrac{\pi}{4} + C \implies C = \dfrac{\pi}{-4},~\therefore \tan^{-1} x = - \tan^{-1}y - \dfrac{\pi}{4}$
	\end{solution}
	
	\newpage
	
	\item $\begin{cases}
		2x(y+1)dx - ydy = 0\\ y(0) = -2
	\end{cases}$
	\begin{solution}
		$\int 2xdx = \int \dfrac{y}{y+1}dy \implies x^2 = y-\ln |y+1| + C$
		
		$y(0) = -2 \implies 0 = -2 + c \implies c = 2 ~\therefore x^2 + y - \ln |y+1| + 2$
	\end{solution}
\end{enumerate}

\textbf{$\S$ Homogeneous equations}

\begin{defn}
	a function $f(x,y)$ is said to be homogeneous of degree $k$ in $x$ and $y$ if and only if
	
	$$f(\lambda x,\lambda y) = \lambda^k f(x,y)$$
\end{defn}

\textbf{Example.} $f(x,y) = x^2 + y^2$

\begin{eqnarray*}
	f(\lambda x , \lambda y) &=& (\lambda x)^2 + (\lambda y)^2\\
	&=& \lambda^2 (x^2 + y^2)\\
	&=& \lambda^2 f(x,y)
\end{eqnarray*}

$\therefore f(x,y)$ is homogeneous  
, $k = 2$

\begin{thm*}
	If $M(x,y)$ and $N(x,y)$ are both homogeneous and of the same degree, then $\dfrac{M(x,y)}{N(x,y)}$ is homogeneous of degree zero.
\end{thm*}
\begin{proof}
	Set $f(x,y) = \dfrac{M(x,y)}{N(x,y)}$. By definition, we assume $M$ and $N$ are homogeneous of degree $k$, so
	
	$$M(\lambda x , \lambda y) = \lambda^k M(x,y) \text{ and } N(\lambda x,\lambda y) = \lambda^kN(x,y)$$
	
	$\therefore f(\lambda x,\lambda y) = \dfrac{M(\lambda x ,\lambda y )}{N(\lambda x, \lambda y)} = \dfrac{\lambda^k}{\lambda^k} \cdot \dfrac{M(x,y)}{N(x,y)} = \lambda^0 \dfrac{M(x,y)}{N(x,y)}$
\end{proof}












\end{document}