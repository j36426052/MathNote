\documentclass[12pt]{article}
\usepackage[margin=1in]{geometry} 
\usepackage{amsmath}
\usepackage{tcolorbox}
\usepackage{amssymb}
\usepackage{amsthm}
\usepackage{mathrsfs}
\usepackage{lastpage}
\usepackage{fancyhdr}
\usepackage{accents}
\usepackage{tasks}
\linespread{1.2}
\parindent = 0pt
%\pagestyle{fancy}
\setlength{\headheight}{40pt}
\everymath{\displaystyle}

% we will modify sections, subsections and sub subsections
\RequirePackage{titlesec}
% Modification of section 
\titleformat{\section}[block]{\normalsize\bfseries\filcenter}{\thesection.}{.3cm}{} 

% modification of subsection and sub sub section
\titleformat{\subsection}[runin]{\bfseries}{ \thesubsection.}
{1mm}{}[.\quad]
\titleformat{\subsubsection}[runin]{\bfseries\itshape}{ \thesubsubsection.}
{1mm}{}[.\quad]

\newenvironment{solution}
  {\renewcommand\qedsymbol{$\blacksquare$}
  \begin{proof}[Solution]}
  {\end{proof}}
\renewcommand\qedsymbol{$\blacksquare$}

\newcommand{\ubar}[1]{\underaccent{\bar}{#1}}

%%%%%%%%%%%%%%%%%%%%%%%%%%%%%% Textclass specific LaTeX commands.
%\theoremstyle{plain}
\newtheorem{thm}{\protect\theoremname}[section]
\newtheorem{cor}[thm]{Corollary}
\newtheorem{lmma}[thm]{Lemma}
\newtheorem*{defn}{\underline{Definition}}
\newtheorem*{ex*}{Example}
\newtheorem*{sol*}{Solution}
\newtheorem*{cor*}{Corollary}
\newtheorem*{thm*}{Theorem}
\newtheorem*{lmma*}{Lemma}
\newtheorem*{rmk*}{Remark}
\newtheorem*{pf*}{\underline{\textbf{Proof\ }}}

%%%%%%%%%%%%%%%%%%%%%%%%%%%%%% User specified LaTeX commands.
\renewcommand{\P}{\mathscr{P}}
\newcommand{\B}{\mathscr{B}}
\newcommand{\A}{\mathscr{A}}
\newcommand{\C}{\mathbb{C}}
\newcommand{\CC}{\mathscr{C}}
\newcommand{\R}{\mathbb{R}}
\newcommand{\Q}{\mathbb{Q}}
\newcommand{\Z}{\mathbb{Z}}
\newcommand{\N}{\mathbb{N}}
\newcommand{\X}{\mathcal{X}}
\newcommand{\T}{\mathscr{T}}
\newcommand{\arbuni}{\bigcup_{\alpha\in I}}
\newcommand{\finint}{\bigcap_{i=1}^n}
\newcommand{\Ua}{{\textsc{U}_\alpha}}
\newcommand{\Ui}{\textsc{U}_i}
\newcommand{\pair}[2]{\left( \,#1\,,\,#2\,\right) }
\newcommand{\dint}[2]{\int_{#1}^{#2}}
\newcommand{\sett}[1]{\left\{ \,#1 \,\right\}}
\newcommand{\linearcombination}[2]{#1_1#2_1+\cdots+#1_n#2_n}
\newcommand{\slinearcombination}[1]{#1_1+\cdots+#1_n}
\newcommand{\spann}[1]{\text{span($#1$)}}
\newcommand{\sub}[1]{\text{sup}}
\newcommand{\inn}[1]{\left< #1 \right>}
%%%%%%%%%%%%%%%%%%%%%%%%%%%%%%

\begin{document}

%\lhead{Linear Algebra} 
%\rhead{Sabrina Edition} 
\cfoot{\thepage} %\ of \pageref{LastPage}}

\section{Introduction to spectral theory}

\subsection{Main definitions}$ $

\begin{tcolorbox}
	Introduce eigenvalue \& eigenvector first, and the method to find these thing.
\end{tcolorbox}

\begin{defn}
	A scalar $\lambda$ is called an eigenvalue of an operator $A:V \rightarrow V$ if there exists a non-zero vector $v \in V$ such that 
	
	$$Av = \lambda v$$
	
	The vector $v$ is called the eigenvector of $A$
\end{defn}

\begin{thm*}[From hamberger Thm 5.2]
	Let $A \in M_{n \times n}(F).$ Then a scalar $\lambda$ is an eigenvalue of $A$ if and only if $\det(A - \lambda I_{n}) = 0$.
\end{thm*}

\begin{proof}
	A scalar $\lambda$ is an eigenvalue of $A$ if and only if there exists a nonzero vector $v \in F^n$ such that $A v = \lambda v$, that is, $(A - \lambda I_n)(v) = 0$. By Theorem 2.5, this is true \textbf{if and only if} $A - \lambda I_n$ is not invertible. However, this result is equivalent to the statement that $\det(A - \lambda I_n)=0$
\end{proof}

\begin{defn}
	Let $A \in M_{n\times n}(\mathrm{F}).$ The polynomial $f(t) = \det (A - tI_n)$ is called the characteristic polynomial of $A$
\end{defn}

\begin{thm*}[From hamberger Thm 5.4]
	Let $T$ be a linear operator on a vector space $V$, and let $\lambda$ be an eigenvalue of $T$. A vector $v \in V$ is an eigenvector of $T$ corresponding to $\lambda$ if and only if $v \neq 0$ and $v \in N(T - \lambda I)$.
\end{thm*}

\begin{defn}
	The nullspace $N(A - \lambda I)$, i.e. the set of all eigenvectors and $0$ vector, is called the eigenspace. The set of all eigenvalues of an operator $A$ is called spectrum of $A$, and is usually denoted $\sigma (A)$.
\end{defn}

\begin{rmk*} $ $

	If the matrix $A$ is ugly, what should we do?
	
	we can use the similar ≤matrices
	
	$A$ and $B$ are called similar if there exists an invertible matrix $S$ such that 
	
	$$ A = SBS^{-1}$$
	
	The determinants of similar matrix is same
	
	$$\det(A) = \det(SBS^{-1}) = \det(S)\det(B)\det(S^{-1}) = \det(B)$$
	
	We can find $A-\lambda I$ and $B -\lambda I$ is similar
	
	$$A - \lambda I = SBS^{-1} - \lambda SIS^{-1} = S(BS^{-1} - \lambda I S^{-1}) = S(B-\lambda I)S^{-1}$$
	
	It same in transform
	
	If $T:V\rightarrow V$ is a linear transform, $\alpha,\beta$ are two bases in $V$, then
	
	$$[T]^{\alpha}_{\alpha} = [I]^{\alpha}_{\beta}[T]^{\beta}_{\beta}[I]^{\beta}_{\alpha} $$
	
	
\end{rmk*}

\begin{tcolorbox}
	Before introducing diagonal, we have these two mutiplicity, 
\end{tcolorbox}

\begin{defn}[algebraic mutiplicity]
	The largest positive integer $k$ such that $(x - \lambda)^k$ divides $p(x)$ is called the multiplicity of the root $\lambda$.
	
	If $\lambda$ is an eigenvalue of an operator (matrix) $A$, then it is a root of the characteristic polynomial $p(z) = \det(A - zI)$. The multiplicity of this root is called the (algebraic) multiplicity of the eigenvalue $\lambda$.
\end{defn}

\begin{defn}[geometric multiplicity]
	The dimension of the eigen space $N(A - \lambda I)$ is called geometric multiplicity of the eigenvalue $\lambda$.
\end{defn}

\subsection{Diagonalization} $ $

\begin{defn}
	A linear operator $T$ on a finite-dimensional vector space $V$ is called diagonalizable if there is an ordered basis $\beta$ for $V$ such that $[T]_\beta$ is a diagonal matrix. A square matrix $A$ is called diagonalizable if $L_A$ is diagonalizable.
\end{defn}

\begin{tcolorbox}
	After have the definition of diagonal, we want to know want matrix can be diagonal.
\end{tcolorbox}



\begin{defn}
	A linear operator $T$ on a finite-dimensional vector space $V$ is called diagonalizable if there is an ordered basis $\beta$ for $V$ such that $[T]_{\beta}$ is a diagonal matrix. A square matrix $A$ is called diagonalizable if $L_A$ is diagonalizable.
\end{defn}

\begin{thm*}
	A matrix $A$ admits a representation $A = SDS^{-1}$, where $D$ is a diagonal matrix and $S$ is an invertible one \textbf{if and only if} there exists a basis in $F^n$ of eigenvectors of $A$.
\end{thm*}

\begin{tcolorbox}
	In this theorem, we will know the relation between diagonal matrix and the eigenvector
\end{tcolorbox}

\begin{proof}
	Let $D = \text{diag}\sett{\lambda_1,\lambda_2,\cdots,\lambda_n}$, and let $b_1,\cdots,b_n$ be the columns of $S$ (note that since $S$ is invertible it's columns form a basis in $\mathrm{F^n}$). Then the identity $A = SDS^{-1}$ means that $D = [A]^\beta_\beta$
	
	Indeed, $S = [I]^{\beta}_{S}$ is the change of the coordinates matrix from $\beta$ to the standard basis $S$ so we get from $A = SDS^{-1}$ that $D = S^{-1}AS = [I]^{\beta}_{S}A[I]^{S}_{\beta}$, which means exactly that $D = [A]_{\beta}$
\end{proof}

\begin{thm*}
	Let $\lambda_1,\lambda_2,\cdots,\lambda_r$ be distinct eigenvalues of $A$, and let $v_1,v_2,\cdots,v_r$ be the corresponding eigenvectors. Then vectors $v_1,v_2,\cdots,v_r$ are linearly independent.
\end{thm*}

\begin{cor*}
	If an operator $A: V \rightarrow V$ has exactly $n = dim V$ distinct eigenvalues, then it is diagonalizable.
\end{cor*}

\begin{tcolorbox}
	This section we will focus the bases of subspace to build the tool to proof the criterion of diagnal.
\end{tcolorbox}

\begin{defn}[Direct sums of subspaces]
	Let $V_1,V_2,\cdots,V_p$ be a subspaces of a vector space $V$. We say that the system of subspace is a basis in $V$ if any vector $v \in V$ admits a unique representation as a sum
	
	$$v = v_1 + v_2 + \cdots + v_p = \sum^p_{k=1}v_k,~v_k \in V_k$$
	
	We also say, that a system of subspaces $V_1,V_2,\cdots,V_p$ is linearly independent if the equation 
	
	$$v_1 + v_2 + \cdots + v_p = 0,~v_k \in V_k$$
	
	has only trivial solution
\end{defn}

\begin{rmk*}
	From the above definition, the system of eigenspaces $E_k$ if an operator $A$
	
	$$E_k = N(A - \lambda_k I)$$
	
	is linearly independent 
\end{rmk*}

\begin{lmma*}
	Let $V_1,V_2,\cdots,V_p$ be a linearly independent family of subspaces, and leu us have in each subspace $V_k$ a linearly independent system $\beta_k$ of vectors. Then the union $B = \cup_k\beta_k$ is a linearly independent system.
\end{lmma*}

\begin{thm*}
	Let $V_1,V_2,\cdots,V_p$ be a basis of subspaces, and let us have in each subspace $V_k$ a basis (of vectors) $\beta_k$. Then the union $\cup_k\beta_k$ of these bases is a basis in $V$.
\end{thm*}

\begin{tcolorbox}
	After we have the tool, we can use it to check the criterion of diagonal.
\end{tcolorbox}

\begin{thm*}
	Let an operator $A:V\rightarrow V$ has exactly $n = \dim V$ eigenvalues (counting multiplicities). Then $A$ is diagonalizable if and only if for each eigenvalue $\lambda$, geometric multiplicity = algebraic multiplicity.
\end{thm*}

\section{Inner Product Spaces}

\subsection{Inner Products \& Norms}

\begin{defn}
	Let $V$ be a vector space over $\mathrm F$. An \textbf{inner product} on $V$ is a function that assigns, to every ordered pair of vectors $x$ and $y$ in $V$, a scalar in $\mathrm F$, denoted $<x,y>$, such that for all $x,y$ and $z$ in $V$ and all $c$ in $\mathrm F$, the following hold:
	
	\begin{enumerate}
		\item[(a)] $\left<x+z,y\right> = \left<x,y\right>+\left<z,y\right>$
		\item[(b)] $\left<cx,y\right> = c\left<x,y\right>$
		\item[(c)] $\overline{\left<x,y\right>} = \left<y,x\right>$
		\item[(d)] $\left<x,x\right> ~>0$ if $x \neq 0$
	\end{enumerate}
	
	if $a_1,a_2,\cdots,a_n \in \mathrm F$ and $y,v_1,v_2,\cdots,v_n \in V$, then
	
	$$\left<\sum^n_{i=1}a_iv_i,y\right> = \sum^n_{i=1}a_i\left<v_i,y\right>$$
	
	vector space $V$ over $\mathrm F$ endowed with a specific inner product is called \textbf{inner product space}
\end{defn}

\begin{tcolorbox}
	\textbf{\color{red}Note}: We just define the \textbf{rule} of the inner product, not define \textbf{what} is inner product.
\end{tcolorbox}


\begin{thm*}
	Let $V$ be an inner product space. Then for $x,y,z \in V$ and $c \in \mathrm F$, the following statements are true.
	
	\begin{enumerate}
		\item[(a)] $\inn{x,y+z} = \inn{x,y} + \inn{x,z}.$
		\item[(b)] $\inn{x,cy} = \overline{c}\inn{x,y}$
		\item[(c)] $\inn{x,0} = \inn{0,x} = 0$
		\item[(d)] $\inn{x,x} = 0$ if and only if $x = 0$
		\item[(e)] If $\inn{x,y} = \inn{x,z}$ for all $x \in V$, then $y=z$ 
	\end{enumerate}
\end{thm*}

\newpage

\begin{defn}
	Let $V$ be an inner product space. For $x \in V$, we define the \textbf{norm} or \textbf{length} of $x$ by $||x|| = \sqrt{\inn{x,x}}$
\end{defn}

\begin{thm*}
	Let $V$ be an inner product space over $\mathrm F$. Then for all $x,y \in V$ and $c \in F$, the following statements are true
	
	\begin{enumerate}
		\item[(a)] $||cx|| = |c|\cdot||x||$
		\item[(b)] $||x|| = 0$ if and only if $x = 0$. In any case, $||x|| \geq 0$
		\item[(c)] (Cauchy-Schwarz Inequality) $|\inn{x,y}| \leq ||x||\cdot||y||.$
		\item[(d)] (Triangle Inequality) $||x+y|| \leq ||x|| + ||y||.$
	\end{enumerate}
\end{thm*}

\begin{tcolorbox}
	After have the definition of inner product and norm, we can define orthogonal.
\end{tcolorbox}


\begin{defn}
	Let $V$ be an inner product space. Vectors $x$ and $y$ in $V$ are \textbf{orthogonal(perpendicular)} if $\inn{x,y} = 0$. A subset $S$ of $V$ is \textbf{orthogonal} if ant two distinct vectors in $S$ are orthogonal. A vector $x$ in $V$ is a \textbf{unit vector} if $||x|| = 1$. Finally, a subset $S$ of $V$ is \textbf{orthonormal} if $S$ is orthogonal and consists entirely of unit vectors.
\end{defn}

\subsection{Gram-Schmidt \& Orthogonal complements} $ $

\begin{tcolorbox}
	The power of the orthogonal subset
\end{tcolorbox}

\begin{thm*}
	Let $V$ be an inner product space and $S = \sett{v_1,v_2,\cdots,v_k}$ be an orthogonal subset of $V$ consisting of nonzero vectors. If $y \in \spann{S}, then$
	
	$$y = \sum^k_{i=1}\dfrac{\inn{y,v_i}}{||v_i||^2}v_i$$
\end{thm*}

\begin{cor*}
	If, in addition to the hypotheses of thm, $S$ is orthonormal and $y \in \spann{S},$ then
	
	$$y = \sum^k_{i=1}\inn{y,v_i}v_i$$.
\end{cor*}

\begin{cor*}
	Let $V$ be an inner product space, and let $S$ be an orthogonal subset of $V$ consisting of nonzero vectors. Then $S$ is linearly independent.
\end{cor*}


\begin{thm*}[Gram-Schmidt]
	Let $V$ be an inner product space and $S = \sett{w_1,w_2,\cdots,w_n}$ be a linearly independent subset of $V$. Define $S' = \sett{v_1,v_2,\cdots,v_n},$ where $v_1 = w_1$ and
	
	$$v_k = w_k - \sum^{k-1}_{j=1}\dfrac{\inn{w_k,v_j}}{||v_j||^2}v_j ~\text{ for $2 \leq k \leq n$}$$ 
	
	Then $S'$ is an orthogonal set of nonzero vectors such that span($S'$) = span($S$).
\end{thm*}

\begin{thm*}
	Let $V$ be a nonzero finite-dimensional inner product space. Then $V$ has an orthonormal basis $\beta$. Furthermore, if $\beta = \sett{v_1,v_2,\cdots,v_n}$ and $x \in V$, then
	
	$$x = \sum^n_{i=1}\inn{x,v_i}v_i$$
\end{thm*}

\begin{cor*}
	Let $V$ be finite-dimensional inner product space with an orthonormal basis $\beta = \sett{v_1,v_2,\cdots,v_n}.$ Let $T$ be a linear operator on $V$, and let $A = [T]_{\beta}$. Then for any $i$ and $j$, $A_{ij} = \inn{T(v_j),v_i}$
\end{cor*}

\begin{defn}
	Let $\beta$ be an orthonormal subset (possibly infinite) of an inner product space $V$, and let $x \in V$. We define the \textbf{Fourier coefficients} of $x$ relative to $\beta$ to  
\end{defn}

\begin{tcolorbox}
	After have good tools, we are going to check the \textbf{orthogonal complement} of $S$.
\end{tcolorbox}

\begin{defn}
	Let $S$ be a nonempty subset of an inner product space $V$. We define $S^{\perp}$ to be the set of all vectors in $V$ that are orthogonal to every vector in $S$; that is, $S^{\perp} = \sett{x \in V~|~\inn{x,y} = 0 \text{ for all } y \in S}$. The set $S^{\perp}$ is called the \textbf{orthogonal complement} of $S$.
\end{defn}

\begin{thm*}
	Let $W$ be a finite-dimensional subspace of an inner product space $V$, and let $y \in V$. Then there exist unique vectors $u \in W$ and $z \in W^{\perp}$. i.e.
	
	$$V = W \oplus W^{\perp}$$
	
	Furthermore, if $\sett{v_1,v_2,\cdots,v_k}$ is an orthonormal basis for $W$, then
	
	$$u = \sum^k_{i=1}\inn{y,v_i}v_i$$
\end{thm*}

\begin{cor*}
	In the notation of Thm, the vector $u$ is the unique vector in $W$ that is "closest" to $y$; that is, for any $x \in W,||y-x||\geq ||y-u||,$ and this inequality is an equality if and only if $x = u$.
\end{cor*}

\begin{thm*}
	Suppose that $S = \sett{v_1,v_2,\cdots,v_k}$ is an orthonormal set in an n-dimensional inner product space $V$. Then
	
	\begin{enumerate}
		\item[(a)] $S$ can be extended to an orthonormal basis $\sett{v_1,v_2,\cdots,v_k,v_{k+1},\cdots,v_n}$ for V
		\item[(b)] If $W = \spann{S}$, then $S_1 = \sett{v_{k+1},v_{k+2},\cdots,v_n}$ is an orthonormal basis for $W^{\perp}$
		\item[(c)] If $W$ is any subspace of $V$, then $\dim (V) = \dim (W) + \dim (W^{\perp})$
	\end{enumerate}
\end{thm*}

\subsection{The Adjoint of a Linear Operator} $ $

\begin{tcolorbox} 
	In this section, we will introduce the adjoint linear operator and it's property.
	
	Before know adjoint operation, we need to know the theorem first.
\end{tcolorbox}

\begin{thm*}
	Let $V$ be a finite-dimensional inner product space over $\mathrm F$, and let $g: V \rightarrow F$ be a linear transformation. Then there exists a \textbf{unique} vector $y \in V$ such that $g(x) = \inn{x,y}$ for all $x \in V$
\end{thm*}

\textbf{\color{red} this proof next time}

\begin{thm*}
	Let V be a finite-dimensional inner product space, and let $T$ be a linear operator on $V$. Then there exists a unique function $T^*:V\rightarrow V$ such that $\inn{T(x),y} = \inn{x,T^*(y)}$ for all $x,y \in V$. Furthermore, $T^*$ is linear.
\end{thm*}

\begin{tcolorbox}
	We define the $T^*$ here, and next we want to know the $*$ in the matrix presentation. 
\end{tcolorbox}

\begin{thm*}
	Let $V$ be a finite-dimensional inner product space, and let $\beta$ be an orthonormal basis for $V$. If $T$ is a linear operator on $V$, then
	
	$$[T^*]_\beta = [T]^*_\beta$$
\end{thm*}

\begin{cor*}
	Let $A$ be an "$n \times n$" matrix. Then $L_{A^*} = (L_A)^*$
\end{cor*}

\begin{thm*}
	Let $V$ be an inner product space, and let $T$ and $U$ be linear operators on $V$. Then
	
	\begin{tasks}(2)
		\task $(T+U)^* = T^* + U^*$
		\task $(cT)^* = \overline{c}T^*$ for any $c \in \mathrm F$
		\task $(TU)^* = U*T*$
		\task $T^{**} = T*$
		\task $I^* = I$
	\end{tasks}
\end{thm*}

\begin{cor*}
	Let $A$ and $B$ be $n \times n$ matrices. Then
	
	\begin{tasks}(2)
		\task $(A + B)^* = A^* + B^*$
		\task $(cA)^* = \overline{c}A^*$ for all $c \in \mathrm F$
		\task $(AB)^{*} = B^*A^*$
		\task $A^{**} = A$
		\task $I^* = I$ 
	\end{tasks}
\end{cor*}

\textbf{\color{red} we skip the minimum function here(LAMMA$\sim$Thm 6.13)}


\subsection{Normal \& Self-Adjoint Operators}


\begin{lmma*}
	Let $T$ be a linear operator on a finite-dimensional inner product space $V$. If $T$ has an eigenvector, then so does $T^*$.
\end{lmma*}

\begin{thm*}[Schur]
	Let $T$ be a linear operator on a finite-dimension inner product space $V$. Suppose that the characteristic polynomial of $T$ splits. Then there exists an orthonormal basis $\beta$ for $V$ such that the matrix $[T]_\beta$ is upper triangular.
\end{thm*}

\begin{tcolorbox}
	I don't know why read Schur here, but we are going to introduce \textbf{normal} and it's property.
\end{tcolorbox}

\begin{defn}
	Let $V$ be an inner product space, and let $T$ be a linear operator on $V$. We say that $T$ is \textbf{normal} if $TT^* = T^*T$. An $n \times n$ real or complex matrix $A$ is \textbf{normal} if $AA^* = A^*A$.
\end{defn}


\begin{thm*}
	Let $V$ be an inner product space, and let $T$ be a \textbf{normal operator} on $V$. Then the following statements are true.
	
	\begin{tasks}
		\task $||T(x)|| = ||T^*(x)||$ for all $x \in V$.
		\task $T - cI$ is normal for every $c \in \mathrm F$
		\task If $x$ is an eigenvector of $T$, then $x$ is also an eigenvector of $T^*$. In fact, if $T(x) = \lambda x$, then $T^*(x) = \overline{\lambda}x$.
		\task If $\lambda_1$ and $\lambda_2$ are distinct eigenvalues of $T$ with corresponding eigenvectors $x_1$ and $x_2$, then $x_1$ and $x_2$ are orthogonal.
	\end{tasks}
\end{thm*}



















\end{document}