\documentclass[12pt]{article}
\usepackage[margin=1in]{geometry} 
\usepackage{amsmath}
\usepackage{tcolorbox}
\usepackage{amssymb}
\usepackage{amsthm}
\usepackage{mathrsfs}
\usepackage{lastpage}
\usepackage{fancyhdr}
\usepackage{accents}
\usepackage{tasks}
\linespread{1.2}
\parindent = 0pt
%\pagestyle{fancy}
\setlength{\headheight}{40pt}
\everymath{\displaystyle}

% we will modify sections, subsections and sub subsections
\RequirePackage{titlesec}
% Modification of section 
\titleformat{\section}[block]{\normalsize\bfseries\filcenter}{\thesection.}{.3cm}{} 

% modification of subsection and sub sub section
\titleformat{\subsection}[runin]{\bfseries}{ \thesubsection.}
{1mm}{}[.\quad]
\titleformat{\subsubsection}[runin]{\bfseries\itshape}{ \thesubsubsection.}
{1mm}{}[.\quad]

\newenvironment{solution}
  {\renewcommand\qedsymbol{$\blacksquare$}
  \begin{proof}[Solution]}
  {\end{proof}}
\renewcommand\qedsymbol{$\blacksquare$}

\newcommand{\ubar}[1]{\underaccent{\bar}{#1}}

%%%%%%%%%%%%%%%%%%%%%%%%%%%%%% Textclass specific LaTeX commands.
%\theoremstyle{plain}
\newtheorem{thm}{\protect\theoremname}[section]
\newtheorem{cor}[thm]{Corollary}
\newtheorem{lmma}[thm]{Lemma}
\newtheorem*{defn}{\underline{Definition}}
\newtheorem*{ex*}{Example}
\newtheorem*{sol*}{Solution}
\newtheorem*{cor*}{Corollary}
\newtheorem*{thm*}{Theorem}
\newtheorem*{lmma*}{Lemma}
\newtheorem*{rmk*}{Remark}
\newtheorem*{pf*}{\underline{\textbf{Proof\ }}}

%%%%%%%%%%%%%%%%%%%%%%%%%%%%%% User specified LaTeX commands.
\renewcommand{\P}{\mathscr{P}}
\newcommand{\B}{\mathscr{B}}
\newcommand{\A}{\mathscr{A}}
\newcommand{\C}{\mathbb{C}}
\newcommand{\CC}{\mathscr{C}}
\newcommand{\R}{\mathbb{R}}
\newcommand{\Q}{\mathbb{Q}}
\newcommand{\Z}{\mathbb{Z}}
\newcommand{\N}{\mathbb{N}}
\newcommand{\X}{\mathcal{X}}
\newcommand{\T}{\mathscr{T}}
\newcommand{\arbuni}{\bigcup_{\alpha\in I}}
\newcommand{\finint}{\bigcap_{i=1}^n}
\newcommand{\Ua}{{\textsc{U}_\alpha}}
\newcommand{\Ui}{\textsc{U}_i}
\newcommand{\pair}[2]{\left( \,#1\,,\,#2\,\right) }
\newcommand{\dint}[2]{\int_{#1}^{#2}}
\newcommand{\sett}[1]{\left\{ \,#1 \,\right\}}
\newcommand{\linearcombination}[2]{#1_1#2_1+\cdots+#1_n#2_n}
\newcommand{\slinearcombination}[1]{#1_1+\cdots+#1_n}
\newcommand{\spann}[1]{\text{span($#1$)}}
\newcommand{\sub}[1]{\text{sup}}
\newcommand{\inn}[1]{\left< #1 \right>}
%%%%%%%%%%%%%%%%%%%%%%%%%%%%%%

\begin{document}

%\lhead{Linear Algebra} 
%\rhead{Sabrina Edition} 
\cfoot{\thepage} %\ of \pageref{LastPage}}

\section{Error Analysis}

\textbf{Definition}:\\
\begin{tcolorbox}

let $x$ is a value, $\tilde{x}$ is a estimated value

\begin{enumerate}
	\item[(1)] absolute   error, $E_a = |x-\tilde{x}|$
	\item[(2)] relation   error, $E_r = |\frac{x-\tilde{x}}{x}|$
	\item[(3)] percentage error, $E_p = 100 \times |\frac{x-\tilde{x}}{x}|$
\end{enumerate}

\end{tcolorbox}

$\exists \epsilon > 0, |x-\tilde{x}| < \epsilon$, Then $\epsilon$ is upper limit of the absolute error measures the absolute accuracy.

\subsection{Error in Implementation of Numerical Methods}

\begin{enumerate}
	\item[(1)] Round-off Error
	\item[(2)] Overflow \& Underflow
	\item[(3)] Floating Point Arithmetic and Error Propagation
	\item[(4)] Truncation Error
	\item[(5)] Machine eps (Epsilon)
\end{enumerate}

\subsection*{(3)Floating Point Arithmetic and Error Propagation}
$ $\\

Let $x_1,x_2$ are values, $E_1,E_2$ are error of $x_1,x_2$,We want to check the change of error in $"+","-","*","/"$

\textbf{"+"}
\begin{tcolorbox}
	Let $x = x_1 + x_2$, error of $x$ is $E$
	
	Then $x+E = x_1 + x_2 + E_1 + E_2 \implies E = E_1 + E_2$
	
	by triangle inequality
	
	Absolute Error = $|E| \leq |E_1|+|E_2|$
	
	Relative Error = $\frac{|E|}{|x|} \leq \frac{|E_1|}{|x|} + \frac{|E_2|}{|x|}$
\end{tcolorbox}

\textbf{"-"}(Similar "+")\\

\newpage

\textbf{"*"}
\begin{tcolorbox}
Let $x = x_1 * x_2$

Then $x + E = (x_1 + E_1)(x_2 + E_2) = x_1x_2 + E_2x_1 + E_1x_2 + E_1E_2$

Absolute Error = $|E| \leq |x_2E_1|+|x_1E_2|$

Relative Error = $\frac{|E_1|}{|x|}  \leq \frac{|E_1|}{|x_1|} + \frac{|E_2|}{|x_2|}$

\end{tcolorbox}

\textbf{"/"}
\begin{tcolorbox}
	Let $x = x_1 /x_2$
	
	$x + E_x = \dfrac{x_1+E_1}{x_2+E_2}\left( \dfrac{x_2 - E_2}{x_2 - E_2}\right) = \dfrac{x_1x_2 + E_1x_2 - x_1E_2}{x_2^2 - E_2^2} + E_1E_2$
	
	Absolute Error = $|E_x| = |\dfrac{E_1x_2 - x_1E_2}{x_2^2}| \leq \dfrac{|E_1|}{|x_2|}+\dfrac{|x_1E_2|}{x_2^2}$
	
	Relative Error = $\dfrac{|E_x|}{|x|} \leq \dfrac{|E_1|}{|x_1 |} + \dfrac{|E_2|}{|x_2|}$
\end{tcolorbox}

\subsection*{(4)Truncation Error}

Cause by approximation infinite with its finite terms.

Use Taylor series ($f(x) \in P(C)$) as example

Let $x = a, f(x) = f(a)+f'(a)(x-a) + f''(a)\dfrac{(x-a)^2}{2!} + \cdots + \dfrac{(x-a)^n}{n!}f^{n}(a)+\cdots + Rn$

$Rn = \int^x_a\dfrac{(x-t)^n}{n!}f^{(n+1)}(t)dt$

\textbf{Thm 1(First Mean Value Theorem)}

\begin{tcolorbox}
	If $g$ is continuous on $[a,x]$, then $\exists~\xi$ between $a$ and $x$ s.t. 

	$$ \int^x_ag(t)~dt = g(\xi)(x-a)$$
\end{tcolorbox}



\textbf{Thm 2(Second Mean Value Theorem)}

\begin{tcolorbox}
	If $g,h$ is differentiable and integrable on $[a,x]$, $h$ does not change sign on $[a,x]$

	then $\exists~\xi$ that $a\leq\xi\leq x$ s.t.

$$ \int^x_ag(t)h(t)~dt = g(\xi)\int^x_ah(t)~dt$$
\end{tcolorbox}

since $t \in [a,x], h(t)=(x-t)^n\dfrac{1}{n!}$, $f^{(n+1)}(t)$ is continuous

$\exists ~\xi \in [a,x]$, $R_n = \dfrac{f^{(n+1)}(\xi)}{(n+1)!}f^{(x+1)}(\xi), ~\xi \in [a,a+h]$

(Ref. Violin page:799)

since power series convergent, $R_n(x) \rightarrow 0,~as_n \rightarrow \infty$

\newpage

Definition

\begin{tcolorbox}
	Given $\sett{a_n}\sett{b_n},~ b_n \geq 0, ~\forall n \geq 1$
	
	$a_n = O(b_n)~$ if $~\exists ~M > 0 \rightarrow |a_w| \leq Mb_n~\forall ~n\geq 1$
	
	$R_n(x)=O(h^{n+1})$
\end{tcolorbox}

\subsection{Condition \& Stability} $ $\\

Condition number is sensitivit of the function

Stability is used to describle the sensitity of the process

\textbf{Condition number of the $f(n)$}

$$\text{CN} = \dfrac{|\dfrac{f(x)-f(\tilde{x})}{x-\tilde{x}} |}{|\dfrac{x-\tilde{x}}{x} |} = |\dfrac{f(x)-f(\tilde{x})}{x-\tilde{x}}|\cdot|\dfrac{x}{f(x)}| = |\dfrac{x}{f(x)} \cdot f'(x) |$$

by Mean Value Theorem,

$$\dfrac{f(x)-f(\tilde{x})}{x-\tilde{x}} \thickapprox f'(x)$$

when CN $\leq 1$ is \textbf{well condition}, other is \textbf{ill condition}

when the function is more sensitive to change, the condition number will be more big.

\newpage

\section{Methods for solutions of the Equation $f(x)=0$}

we have four way to deal this problem

\begin{enumerate}
	\item[(1)] Direct analytical Method
	\item[(2)] Graphical
	\item[(3)] Trial and Error Method
	\item[(4)] Iterative Method
\end{enumerate}

\textbf{Thm. 3(Mean Value Theorem)}

\begin{tcolorbox}
	Let $f$ be a continuous function on $[a,b] = I$(connected),
	
	if $f(a) \leq c \leq f(b)$ that $\exists~\xi \in [a,b] \rightarrow f(\xi) = c$
\end{tcolorbox}

\textbf{Corollary}


\begin{tcolorbox}
	Let $f$ be a continuous function on $[a,b] = I$(connected)
	
	i.e. $f(a) \cdot f(b) < 0 ~\ni~ \exists c \in (a,b) ~\ni~ f(c) = 0$
	
	$c$ is a root of f(t)
\end{tcolorbox}

\section*{Iterative Method}
\subsection{Bisection Method}$ $\\

Let $a,b$ be fixed satisfying Thm.3

$\therefore f(a) \cdot f(b) < 0,f$ is continuous on $[a,b]$.
The first approximation is $x_0 = \dfrac{a+b}{2}$

if $f(a) \cdot f(x_0) \leq 0$, then By Thm. 3 the root will lie on $(a,x_0)$ and $x_1 = \dfrac{a + x_0}{2}$

continue the process, let $x_{n-3},x_{n-2},x_{n-1}$ be same step, then nth approximation

if $f(x_n-1) \cdot f(x_{n-3}) \leq 0$, then $x_n = \dfrac{x_{n-1}+x_{n-2}}{2}$

else $f(x_n-1) \cdot f(x_{n-3}) \geq 0$, then $x_n = \dfrac{x_{n-1}+x_{n-3}}{2}$

we shall label the interval by algorithm

$$[a,b] = [a_0,b_0],[a_1,b_1][a_2,b_2],\cdots$$

by construction $b_na_n = \frac{1}{2}(b_{n-1} - a_{n-1})$, Hence $b_n - a_n = \frac{1}{2^{n}}[b_0-a_0],~\forall n \geq 1$

Clearly $a_0 \leq a_1 \leq \cdots \leq b, b_0 \geq b_1 \geq \cdots \geq a, \sett{a_n},\sett{b_n}$ is bdd and monotonic

$$\lim_{n \rightarrow \infty}a_n  = \lim_{n \rightarrow \infty}b_n = f(r)$$

by assumption $f(a_n)f(b_n) < 0, \lim_{n \rightarrow \infty}f(a_n) = f(\lim_{n \rightarrow \infty}a_n) = f(r)$

$\therefore f(b_n) = f(r), 0 \leq [f(r)]^2 \leq 0 \implies f(r) = 0$


The process is called \textbf{nested internal property}

\newpage

Let $\sett{C_k}^{\infty}_{k=1}$ is a $\downarrow$ sequence of nonempty closed compact subset of X, then $\cap k \subset k\neq \varnothing $

if $c_k \rightarrow 0,$ then $\cap_kc_k = \sett{r}$

Let $\xi$ be the solution $f(x) = 0$, then $\sett{x_0-\xi} \leq \dfrac{b-a}{2},\cdots,\sett{x_n-\xi}\leq \dfrac{b-a}{2^{n+1}}$

\textbf{Definition(p-order-convergence)}\\
\begin{tcolorbox}
	$\sett{x_n}:\text{seq}, x_n \rightarrow z, s_n \rightarrow \infty$, define $\epsilon_n = z - x_n$, ~ if $\exists c > 0, p\geq 1$
	
	$$\lim_{n \rightarrow \infty}\dfrac{|\epsilon_{n+1}|}{|\epsilon_n|^p} = c$$
	
	we call $\sett{x_n}$ is $p$ order convergence
	
	if $c \leq 1,$ then it's good(only check this when  it's a first order convergence)
\end{tcolorbox}

Let $\epsilon_n$ be the error i.e. $\epsilon_n = |x_n - \xi|$, $\epsilon_n \leq \dfrac{b-a}{2^{n+1}} \leq \epsilon$, i.e. $h \geq \dfrac{ln(b-a) - ln\epsilon}{ln2}-1$

$\epsilon_n = |x_n - \xi| \leq \frac{1}{2}(\frac{b-a}{2^n}) \approx \frac{1}{2}\epsilon_n - 1 \implies \lim_{n \rightarrow \infty}|\dfrac{\epsilon_n}{\epsilon_n - 1}| = \dfrac{1}{2}$

Then Bisection Method is first order convergence

\subsection{Newton-Taphson Method}$ $\\
observation:

Let $x_0$ be an initial approximate to the root of $f(x) = 0$, then $x_0 + h$ is the exact root of $f(x)=0$, i.e. $f(x_0 + h) = 0$, from Taylor series, $f(x_0+h) = f(x_0)+h\cdot f(x_0)+\cdots$\\i.e. $x_0 \approx x_0 + h$

the first order approximation, $f(x_0+h) = f(x_0)+h\cdot f'(x_0)=0$





































\end{document}