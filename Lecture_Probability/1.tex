\documentclass[12pt]{article}
\usepackage[margin=1in]{geometry} 
\usepackage{amsmath}
\usepackage{tcolorbox}
\usepackage{amssymb}
\usepackage{amsthm}
\usepackage{mathrsfs}
\usepackage{lastpage}
\usepackage{fancyhdr}
\usepackage{accents}
\usepackage{tasks}
\linespread{1.2}
\parindent = 0pt
%\pagestyle{fancy}
\setlength{\headheight}{40pt}
\everymath{\displaystyle}

% we will modify sections, subsections and sub subsections
\RequirePackage{titlesec}
% Modification of section 
\titleformat{\section}[block]{\normalsize\bfseries\filcenter}{\thesection.}{.3cm}{} 

% modification of subsection and sub sub section
\titleformat{\subsection}[runin]{\bfseries}{ \thesubsection.}
{1mm}{}[.\quad]
\titleformat{\subsubsection}[runin]{\bfseries\itshape}{ \thesubsubsection.}
{1mm}{}[.\quad]

\newenvironment{solution}
  {\renewcommand\qedsymbol{$\blacksquare$}
  \begin{proof}[Solution]}
  {\end{proof}}
\renewcommand\qedsymbol{$\blacksquare$}

\newcommand{\ubar}[1]{\underaccent{\bar}{#1}}

%%%%%%%%%%%%%%%%%%%%%%%%%%%%%% Textclass specific LaTeX commands.
%\theoremstyle{plain}
\newtheorem{thm}{\protect\theoremname}[section]
\newtheorem{cor}[thm]{Corollary}
\newtheorem{lmma}[thm]{Lemma}
\newtheorem*{defn}{\underline{Definition}}
\newtheorem*{ex*}{Example}
\newtheorem*{sol*}{Solution}
\newtheorem*{cor*}{Corollary}
\newtheorem*{thm*}{Theorem}
\newtheorem*{lmma*}{Lemma}
\newtheorem*{rmk*}{Remark}
\newtheorem*{pf*}{\underline{\textbf{Proof\ }}}

%%%%%%%%%%%%%%%%%%%%%%%%%%%%%% User specified LaTeX commands.
\renewcommand{\P}{\mathscr{P}}
\newcommand{\B}{\mathscr{B}}
\newcommand{\A}{\mathscr{A}}
\newcommand{\C}{\mathbb{C}}
\newcommand{\CC}{\mathscr{C}}
\newcommand{\R}{\mathbb{R}}
\newcommand{\Q}{\mathbb{Q}}
\newcommand{\Z}{\mathbb{Z}}
\newcommand{\N}{\mathbb{N}}
\newcommand{\X}{\mathcal{X}}
\newcommand{\T}{\mathscr{T}}
\newcommand{\arbuni}{\bigcup_{\alpha\in I}}
\newcommand{\finint}{\bigcap_{i=1}^n}
\newcommand{\Ua}{{\textsc{U}_\alpha}}
\newcommand{\Ui}{\textsc{U}_i}
\newcommand{\pair}[2]{\left( \,#1\,,\,#2\,\right) }
\newcommand{\dint}[2]{\int_{#1}^{#2}}
\newcommand{\sett}[1]{\left\{ \,#1 \,\right\}}
\newcommand{\linearcombination}[2]{#1_1#2_1+\cdots+#1_n#2_n}
\newcommand{\slinearcombination}[1]{#1_1+\cdots+#1_n}
\newcommand{\spann}[1]{\text{span($#1$)}}
\newcommand{\sub}[1]{\text{sup}}
\newcommand{\inn}[1]{\left< #1 \right>}
%%%%%%%%%%%%%%%%%%%%%%%%%%%%%%

\begin{document}

\title{Probability}

\author{QSnake Edition}

\maketitle

%\section*{Probability Lecture 1}

%\lhead{Linear Algebra} 
%\rhead{Sabrina Edition} 
%\cfoot{\thepage} %\ of \pageref{LastPage}}

\subsection*{1. Review}$ $

\begin{enumerate}[wide, label = $\bullet$]
	\item experiment, trial: 
	 
	the process of obtaining an observed result of some phenomenon.
	\item outcome : observed result
\end{enumerate}

\begin{defn}
	The set of all possible outcomes of an experiment is called the sample space, denoted by $S$.
\end{defn}

\begin{defn}
	If a sample space $S$ is either finite or countably, then it is called a discrete sample space. Otherwise, is called a continuous sample space.
\end{defn}

\begin{defn}
	An event is a subset of the sample space $S$. If $A$ is an event, then "$A$ occurred" if "$A$ contains the out come that occurred"
\end{defn}

%\begin{defn}
%	An event is called and elementary event if it contains exactly one outcome of the experiment.
%\end{defn}
%
%\begin{defn}$ $
%	\begin{enumerate}[wide, label = $\bullet$]
%		\item Two events $A$ and $B$ are called mutually exclusive if $A \cap B = \emptyset$
%		\item Events $A_1,A_2,A_3,\cdots$ are said to be mutually exclusive if they are pairwise mutually exclusive. That is, if $A_i \cap A_j = \emptyset$ whenever $i \neq j$.
%	\end{enumerate}
%\end{defn}

\begin{defn}
	For a given experiment, $S$ denotes the sample space and $A_1,\cdots$ represent possible events. A set function that associates a real value $P(A)$ with each event $A$ is called a probability set function, and $P(A)$ is called the probability of $A$, if the following properties are satisfied:
	
	\begin{enumerate}[wide, label = $\roman*)$]
		\item $0 \leq p(A)$ for every $A$
		\item $P(S) = 1$
		\item $P(\bigcup_{i=1}^{\infty} A_i) = \sum^{\infty}_{i = 1}P(A_i)$ if $A_1,\cdots$ are pairwise mutually exclusive events.
	\end{enumerate}
\end{defn}

%\begin{defn}
	%If an object is chosen from a finite collection of distinct objects in such a manner that each object has the same probability of begin chosen, then we say that the object was chosen at random.
%\end{defn}

%\begin{thm*}$ $
%	\begin{enumerate}[wide,label = $\bullet$]
%		\item $P(A) = 1 - P(A')$
%		\item $P(A) \leq 1$, for any event $A$
%		\item For any two event $A$ and $B$, $P(A \cup B) = P(A) + P(B) - P(A \cap B)$
%		\item If $A \subset B,$ then $P(A) \leq P(B)$
%		\item $P\left( \bigcup^{\infty}_{i = 1} A_i \right) \leq \sum^{\infty}_{i = 1}P(A_i)$ If $A_1,\cdots$ is a sequence of events
%		\item If $A_1,A_2,\cdots,A_k$ are events, then $P(\bigcap_{i=1}^{k}A_i) \geq 1 - \sum_{i=1}^kP(A_i')$ 
%	\end{enumerate}
%\end{thm*}

%\begin{defn}
%	The conditional probability of an event $A$, given the event $B$, is defined by 
%	
%	$$P(A~|~B) = \frac{P(A \cap B)}{P(B)}$$
%	
%	if $P(B) \neq 0$
%\end{defn}

%\begin{thm*}
%	For any events $A$ and $B$,
%	
%	$$P(A \cap B) = P(B)P(A~|~B) = P(A)P(B~|~A)$$
%\end{thm*}
%
%\begin{thm*}
%	If $B_1,\cdots,B_k$ is a collection of mutually exclusive and exhaustive events, then for any event $A$
%	
%	$$P(A) = \sum^{k}_{i = 1}P(B_i)P(A~|~B_i)$$
%\end{thm*}

%\textbf{Note.} exhaustive events:  A collection of event which union is sample space.
%
%\begin{thm*}
%	If $B_1,\cdots,B_k$ is mutually exclusive and exhaustive events, then for any event $A$ and each $j = 1,\cdots,k$
%	
%	$$P(B_j ~|~ A) = \dfrac{P(B_j)P(A~|~B_j)}{\sum^{k}_{i=1}P(B_i)P(A~|~B_i)} \left(  = \dfrac{P(A \cap B_j)}{P(A)}\right)$$
%\end{thm*}
%
%\begin{defn}
%	Two events $A$ and $B$ are called independent events if
%	
%	$$P(A \cap B) = P(A)P(B)$$
%	
%	Otherwise, $A$ and $B$ are called dependent event
%\end{defn}
%
%\begin{thm*}
%	If $A$ and $B$ are events such that $P(A) > 0$ and $P(B) > 0$, and $A$ and $B$ are independent, we get
%	
%	$$P(A \cap B) = P(A)P(B) \Leftrightarrow P(A~|~B) = P(A) \Leftrightarrow P(B~|~A) = P(B)$$
%\end{thm*}
%
%\begin{thm*}
%	\begin{eqnarray*}
%		P(A \cap B) &=& P(A)P(B)\\
%		\Leftrightarrow P(A' \cap B) &=& P(A')P(B)\\
%		\Leftrightarrow P(A \cap B') &=& P(A)P(B')\\ 
%		\Leftrightarrow P(A' \cap B') &=& P(A')P(B')
%	\end{eqnarray*}
%\end{thm*}


%\begin{defn}
%	The $k$ events $A_1,\cdots,A_k$ are said to b independent or mutually independent if for every $j = 2,3,\cdots ,k$ and every subset of distinct indices $i_1,i_2,\cdots,i_j$
%	
%	$$P(A_{i1} \cap A_{i2} \cap \cdots \cap A_{ij}) = P(A_{i1})P(A_{i2})\cdots P(A_{ij})$$
%\end{defn}

\textbf{Example.}

Throwing a coin, the outcome is "head" or "tail",  the sample space is $\{head,tail\}$, we can give it an event A like "a head exists", the $A = \{head\} \subseteq S$

\newpage

\subsection*{2. Discrete Random Variable}

\begin{defn}
	A random variable, say $X$, is a function defined over a sample space $S$, that associated a real number with each possible outcome in $S$
	
	$$X(e) = x, \text{ where } e \in S$$
\end{defn}

\begin{defn}
	A random variable that can take on at most a countable number of possible values is said to be discrete. For a discrete random variable $X$, we define the probability mass function $p(a)$ of $X$ by
	
	$$p(a) = P\{X = a\}$$
\end{defn}

\textbf{Property.} $f(x_i) \geq 0,~\sum_{\text{all }x_i}f(x_i) = 1$
%\begin{defn}
%	The cumulative distribution function(CDF) of a random variable $X$ is defined for any real $x$ by
%	
%	$$F(x) = P[X \leq x]$$
%\end{defn}
%
%
%\begin{thm*}
%	Let $X$ be a discrete random variable with pdf $f(x)$ and CDF $F(x)$. If the possible values of $X$ are indexed in increasing order, $x_1 < x_2 < x_3 < \cdots ,$ then:
%	
%	\begin{enumerate}[wide,label = ($\roman*$)]
%		\item $f(x_1) = F(x_i)$
%		\item for any $i>1,~f(x_i) = F(x_i) - F(x_{i-1})$
%		\item if $x < x_1$ then $F(x) = 0$
%		\item $F(x) = \sum_{x_i \leq x}f(x_i)$
%	\end{enumerate}
%\end{thm*}

%\begin{thm*}
%	A function $F(x)$ is a CDF for some random variable $X$ if and only if it satisfies: 
%	
%	\begin{enumerate}[wide, label = ($\roman*$)]
%		\item $\lim_{x \rightarrow -\infty}F(x) = 0$
%		\item $\lim_{x \rightarrow \infty}F(x) = 1$
%		\item $\lim_{h \rightarrow 0^+} F(x+h) = F(x)$
%		\item $a < b$ implies $F(a) \leq F(b)$
%	\end{enumerate}
%\end{thm*}

\begin{defn}
	If $X$ is a discrete random variable having a probability mass function $p(x)$, then the expectation, or the expected value, of $X$, denoted by $E\left[X\right]$, is defined by	
	$$E(X) = \sum_{x}xf(x)$$
	
	and we some times denote $E(X)= \mu$
\end{defn}

\textbf{Note.} Don't just remember that the expected value just the mean.

$ $

\textbf{Example.} If we define $X("head") = 20$, $X("tail") = 100$, we get $E\left[X\right] =60$ but it cannot give us more information.

$ $

\begin{thm*}
	If $X$ is a discrete random variable that takes on one of the values $x_i,~i\geq 1$, with respective probabilities $p(x_i)$, then, for any real-valued function $g$,
	
	$$E[g(X)] = \sum_{i}g(x_i)p(x_i)$$
\end{thm*}

\newpage

\begin{proof}
	By grouping together all the terms in $\sum_ig(x_i)p(x_i)$
	
	\begin{eqnarray*}
		\sum_ig(x_i)p(x_i) &=& \sum_j \sum_{i:g(x_i)=y_j}g(x_i)p(x_i)\\
		&=& \sum_j\sum_{i:g(x_i)=y_j}y_jp(x_i)\\
		&=& \sum_jy_j\sum_{i:g(x_i)=y_j}p(x_i)\\
		&=&\sum_{j}y_jP\{g(X) = y_j\}\\
		&=& E[g(X)] 
	\end{eqnarray*}
\end{proof}

\begin{cor*}
	If $a$ and $b$ are constants, then 
	
	$$E[aX + b] = aE[X] + b$$
\end{cor*}

\begin{defn}
	If $X$ is a random variable with mean $\mu$, then the variance of $X$, denoted by $\text{Var}(X)$, is defined by
	
	$$\text{Var}(X) = E[(X - \mu)^2]$$
\end{defn}

\textbf{Prop.} An alternative formula for Var($X$) is derived as follows:

$$\text{Var}(X) = E[(X - \mu)^2] = E[X^2] - (E[X])^2$$

\begin{proof}
	think $(X - \mu)^2$ as $g(X),$ use the theorm above you get
	
	\begin{eqnarray*}
		\text{Var}(X) &=& E[(X - \mu)^2]\\
		&=& \sum_x(x - \mu)^2p(x)\\
		&=& \sum_x(x^2 - 2\mu x + \mu^2)p(x)\\
		&=&\sum_x x^2p(x) - 2\mu \sum_x xp(x) + \mu^2\sum_x p(x)\\
		&=& E[X^2] - 2\mu^2 + \mu^2\\
		&=& E[X^2] - \mu^2
	\end{eqnarray*}
\end{proof}

\newpage

\subsection*{3. Bernoulli and Binomial Random Variables}$ $

$ $

\begin{defn}
	A trial whose outcome can be classified as either a success or a failure is performed. If we let $X = 1$ when the outcome is a \textbf{success} and $X = 0$ when it is a failure, then the probability mass function of $X$ is given by
	
	\begin{eqnarray*}
		p(0) &=& P\{X = 0\} = 1 - p\\
		p(1) &=& P\{X = 1\} = p
	\end{eqnarray*}
	
	and the random variable $X$ is said to be a Bernoulli random variable.
\end{defn}


\begin{defn}
	Suppose now that $n$ independent trials, each of which results in a success with probability $p$ or in a failure with probability $1 - p$, are to be performed. If $X$ represents the \textbf{number of successes} that occur in the $n$ trials, then $X$ is said to be a binomial random variable with parameters $(n,p)$.
	
	$ $
	
		The probability mass function of a binomial random variable having parameters $(n,p)$ is given by
	
	$$p(i) = \left(\begin{matrix}
		n \\ i
	\end{matrix}\right) p^i(1 - p)^{n-i} ~|~i = 0,1,\cdots,n$$
\end{defn}

$ $

\textbf{Note.} A Bernoulli random variable is just a binomial random variable with parameters $(1,p)$.

$ $

\textbf{Example.} We have 10 bulbs in the cases, we know that 2 of 10 is broken. If we pick the bulbs randomly (put it back after pick), what is the probability distribution?

\begin{solution} $ $
	\begin{center}
	\begin{tabular}{|c|c|c|c|c|}
	\hline
	x & 0 & 1 & 2 & 3\\
	\hline
	p(x) & 
	$C^3_0 \left( \frac{2}{10}\right)^{0}\left(\frac{8}{10}\right)^3$ & 
	$C^3_1 \left( \frac{2}{10}\right)^{1}\left(\frac{8}{10}\right)^2$ & 
	$C^3_2 \left( \frac{2}{10}\right)^{2}\left(\frac{8}{10}\right)^1$ & 
	$C^3_3 \left( \frac{2}{10}\right)^{3}\left(\frac{8}{10}\right)^0$\\
	\hline
\end{tabular}
\end{center}
\end{solution}

\newpage

\textbf{The Expect Value \& Variance of Binomial}
$ $

X is a binomial random variable with parameters $(n,p)$.

\begin{center}
	$E(X) = np~$ \& $~\text{Var}(x) = np(1-p)$
\end{center}

\begin{proof}
	Just use the definition of $E(X)$ and Var($X$)
	
	\begin{eqnarray*}
		E(X) &=& \sum_i^n x_ip(x_i)\\
		&=& \sum_i^n i \left(\begin{matrix}
		n \\ i
	\end{matrix}\right) p^i(1 - p)^{n-i}\\
	&=& \sum^n_i i\dfrac{n!}{i!(n-i)!}p^i(1-p)^{n-i}\\
	&=& \sum^n_{i=1}\dfrac{n!}{(i-1)!(n-i)!}p^i(1-p)^{n-i}\\
	&=& n\sum^n_{i=1} \dfrac{(n-1)!}{(i-1)!(n-i)!}p^i(1-p)^{n-i}\\
	&=& np\sum^n_{i=1}\dfrac{(n-1)!}{(i-1)(n-i)!}p^{i-1}q^{n-i}\\
	&=& np\sum^n_{i=1}\left(\begin{matrix}
		n-1 \\ i-1
	\end{matrix}\right)p^{i-1}(1-p)^{n-i}\\
	&=& np
	\end{eqnarray*}
\end{proof}

\textbf{Example.} A shooter whose shooting rate is $0.6$, today he shoot $100$ times, what is the expect value and variance?

\begin{solution}
	$E[X] = 0.6 \times 100 = 60,~\text{Var}(X) = 100 \times (1-0.6) \times 0.6$
\end{solution}






\newpage

\subsection*{$\text{4}^*$. Moment generating function (Discrete)}$ $

\begin{defn}
	The moment generating function $M(t)$ of the random variable $X$ is defined for all real values of $t$ by 
	
	$$M(t) = E\left[e^{tX}\right] = \sum_x e^{tx}p(x)$$
	
	if $X$ is discrete with mass function $p(x)$
\end{defn}

$ $

\textbf{Mean.} We call $M(t)$ the moment generating function because all of the moments of $X$ can be obtained by successively differentiating $M(t)$ and then evaluating the result at $t = 0$. For example,

\begin{eqnarray*}
	M'(t) &=& \frac{d}{dt}E[e^{tX}]\\
	&=& E[\frac{d}{dt}(e^{tX})]\\
	&=& E[Xe^{tX}]
\end{eqnarray*}

where we have assumed that the interchange of the differentiation and expectation operators is legitimate. That is, we have assumed that

$$\frac{d}{dt}\left[\sum_xe^{tx}p(x)\right] = \sum_x \frac{d}{dt}\left[e^{tx}p(x)\right]dx$$

in the discrete case

$ $

In general, the $n$th derivative of $M(t)$ is given by

$$M^n(t) = E\left[X^ne^{tX}\right]$$

implying that 

$$M^n(0) = E\left[X^n\right] ~ n\geq 1$$

\newpage

\textbf{Example.} If $X$ is a binomial random variable with parameters $n$ and $p$, then 

\begin{eqnarray*}
	M(t) &=& E\left[e^{tX}\right]\\
	&=&\sum^n_{k=0}e^{tk}\left(\begin{matrix}
		n \\ k
	\end{matrix}\right) p^k( 1- p)^{n-k}\\
	&=& \sum^n_{k=0}\left(\begin{matrix}
		n\\k
	\end{matrix}\right)(pe^t)^k(1-p)^{n-k}\\
	&=&(pe^t + 1 - p)^n
\end{eqnarray*}

where the last equality follows from the binomial theorem. Differentiation yields

$$M'(t) = n(pe^t + 1 - p)^{n-1}pe^t$$

Thus,

$$E[X] = M'(0) = np$$

Differentiating a second time yields

$$M''(t) = n(n - 1)(pe^t + 1 - p)^{n-2}(pe^t)^2 + n(pe^t + 1 - p)^{n-1}pe^t$$

so

$$E[X^2] = M''(0) = n(n-1)p^2 + np$$

The variance of $X$ is given by

\begin{eqnarray*}
	\text{Var}(X) &=& E[X^2] - (E[X])^2\\
	& = & n(n - 1)p^2 + np - n^2p^2\\
	& = & np(1 - p)
\end{eqnarray*}
\newpage

\subsection*{5. Poisson Random Variable}$ $

$ $

\begin{defn}
	A random variable $X$ that takes on one of the values $0,1,2,\cdots$ is said to be a Poisson random variable with parameter $\lambda$ if, for some $\lambda > 0$,
	
	$$p(i) = P\{X = i\} = e^{-\lambda}\dfrac{\lambda^i}{i!} ~|~ i = 0,1,2,\cdots$$
	
	and above defines a probability mass function, since
	
	$$\sum^{\infty}_{i = 0}p(i) = e^{-\lambda}\sum^{\infty}_{i=0}\dfrac{\lambda^i}{i!} = e^{-\lambda}e^{\lambda} = 1$$
\end{defn}

\textbf{The Expect Value \& Variance of Poisson}

X is a Poisson random variable with parameter $\lambda$

$$E(X) = \lambda ~\&~ \text{Var}(X) = \lambda$$

\begin{thm*}
	We can think poisson is a binomial distribution with parameter (n,p) where $n \rightarrow \infty$
\end{thm*}

\begin{proof}

To see this, suppose that $X$ is a binomial random variable with parameters $(n,p),$ and let $\lambda = np$.Then
	\begin{eqnarray*}
		P\{X = i\} &=& \dfrac{n!}{(n-i)!i!}p^i(1 - p)^{n - i}\\
		&=& \dfrac{n!}{(n-i)!i!}\left(\dfrac{\lambda}{n}\right)^i\left(1 - \dfrac{\lambda}{n}\right)^{n - i}\\
		&=& \dfrac{n(n-1)\cdots(n-i+1)}{n^i}\dfrac{\lambda^i}{i!}\dfrac{(1 - \lambda /n)^n}{(1 - \lambda /n)^i}
	\end{eqnarray*}
	
	Now, for $n$ large and $\lambda$ moderate,
	
	\begin{tasks}(2)
		\task[$\bullet$] $\dfrac{n(n-1)\cdots(n-i+1)}{n^i} \approx 1$
		\task[$\bullet$] $\left( 1 -\dfrac{\lambda}{n}\right)^n\approx e^{-\lambda}$
		\task[$\bullet$] $\left( 1 -\dfrac{\lambda}{n}\right)^i \approx 1$
	\end{tasks}
	
	Hence, for $n$ large and $\lambda$ moderate,
	
	$$P\{X = i\} \approx e^{-\lambda}\dfrac{\lambda^i}{i!}$$
\end{proof}

\textbf{Example.}

In average, it's will has three cars pass the free way, then what is the probability that exactly two cars passing the freeway.

\begin{solution}
	let $\lambda = 3$, what is $P(x=2)$?
	
	$P(x = 2) = \dfrac{e^{-3}\cdot 3^2}{2!}$
\end{solution}




%\subsection*{6. }



















\end{document}