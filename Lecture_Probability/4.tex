\documentclass[12pt]{article}
\usepackage[margin=1in]{geometry} 
\usepackage{amsmath}
\usepackage{tcolorbox}
\usepackage{amssymb}
\usepackage{amsthm}
\usepackage{mathrsfs}
\usepackage{lastpage}
\usepackage{fancyhdr}
\usepackage{accents}
\usepackage{tasks}
\linespread{1.2}
\parindent = 0pt
%\pagestyle{fancy}
\setlength{\headheight}{40pt}
\everymath{\displaystyle}

% we will modify sections, subsections and sub subsections
\RequirePackage{titlesec}
% Modification of section 
\titleformat{\section}[block]{\normalsize\bfseries\filcenter}{\thesection.}{.3cm}{} 

% modification of subsection and sub sub section
\titleformat{\subsection}[runin]{\bfseries}{ \thesubsection.}
{1mm}{}[.\quad]
\titleformat{\subsubsection}[runin]{\bfseries\itshape}{ \thesubsubsection.}
{1mm}{}[.\quad]

\newenvironment{solution}
  {\renewcommand\qedsymbol{$\blacksquare$}
  \begin{proof}[Solution]}
  {\end{proof}}
\renewcommand\qedsymbol{$\blacksquare$}

\newcommand{\ubar}[1]{\underaccent{\bar}{#1}}

%%%%%%%%%%%%%%%%%%%%%%%%%%%%%% Textclass specific LaTeX commands.
%\theoremstyle{plain}
\newtheorem{thm}{\protect\theoremname}[section]
\newtheorem{cor}[thm]{Corollary}
\newtheorem{lmma}[thm]{Lemma}
\newtheorem*{defn}{\underline{Definition}}
\newtheorem*{ex*}{Example}
\newtheorem*{sol*}{Solution}
\newtheorem*{cor*}{Corollary}
\newtheorem*{thm*}{Theorem}
\newtheorem*{lmma*}{Lemma}
\newtheorem*{rmk*}{Remark}
\newtheorem*{pf*}{\underline{\textbf{Proof\ }}}

%%%%%%%%%%%%%%%%%%%%%%%%%%%%%% User specified LaTeX commands.
\renewcommand{\P}{\mathscr{P}}
\newcommand{\B}{\mathscr{B}}
\newcommand{\A}{\mathscr{A}}
\newcommand{\C}{\mathbb{C}}
\newcommand{\CC}{\mathscr{C}}
\newcommand{\R}{\mathbb{R}}
\newcommand{\Q}{\mathbb{Q}}
\newcommand{\Z}{\mathbb{Z}}
\newcommand{\N}{\mathbb{N}}
\newcommand{\X}{\mathcal{X}}
\newcommand{\T}{\mathscr{T}}
\newcommand{\arbuni}{\bigcup_{\alpha\in I}}
\newcommand{\finint}{\bigcap_{i=1}^n}
\newcommand{\Ua}{{\textsc{U}_\alpha}}
\newcommand{\Ui}{\textsc{U}_i}
\newcommand{\pair}[2]{\left( \,#1\,,\,#2\,\right) }
\newcommand{\dint}[2]{\int_{#1}^{#2}}
\newcommand{\sett}[1]{\left\{ \,#1 \,\right\}}
\newcommand{\linearcombination}[2]{#1_1#2_1+\cdots+#1_n#2_n}
\newcommand{\slinearcombination}[1]{#1_1+\cdots+#1_n}
\newcommand{\spann}[1]{\text{span($#1$)}}
\newcommand{\sub}[1]{\text{sup}}
\newcommand{\inn}[1]{\left< #1 \right>}
%%%%%%%%%%%%%%%%%%%%%%%%%%%%%%

\begin{document}

\title{Probability}

\author{QSnake Edition}

\maketitle

\subsection*{0. Property of Expectation}$ $

Since $X$ is a weighted average of the possible values of $X$, it follows that if $X$ must lie between $a$ and $b$, then so must its expected value. That is, if 

$$P\{a \leq X \leq b\} = 1$$

then $a \leq E[X] \leq b$

\begin{proof}
	Suppose $X$ is a discrete random variable for which $P\{a \leq X \leq b\} = 1$. Since this implies that $p(x) = 0$ for all $x$ outside of the interval $[a,b]$, it follows that
	
	\begin{eqnarray*}
		E[x] &=& \sum_{x:p(x)>0}xp(x) \\
		&\geq& \sum_{x:p(x)>0} ap(x) \\
		&=& a\sum_{x:p(x)>0} p(x) \\
		&=& a
	\end{eqnarray*}
	
	and the proof in the continuous case is similar, the result follows. 
\end{proof}

\subsection*{1. Expectation of Sums of Random Variables} $ $

\textbf{Recall.} $X$ is a non-negative random variable, we have

$$E[X] = \int^{\infty}_0 P\{X > t\}dt$$

\begin{proof}
	\begin{eqnarray*}
		E[X] &=& \int^{\infty}_0 xf(x)dx\\
		&=& \int^{\infty}_{0}\left(\int^x_0 1 dt \right)f(x)dx\\
		&=& \int^{\infty}_{0}\int^x_0 f(x) dt dx\\
		&=& \int^{\infty}_{0}\left(\int^{\infty}_{t}f(x)dx\right)dt\\
		&=& \int^{\infty}_{0}P\{X > t\}dt
	\end{eqnarray*}
\end{proof}

\begin{prop*}
	If $X$ and $Y$ have a joint probability mass function $p(x,y)$ then 
	
	$$E[g(X,Y)] = \sum_y\sum_x g(x,y)p(x,y)$$
	
	If $X$ and $Y$ have a joint probability density function $f(x,y)$, then
	
	$$E[g(X,Y)] = \int^{\infty}_{-\infty}\int^{\infty}_{-\infty} g(x,y)f(x,y)dxdy$$
\end{prop*}

\begin{proof}
	Use the recall above we get
	$$E[g(X,Y)] = \int^{\infty}_0 P\{g(X,Y) > t\}dt$$
	Writing 
	$$P\{g(X,Y) > t\} = \iint_{(x,y):g(x,y) > t}f(x,y)dydx$$
	shows that
	$$E[g(X,Y)] = \int^{\infty}_{0}\iint_{(x,y):g(x,y)>t}f(x,y)dydx$$
	Interchanging the order of integration gives
	
	\begin{eqnarray*}
		E[g(X,Y)] &=& \int_x\int_y\int^{g(x,y)}_{t = 0}f(x,y)dtdydx\\
		&=& \int_x\int_yg(x,y)f(x,y)dydx
	\end{eqnarray*}
\end{proof}

\begin{cor*}
	$E[X]$ and $E[Y]$ are finite,
	
	$$E[X + Y] = E[X] + E[Y]$$
	
\end{cor*}

\begin{proof}
	Suppose $E[X]$ and $E[Y]$ are both finite and let $g(X,Y) = X + Y$. Then, in the continuous case,
	
	\begin{eqnarray*}
		E[X + Y] &=& \int^{\infty}_{\infty}\int^{\infty}_{\infty}(x + y)f(x,y)dxdy\\
		&=& \int^{\infty}_{-\infty}\int^{\infty}_{-\infty}xf(x,y)dydx + \int^{\infty}_{-\infty}\int^{\infty}_{-\infty}yf(x,y)dxdy\\
		&=& \int^{\infty}_{-\infty}xf_X(x)dx + \int^{\infty}_{-\infty}yf_Y(y)dy\\
		&=& E[X] + E[Y]
	\end{eqnarray*}
\end{proof}

\newpage

\subsection*{2*. Moments of the Number of Events that Occur} $ $

For given events $A_1,\cdots,A_n$, find $E[X]$, where $X$ is the number of these events that occur. The solution then involved defining an indicator variable $I_i$ for event $A_i$ such that

$$I_i = \begin{cases}
	1 & $if $ A_i $ occurs$ \\
	0 & $otherwise$
\end{cases}$$

Because

$$X = \sum^n_{i = 1}I_i$$

we obtained the result

$$E[X] = E\left[ \sum^n_{i = 1} I_i \right] = \sum^n_{i = 1}E[I_i] = \sum^n_{i = 1}1 \cdot P(A_i) + 0 \cdot P(A_i^c) =  \sum^n_{i = 1}P(A_i)$$

Consider the number of pairs of events that occur. Because $I_i I_j$  will equal $1$ if both $A_i$ and $A_j$ occur and will equal $0$ otherwise, it follows that the number of pairs is equal to $\sum_{i < j}I_iI_j$. But because $X$ is the number of events that occur, it also follows that the number of pairs of events that occur is $(\begin{matrix}
	X \\ 2
\end{matrix})$. Consequently,

$$\left( \begin{matrix}
	X \\ 2 
\end{matrix} \right) = \sum_{i < j}I_iI_j$$

and taking expectations yields

$$E\left[ \left( \begin{matrix} X \\ 2\end{matrix} \right) \right] = \sum_{i < j}E[I_iI_j] = \sum_{i < j}P(A_iA_j)$$

or 

$$E\left[ \dfrac{X(X - 1)}{2} \right] = \sum_{i < j}P(A_iA_j)$$

giving that 

$$E[X^2] - E[X] = 2\sum_{i < j}P(A_iA_j)$$

Moreover, by considering the number of distinct subsets of $k$ events that all occur, we see that

$$\left( \begin{matrix}
	X \\ k
\end{matrix}\right) = \sum_{i_1 < i_2 < \cdots < i_k} I_{i_1}I_{i_2}\cdots I_{i_k}$$

Taking expectations gives the identity

$$E\left[ \left( \begin{matrix}
	X \\ k
\end{matrix} \right) \right] = \sum_{i_1 < i_2 < \cdots < i_k}E[I_{i_1} I_{i_2}\cdots I_{i_k}] = \sum_{i_1 < i_2 < \cdots < i_k} P{A_{i_1}A_{i_2} \cdots A_{i_k}}$$

\newpage

\subsection*{3. Covariance, Variance of Sums, and Correlations}$ $

\begin{prop*}
	If $X$ and $Y$ are independent, then for any functions $h$ and $g$, 
	
	$$E[g(X)h(Y)] = E[g(X)]E[h(Y)]$$
\end{prop*}

\begin{proof}
	Suppose that $X$ and $Y$ are jointly continuous with joint density $f(x,y)$. Then
	
	\begin{eqnarray*}
		E[g(X)h(Y)] &=& \int^{\infty}_{\infty} \int^{\infty}_{-\infty} g(x)h(x)f(x,y)dxdy\\
		&=& \int^{\infty}_{\infty} \int^{\infty}_{-\infty} g(x)h(x)f_X(x)f_Y(y)dxdy \\
		&=& \int^{\infty}_{-\infty} h(y)f_Y(y)dy \int^{\infty}_{-\infty}g(x)f_X(x)dx \\
		&=& E[h(Y)]E[g(X)]
	\end{eqnarray*}
	
	The proof in the discrete case is similar.
\end{proof}

\begin{defn}
	The covariance between $X$ and $Y$, denote by Cov$(X,Y)$, is defined by
	
	$$ \text{Cov}(X,Y) = E[(X - E[X])(Y - E[Y])] $$
\end{defn}

Upon expanding the right side of the preceding definition, we see that

\begin{eqnarray*}
	\text{Cov}(X,Y) &=& E[XY - E[X] Y - XE[Y] + E[Y]E[E]] \\
	&=& E[XY] - E[X]E[Y] - E[X]E[Y] + E[X]E[Y] \\
	&=& E[XY] - E[X]E[Y]
\end{eqnarray*}

\begin{rmk*}
	If $X$ and $Y$ are independent, then Cov$(X,Y) = 0$. However, the converse is not true.
\end{rmk*}

\textbf{Example.}

$$P\{X = 0\} = P\{X = 1\} = P\{X = -1\} = \dfrac{1}{3}$$

and defining 

$$Y = \begin{cases}
	0 & $if $ X \neq 0\\
	1 & $if $ X = 0
\end{cases}$$ 

Now, $XY = 0$, so $E[XY] = 0$. Also, $E[X] = 0$.

\begin{prop*} $ $
	\begin{enumerate}[label = (\roman*)]
		\item Cov$(X,Y) =$ Cov$(Y,X)$
		\item Cov$(X,X) =$ Var$(X)$
		\item Cov$(aX, Y = a$Cov$(X,Y)$ 
		\item Cov$\left(  \sum^{n}_{i = 1}X_i,\sum^m_{j = 1} \right) = \sum^n_{i = 1}\sum^m_{j = 1}$Cov $(X_i,Y_j)$
	\end{enumerate}
\end{prop*}

\begin{proof}
	We only proof (iv) here, let $\mu_i = E[X_i]$ and $v_j = E[Y_j]$. Then
	
	$$E\left[ \sum^n_{i = 1}X_i \right] = \sum^n_{i = 1}\mu_i,\quad E\left[ \sum^m_{j = 1}Y_i \right] = \sum^m_{j = 1}v_j$$
	
	and 
	
	\begin{eqnarray*}
		Cov\left( \sum^n_{i = 1}X_i,\sum^m_{j=1}Y_j \right) &=& E\left[ \left( \sum^n_{i = 1}X_i - \sum^n_{i = 1}\mu_i \right)\left( \sum^m_{j = 1}Y_j - \sum^m_{j =1}v_j \right) \right] \\
		&=& E\left[ \sum^n_{i = 1}(X_i - \mu_i)\sum^m_{j = 1}(Y_j - v_j) \right] \\
		&=& E\left[ \sum^n_{i = 1}\sum^{m}_{j = 1}(X_i - \mu_i)(Y_j - v_j) \right]\\
		&=& \sum^n_{i = 1}\sum^m_{j = 1}E[(X_i - \mu_i)(Y_j - v_j)]
	\end{eqnarray*}
\end{proof}

\begin{cor*}
	$$Var\left( \sum^n_{i = 1}X_i \right) = \sum^n_{i = 1}Var(X_i) + 2{\sum\sum}_{i < j}Cov(X_i,X_j)$$
\end{cor*}

\begin{proof}
	\begin{eqnarray*}
		Var\left(\sum^n_{i = 1}X_i\right) &=& Cov\left(\sum^n_{i = 1}X_i,\sum^n_{j = 1}X_j\right) \\
		&=& \sum^{n}_{i = 1}\sum^{n}_{j = 1}Cov(X_i,X_j)\\
		&=& \sum^n_{i = 1}Var(X_i) + \sum\sum_{i \neq j}Cov(X_i,X_j)
	\end{eqnarray*}
\end{proof}

\begin{cor*}
	If $X_1$ and $X_2$ are random variables with joint pdf $f(x_1,x_2)$, then 
	
	$$Var(X_1 + X_2) = Var(X_1) + Var(X_2) + 2Cov(X_1,X_2)$$
\end{cor*}


\newpage

\subsection*{4. Conditional Expectation}

\begin{defn}
	If $X$ and $Y$ are jointly discrete random variables, then the conditional probability mass function of $X$, given that $Y = y$, is defined for all $y$ such that $P\{Y = y \} > 0$, by
	
	$$p_{X|Y}(x|y) = P\{X = x ~|~ Y = y\} = \dfrac{p(x,y)}{p_Y(y)}$$
	
	It is therefore natural to define, in this case, the conditional expectation of $X$ given that $Y = y$, for all value of $y$ such that $p_Y(y) > 0$, by
	
	\begin{eqnarray*}
		E[X|Y =  y] &=& \sum_x xP\{X = x~|~ Y = y\}\\
		&=& \sum_x xp_{X|Y}(x|y)
	\end{eqnarray*}
\end{defn}

\begin{prop*}
	$$E[X] = E[E[X|Y]]$$
\end{prop*}

\begin{proof}
	We only proof the equation when $X$ and $Y$ are discrete
	
	Observe that $E[X|Y]$ is a random variable of $Y$, so we can rewrite
	
	$$E[E[X|Y]] = \sum_yE[X|Y = y]P\{Y = y\}$$
	
	Now, the equation can be written as
	
	\begin{eqnarray*}
		\sum_yE[X|Y = y]P\{Y = y\} &=& \sum_y\sum_xxP\{X = x ~|~ Y = y\}P\{Y = y\}\\
		&=& \sum_y\sum_xx\dfrac{P\{X = x,Y = y\}}{P\{Y = y\}}P\{Y = y\}\\
		&=& \sum_y\sum_xxP\{X = x,Y = y\}\\
		&=& \sum_x x \sum_y P\{X = x,Y = y\}\\
		&=& \sum_x xP\{ X = x\} = E[X]
	\end{eqnarray*}
\end{proof}

\begin{defn}[Conditional Variance]

$$Var(X~|~Y) = E[(X - E[X|Y])^2|Y]$$
	
\end{defn}

\begin{prop*}
	$$Var(X) = E[Var(X|Y)] + Var(E[X|Y])$$
\end{prop*}

\begin{proof}$ $
	\begin{enumerate}
		\item Observe $E[Var(X|Y)]$ first, rewrite 
		
		$$Var(X|Y) = E[X^2 | Y] - (E[X|Y])^2$$
		
		so 
		
		\begin{eqnarray*}
			E[Var(X|Y)] &=& E[E[X^2|Y]] - E[(E[X|Y])^2] \\
			&=& E[X^2] - E[(E[X|Y])^2] \quad(\star_1)
		\end{eqnarray*}
		\item Observe $Var(E[X|Y])$
		
		\begin{eqnarray*}
			Var(E[X|Y]) &=& E[(E(X|Y))^2] - (E[E[X|Y]])^2\\
			&=& E[(E[X|Y])^2] - (E[X])^2 \quad(\star_2)
		\end{eqnarray*}
		
	\end{enumerate}
	
	Combine $(\star_1)$ and $(\star_2)$, we get
	
	\begin{eqnarray*}
		E[Var(X|Y)] + Var(E[X|Y]) &=& E[X^2] - (E[X])^2\\
		&=& Var(X)
	\end{eqnarray*}
\end{proof}

\begin{thm*}
	If $X$ and $Y$ are jointly distributed random variables, and $g(x)$ is a function, then
	
	$$E[g(X)Y~|~x] = g(x)E(Y~|~x)$$
\end{thm*}



\newpage

\subsection*{5. Moment Generating Function}

\begin{defn}
	The moment generating function $M(t)$ of the random variable $X$ is defined for all real values of $t$ by
	
	\begin{eqnarray*}
		M(t) &=& E[e^{eX}] \\
		&=& \begin{cases}
			\sum_x e^{tx}p(x) & $if $ X $ is discrete with mass function $ p(x)\\
			\int^{\infty}_{-\infty}e^{tx}f(x)dx & $if $ X $ is continuous with density$ f(x)
		\end{cases}
 	\end{eqnarray*}
 	
 	We call $M(t)$ the moment generating function because all of the moments of $X$ can be obtained by successively differentiating $M(t)$ and then evaluating the result at $t = 0$. For example.
 	
 	\begin{eqnarray*}
 		M'(t) &=& \dfrac{d}{dt}E[e^{tX}] \\
 		&=& E\left[\dfrac{d}{dt}(e^{tX})\right]\\
 		&=& E[Xe^{tX}]
 	\end{eqnarray*}
 	
 	where we have assumed that the interchange of the differentiation and expectation operator is legitimate. That is, we have assumed that
 	
 	$$\dfrac{d}{dt}\left[ \sum_xe^{tx}p(x) \right] = \sum_x \dfrac{d}{dt}[e^{tx}p(x)]$$
 	
 	in the discrete case and
 	
 	$$\dfrac{d}{dt}\left[ \int e^{tx}f(x)dx \right] = \int \dfrac{d}{dt}[e^{tx}f(x)]dx$$
 	
 	in the continuous case. This assumption can almost always be justified. Hence, evaluated at $t = 0$, we obtain
 	
 	$$M'(0) = E[X]$$
 	
 	Similarly,
 	
 	\begin{eqnarray*}
 		M''(t) &=& \dfrac{d}{dt}M'(t) \\
 		&=& \dfrac{d}{dt}E[Xe^{tX}] \\
 		&=& E\left[ \dfrac{d}{dt}(Xe^{tX}) \right] \\
 		&=& E[X^2 e^{tX}]
 	\end{eqnarray*}
 	
 	Thus,
 	
 	$$M''(x) = E[X^2]$$
 	
 	In general, the $n$th derivative of $M(t)$ is given by
 	
 	$$M^n(t) = E[X^ne^{tX}] \quad n \geq 1$$
 	
 	implying that
 	
 	$$M^n(0) = E[X^n] \quad n \geq 1$$
\end{defn}

\begin{defn}
	For any $n$ random variables $X_1,\cdots,X_n$ the joint moment generating function, $M(t_1,\cdots,t_n)$, is defined, for all real values of $t_1,\cdots,t_n$, by
	
	$$M(t_1,\cdots,t_n) = E[e^{t_1X_1 + \cdots + t_n X_n}]$$
	
	The individual moment generating functions can be obtained from $M(t_1,\cdots,t_n)$ by letting all but one of the $t_j$'s be $0$. That is,
	
	$$M_{X_i}(t) = E[e^{tX_i}] = M(0,\cdots,0,t,0,\cdots,0)$$
	
	where the $t$ is in the $i$th place.
\end{defn}

\begin{prop*}
	the joint moment generating function $M(t_1,\cdots,t_n)$ uniquely determines the joint distribution of $X_1,\cdots,X_n$. and the $n$ random variables $X_1,\cdots, X_n$ are independent if and only if 
	
	$$M(t_1,\cdots,t_n) = M_{X_1}(t_1)\cdots M_{X_n}(t_n)$$
\end{prop*}

\newpage

\subsection*{6*. General Definition of Expectation}

There also exist random variables that are neither discrete nor continuous, and they too, may possess an expectation.

\textbf{Example.} $X$ be a Bernoulli random variable with parameter $p = \frac{1}{2}$, and let $Y$ be a uniformly distributed random variable over the interval $[0,1]$. and suppose that $X$ and $Y$ are independent, and define the new random variable $W$ by

$$W = \begin{cases}
	X & $if $ X = 1\\
	Y & $if $ X \neq 1
\end{cases}$$

Clearly, $W$ is neither a discrete (since its set of possible values, [0,1], is uncountable) nor a continuous (since $P\{W = 1\} = \frac{1}{2}$)

In order to define the expectation of an arbitrary random variable, we require the notion of a Stieltjes integral. For instance, for any function $g$, $\int^b_a g(x)dx$ is defined by

$$\int^b_a g(x)dx = \lim \sum^n_{i = 1}g(x_i)(x_i - x_{i-1})$$

where the limit is taken over all $a = x_0 < x_1 < x_2 < \cdots < x_n = b$ as $n \rightarrow \infty$ and where $\max_{i = 1,\cdots,n}(x_i - x_{i - 1} \rightarrow 0)$.

For any distribution function $F$, we define the Stieltjes integral of the nonnegative function $g$ over the interval $[a,b]$ by

$$\int^b_a g(x)dF(x) = \lim \sum^n_{i = 1}g(x_i)[F(x_i) - F(x_{i - 1})]$$



\end{document}