\documentclass[12pt,reqno]{amsart}
%\usepackage[margin=1in]{geometry}
\usepackage{tcolorbox}
\usepackage{amssymb}
\usepackage{amsthm}
\usepackage{amsmath}
\usepackage{amssymb}
\usepackage{mathrsfs}
\usepackage{centernot}
\usepackage{lastpage}
\usepackage{fancyhdr}
\usepackage{accents}
\usepackage{tasks}
\usepackage{graphicx}
\usepackage{natbib}
\usepackage{tabularx}
\usepackage{multirow}
\usepackage{booktabs}
\usepackage{hyperref}
\usepackage{bm}
\usepackage{float}
\theoremstyle{plain}
\usepackage{multicol}
\usepackage{enumitem,kantlipsum}

% 加上浮水印
%\usepackage{wallpaper}
%\CenterWallPaper{.180}{../qsnake-logo.jpg}


\linespread{1.2}
\parindent = 0pt
\pagestyle{fancy}
\setlength{\parindent}{0pt}
%\everymath{\displaystyle}
%new area
%\usepackage[utf8]{inputenc}
%\usepackage{CJKutf8}
%\xeCJKsetup{AutoFakeBold=true, AutoFakeSlant=true}

% 設定頭部
\fancyhead[L]{Midterm} % 左邊頭部清空
\fancyhead[C]{} % 中間頭部清空
\fancyhead[R]{} % 右邊頭部顯示頁碼

% Adjust the footer as desired:
\fancyfoot[L]{} % Left footer: Empty.
\fancyfoot[C]{\thepage} % Center footer: Empty.
\fancyfoot[R]{} % Right footer: Empty.



% we will modify sections, subsections and sub subsections
\RequirePackage{titlesec}
% Modification of section 
\titleformat{\section}[block]{\normalsize\bfseries\filcenter}{\thesection.}{.3cm}{} 


% modification of subsection and sub sub section
\titleformat{\subsection}[runin]{\bfseries}{ \thesubsection.}
{1mm}{}[.\quad]
\titleformat{\subsubsection}[runin]{\bfseries\itshape}{ \thesubsubsection.}
{1mm}{}[.\quad]

\newenvironment{solution}
  {\renewcommand\qedsymbol{$\blacksquare$}
  \begin{proof}[Solution]}
  {\end{proof}}
\renewcommand\qedsymbol{$\blacksquare$}

\newcommand{\ubar}[1]{\underaccent{\bar}{#1}}

%%%%%%%%%%%%%%%%%%%%%%%%%%%%%% Textclass specific LaTeX commands.
%\theoremstyle{plain}
%\newtheorem{thm}{\protect\theoremname}[section]
\newtheorem{thm}{\textbf{Theroem}}[section]
\newtheorem{cor}[thm]{Corollary}
\newtheorem{lmma}[thm]{Lemma}
\newtheorem*{defn}{\underline{Definition}}
\newtheorem*{prop*}{Proposition}
\newtheorem*{ex*}{Example}
\newtheorem*{sol*}{Solution}
\newtheorem*{cor*}{Corollary}
\newtheorem*{thm*}{Theorem}
\newtheorem*{lmma*}{Lemma}
\newtheorem*{rmk*}{Remark}
\newtheorem*{pf*}{\underline{\textbf{Proof\ }}}

%%%%%%%%%%%%%%%%%%%%%%%%%%%%%% User specified LaTeX commands.
\renewcommand{\P}{\mathscr{P}}
\newcommand{\B}{\mathscr{B}}
\newcommand{\A}{\mathscr{A}}
\newcommand{\C}{\mathbb{C}}
\newcommand{\CC}{\mathscr{C}}
\newcommand{\R}{\mathbb{R}}
\newcommand{\Q}{\mathbb{Q}}
\newcommand{\Z}{\mathbb{Z}}
\newcommand{\N}{\mathbb{N}}
\newcommand{\X}{\mathcal{X}}
\newcommand{\T}{\mathscr{T}}
\newcommand{\arbuni}{\bigcup_{\alpha\in I}}
\newcommand{\finint}{\bigcap_{i=1}^n}
\newcommand{\Ua}{{\textsc{U}_\alpha}}
\newcommand{\Ui}{\textsc{U}_i}
\newcommand{\pair}[2]{\left( \,#1\,,\,#2\,\right) }
\newcommand{\dint}[2]{\int_{#1}^{#2}}
\newcommand{\sett}[1]{\left\{ \,#1 \,\right\}}
\newcommand{\linearcombination}[2]{#1_1#2_1+\cdots+#1_n#2_n}
\newcommand{\slinearcombination}[1]{#1_1+\cdots+#1_n}
\newcommand{\spann}[1]{\text{span($#1$)}}
\newcommand{\sub}[1]{\text{sup}}
\newcommand{\inn}[1]{\left< #1 \right>}
\newcommand{\kernal}[1]{Ker(#1)}
\newcommand{\image}[1]{Im(#1)}
\newcommand{\norm}[1]{\parallel #1 \parallel}
\newcommand{\dia}[0]{\text{dia}}
\newcommand{\marking}[1]{\text{\color{red} #1}}
%%%%%%%%%%%%%%%%%%%%%%%%%%%%%%

\begin{document}

%\lhead{Linear Algebra} 
%\rhead{Sabrina Edition} 
\cfoot{\thepage} %\ of \pageref{LastPage}}

\section{Introduction to spectral theory}

\subsection{Main definitions}

\begin{defn}
	A scalar $\lambda$ is called an eigenvalue of an operator $A:V \rightarrow V$ if there exists a non-zero vector $v \in V$ such that 
	
	$$Av = \lambda v$$
	
	The vector $v$ is called the eigenvector of $A$
\end{defn}

\begin{thm*}[From hamberger Thm 5.2]
	Let $A \in M_{n \times n}(F).$ Then a scalar $\lambda$ is an eigenvalue of $A$ if and only if $\det(A - \lambda I_{n}) = 0$.
\end{thm*}

\begin{proof}
	A scalar $\lambda$ is an eigenvalue of $A$ if and only if there exists a nonzero vector $v \in F^n$ such that $A v = \lambda v$, that is, $(A - \lambda I_n)(v) = 0$. By Theorem 2.5, this is true \textbf{if and only if} $A - \lambda I_n$ is not invertible. However, this result is equivalent to the statement that $\det(A - \lambda I_n)=0$
\end{proof}

\begin{defn}
	Let $A \in M_{n\times n}(\mathrm{F}).$ The polynomial $f(t) = \det (A - tI_n)$ is claaed the characteristic polynomial of $A$
\end{defn}

\begin{thm*}[From hamberger Thm 5.4]
	Let $T$ be a linear operator on a vector space $V$, and let $\lambda$ be an eigenvalue of $T$. A vector $v \in V$ is an eigenvector of $T$ corresponding to $\lambda$ if and only if $v \neq 0$ and $v \in N(T - \lambda I)$.
\end{thm*}

\begin{defn}
	The nullspace $N(A - \lambda I)$, i.e. the set of all eigenvectors and $0$ vector, is called the eigenspace. The set of all eigenvalues of an operator $A$ is called spectrum of $A$, and is usually denoted $\sigma (A)$.
\end{defn}

\begin{rmk*} $ $

	If the matrix $A$ is ugly, what should we do?
	
	we can use the similar ≤matrices
	
	$A$ and $B$ are called similar if there exists an invertible matrix $S$ such that 
	
	$$ A = SBS^{-1}$$
	
	The determinants of similar matrix is same
	
	$$\det(A) = \det(SBS^{-1}) = \det(S)\det(B)\det(S^{-1}) = \det(B)$$
	
	We can find $A-\lambda I$ and $B -\lambda I$ is similar
	
	$$A - \lambda I = SBS^{-1} - \lambda SIS^{-1} = S(BS^{-1} - \lambda I S^{-1}) = S(B-\lambda I)S^{-1}$$
	
	It same in transform
	
	If $T:V\rightarrow V$ is a linear transform, $\alpha,\beta$ are two bases in $V$, then
	
	$$[T]^{\alpha}_{\alpha} = [I]^{\alpha}_{\beta}[T]^{\beta}_{\beta}[I]^{\beta}_{\alpha} $$
	
	
\end{rmk*}

\begin{defn}[algebraic mutiplicity]
	The largest positive integer $k$ such that $(x - \lambda)^k$ divides $p(x)$ is called the multiplicity of the root $\lambda$.
	
	If $\lambda$ is an eigenvalue of an operator (matrix) $A$, then it is a root of the characteristic polynomial $p(z) = \det(A - zI)$. The multiplicity of this root is called the (algebraic) multiplicity of the eigenvalue $\lambda$.
\end{defn}

\begin{defn}[geometric multiplicity]
	The dimension of the eigen space $N(A - \lambda I)$ is called geometric multiplicity of the eigenvalue $\lambda$.
\end{defn}

\subsection{Diagonalization}

\begin{defn}
	A linear operator $T$ on a finite-dimensional vector space $V$ is called diagonalizable if there is an ordered basis $\beta$ for $V$ such that $[T]_{\beta}$ is a diagonal matrix. A square matrix $A$ is called diagonalizable if $L_A$ is diagonalizable.
\end{defn}

\begin{thm*}
	A matrix $A$ admits a representation $A = SDS^{-1}$, where $D$ is a diagonal matrix and $S$ is an invertible one \textbf{if and only if} there exists a basis in $F^n$ of eigenvectors of $A$.
\end{thm*}

\begin{proof}
	Let $D = \text{diag}\sett{\lambda_1,\lambda_2,\cdots,\lambda_n}$, and let $b_1,\cdots,b_n$ be the columns of $S$ (note that since $S$ is invertible it's columns form a basis in $\mathrm{F^n}$). Then the identity $A = SDS^{-1}$ means that $D = [A]$
\end{proof}









\end{document}