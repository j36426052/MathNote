\documentclass[12pt]{article}
\usepackage[margin=1in]{geometry} 
\usepackage{amsmath}
\usepackage{tcolorbox}
\usepackage{amssymb}
\usepackage{amsthm}
\usepackage{mathrsfs}
\usepackage{lastpage}
\usepackage{fancyhdr}
\usepackage{accents}
\usepackage{tasks}
\linespread{1.2}
\parindent = 0pt
%\pagestyle{fancy}
\setlength{\headheight}{40pt}
\everymath{\displaystyle}

% we will modify sections, subsections and sub subsections
\RequirePackage{titlesec}
% Modification of section 
\titleformat{\section}[block]{\normalsize\bfseries\filcenter}{\thesection.}{.3cm}{} 

% modification of subsection and sub sub section
\titleformat{\subsection}[runin]{\bfseries}{ \thesubsection.}
{1mm}{}[.\quad]
\titleformat{\subsubsection}[runin]{\bfseries\itshape}{ \thesubsubsection.}
{1mm}{}[.\quad]

\newenvironment{solution}
  {\renewcommand\qedsymbol{$\blacksquare$}
  \begin{proof}[Solution]}
  {\end{proof}}
\renewcommand\qedsymbol{$\blacksquare$}

\newcommand{\ubar}[1]{\underaccent{\bar}{#1}}

%%%%%%%%%%%%%%%%%%%%%%%%%%%%%% Textclass specific LaTeX commands.
%\theoremstyle{plain}
\newtheorem{thm}{\protect\theoremname}[section]
\newtheorem{cor}[thm]{Corollary}
\newtheorem{lmma}[thm]{Lemma}
\newtheorem*{defn}{\underline{Definition}}
\newtheorem*{ex*}{Example}
\newtheorem*{sol*}{Solution}
\newtheorem*{cor*}{Corollary}
\newtheorem*{thm*}{Theorem}
\newtheorem*{lmma*}{Lemma}
\newtheorem*{rmk*}{Remark}
\newtheorem*{pf*}{\underline{\textbf{Proof\ }}}

%%%%%%%%%%%%%%%%%%%%%%%%%%%%%% User specified LaTeX commands.
\renewcommand{\P}{\mathscr{P}}
\newcommand{\B}{\mathscr{B}}
\newcommand{\A}{\mathscr{A}}
\newcommand{\C}{\mathbb{C}}
\newcommand{\CC}{\mathscr{C}}
\newcommand{\R}{\mathbb{R}}
\newcommand{\Q}{\mathbb{Q}}
\newcommand{\Z}{\mathbb{Z}}
\newcommand{\N}{\mathbb{N}}
\newcommand{\X}{\mathcal{X}}
\newcommand{\T}{\mathscr{T}}
\newcommand{\arbuni}{\bigcup_{\alpha\in I}}
\newcommand{\finint}{\bigcap_{i=1}^n}
\newcommand{\Ua}{{\textsc{U}_\alpha}}
\newcommand{\Ui}{\textsc{U}_i}
\newcommand{\pair}[2]{\left( \,#1\,,\,#2\,\right) }
\newcommand{\dint}[2]{\int_{#1}^{#2}}
\newcommand{\sett}[1]{\left\{ \,#1 \,\right\}}
\newcommand{\linearcombination}[2]{#1_1#2_1+\cdots+#1_n#2_n}
\newcommand{\slinearcombination}[1]{#1_1+\cdots+#1_n}
\newcommand{\spann}[1]{\text{span($#1$)}}
\newcommand{\sub}[1]{\text{sup}}
\newcommand{\inn}[1]{\left< #1 \right>}
%%%%%%%%%%%%%%%%%%%%%%%%%%%%%%

\begin{document}

%\lhead{Linear Algebra} 
%\rhead{Sabrina Edition} 
\cfoot{\thepage} %\ of \pageref{LastPage}}


\textbf{Classification}
\begin{tcolorbox}
	\textbf{order}

	$$\dfrac{dy}{dx} = y^2$$
	
	which is $1$st order,  $x$ independent variable, $y$ dependent variable
	
	$$\dfrac{d^4y}{dt^4} + 5\dfrac{d^2x}{dt^2} + 3x \sin{t}$$
	
	which is $4$th order
	
	because above equation only have $1$ independent variable, they are ordinary differential equations(ODEs).
	
	$$\dfrac{\partial v}{\partial s} + \dfrac{\partial v}{\partial t} = r$$
	
	which is $1$st order
	
	$$\dfrac{\partial^2 u}{\partial x^2} + \dfrac{\partial^2 u}{\partial y^2} + \dfrac{\partial^2 u}{\partial z^2} = 0$$
	
	which is $2$nd order
	
	above equation have more than one independent variable, they are partial differential equations(PDEs).
\end{tcolorbox}

$n$th-order ODE: $F(x,y,y',\cdots,y^{(n)}) = 0$

In certain condition on $F$, it can be written as

$$y^{(n)} = f(x,y,y',\cdots,y^{(n)}) = 0 (\star)$$

Example $(y')^2 + y' + xy = 0$

$$y' = \dfrac{-1 \pm \sqrt{1 - 4xy}}{2}$$

\begin{defn}
	a function $\phi(x)$ is called a solution of ($\star$) on $a<x<b$ if $\phi^{(n)}$ exists on $a<x<b$ and 
	
	$$\phi^{(n)}(x) = f(x,\phi (x),\phi'(x),\cdots,\phi^{(n-1)}(x) )~\forall a<x<b$$
\end{defn}

\textbf{Example.} Verify that $y = e^{2x}$ is a solution of $y'' + y' - 6y = 0$

\begin{proof}
	$y''+y' - 6y = 4e^{3x} + 2e^{2x} - 6e^{3x} = 0 ~\forall -\infty < x < \infty$
	
	$\therefore y = e^{2x}$ is a solution on $-\infty < x < \infty$
\end{proof}

\newpage

\textbf{Note.} $y' = \dfrac{xy}{x + y + 1}$ is derivative form $\Leftrightarrow dy = \dfrac{xy}{x+y+1}dx$ or $xydx - (x + y + 1)dy = 0$ is differential form.

\begin{defn}
	An ODE of order $n$ is called linear if it may be written in the form
	
	$$b_0(x)y^{(n)} + b_1(x)y^{(n-1)} + \cdots + b_{n-1}(x)y' + b_n(x)y = R(x)$$
	
	where $b_0 \neq 0$, An ODE that is not linear is called nonlinear ODE.
\end{defn}

\textbf{Example}
\begin{tcolorbox}
linear 
	\begin{eqnarray*}
		y'(x) + 5y'(x) + 6y(x) &=& 0\\
		y'''(x) + x^2 y''(x) + x^3y'(x) &=& xe^x
	\end{eqnarray*}
non linear
	\begin{eqnarray*}
		y''(x) + 5y'(x) + 6y^2(x) &=& 0\\
		y''(x) + 5(y'(x))^3 + 6y(x) &=& 0
	\end{eqnarray*}	
\end{tcolorbox}

Initial-Value Problem(IVP): same point (and $1$st order)
\begin{tcolorbox}
	$\begin{cases}
		\dfrac{d^2 y}{dx^2} + y = 0\\ y(1) = 3 \\ y(1) = 2
	\end{cases}$
\end{tcolorbox}

Boundary-Value Problem(BVP): two or more different points

\begin{tcolorbox}
	$\begin{cases}
		\dfrac{d^2 y}{dx^2} + y = 0\\ y(1) = 3\\ y(2) = 2
	\end{cases}$
\end{tcolorbox}

\newpage

\begin{thm*}[Existence and uniqueness]
	Consider
	
	$$\begin{cases}y' = f(x,y) \\ y(x_0) = y_0\end{cases}$$
	
	where $x_0,y_0 \in \R$ are given
	
	Let $T = \{(x,y)~|~ |x-x_0| \leq a,|y-y_0| \leq b\}$, where $a,b > 0$. Suppose that $f$ and $fy$ are continuous in $T$. Then (IVP) has a unique solution defined on $[x_0 - h,x_0 + h]$ for some $h > 0$
\end{thm*}

\textbf{$\S$ Separable equation} $A(x)dx = B(y)dy$

$ $

\textbf{Example.}

\begin{enumerate}[wide]
	\item $\dfrac{dy}{dx} = \dfrac{2y}{x}$
	\begin{solution}
		\begin{eqnarray*}
			\dfrac{1}{y}dy &=& \dfrac{2}{x}dx\\
			\implies \int \dfrac{1}{y}dy &=& \int \dfrac{2}{x}dx\\
			\implies \ln |y| &=& 2\ln|x| + C
		\end{eqnarray*}
	\end{solution}
	
	\item $\begin{cases}
		(1+y^2)dx + (1+x^2)dy = 0\\ y(0) = -1
	\end{cases}$

	\begin{solution}
		\begin{eqnarray*}
			(1+y^2)dx &=& -(1+x^2)dy\\
			\implies \dfrac{dx}{-(1+x^2)} &=& \dfrac{dy}{(1 + y^2)}\\
			\implies \int \dfrac{1}{1+x^2}dx &=& - \int \dfrac{1}{1+y^2}dy
		\end{eqnarray*}
		
		you can let $x = \tan\theta \implies dx = \sec^2\theta d\theta $
		
		$\therefore \int\dfrac{1}{1+x^2}dx = \int \cos^2\theta \sec^2\theta d\theta = \theta + C = \tan^{-1}x + C$
		
		$\implies \tan^{-1}x = -\tan^{-1}y + C$
		
		$y(0) = -1 \implies 0 = \dfrac{\pi}{4} + C \implies C = \dfrac{\pi}{-4},~\therefore \tan^{-1} x = - \tan^{-1}y - \dfrac{\pi}{4}$
	\end{solution}
	
	\newpage
	
	\item $\begin{cases}
		2x(y+1)dx - ydy = 0\\ y(0) = -2
	\end{cases}$
	\begin{solution}
		$\int 2xdx = \int \dfrac{y}{y+1}dy \implies x^2 = y-\ln |y+1| + C$
		
		$y(0) = -2 \implies 0 = -2 + c \implies c = 2 ~\therefore x^2 + y - \ln |y+1| + 2$
	\end{solution}
\end{enumerate}

\textbf{$\S$ Homogeneous equations}

\begin{defn}
	a function $f(x,y)$ is said to be homogeneous of degree $k$ in $x$ and $y$ if and only if
	
	$$f(\lambda x,\lambda y) = \lambda^k f(x,y)$$
\end{defn}

\textbf{Example.} $f(x,y) = x^2 + y^2$

\begin{eqnarray*}
	f(\lambda x , \lambda y) &=& (\lambda x)^2 + (\lambda y)^2\\
	&=& \lambda^2 (x^2 + y^2)\\
	&=& \lambda^2 f(x,y)
\end{eqnarray*}

$\therefore f(x,y)$ is homogeneous  
, $k = 2$

\begin{thm*}
	If $M(x,y)$ and $N(x,y)$ are both homogeneous and of the same degree, then $\dfrac{M(x,y)}{N(x,y)}$ is homogeneous of degree zero.
\end{thm*}
\begin{proof}
	Set $f(x,y) = \dfrac{M(x,y)}{N(x,y)}$. By definition, we assume $M$ and $N$ are homogeneous of degree $k$, so
	
	$$M(\lambda x , \lambda y) = \lambda^k M(x,y) \text{ and } N(\lambda x,\lambda y) = \lambda^kN(x,y)$$
	
	$\therefore f(\lambda x,\lambda y) = \dfrac{M(\lambda x ,\lambda y )}{N(\lambda x, \lambda y)} = \dfrac{\lambda^k}{\lambda^k} \cdot \dfrac{M(x,y)}{N(x,y)} = \lambda^0 \dfrac{M(x,y)}{N(x,y)}$
\end{proof}

\begin{thm*}
	If $f(x,y)$ is homogeneous of degree zero in $x$ and $y$, then $f(x,y) = g(\frac{y}{x})$ for some function $y$.
\end{thm*}

\begin{proof}
	By assumption,
	
	$$f(\lambda x, \lambda y) = \lambda^0f(x,y) = f(x,y)$$
	
	Take $\lambda = \frac{1}{x}$. Then $f(x,y) = f(1,\frac{y}{x}) = g(\frac{y}{x})$, where $g(v) = f(1,v).$
\end{proof}

\begin{cor*}
	If $M(x,y)$ and $N(x,y)$ are both homogeneous and of the same degree, then $\dfrac{M(x,y)}{N(x,y)} = g(\frac{y}{x})$ for some function $g$.
\end{cor*}

\begin{defn}
	$M(x,y) + N(x,y)dy = 0$ is said to be homogeneous if it can be written as the form $\dfrac{dy}{dx} = g(\frac{y}{x})$ for some function $g$
\end{defn}

\textbf{Example.} $(x^2 - 3y^2)dx + 2xydy = 0 (\star)$

$$\dfrac{dy}{dx} = -\dfrac{x^2 - 3y^2}{2xy} = - \dfrac{1 - 3(\frac{y}{x})^2}{2 \cdot \frac{y}{x}} = g(\dfrac{y}{x}) \text{ where } g(v) = \dfrac{1 - 3v^2}{-2v}$$

\begin{rmk*}
	If $M(x,y)$ and $N(x,y)$ are homogeneous of the same degree, then $M(x,y)dx + N(x,y)dy = 0$ is homogeneous.
\end{rmk*}

\begin{proof}
	By assumption and corollary, $\dfrac{M(x,y)}{N(x,y)} = g(\dfrac{y}{x})$ for some function $g$. $\therefore Mdx + Ndy = 0 \implies \dfrac{dy}{dx} = -\dfrac{M(x,y)}{N(x,y)} = -g(\dfrac{y}{x})$
	
	$\therefore Mdx + Ndy$ is homogeneous.
\end{proof}

\textbf{How to solve homogeneous equation}

Suppose$M(x,y)dx + N(x,y)dy = 0 (\star)$ is homogeneous.

Let $y = vx \implies \dfrac{dy}{dx} = \dfrac{dv}{dx}x + v (1)$

$\because (\star)$ is homogeneous $\therefore$ By definition, $(\star) \Leftrightarrow \dfrac{dy}{dx} = g(\dfrac{y}{x})(2)$

where $g$ is a function.

\begin{eqnarray*}
	\text{put (1) to (2) } &\implies& \dfrac{dv}{dx}x + v = g(v)\\
	&\implies& \dfrac{dv}{dx}x = g(v) - v\\
	&\implies& \dfrac{1}{g(v) - v}dv = \dfrac{1}{x}dx, \text{ which is separable}
\end{eqnarray*}

$\therefore$ The solution is $\int\dfrac{1}{g(v) - b}dv = \int \dfrac{1}{x}dx$

\textbf{Example.} $(x^2 - xy + y^2)dx - xydy = 0$---(1)

\begin{solution}
	\begin{eqnarray*}
		M(\lambda x, \lambda y) &=& (\lambda x)^2 - (\lambda x)(\lambda y) + (\lambda y)^2\\
		&=& \lambda^2(x^2 - xy + y^2)\\
		&=& \lambda^2 M(x,y)
	\end{eqnarray*}
	
	\begin{eqnarray*}
		N(\lambda x,\lambda y) &=& -(\lambda x)(\lambda y)\\
		&=& - \lambda^2 xy\\
		&=& \lambda^2 N(x,y) 
	\end{eqnarray*}
\end{solution}

so  (1) is homogeneous.

Let $y = vx \implies \dfrac{dy}{dx} = \dfrac{dv}{dx}x + v$

\begin{eqnarray*}
	\text{(1)} &\implies& \dfrac{dy}{dx} = -\dfrac{x^2 - xy + y^2}{xy} = \dfrac{1 - \frac{y}{x} + (\frac{y}{x})^2}{\frac{y}{x}}\\
	&\implies& \dfrac{dv}{dx}x + v = \dfrac{1 - v + v^2}{v}\\
	&\implies& \dfrac{dv}{dx}x = \dfrac{1 - v + v^2}{v} - v = \dfrac{1 - v}{v}\\
	&\implies&\int \dfrac{v}{1-v}dv = \int \dfrac{1}{x}\\
	&\implies& \dfrac{v-1+1}{1-v} = -1 - \dfrac{1}{v-1}\\
	&\implies& -v-\ln|v - 1| = \ln |x| + c
\end{eqnarray*}


\textbf{Example.} $xydx + (x^2 + y^2)dy = 0$

\begin{solution}
	$\dfrac{dy}{dx} = \dfrac{xy}{-(x^2 + y^2)} = -\dfrac{\frac{y}{x}}{1 + (\frac{y}{x})^2} = g(\frac{y}{x}) $ ---(1), 
	
	where $g(v) = \dfrac{-v}{1+v^2}$, so the equation is homogeneous.
	
	Let $y = vx \implies \dfrac{dy}{dx} = \dfrac{dv}{dx}x + v$
	
	\begin{eqnarray*}
		\text{(1) } &\implies& \dfrac{dv}{dx}x + v = -\dfrac{v}{1+v^2}\\
		&\implies& \dfrac{dv}{dx} \cdot x = -\dfrac{v}{1+v^2} - v = -\dfrac{2v + v^3}{1 + v}\\
		&\implies& \int \dfrac{1 + v^2}{2v + v^3}dv = -\int \dfrac{1}{x}dx\\
		&\implies& \int (\dfrac{0.5}{v} + \dfrac{0.5v + 0}{2 + v^2}) = - \int \dfrac{1}{x}dx\\
		&\therefore & 0.5 \int \dfrac{1}{v} + \dfrac{v}{2+v^2}dv = - \ln |x| + c\\
		&\implies& 0.5 \ln|v| + 0.25\ln |2 + v^2| + -\ln|x| + c\\
		&\implies& 0.5 \ln |\frac{y}{x}| + 0.25 \ln |2 + \dfrac{y^2}{x^2}| = -\ln |x| + c
	\end{eqnarray*}
\end{solution}

\textbf{$\S$ Exact equation}

\begin{defn}
	$M(x,y)dx + N(x,y)dy=0$ is called an exact equation if there exists a function $F(x,y)$ such that $F_x = M$ and $F_y = N$
\end{defn}

\textbf{Example.}

$y^2 dx + 2xy dy = 0$ Set $F(x,y) = xy^2 \implies Fx = y^2$ and $Fy = 2xy$

$\therefore$ exact equation.

\textbf{How to solve homogeneous equation}

Suppose $M(x,y)dx + N(x,y) = 0$ ---($\star$) is exact

$\implies \exists$ a function $F(x,y)$ such that $Fx = M$ and $Fy = N$

\begin{eqnarray*}
	(\star) &\implies& Fxdx + Fydy = 0\\
	&\implies& dF = 0\\
	&\implies& F = C, \text{ where $C$ is an arbitrary constant}
\end{eqnarray*}

\begin{thm*}
	Suppose $M,N,My,Nx$ are continuous. Then $Mdx + Ndy = 0$ is an exact equation $\Leftrightarrow ~ My = Nx$
\end{thm*}

\begin{proof}
	($\Leftarrow$) Suppose $My = Nx$. Claim $(\star)$ is exact
	$$\begin{cases}
		Fx = M \text{---(1)}\\
		Fy = N \text{---(2)}
	\end{cases}$$
	
	(1) $\Leftrightarrow F(x,y) = \int M(x,y)\partial x + \phi(y)$ for some function $\phi$
	
	\begin{eqnarray*}
		(1)(2) &\Leftrightarrow& \dfrac{\partial}{\partial y} \int M(x,y)\partial x + \phi'(y) = N(x,y)\\
	&\Leftrightarrow& \phi'(y) = N(x,y) - \dfrac{\partial}{\partial y}\int M(x,y)\partial x = N(x,y) - \int My(x,y)\partial x
	\end{eqnarray*}
	
	We compute
	
	$$\dfrac{\partial}{\partial}\left[N(x,y)- \int My(x,y)\partial x\right] = Nx(x,y) - My(x,y) = 0$$
	
	This implies $N(x,y) - \int My(x,y)\partial x$ is independent of $x$
	
	$\therefore \phi(y) = \int\left[N(x,y) - \int My(x,y)\partial x\right]dy$
	
	$\therefore F(x,y) = \int M(x,y)\partial x + \int \left[N(x,y) - \int My(x,y)\partial x \right]dy$ satisfy (1)(2) $\therefore Mdx + Ndy = 0$ is exact. 
	
\end{proof}

\textbf{Example.} $3x(xy - 2)dx + (x^3 + 2y)dy = 0$

\begin{solution}
	$\therefore My = 3x^2 = Nx ~\therefore$ exact
	
	Find a function $F$ s.t. $\begin{cases}
		Fx = M = 3x^2y - 6x \text{--- (1)}\\
		Fy = N = x^3 + 2y ~\text{   --- (2)}
	\end{cases}$ 
	
	\begin{eqnarray*}
		(1) \implies F &=& \int (2x^3 - xy^2 - 2y + 3) \partial x + \phi (y)\\
		&=& \dfrac{1}{3}x^4 - \dfrac{1}{2}x^2y^2 - 2xy + 3x + \phi(y)\\
		&\implies& -x^2y-2x=Fy = -x^2y - 3x + \phi'(y) \implies \phi'(y) = c\\
		&\implies& \phi (y) = 0 \implies F = x^4 - \dfrac{1}{2}x^2y^2 - 2xy + 3x
	\end{eqnarray*}
	
	$\therefore \dfrac{1}{2}x^4 - \dfrac{1}{2}x^2y^2 - 2xy + 3x = C$ is the general solution
\end{solution}

\textbf{$\S$ Linear equations}: $A(x)y' + B(x)y = c(x)$, where $A(x) \neq 0$


Suppose $c(x) = 0$
\begin{eqnarray*}
	&\implies& A(x)y' + B(x)y = 0\\
	&\implies& A(x)y' = -B(x)y \implies \dfrac{1}{y}dy = \dfrac{B(x)}{-A(x)}dx, \text{ which is separable}
\end{eqnarray*}

Suppose $B(x) = 0 \implies$ it's easy to solve

Suppose $c(x) \neq 0$ and $B(x) \neq 0$

$A(x)y' + B(x)y = c(x) \implies y' + p(x)y = Q(x)$, where $p(x) = \dfrac{B(x)}{A(x)}$ and $Q(x) = \dfrac{C(x)}{A(x)}$

$\Leftrightarrow dF = 0 \Leftrightarrow Fxdx + Fydy = 0$

$\Leftrightarrow (P(x)y - Q(x))dx + 1dy = 0$---(1) is not exact.

Let $v = v(x)$, (1) $\times v \implies v(P(x)y - Q(x))dx + v(x)dy$ --- (2)

\begin{eqnarray*}
	(2) \text{ is exact} &\Leftrightarrow& \dfrac{\partial}{\partial y} \left[ v(x)P(x)y - Q(x)\right] = v'(x)\\
	&\Leftrightarrow& v(x)P(x) = v'(x) \Leftrightarrow P(x)dx = \dfrac{1}{v}dv\\
	&\Leftrightarrow& \ln|v| = \int P(x)dx \Leftrightarrow |v| = e
\end{eqnarray*}

Let $v(x) = e^{\int p(x)dx}$, which is called the integrating factor of (1)

$A(x)y' + B(x)y = C(x) \implies y' + \dfrac{B(x)}{A(x)}y = \dfrac{C(x)}{A(x)}$, and let $\dfrac{B(x)}{A(x)} = p(x)$

$$\mu(x) = e^{\int p(x)dx}$$

$$e^{\int p(x)dx}y' + p(x)e^{\int p(x)dx}y = Q(x) e^{\int p(x) dx}$$

\textbf{Example.} $2(y - 4x^2)dx + xdy = 0$

\begin{solution}
	$x \dfrac{dy}{dx} + 2y = 8x^2 \implies \dfrac{dy}{dx} + \dfrac{2}{x}y = 8x$ -- (1)
	
	$$\mu(x) = e^{\int \dfrac{2}{x}dx} = e^{2\ln|x|} = e^{\ln x^2} = x^2$$
	
	\begin{eqnarray*}
		\text{(1)} \times \mu(x) &\implies& x^2 \dfrac{dy}{dx} + 2xy + 8x^3\\
		&\implies& [x^2y]' = 8x^3\\
		&\implies& x^2 y = \int 8x^3dx = 2x^4 + c
	\end{eqnarray*} 
\end{solution}

\textbf{Example} $ydx + (3x - xy + 2)dy = 0$

\begin{solution}
	$(3x - xy + 2)\dfrac{dy}{dx} + y = 0$ which is not linear
	
	$y \dfrac{dx}{dy} + (3 - y)x = -2$, which is linear
	
	$\implies \dfrac{dx}{dy} + \dfrac{3 - y}{y}x = -\dfrac{2}{y}$ -- (1)
	
	$\mu(y) = e^{\int \dfrac{3 - y}{y}dy} = e^{3\ln|y| - y} = |y|^3e^{-y} = \pm y^3 e^{-y} = y^3 e^{-y}$
	
	(1) $\times \mu (y) \implies y^3 e^{-y}\dfrac{dx}{dy} + y^3e^{-y}\left(\dfrac{3-y}{y}\right)x = -2y^2e^{-y}$
	
	$\implies \dfrac{d}{dy}\left[y^3e^{-y}x\right] = -2y^2 e^{-y}$
	
	$\implies y^3 e^{-y}x = -2\int y^2e^{-y}dx$
\end{solution}

\section*{5. Additional topics on equations of order one}

(A) Find an integrating factor

Consider

$$Mdx + Ndy = 0 \text{ -- (1)}$$

Suppose (1) is not exact, let $\mu = \mu (x,y)$

(1) $\times \mu \implies \mu M dx + \mu N dy = 0$ -- (2)

(2) is exact iff $\dfrac{\partial}{\partial y}(\mu , M) = \dfrac{\partial}{\partial x}(\mu N)$ By thm 2.3, iff $\mu_y M + \mu M_y= \mu_xN + \mu N_x$ -- (3)

Note that (3) is a $1$st linear PDE and connot be solved in general

Suppose $\mu = \mu(x):$

\begin{eqnarray*}
	(3) &\Leftrightarrow& \mu M_y =  \mu_x N + \mu N_x\\
	&\Leftrightarrow & N - \dfrac{du}{dx} = (My - Nx)\mu\\
	&\Leftrightarrow & \dfrac{1}{\mu}\dfrac{du}{dx} = \dfrac{My - Nx}{N}
\end{eqnarray*}

Suppose $\dfrac{My - Nx}{N}$ depends only on $x$. Then

$\int \dfrac{1}{\mu}du = \int \dfrac{My - Nx}{N}dx \implies \ln|\mu| \implies e^{\int \dfrac{My - Nx}{N}dx} \implies$

$\mu = \pm e^{\int \dfrac{My - Nx}{N}dx} = e^{\int \dfrac{My - Nx}{N}dx}$ Take positibity

$\therefore \mu(x) = e^{\dfrac{My - Nx}{N}dx}$ is an integrating factor of (1)

Suppose $u = \mu(y)$,

\begin{eqnarray*}
	\text{(2)} &\Leftrightarrow & u_yM + u M_y = u N_x\\
	&\Leftrightarrow & \dfrac{du}{dy}M = (Nx - My)u\\
	&\Leftrightarrow & \dfrac{1}{u}\dfrac{du}{dy} = \dfrac{Nx - My}{M}
\end{eqnarray*}

and $\dfrac{1}{u}\dfrac{du}{dy}$ depending only on $y$.

Suppose $\dfrac{Nx - My}{M}$ depends only on $y$, then

\begin{eqnarray*}
	\int \dfrac{1}{u}du = \int \dfrac{Nx - My}{M}dy\\
	\implies |u| = e^{\int \dfrac{Nx - My}{M}dy}\\
	\implies u = \pm e^{\int \dfrac{Nx - My}{M}dy}
\end{eqnarray*}

Take positivity of $u \implies u(y) = e^{\int \dfrac{Nx - My}{M} dy}$ is an integrating factor of (1)

Conclusion

\begin{enumerate}
	\item Suppose $\dfrac{My - Nx}{N}$ depends only on $x$, then $u(x) = e^{\int \dfrac{My - Nx}{N}dx}$ is an integrating factor of (1)
	\item Suppose $\dfrac{Nx - My}{M}$ depends only on $y$, then $u(y) = e^{\int \dfrac{Nx - My}{M} dy}$ is an integrating factor of (1)
\end{enumerate} 

\textbf{Example} $(4xy + 3y^2 - x)dx + x(x + 2y)dy = 0$

\begin{solution}
	$My = 4x + 6y,~Nx = 2x+3y$
	
	$\because My \neq Nx \therefore$ not exact
	
	$\dfrac{My - Nx}{N} = \dfrac{2x + 4y}{x(x + 2y)} = \dfrac{2}{x}$, which depends only on $x$
	
	$\mu (x) = e^{\int \dfrac{2}{x} dx} = e^{2\ln|x|} = x^2$ is an integrating factor
	
	$\mu \times (1) \implies (vx^3y + 3x^2y^2 - x^3)dx + (x^4 + 3x^3y)dy = 0$, which is exact
	
	$\begin{cases}
		Fx = 4x^3y + 3x^2y^2 - x^3\text{ -- (2)}\\
		Fy = x^4 + 2x^3y \text{ -- (3)}
	\end{cases}$
\end{solution}

(2) $\implies F = x^4y + x^3y^2 - \dfrac{1}{4}x^4 + \phi(y) \implies Fy = x^4 + 2x^3y + \phi'(y) \implies \phi(y) = 0 \implies \phi(y) = 0 \implies F = x^4y + x^3 y^2 - \dfrac{1}{4}x^4$

$\therefore x^4 y + x^2y^2 - \dfrac{1}{4}x^4 = C$ is the general solution.

\textbf{Example.} $y(x+y+1)dx + x(x + 3y + 2)dy = 0$

\begin{solution}
	$My = x + 2y + 1,~Nx = 2x + 3y + 2$
	
	$\because My \neq Nx \therefore$ not exact $\dfrac{My - Nx}{N} = \dfrac{-x-y-1}{x(x+3y+2)}$, which depends on both $x$ and $y$. $\dfrac{Nx - My}{M} = \dfrac{x + y + 1}{y(x+y+1)} = \dfrac{1}{-y}$, which depends only on $y$, $\therefore \mu(y) = e^{\int \dfrac{1}{y}dy} = e^{\ln|y|} = |y| = y$ taking positive
	
	(1) $\times \mu(y) \implies (xy^2 + y^3 + y^2)dx + (x^2y+3xy^2 + 2xy)dy = 0$, which is exact.
	
	$\begin{cases}
		Fx = xy^2 + y^3 + y^2 \\
		Fy = x^2y + 2xy^2 + 2xy \text{ --(3)}
	\end{cases}$
	
	$\implies F = \dfrac{1}{2}x^2y^2 + xy^3 + xy^2 + \phi(y) $
	
	$\implies Fy = x^2y + 3xy^2 + 2xy + \phi'(y)$ -- (2)
	
	(2)(3) $\implies \phi'(y) = 0 \implies \phi(y) = 0$
	
	$\therefore F = \dfrac{1}{2}x^2y^2 + xy^3 + xy^2$
	
	$\therefore \dfrac{1}{2}x^2y^2 + xy^3 + xy^2 = 0$ is the general solution
\end{solution}

(B) substitution

\textbf{Example.} $(x + 2y - 1)dx + 3(x + 2y)dy = 0$

\begin{solution}
	Let $v = x + 2y \implies dv = dx + 2dy$
	
	$\implies (v - 1)(dv - 2dy) + 3vdy = 0 \implies (v-1)dv - 2(v-1)dy + 3vdy$
	
	$\implies (v-1)dv + (v+2)dy = 0 \implies (v-1)dv = -(v+2)dy$
	
	$\implies \int \dfrac{v - 1}{v+2}dv = - \int 1 dy \implies v - 3\ln |v+2| = -y + c$
	
	$\implies x+2y - 3\ln |x+2y+2| = -y+c$
\end{solution}

\textbf{Example.} $(1 + 3x \sin y)dx - x^2 \cos y dy = 0$

\begin{solution}
	Let $v = \sin y \implies dv = \cos y dy $
	
	$\implies (1 + 3xv)dx - x^2dv = 0 \implies -x^2 \dfrac{dv}{dx} + 3xv = -1$ is linear
	
	$\implies \dfrac{dv}{dx} - \dfrac{3}{x}v = \dfrac{1}{x^2}$ --(1)
	
	$\mu (x) = e^{-\int \dfrac{3}{x}dx} = e^{-3\ln|x|} = |x|^{-3} = x^{-3}$ taking positivity
	
	$\implies x^{--3}\dfrac{dv}{dx} - 3x^{-4}v = x^{-5} \implies x^{-3}v = \int x^{-5}dx = \dfrac{1}{-4}x^{-4} + C$
	
	$\implies x^{-3}\sin y = \dfrac{1}{-4}x^{-4} + C$
\end{solution}

(C) Benoull's equation's: $y' + p(x)y = Q(x)y^n$ -- (1)

Suppose $n = 1$ (1) is linear and separable,

Suppose $n \neq 1$, (1) $\implies y^ny^1 + p(x)y^{1-n} = Q(x)$ -- (2)

$\implies $ Set $z = y^{1-n} \implies z' = (1-n)y^{-n}y'$

(2) $\implies \dfrac{1}{1-n}z' + p(x)z = Q(x)$, which is linear.

\textbf{Example.} $y(6y^2 + x -1)dx + 2xdy = 0$

\begin{solution}
	$2x\dfrac{dy}{dx} - (x+1)y = -6y^3 \implies 2xy^{-3}y' - (x+1)y^{-2} = -6$
	
	Set $z = y^{-2} \implies z' = -2y^{-3}y' \therefore -xz' - (x+1)z = -6$ is linear
	
	$\implies z' + \dfrac{x+1}{x}z = \dfrac{6}{x},~\mu(x) = e^{\int \dfrac{x+1}{x}dx} = e^{x + \ln|x|} = |x|e^x = xe^x$
	
	(1) $\times \mu \implies (xe^xz)' = 6e^x \implies xe^xz = 6x^x + c$
\end{solution}


\textbf{Example} $6y^2 dx - x(2x^3 + y)dy = 0$

\begin{solution}
	$-x(2x^3 + y)\dfrac{dy}{dx} + 6y^2 = 0$ is not Bonuliy
	
	$6y^2 \dfrac{dx}{dy} - xy = 2x^4$, which is a Berrnoulli's equation
	
	$\implies 6y^2x^{-4}\dfrac{dx}{dy} - yx^{-3} = 2$
	
	Set $z = x^{-3} \implies z' = -3x^{-4}x' \implies -2y^2z' - yz = 2$, is linear
	
	$\implies z' + \dfrac{1}{2y}z = -y^{-2}$
	
	$\mu(y) = e^{\int \dfrac{1}{2y}dy} = e^{\dfrac{1}{2}\ln|y|} = \sqrt{|y|} = \sqrt{y}$ (taking positive)
	
	$\implies y^{\frac{1}{2}}z' + \frac{1}{2}y^{\frac{1}{-2}}z = -y^{\frac{3}{-2}} \implies y^{\frac{1}{2}}z = 2y^{-\frac{1}{2}} + c $
	
	$\implies y^{\frac{1}{2}}x^{-3} = 2y^{-\frac{1}{2}} + c$
\end{solution}

(D) $(a_1x+b_1y+c_1)dx + (a_2x+b_2y+c_2)dy = 0~|~c_1^2+c_2^2 \neq 0$

Case I: $\frac{a_2}{a_1} \neq \frac{b_2}{b_1}$

Let $x = u+h$ and $y = v + k$, $h,k$ are constants to be determined later.

$\implies dx = du $ and $dy = dv $

$\implies [a_1u+b_1v+(a_1h+b_1k+c_1)]du + [a_2(u+h)+b_2(v+k)+c_2]dv = 0$

$\implies [a_1u + b_1 v + (a_1h+b_1+c_1)]du + [a_2u+b_2v+(a_2h+b_2k+c_2)]dv = 0$

Take $h,k$ such that $\begin{cases}
	a_1 h + b_1 k + c_1 = 0\\
	a_2 h + b_22k + c_2 = 0
\end{cases}$

$\implies (a_1u + b_1v)du + (a_2u+b_2v)dv = 0$,which is homogeneous.

Case II $\frac{a_2}{a_1} = \frac{b_2}{b_2} = l \implies a_2 = la_1 $ and $ b_2 = lb_1$

$\implies (a_1 x + b_1y + c_1)dx + (la_1x + b_1y + c_2)$dy = 0

Let $v = a_1 x + b_1y \implies dv = a_1 dx + b_1dy $

$\implies \frac{1}{a_1}(v+c_1)(dv - b_1dy) + (lv + c_2)dy = 0$

$\implies \frac{1}{a_1}(v+c_1)dv - \frac{b_1}{a_1}(v + c_1)dy + (lv + c_2)dy = 0$

$\implies \frac{1}{a_1}(v+c_1)dv + [lv + c_2 - \frac{b_1}{a_1}(v + c_1)]dy = 0$ which is separable.

\textbf{Example.} $(x+2y - 4)dx - (2x + y-5)dy = 0$

\begin{solution}
	$\begin{cases}
		h + 2k - 4 = 0 \text{--(1)}\\
		2h + k - 5 = 0 \text{--(2)}
	\end{cases}$
	
	$(1) \times (2) - (2) \implies 3k - 3 = 0 \implies k = 1,h = 2$
	
	Let $x = u+2,~y=v+1 \implies dx = du $ and $dy = dv$
	
	$\therefore [(u+2)+ 2(r+x) - 4]du - [2(u+2) + (v + x) - 5]dv = 0$
	
	$\implies (u+2v)du - (2u + v)du = 0 \implies \dfrac{dv}{du} = \dfrac{u + 2v}{2u + v} = \dfrac{1 + 2\frac{v}{u}}{2 + \frac{v}{u}}$
	
	Let $v = uw \implies \dfrac{dv}{du} = w+u\dfrac{dw}{wu}~\therefore w + u\dfrac{dw}{du} = \dfrac{1 + 2w}{2 + w} $
	
	$\implies u \dfrac{dw}{du} = \dfrac{1 + 2w - 2w - w^2}{2 + w} = \dfrac{1 - w^2}{2+2} \implies \int \dfrac{2 + w}{1-w^2} = \int \dfrac{1}{u}du$
	
	$\implies \dfrac{3}{-2}\ln|w-1| + \dfrac{1}{2}\ln|w+1| = \ln |u| + c$
	
	$\implies \dfrac{3}{-2}\ln|\dfrac{y-1}{x-2} - 1| + \dfrac{1}{2} \ln|\dfrac{y-1}{x-2} + 1 | = \ln|x-2| + c$
\end{solution}

\textbf{Example.} $(2x + 3y - 1)dx + (2x + 3y + 2)dy = 0$

Let $v = 2x + 3y \implies dv = 2dx + 3dy$

$\therefore \dfrac{1}{2}(v-1)(dv - 3dy) + (v + 2)dy = 0$

$\implies \dfrac{1}{2}(v-1)dv - \dfrac{3}{2}(v-1)dy + (v+2)dy = 0$

$\implies \dfrac{1}{2} (v-1)dv = [\dfrac{3}{2}(v-1) - 2(v+2)]dy \implies \int \dfrac{v-1}{v-7}dv = \int 1 dy$

$\implies v + 6 \ln |v - 7| = y + c \implies 2x+3y + 6\ln |2x+3y -7| = y+c$

\newpage

\section*{6. Linear Differential}

Form ($n$th-order)

($\star$) $b_0(x)y^n(x) + b_1(x)y^{n-1}(x) + \cdots + b_{n-1}(x)y'(x) + b_n(x)y(x) = R(x)$ where $b_0(x) \neq 0$

\begin{defn}
	If $R(x) = 0$, then $(\star)$ is said to be homogeneous. Otherwise, it is said to be nonhomogeneous
\end{defn}

\begin{defn}
	Let $I$ be an interval. If $b_0(x) \neq 0,~\forall x \in I$ and $b_0,b_1,\cdots,b_n,R \in C(I)$, then $(\star)$ is said to be normal on $I$.
\end{defn}

\textbf{Example.} $(x - 1)y' + y = \sin x$

$1$st order, linear, nonhomogeneous equation. normal on any interval $I$ where $1 \notin I$

\textbf{Example.} $3y'' + xy = 0$

$2$nd order, linear, homogeneous equation. normal on any interval $I$

\begin{thm*}
		Let $y_1,\cdots,y_k$ be solution of
		
		$$(\star) b_0 (x)y^{(n)} + b_1(x)y^{(n-1)} + \cdots + b_n (x)y$$
		
		in $I$. Then, $\forall c_1,\cdots,c_n \in \R$, $y = c_1y_1 + \cdots + c_n y+n$ is also a solution of $(\star)$ on $I$
\end{thm*}

\begin{solution}
	$b_0(x)y^{(n)} + b_1(x)y^{(n-1)} + \cdots + b_{n-1}(x)y' + b_n(x)y$
	
	$ = b_0(x)[c_1y_1^{(n)} + \cdots + c_ky_k^{(n)}] + \cdots + b_1(x)[c_1y_1 + \cdots + c_ky_k]$
	
	$ = c_1[b_0y_1^{(n)} + b_1(x)y_1^{(n-1)} + \cdots + b_{n-1}(x)y_1' + b_n(x)y_1] + \cdots + c_k [b_0(x)y_k^{(n)} + b_1(x)y_k^{n-1} + \cdots + b_{n-1}(x)y_k + b_n(x)y_k] = 0$
	
	$\therefore y= c_1y_1+\cdots + c_ky_k$ is also a solution of $(\star)$ on $I$
\end{solution}

\begin{defn}
	Let $y_1,\cdots,y_k$ be functions. Then $\forall c_1,\cdots,c_k \in \R$, $c_1y_1 + \cdots + c_ky_k$ is called a linear combination of $y_1,\cdots,y_k$
\end{defn}

\begin{rmk*}
	We may restate Thm 6.1 as follows:
	
	Any linear combination of solutions of $(\star)$ is also a solution of $(\star)$
\end{rmk*}

\textbf{Example. } $y'' + y = 0$

$y = \sin x$ and $y = \cos x$ are solutions on $\R$. By Thm 6.1, $\forall c_1,c_2 \in \R,~ y = c_1\sin x + c_2 \cos x$ is also a solution on $\R$

\begin{thm*}
	Consider $(\star)$ is normal in $I$ and $x_0 \in I$. For any given $y_0,\cdots,y_{n-1} \in \R$, $(\star)$ has a unique solution $y = y(x)$ in $I$ satisfying
	
	$$y(x_0 = y_0),~y'(x_0) = y_1,\cdots,y^{n-1}(x_0) = y_{n-1}$$,
\end{thm*}

\textbf{Example} $\begin{cases}
	y'' + y = 0 \text{ -- (1)}\\
	y(0) = 0,~y'(0) = 1 \text{ -- (2)}
\end{cases}$

$\because$ (1) is normal on $\R$ and $0 \in \R$

$\therefore$ By Thm 6.2, (1) has a unique solution on $\R$ satisfying (2). Indeed, $y = c_1\sin x + c_2 \cos x$ is a solution of (1) on $\R$, $\forall c_1,c_2 \in \R$

$y(0) = 0 \implies c_2 = 0 \implies y = c_2 \sin x \implies y' = c_1 \cos x$

$y'(0) = 1 \implies c_1 = 1 \therefore y = \sin x$ is the unique solution.

\textbf{Example.}

$\begin{cases}
	x^2y'' + 2xy' - 12y = 0 \text{ ---(1)}\\
	y(1) = 4,~y'(1) = 5 \text{ ---(2)}
\end{cases}$

$\because$ (1) is normal on $(-\infty,0)$ or $(0,\infty)$, and $1 \in (0,\infty)$

$\therefore$ By Thm 6.2, (1) has a unique solution in $(0,\infty)$ satisfying (2)

\begin{defn}
	Let $f_1,\cdots,f_n$ be functions in $[a,b]$. If there exists $c_1,\cdots,c_n \in \R$, not all zero, such that
	
	$$c_1f_1(x) + \cdots + c_nf_n(x) = 0,\forall x \in [a,b],$$
	
	then $f_1,\cdots,f_n$ are said to be linearly dependent on $[a,b]$. Otherwise, they are said to be linearly independent on $[a,b]$.
\end{defn}

\begin{rmk*}
	If $f_1,\cdots,f_n$ are linear dependent on $[a,b]$, then
	
	$$c_1f_1(x) + \cdots + c_nf_n(x) = 0,~\forall[a,b]$$
	
	implies $c_1 = \cdots = c_n = 0$
\end{rmk*}

\textbf{Example.} $x,2x$ are linear dependent on $\R$

\begin{defn}
	let $f_1,\cdots , f_n$ be $n$ functions
	
	$\left|\begin{matrix}
		f_1(x) & f_2(x) & \cdots & f_n(x)\\
		f_1'(x) & f_2'(x) & \cdots & f_n'(x) \\
		\vdots & \vdots & & \vdots \\
		f_1^{n-1}(x) & f_2^{n-1}(x) & \cdots & f_n^{(n-1)}(x)
	\end{matrix}\right|$ is called the Wronskian of $f_1,\cdots,f_n$
\end{defn}

\begin{thm*}
	Suppose $(\star)$ is normal on $[a,b]$, and suppose $y_1,\cdots,y_n$ are solutions of $(\star)$ on $[a,b]$. Then $y_1,\cdots,y_n$ are linear independent on $[a,b]$ iff $w[y_1,\cdots,y_n](x_0)\neq 0$ for some $x_0 \in [a,b]$
\end{thm*}

\begin{proof}
	$(\Rightarrow)$ Suppose $w[y_1,\cdots,y_n](x) = 0,~\forall x \in [a,b]$. Pick any point $x_0 \in [a,b]$. Then $w[y_1,\cdots,y_n](x_0) = 0$, i.e.
	
	$$\left|\begin{matrix}
		y_1(x_0) & \cdots & y_n(x_0)\\
		y_1'(x_0) & \cdots & y_n'(x_0)\\
		\vdots & & \vdots\\
		y^{n-1}_1(x_0) & \cdots & y^{n-1}_n(x_0)
	\end{matrix}\right| = 0$$
	
	Thus, $\exists c_1,\cdots,c_n \in \R$, not all are zero such that
	
	$$\begin{cases}
		y_1(x_0)c_1 + y_2(x_0)c_2 + \cdots + y_n(x_0)c_n = 0\\
		Y_1'(x_0)c_1 + y_2'(x_0)c_2 + \cdots + y_n'(x_0)c_n = 0\\
		\vdots \\
		y_1^{n - 1}(x_0) + y_2^{(n-1)}(x_0)c_2 + \cdots + y_n^{n-1}(x_0)c_n = 0
	\end{cases}$$
	
	Let $y(x) = c_1y_1 (x) + \cdots + c_ny_n(x)$
	
	$\because y_1,\cdots,y_n$ are solutions of $(\star)$ on $[a,b]$ $\therefore$ By Thm 6.1, $y$ is also a solution of $\star$ on $[a,b]$
	
	In addition, by (1), $y(x_0) = y'(x_0) = \cdots = y^{n-1}(x_0) = 0$. By Thm 6.2, $y = 0$ in $[a,b]$ $\therefore c_1,\cdots,c_n$ are not all zero such that (2) holds.
	
	$\therefore y_1,\cdots,y_n$ are linear dependent on $[a,b]$
	
	$(\Leftarrow)$ Suppose $w[y_1,\cdots,y_n](x_0) \neq 0$ for some $x_0 \in [a,b]$
	
	$\implies \left| \begin{matrix}
		y_1(x_0) & \cdots & y_n(x_0) \\
		\vdots & & \vdots \\
		y^{(n-1)}_1(x_0) & \cdots & y_n^{(n-1)}
	\end{matrix}\right| \neq 0$ --- (3)
	
	Let $c_1,\cdots,c_n \in \R$ such that $c_1y_1(x) + c_2y_2(x) + \cdots + c_ny_n(x) = 0,~\forall x \in [a,b],$
	
	$\implies c_1y_1'(x) + c_2y_2'(x) + \cdots + c_ny_n'(x) = 0,~\forall x \in [a,b]$
	
	$\implies c_1y_1''(x) + c_2y_2''(x) + \cdots + c_ny_n''(x) = 0,~\forall x \in [a,b]$
	
	$\vdots$
	
	$\implies c_1y_1^{(n-1)}(x) + c_2y_2^{(n-1)}(x) + \cdots + c_ny_n^{(n-1)} = 0 ,~\forall x \in [a,b]$
	
	$\because x_0 \in [a,b]$
	
	$\begin{cases}
		c_1y_1(x_0) + c_2y_2(x_0) + \cdots + c_ny_n(x_0) = 0\\
		c_1y_1'(x_0) + c_2y_2'(x_0) + \cdots + c_ny_n'(x_0) = 0\\
		\vdots \\
		c_1y^{n-1}(x_0) + c_2y^{(n-1)}(x_0) + \cdots + c_ny_n^{(n-1)}(x_0) = 0
	\end{cases}$
	
	By (3), $c_1 = \cdots = c_n = 0$. So $y_1,\cdots,y_n$ are linear independent on $[a,b]$
\end{proof}

\textbf{Example.} $y'' + y = 0$ has two solutions $\sin x$ and $\cos x$ on $\R$

$w[\sin x , \cos x] = \left|\begin{matrix}
	\sin x & \cos x\\
	\cos x & -\sin x
\end{matrix}\right| = -1 \neq 0,~\forall x \in \R$

$\therefore$ By Thm 6.3, $\sin x$ and $\cos x$ are linear independent on $\R$

\textbf{Example.} $y''' - 2y'' - y' + 2y = 0$ has solution: $e^x ,~ e^{-x},e^{2x}$ on $\R$

$w[e^x,e^{-x}] = \left| \begin{matrix}
	e^x & e^{-x} & e^{2x} \\
	e^x & -e^{-x} & 2e^{2x} \\
	e^x & e^{-x} & 4e^{2x}
\end{matrix} \right| = e^{3x} \left| \begin{matrix}
	1 & 1 & 1\\
	1 & -1 & 2\\
	1 & 1 & 4
\end{matrix}\right| = -6e^{2x} \neq 0$

$\therefore e^{-x},e^{-x},e^{2x}$ are linear independent on $\R$

\begin{thm*}
	Suppose $(\star)$ is normal on $[a,b]$, and $y_1,\cdots,y_n$ are linear independent solution of $(\star)$ on $[a,b]$. For any solution $\phi$ of $(\star)$ on $[a,b]$, $\exists \overline{c_1},\cdots,\overline{c_n} \in \R$, such that $\phi (x) = \overline{c_1}y_1(x) + \cdots + \overline{c_n}y_n(x), \forall x \in [a,b]$
\end{thm*}

\begin{proof}
	$\because y_1,\cdots,y_n$ are linear independent solution of $(\star)$ on $[a,b]$. By Thm 6.3, $\exists x_0 \in [a,b]$ such that $w[y_1,\cdots,y_n](x_0) \neq 0,$
	
	i.e. $ \left| \begin{matrix}
		y_1(x_0) & \cdots & y_n(x_0) \\
		\vdots & & \vdots \\
		y^{(n-1)}_1(x_0) & \cdots & y_n^{(n-1)}
	\end{matrix}\right| \neq 0$
	
	$\implies \exists \overline{c_1},\cdots,\overline{c_n} \in \R$ such that
	
	$\begin{cases}
		\overline{c_1}y_1(x_0) + \overline{c_2}y_2(x_0) + \cdots + \overline{c_n}y_n(x_0) = 0\\
		\overline{c_1}y_1'(x_0) + \overline{c_2}y_2'(x_0) + \cdots + \overline{c_n}y_n'(x_0) = 0\\
		\vdots \\
		\overline{c_1}y^{n-1}(x_0) + \overline{c_2}y^{(n-1)}(x_0) + \cdots + \overline{c_n}y_n^{(n-1)}(x_0) = 0
	\end{cases}$
\end{proof}

let $y(x) = \overline{c_1}y_1(x) + \cdots + \overline{c_n}y_n(x)$

$\therefore y_1,\cdots,y_n$ are solutions of $(\star)$ on $[a,b]$

$\because$ By Thm 6.1, $y$ is also a solution of $(\star)$ on $[a,b]$.

By (1), $y(x_0) = \phi (x_0),~y'(x_0) = \phi'(x_0),\cdots,y^{(n-1)}(x_0) = \phi^{(n-1)}(x_0)$.

By Thm 6.2, $y(x) = \phi(x),~\forall x \in [a,b]$,

i.e. $\phi(x) = c_1y_1(x) + \cdots + c_ny_n(x) \forall x \in [a,b]$

\begin{defn}
	Let $(\star)$ be normal in an interval $I$. Suppose $y_1,\cdots,y_n$ are linear independent solutions of $(\star)$ on $I$. Then $y = c_1y_1 + \cdots + c_ny_n$ is called the general solution of $(\star)$ on $I$, where $c_1,\cdots,c_n$ are arbitrary constants.
\end{defn}

\textbf{Example.} $y'' + y = 0$ has solutions linear independent $\sin x$ and $\cos x$

$\therefore$ The general solutions is $y = c_1\sin x + c_2 \cos x$, where $c_1$ and $c_2$ are arbitrary constants. Consider the nonhomogeneous equation

(NH) $b_0(x)y^n(x) + b_1(x)y^{(n-1)}(x) + \cdots + b_{n-1}(x)y'(x) + b_n(x)y(x) = R(x)$ and its corresponding homogeneous equation.

(H) $b_0(x)y^{(n)}(x) + b_1(x)y^{n-1}(x) + \cdots + b_{n-1}(x)y'(x) + b_n(x)y(x) = 0$

\begin{thm*}
	Let $v$ be any solution of (NH) and let $u$ be any solution of (H). Then $u + v$ is also a solution of (NH).
\end{thm*}

\newpage

\begin{proof} $ $


	$b_0(x)[u + v]^{(n)} + b_1(x)[u + v]^{(n-1)} + \cdots + b_{n -1}(x)[u + v]' + b_n(x)[u + v]$
	
	$ = b(x)[u^{(n)} + v^{(n)}] + b_1(x)[u^{(n-1)} + v^{(n-1)}] + \cdots + b_{n-1}(x)[u' + v'] + b_n(x)[u + v]$
	
	$ = [b_0(x)u^{(n)} + b_1(x)u^{n-1} + \cdots + b_{n-1}(x)u' + b_n(x)u] + [b_0(x)v^{(n)} + b_1(x)u^{(n-1)} + \cdots + b_{n-1}(x)v' + b_n(x)v]$
	
	$ = 0 + R(x)$ ($\because ~ u$ is an root of (H) and $v$ is an root of (NH))
	
	$\therefore u+v$ is a solution of (NH)
\end{proof}

\textbf{Example.} $y'' + y = x$ has a solution $x$, $y'' + y = 0$ has a solution $\sin x$.

By Thm 6.8 $x + \sin x$ is a solution of $y'' + y = x$

\begin{rmk*}
	Let $y_p$ be a particular solution of (NH) and $y_c = c_1y_1+\cdots + c_ny_n$ be the general solution of (H). Then, $\forall c_1,\cdots,c_n \in \R,~y_c + y_p$ is a solution of (NH).
\end{rmk*}

\begin{thm*}
	Let $y_p$ be a particular solution of (NH) and $y_c = c_1y_1 + \cdots + c_ny_n$ be the general solution of (H). Then every solution $y$ of (NH) can be expressed in the form $y = y_c + y_p$ for suitable choice of $c_1,\cdots,c_n$
\end{thm*}

\begin{proof}
	$\because~y$ and $y_p$ are solution of (NH).
	
	$\therefore b_0(x)y^{(n)} + b_1(x)y^{(n-1)} + \cdots + b_{n-1}(x)y' + b_n(x)y = R(x)$
	
	$b_0(x)y_p^{(n)} + b_1(x)y_p^{(n-1)} + \cdots + b_{n-1}(x)y_p' + b_n(x)y_p = R(x)$
	
	(1)-(2) $\implies b_0(x)[y-y_p]^{(n)} + b_1(x)[y-y_p]^{(n-1)} + \cdots + b_{n-1}(x)[y - y_p] + b_n(x)[y - y_p] = 0 $
	
	$\implies y - y_p$ is a solution of (H).
	
	By Thm 6.4, $\exists c_1,\cdots,c_n \in \R \ni [y - y_p] = c_1y_1 + \cdots + c_ny_n$
	
	$y = c_1y_1 + \cdots + c_ny_n + y_p = y_c + y_p$
\end{proof}

\begin{defn}$ $
	\begin{enumerate}
		\item The general solution of (H) is called the complementary function of (NH). We denote it by $y_c$
		\item The general solution of (NH) is $y = y_c + y_p$, where $y_p$ is any particular solution of (NH).
	\end{enumerate}
\end{defn}

\textbf{Example.} $y'' = 4$ and $y'' = 0 \implies y' = c_1 \implies y = y_c = c_1x + c_2$, where $c_1,c_2$ are arbitrary constant

$y'' = 4 \implies y' = 4x = y = y_p = 2x^2$

$\therefore$ The general solution of $y'' = 4$ is $y = y_c + y_p = c_1x + c_2 + 2x^2$

\begin{defn}$ $

	Let $A = a_0D^n + a_1D^{n-1} + \cdots + a_{n-1}D + a_n$
	
	$B = b_0 D^n + b_1 D^{n-1} + \cdots + b_{n-1}D + b_n$
	
	we define $A + B = (a_0 + b_0)D^n + \cdots + (a_{n - 1} + b_{n-1})D + (a_n + b_n)$
\end{defn}

\textbf{Example.} $A = 3D^2 - D + x - 2,~ B = x^2D^2 + 4D = 7$

$\implies A + B = (3 + x^2)D^2 + 3D + x + 5$

\begin{rmk*}
	Let $A$ be a $n$th order linear differential operator, $c_1,c_2$ be constant, $f_1,f_2$ be two functions with $f_1^{(n)}$ and $f_2^{(n)}$ exists.
	
	Then $A(c_1f_1 + c_2f_2) = c_1Af_1 + c_2Af_2$(i.e. $A$ is linear)
\end{rmk*}

\begin{proof}
	Write $A = a_0D^n + a_1D^{n-1} + \cdots + a_{n-1}D + a_n$
	
	$\implies A(c_1f_1 + c_2f_2) = a_0(c_1f_1 + c_2f_2)^{(n)} + a_1 (c_1f_1 + c_2f_2)$
	
	$= a_0 (c_1f_1 + c_2f_2)^{(n)} + a_1(c_1f_1 + c_2f_2)^{(n-1)} + \cdots + a_{n-1}(c_1f_1 + c_2f_2)' + a_n(c_1f_1 + c_2 f_2) $
	
	$= a_0 (c_1f_1^{(n)} + c_2f_2^{n-1}) + a_1(c_1f_1^{n-1} + c_2f_2^{n-1}) + \cdots + a_{n-1}(c_1f_1' + c_2f_2') + a_n(c_1f_1 + c_2f_2)$
\end{proof}

The fundamental low:

Let $A,B,C$ be linear differential operators. Then

\begin{enumerate}
	\item $A + B = B + A$
	\item $(A + B) + C = A + (B + C)$
	\item $(AB)C = A(BC)$
	\item $A(B + C) = AB + AC$
	\item $AB = BA$ if $A,B$ are with  constant cofficients
\end{enumerate}

Let $a_0,a_1,\cdots , a_n$ be constants

$a_ny^{(n)} + a_1y^{(n-1)} + \cdots + a_{n-1}y' + a_ny = 0$ -- (1) 

$\Leftrightarrow (a_0D^n + a_1D^{n-1} + \cdots + a_{n - 1}D + a_n)y = 0$

Let $y = e^{nx}$

Put $y = e^{nx}$ into (1) $\implies a_0 (m^ne^{mx}) + a_1(m^{n-1}e^{mx}) + \cdots + a_{n-1}(me^{mx}) + a_n e^{mx} = 0$









\end{document}