\documentclass[12pt,reqno]{amsart}
%\usepackage[margin=1in]{geometry}
\usepackage{tcolorbox}
\usepackage{amssymb}
\usepackage{amsthm}
\usepackage{amsmath}
\usepackage{amssymb}
\usepackage{mathrsfs}
\usepackage{centernot}
\usepackage{lastpage}
\usepackage{fancyhdr}
\usepackage{accents}
\usepackage{tasks}
\usepackage{graphicx}
\usepackage{natbib}
\usepackage{tabularx}
\usepackage{multirow}
\usepackage{booktabs}
\usepackage{hyperref}
\usepackage{bm}
\usepackage{float}
\theoremstyle{plain}
\usepackage{multicol}
\usepackage{enumitem,kantlipsum}

% 加上浮水印
%\usepackage{wallpaper}
%\CenterWallPaper{.180}{../qsnake-logo.jpg}


\linespread{1.2}
\parindent = 0pt
\pagestyle{fancy}
\setlength{\parindent}{0pt}
%\everymath{\displaystyle}
%new area
%\usepackage[utf8]{inputenc}
%\usepackage{CJKutf8}
%\xeCJKsetup{AutoFakeBold=true, AutoFakeSlant=true}

% 設定頭部
\fancyhead[L]{Midterm} % 左邊頭部清空
\fancyhead[C]{} % 中間頭部清空
\fancyhead[R]{} % 右邊頭部顯示頁碼

% Adjust the footer as desired:
\fancyfoot[L]{} % Left footer: Empty.
\fancyfoot[C]{\thepage} % Center footer: Empty.
\fancyfoot[R]{} % Right footer: Empty.



% we will modify sections, subsections and sub subsections
\RequirePackage{titlesec}
% Modification of section 
\titleformat{\section}[block]{\normalsize\bfseries\filcenter}{\thesection.}{.3cm}{} 


% modification of subsection and sub sub section
\titleformat{\subsection}[runin]{\bfseries}{ \thesubsection.}
{1mm}{}[.\quad]
\titleformat{\subsubsection}[runin]{\bfseries\itshape}{ \thesubsubsection.}
{1mm}{}[.\quad]

\newenvironment{solution}
  {\renewcommand\qedsymbol{$\blacksquare$}
  \begin{proof}[Solution]}
  {\end{proof}}
\renewcommand\qedsymbol{$\blacksquare$}

\newcommand{\ubar}[1]{\underaccent{\bar}{#1}}

%%%%%%%%%%%%%%%%%%%%%%%%%%%%%% Textclass specific LaTeX commands.
%\theoremstyle{plain}
%\newtheorem{thm}{\protect\theoremname}[section]
\newtheorem{thm}{\textbf{Theroem}}[section]
\newtheorem{cor}[thm]{Corollary}
\newtheorem{lmma}[thm]{Lemma}
\newtheorem*{defn}{\underline{Definition}}
\newtheorem*{prop*}{Proposition}
\newtheorem*{ex*}{Example}
\newtheorem*{sol*}{Solution}
\newtheorem*{cor*}{Corollary}
\newtheorem*{thm*}{Theorem}
\newtheorem*{lmma*}{Lemma}
\newtheorem*{rmk*}{Remark}
\newtheorem*{pf*}{\underline{\textbf{Proof\ }}}

%%%%%%%%%%%%%%%%%%%%%%%%%%%%%% User specified LaTeX commands.
\renewcommand{\P}{\mathscr{P}}
\newcommand{\B}{\mathscr{B}}
\newcommand{\A}{\mathscr{A}}
\newcommand{\C}{\mathbb{C}}
\newcommand{\CC}{\mathscr{C}}
\newcommand{\R}{\mathbb{R}}
\newcommand{\Q}{\mathbb{Q}}
\newcommand{\Z}{\mathbb{Z}}
\newcommand{\N}{\mathbb{N}}
\newcommand{\X}{\mathcal{X}}
\newcommand{\T}{\mathscr{T}}
\newcommand{\arbuni}{\bigcup_{\alpha\in I}}
\newcommand{\finint}{\bigcap_{i=1}^n}
\newcommand{\Ua}{{\textsc{U}_\alpha}}
\newcommand{\Ui}{\textsc{U}_i}
\newcommand{\pair}[2]{\left( \,#1\,,\,#2\,\right) }
\newcommand{\dint}[2]{\int_{#1}^{#2}}
\newcommand{\sett}[1]{\left\{ \,#1 \,\right\}}
\newcommand{\linearcombination}[2]{#1_1#2_1+\cdots+#1_n#2_n}
\newcommand{\slinearcombination}[1]{#1_1+\cdots+#1_n}
\newcommand{\spann}[1]{\text{span($#1$)}}
\newcommand{\sub}[1]{\text{sup}}
\newcommand{\inn}[1]{\left< #1 \right>}
\newcommand{\kernal}[1]{Ker(#1)}
\newcommand{\image}[1]{Im(#1)}
\newcommand{\norm}[1]{\parallel #1 \parallel}
\newcommand{\dia}[0]{\text{dia}}
\newcommand{\marking}[1]{\text{\color{red} #1}}
%%%%%%%%%%%%%%%%%%%%%%%%%%%%%%

\begin{document}

\title{Probability Exercise}

\author{Week 1}

\maketitle

\begin{enumerate}
	\item Two fair dice are rolled. Let $X$ equal the product of the $2$ dice. Compute $P\{X = i\}$ for $i = 1,\cdots,36$
	\item Let $X$ represent the difference between the number of heads and the number of tails obtained when a coin is tossed $n$ times. What are the possible values of $X$?
	\item Suppose that the distribution function of $X$ is given by
		$$F(b) = \begin{cases}
			0 & b<0\\
			\frac{b}{4} & 0 \leq b < 1\\ 
			\frac{1}{2} + \frac{b-1}{4} & 1\ \leq b < 2\\
		 	\frac{11}{12} & 2 \leq b < 3\\
		  	1 & 3 \leq b�
		\end{cases}$$
		\begin{enumerate}
		\item Find $P\{X = i\},~i = 1,2,3$
		\item Find $P\{\frac{1}{2} < X < \frac{3}{2�}\}$
		\end{enumerate}
	\item Four buses carrying $148$ students from the same school arrive at a football stadium. The buses carry, respectively, $40,33,25$, and $50$ students. One of the students is randomly selected. Let $X$ denote the number of students who were on the bus carrying the randomly selected student. One of the $4$ bus drivers is also randomly selected. Let $Y$ denote the number of students on her bus.
	
		\begin{enumerate}
		\item Which of $E[X]$ or $E[Y]$ do you think is larger? Why?
		\item Compute $E[X]$ and $E[Y]$
		\end{enumerate}
	\newpage
	\item Two coins are to be flipped. The first coin will land on heads with probability $0.6$, the second with probability $0.7$. Assume that the results of the flips are independent, and let $X$ equal the total number of heads that result.
		\begin{enumerate}
		\item Find $P\{X = 1\}$
		\item Determine $E[X]$
		\end{enumerate}
	\item If $E[X] = 1$ and $Var(X) = 5$, find
		\begin{enumerate}
		\item $E[(2+X)^2]$
		\item $Var(4 + 3X)$
		\end{enumerate}
	\item Compare the Poisson approximation with the correct binomial probability for the following cases:
		\begin{enumerate}
		\item $P\{X = 2\}$ when $n = 8,p = 0.1$;
		\item $P\{X = 9\}$ when $n = 10,p = 0.95$
		\item $P\{X = 0\}$ when $n = 10,p = 0.1$
		\item $P\{X = 4\}$ when $n = 9,p = 0.2$
		\end{enumerate}
	\item Let $N$ be a nonnegative integer-valued random variable. For nonnegative values $a_j, j \geq 1$, show that
	$$\sum^{\infty}_{j = 1}(a_1 + \cdots + a_j)P\{N = j\} = \sum^{\infty}_{i = 1}a_iP\{N \geq i\}$$
	
	Then show that
	
	$$E[N] = \sum^{\infty}_{i = 1}P\{N \geq i\}$$
	
	and
	
	$$E[N(N+1)] = 2\sum^{\infty}_{i = 1}iP\{N \geq i\}$$
	\item Let $X$ be a Poisson random variable with parameter $\lambda$.
		\begin{enumerate}
		\item Show that 
		$$P \{X \text{ is even}\} = \frac{1}{2}[1 + e^{-2\lambda}]$$
		By using the result of Theoretical Exercise $15$ and the relationship between Poisson and binomial random variables.
		\item Verify the formula in part(a) directly by making use of the expansion of $e^{-\lambda} + e^{\lambda}$
		\end{enumerate}
	\item Show that $X$ is a Poisson random variable with parameter $\lambda$, then
	
	$$E[X^n] = \lambda E[(X + 1)^{n-1}]$$
	Now use this result to compute $E[X^3]$
	\item Prove
	
	$$\sum^n_{i=0}e^{-\lambda}\frac{\lambda^i}{i!} = \frac{1}{n!} \int^{\infty}_{\lambda}e^{-x}x^n dx$$
	Hint: Use integration by parts.
\end{enumerate}

















\end{document}