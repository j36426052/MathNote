\documentclass[12pt,reqno]{amsart}
%\usepackage[margin=1in]{geometry}
\usepackage{tcolorbox}
\usepackage{amssymb}
\usepackage{amsthm}
\usepackage{amsmath}
\usepackage{amssymb}
\usepackage{mathrsfs}
\usepackage{centernot}
\usepackage{lastpage}
\usepackage{fancyhdr}
\usepackage{accents}
\usepackage{tasks}
\usepackage{graphicx}
\usepackage{natbib}
\usepackage{tabularx}
\usepackage{multirow}
\usepackage{booktabs}
\usepackage{hyperref}
\usepackage{bm}
\usepackage{float}
\theoremstyle{plain}
\usepackage{multicol}
\usepackage{enumitem,kantlipsum}

% 加上浮水印
%\usepackage{wallpaper}
%\CenterWallPaper{.180}{../qsnake-logo.jpg}


\linespread{1.2}
\parindent = 0pt
\pagestyle{fancy}
\setlength{\parindent}{0pt}
%\everymath{\displaystyle}
%new area
%\usepackage[utf8]{inputenc}
%\usepackage{CJKutf8}
%\xeCJKsetup{AutoFakeBold=true, AutoFakeSlant=true}

% 設定頭部
\fancyhead[L]{Midterm} % 左邊頭部清空
\fancyhead[C]{} % 中間頭部清空
\fancyhead[R]{} % 右邊頭部顯示頁碼

% Adjust the footer as desired:
\fancyfoot[L]{} % Left footer: Empty.
\fancyfoot[C]{\thepage} % Center footer: Empty.
\fancyfoot[R]{} % Right footer: Empty.



% we will modify sections, subsections and sub subsections
\RequirePackage{titlesec}
% Modification of section 
\titleformat{\section}[block]{\normalsize\bfseries\filcenter}{\thesection.}{.3cm}{} 


% modification of subsection and sub sub section
\titleformat{\subsection}[runin]{\bfseries}{ \thesubsection.}
{1mm}{}[.\quad]
\titleformat{\subsubsection}[runin]{\bfseries\itshape}{ \thesubsubsection.}
{1mm}{}[.\quad]

\newenvironment{solution}
  {\renewcommand\qedsymbol{$\blacksquare$}
  \begin{proof}[Solution]}
  {\end{proof}}
\renewcommand\qedsymbol{$\blacksquare$}

\newcommand{\ubar}[1]{\underaccent{\bar}{#1}}

%%%%%%%%%%%%%%%%%%%%%%%%%%%%%% Textclass specific LaTeX commands.
%\theoremstyle{plain}
%\newtheorem{thm}{\protect\theoremname}[section]
\newtheorem{thm}{\textbf{Theroem}}[section]
\newtheorem{cor}[thm]{Corollary}
\newtheorem{lmma}[thm]{Lemma}
\newtheorem*{defn}{\underline{Definition}}
\newtheorem*{prop*}{Proposition}
\newtheorem*{ex*}{Example}
\newtheorem*{sol*}{Solution}
\newtheorem*{cor*}{Corollary}
\newtheorem*{thm*}{Theorem}
\newtheorem*{lmma*}{Lemma}
\newtheorem*{rmk*}{Remark}
\newtheorem*{pf*}{\underline{\textbf{Proof\ }}}

%%%%%%%%%%%%%%%%%%%%%%%%%%%%%% User specified LaTeX commands.
\renewcommand{\P}{\mathscr{P}}
\newcommand{\B}{\mathscr{B}}
\newcommand{\A}{\mathscr{A}}
\newcommand{\C}{\mathbb{C}}
\newcommand{\CC}{\mathscr{C}}
\newcommand{\R}{\mathbb{R}}
\newcommand{\Q}{\mathbb{Q}}
\newcommand{\Z}{\mathbb{Z}}
\newcommand{\N}{\mathbb{N}}
\newcommand{\X}{\mathcal{X}}
\newcommand{\T}{\mathscr{T}}
\newcommand{\arbuni}{\bigcup_{\alpha\in I}}
\newcommand{\finint}{\bigcap_{i=1}^n}
\newcommand{\Ua}{{\textsc{U}_\alpha}}
\newcommand{\Ui}{\textsc{U}_i}
\newcommand{\pair}[2]{\left( \,#1\,,\,#2\,\right) }
\newcommand{\dint}[2]{\int_{#1}^{#2}}
\newcommand{\sett}[1]{\left\{ \,#1 \,\right\}}
\newcommand{\linearcombination}[2]{#1_1#2_1+\cdots+#1_n#2_n}
\newcommand{\slinearcombination}[1]{#1_1+\cdots+#1_n}
\newcommand{\spann}[1]{\text{span($#1$)}}
\newcommand{\sub}[1]{\text{sup}}
\newcommand{\inn}[1]{\left< #1 \right>}
\newcommand{\kernal}[1]{Ker(#1)}
\newcommand{\image}[1]{Im(#1)}
\newcommand{\norm}[1]{\parallel #1 \parallel}
\newcommand{\dia}[0]{\text{dia}}
\newcommand{\marking}[1]{\text{\color{red} #1}}
%%%%%%%%%%%%%%%%%%%%%%%%%%%%%%

\begin{document}

%\lhead{Linear Algebra} 
%\rhead{Sabrina Edition} 
\cfoot{\thepage} %\ of \pageref{LastPage}}

In calculus

\begin{enumerate}
	\item Extreme Value Theorem: Every continuous function $f:[a,b]\rightarrow\mathrm{R}$ admit both max and min value $\Rightarrow$ Compact set
	\item Intermediate value Theorem: Given continous function $f:[a,b]\rightarrow \mathrm{R}$ for all $f(a) \leq \lambda \leq f(b) \exists c \in [a,b] \ni f(c) = \lambda \Rightarrow$ connected set
\end{enumerate}

How to prove a statement: $H P,$then $Q$, $P \Rightarrow Q$

$\begin{cases}
	\begin{cases}
	\text{Direct Proof}\\\text{Indirect Proof}\begin{cases}\text{contrapositive $\sim Q \Rightarrow \sim P$} \\ \text{by contradiction}\end{cases}
	\end{cases}\\
	Mathematical Induction
\end{cases}$

\section{Some preliminary}

\subsection{Set Theory}

We will assume that you are familiar with some basic set theory e.g. union, intersection, difference

\subsection{The Number System} $ $ 

$\mathbb{N}=\sett{1,2,3,\cdots}$ the set of all positive integers $n$ natural numbers

$\mathbb{Z}=\sett{\cdots,-2,-1,0,-1,-2,\cdots}$ the set of all integers called the ring of intepus

$\mathbb{Q} = \sett{\dfrac{m}{n}~:~n,m \in \mathbb{Z},n\neq 0}$ the set of all rational numbers of the national number field on real line

$\mathbb{R}$ the set all of real numbers on the real number field on real line

$\mathbb{C} = \sett{z=a+ib~|~a,b\in\mathbb{R}}$ the set of all complex numbers or the complex number filed on complex plane, where $i = \sqrt{-1}$

 \begin{rmk*}$ $
 	\begin{enumerate}
 		\item $x + 2 = 0$ no root in $\mathbb{N}$\\$3x - 5 = 0$ no root in $\mathbb{Z}$\\$x^2 + 1 = 0$ no root in $\mathbb{R}$
 		\item One can construct $\mathbb{Q}$ from $\mathbb{Z}$ in algebraic way, called the fraction field of $\mathbb{Z}$
 		\item One can construct $\mathbb{R}$ from $\mathbb{Q}$ in two ways:
 		
 		$\cdot$ Using Dedekind cut which is given in the appendix of Rudin p17-21
 		
 		$\cdot$ Using completion of matrix space
 		\item One can construct $\mathbb{C}$ from in complex analysis
 	\end{enumerate}
 \end{rmk*}
 
 \newpage

\begin{ex*} $ $
	\begin{enumerate}
		\item Between any two rational numbers, there is another one
		
		\begin{proof}
			Let $r,s \in \mathbb{Q}$ with $r<s$, then $\dfrac{r+s}{2} \in \mathbb{Q}$ and $r < \dfrac{r+s}{2} < s$
			
			\begin{tcolorbox}
				$r = \dfrac{m_1}{n_1}, s = \dfrac{m_2}{n_2}, \dfrac{r+s}{2} = \dfrac{\frac{m_1}{n_1}+\frac{m_2}{n_2}}{{2}} = \dfrac{m_1n_2 + n_1m_2}{2n_1m_1} 
				\in Q$
				
				$s = \dfrac{s+s}{2} > \dfrac{r + s}{2} > \dfrac{r+r}{2} = r$
			\end{tcolorbox}
		\end{proof}
		
		\item $x^2 = \dfrac{4}{9}$ has exactly two rational solutions, namely, $\pm \frac{2}{3}$
		\item $x^2 = 2$ has exactly two real root, namely, $\pm \sqrt{2}$
		\item Is there any rational roots of $x^2 = 2$? i.e., is $\sqrt{2}$ rational?
		\begin{tcolorbox}
			Suppose $r = \dfrac{m}{n} \in \mathrm{Q}$, is a root of $x^2 = 2$, where $(m,n) = 1$
			
			Then $\dfrac{m^2}{n^2} = 2 \implies m^2 = 2n^2 \implies 2~|~m^2 \implies 2~|~m \implies 4~|~m^2 \implies 4~|~2n^2$
			
			$\implies 2 ~|~ n^2 \implies 2 ~|~ n \implies (n,m) \neq 1$
		\end{tcolorbox}
		\item Let $A = \sett{r \in \mathrm{Q}~|~r > 0 ~\& ~r^2 < 2}, B = \sett{r \in \mathrm{Q}~|~r>0 ~\& ~r^2 > 2}$
		
		Then $A$ contains no largest numbers, i.e. max element $\& ~B$ contains no smallest numbers, i.e. min element
		\begin{tcolorbox}
			\begin{proof}
				$A$ contains no largest numbers $\Leftrightarrow$ given $r \in A$, $\exists s \in A \ni s>r$
				
				Now, given $r \in A$, Let $s = r - \dfrac{r^2 - 2}{r+2} = \dfrac{2r + 2}{r+2} ~(\star_1)$
				
				$\implies s^2 - 2 = \dfrac{2(r^2 - 2)}{(r+2)^2} ~(\star_2)$
				
				Now, $r \in A, r^2 < 2 \implies r^2-2 <0 \therefore $
				
				$(\star_1)\&(\star_2) \implies s > r ~\& ~s^2 < 2 \implies s \in A$
			\end{proof}
		\end{tcolorbox}
		\item As you know, in calculus, the sequence $\sett{1,1.4,1.41,1.414,1.4142,\cdots}$ does not converge in $\mathrm{Q}$, but it converges to $\sqrt{2}$ in $\mathrm{R}$
	\end{enumerate}
\end{ex*}

\newpage

\subsection{Order Sets}

\begin{defn}$ $


	Let $X$ be a nonempty set $A$, relation on $X$ is a subset $\mathrm{R}$ of $X \times X = \sett{(x,y)~|~x,y \in X}$
	
	Let $\mathrm{R}$ be a relation on $X$, if $(x,y) \in \mathrm{R}$, then we say that $x$ is retaliated to $y$, and is written as $xRy(x\sim y)$
\end{defn}

\begin{defn}
	An ordered set on $S$, is a relation denoted by $"<"$ on $S$, satisfy:
	
	\begin{enumerate}
		\item[(i)] The low of trichonomy
		
		Given $x,y \in S$, one and only one of the following holds: $x<y,x=y,y<x$
		
		\item[(ii)] Transitivity: if $x < y \& y < z,$ than $x<z$
	\end{enumerate} 
	
	\textbf{Notation}
	
	\begin{enumerate}
		\item[(1)] $x<y$ means "$x$ is less than $y$" or "$x$ is smaller than $y$"
		\item[(2)] $y > x$ means $x<y$
		\item[(3)] $x \leq y$ means $x < y$ or $x = y$, i.e. the negative of $x > y$
	\end{enumerate}
\end{defn}


\begin{defn}
	Let "$<$" lie an order on a set $S$, then pass $(S,<)$ on simply $S$ is called an ordered set
\end{defn}

\begin{defn}
	Let $S$ is an ordered set $\&~ E \subseteq S(E \neq \emptyset)$
	
	\begin{enumerate}
		\item[$\bullet$] $E$ is bounded above if $\exists~ \alpha \in S \implies x \leq \alpha~\forall ~x \in E$
		
		such $\alpha$ is called an upper bound of $E$
		 
		\item[$\bullet$] $E$ is bounded below if $\exists~ \beta \in S \ni \beta \leq x, \forall~x\in E$, such $\beta$ is called a lower bdd of $E$
		\item[$\bullet$] $E$ is bdd is $E$ is both bdd above and below. 
	\end{enumerate}
\end{defn}

\begin{defn}
	Let $S$ be an ordered set and $E \subseteq S (E \neq \emptyset)$ bdd above. An element $\alpha \in S$ is called the last upper bound or supremum of $E$ if
	
	\begin{enumerate}
		\item[(i)] $\alpha$ is an upper bound of $E$
		\item[(ii)] $\alpha$ is the smallest such one. 
	\end{enumerate}
	
	Equivalently, 
	
	\begin{enumerate}
		\item[(i')] $x \leq \alpha, \forall x \in E$
		\item[(ii')] if $\beta < \alpha$, then $\beta$ is not an upper bdd of $E$, i.e. $\exists~ x \in E \ni x > \beta$
	\end{enumerate}
	
	Such $\alpha$(if exists) is denoted by
	
	$$\alpha = \text{sup}(E)$$
	
	similarly, one can defined the greatest lower bdd of infimum of $E$
\end{defn}

\newpage

\begin{rmk*}
	if $\sup (E)$ exists then it is unique
	
	suppose $\alpha \neq \alpha'$ both lub of $E$
	
	$\because$ by trichotomy ,$\alpha > \alpha'$ or $\alpha = \alpha'$ or $\alpha < \alpha' (\rightarrow\leftarrow)$
\end{rmk*}

\begin{defn}
	A ordered set $S$ is said to have the least upper bdd property if $E \subseteq S,~E \neq \emptyset$ and $E$ is bdd above, then $\sup (E)$ exists in $S$
\end{defn}

\begin{ex*}$ $
	\begin{enumerate}
		\item In $\mathrm{Q}$ with the normal ordining
		 
		$A = \sett{r \in \mathrm{Q}~|~r>0,~r^2<2} \& B = \sett{r\in \mathrm{Q}~|~r>0,~r^2>2}$
		
		Then $A$ is bdd above, in fact, bdd by every element in $B$, but $\sup (A)$ does not exist in $\mathrm{Q}$($\because$ by Ex1.5)
		\item $B$ is bdd below by every element of $A$ and $\inf B$ does not exists
		\item Note that $\sup (E) \& \inf (E)$ may not in $E$ even if exist
	\end{enumerate}
\end{ex*}

\begin{rmk*}$ $
	\begin{enumerate}
		\item By the Example above, $\mathrm{Q}$ with the usual ordering has no l.u.b property
		\item In $1.5$ we will explain that $\mathrm{R}$ with usual ordering has the l.u.b. property.
		However, we usually adopt the follwing
	\end{enumerate}
\end{rmk*}

\textbf{The Axiom of Completence or Least upper bdd property}:

Every nonempty subset $E$ of $\mathrm{R}$ which is bdd above has l.u.b

\begin{thm*}
	Let $S$ is an ordered set if $S$ has the l.u.b. property, then $S$ has the g.l.b. property, i.e. if $\emptyset \neq B \subseteq S$ is bdd below, then $\inf (B)$ exists in $S$ 
\end{thm*}

\begin{proof}
	Given $B(\neq \emptyset) \subseteq S$ which is bdd below
	
	Let $L = \sett{a \in S ~|~ a \text{ is a lower bdd of }B}$
	
	\begin{enumerate}
		\item[$\bullet$] $L \neq \emptyset(\because B\text{ is bdd below})$
		\item[$\bullet$] $L$ is bdd above (in fact, every element in $B$ is on upper bound of $L$)
		
		$\implies \forall a \in L \implies a \leq x ,~\forall x \in B \implies x$ is an upper bound of $L$ 
		\item[$\bullet$] $\sup (L) = \alpha$ exists by assumption
		
	\end{enumerate}
	
	\textbf{Claim} $\alpha = \inf B$
	
	\begin{enumerate}
		\item[(i)] $\alpha$ is a lower bdd of $B$, i.e. $\alpha \leq x,~\forall x \in B$
		
		By $\alpha = \sup L$, if $r < \alpha$, them $r$ is not an upper bdd of $L$($\because \alpha$ is the smallest one).Hence,$r \notin B$($\because$ every element of $B$ is an upper bdd of $L$), so $\alpha \leq x, \forall x \in B$ 
		
		We have proved ($r<\alpha \implies r \notin B$) $\implies$ ($r\in B \implies r \geq \alpha$)
		\item[(ii)] $\alpha$ is the greated one
		
		if $\alpha < \beta$ and $\beta$ is a lower bdd of $B$, then $\beta \notin L$, i.e. $\beta$ is not a lower bdd of $B$, so $\alpha$ is the greatest one. Therefore, $\alpha = \inf (B)$
	\end{enumerate}
	
\end{proof}

\newpage

\begin{rmk*}
	Let $E(\neq \emptyset) \subseteq \mathrm{R}$ be bdd below, then $\inf (E)$ exists and $\inf(E) = -\sup(-E)$, where $-E = \sett{-x ~|~x \in E}$
\end{rmk*}

\subsection{Field} $ $
 
Recall the addition $\&$ multiplication in $\mathrm{R}$

$+ : \mathrm{R} \times \mathrm{R} \rightarrow \mathrm{R}((a,b)\mapsto a+b)$

$\times : \mathrm{R} \times \mathrm{R} \rightarrow \mathrm{R}((a,b)\mapsto a \cdot b = ab)$

\begin{defn}
	Let $X$ are a nonempty set $A$, binary operation on $X$ is a function, $o:X\times X \rightarrow X$
\end{defn}


\begin{defn}
	Let $\mathrm{F}$ be a nonempty set, we say that $\mathrm{F}$ is a field ($(F,+,\cdot)$ is a field) if there are two binary operator called addition $"+"$ and multiplication $"\cdot"$ on $\mathrm{F}$ property
	
	\textbf{Axioms for $"+"$}
	\begin{enumerate}
		\item[(A1)] Commutative: $\forall x,y \in \mathrm{F},~x+y=y+x$
		\item[(A2)] Associative: $\forall x,y,z \in F, (x+y)+z = x+(y+z)$
		\item[(A3)] Additive identity or zero element: $\exists~ 0 \in F \implies x + 0 = 0 + x = x ,~\forall x \in \mathrm{F}$
		\item[(A4)] Additive inverse on negative: For each $x \in X,~ \exists~-x \in F \implies x + (-x) = (-x) + x = 0$
	\end{enumerate}
	i.e. $(\mathrm{F},+)$ is an abelian group
	\textbf{Axioms for multiplication}
	\begin{enumerate}
		\item[(M1)] Commutative: $\forall x,y \in \mathrm{F},~xy=yx$
		\item[(M2)] Associative: $\forall x,y,z \in \mathrm{F},~(xy)z = x(yz)$
		\item[(M3)] Muti identity: $\exists 1 \neq 0$ in $\mathrm{F} \ni x1 = 1x = x$
		\item[(M4)] Multiplicative inverse: For each $x \neq 0, \exists x^{-1} \in \mathrm{F} \implies xx^{-1} = x^{-1}x = 1$
	\end{enumerate}
	i.e. $(F = F\cdot \sett{0},\cdot)$ is an abelian group
	
	\textbf{Distributive Law}
	\begin{enumerate}
		\item[(D1)] $\forall x,y,z \in \mathrm{F},~(x,y)z = xz+yz ~\&~ x(y+z) = xy+xz$
	\end{enumerate}
	
	
\end{defn}

\newpage

let $(\mathrm{F},+,\cdot)$ be a field, we list a series of basic identity as you learn in high school in the real number system

\begin{enumerate}
	\item[(a)] Cancellation law for $"+":x+y = x+z \implies y = z$
	
	$\because x+y = x+z \implies (-x)+(x+y)=(-x)+(x+z) \implies ((-x)+x)+y = ((-x)+x)+z$
	
	$\implies 0+y = 0+z \implies y=z$
	
	\item[(b)] $0$ is $"1"$
	
	suppose $0' \in \mathrm{F}$ is another element satisfy $A_3$, then $0 = 0+0'=0'$
	\item[(c)] $x+y = x \implies y = 0$ by (a) $\because x+y = x+0 \implies y=0$
	
	\item[(d)] negative $-x$ of $x$ is $"1"$
	
	if $x' \in F$, is another negative of $x$, them $x+x' = x'+x = 0$
	
	From $x + x' = 0 \implies (-x)+(x + x')=-x+0=-x$
	
	\item[(e)] $x+y = 0 \implies y = -x$
	
	$x +y=0 \implies (-x)+(x+y) = (-x)+0 \implies ((-x)+x)+y = -x $
	
	$\implies 0+y = -x \implies y = -x$
	
	\item[(f)] $-(-x) = x $
	
	$-(-x)+(-x) = 0,$ By (d) $x = -(x)$
	
	\item[(a')]cancellation law
	
	if $x \neq 0$, then $xy = xz \implies y = z,~\because (x^{-1})(xy) = (x^{-1})(xz)$
	
	$\implies (x^{-1})(xy) = (x^{-1}x)z \implies 1y = 1z \implies y = z$
	\item[(b')]$1$ is $"1"$
	
	if $1'$ is another identity, then $1 = 11' = 1'$
	
	\item[(c')] $x \neq 0 ~\&~ xy = x \implies y = 1$
	
	$xy = x1 \implies y = 1$
	
	\item[(d')]For $x \neq 0$ in $\mathrm{F}, x^{-1}$ is $"1"$
	
	if $x$ is another one, i.e. $x'x = xx' = 1 \implies (x^{-1})(xx') = (x^{-1})1 = x^{-1}$
	
	\item[(f')]$x \neq 0 \implies (x^{-1})^{-1} = x$
	
	$(x^{-1})^{-1}(x^{-1}) = 1 \implies x = (x^{-1})^{-1}$
	
	\item[(g')]$0x = x0 = 0$
	
	$(0+0)x = 0x+0x \implies 0x = 0$
	
	\item[(h')] $x \neq 0 ~\&~ y\neq 0 \implies xy \neq 0,$ equivalently $xy =0 \implies x=0$ or $y = 0$
	
	$\because xy = 0$ then $(x^{-1})(xy) = ((x^{-1})x)y=1y=y (\rightarrow\leftarrow)$
	
	\item[(i')] $(-x)y = -(xy) = x(-y)$
	
	$\because [(-x)+x]y = 0y = 0 = (-x)y = -(xy) \implies (-x)y = -(xy)$
	
	\item[(j')] $(-x)(-y) = xy$
	
	$\because(-x)(-y) = -(x(-y))$ by (i)
	
	$= -(-(xy)) = xy$
	\item[(k)] $-x = (-1)x$
	
	$\because (1-1)x = 0x = 0 = 1x + (-1)x = x + (-1)x \implies (-1)x = -x$
\end{enumerate}


\begin{defn}[Order Field]
	Let $\mathrm{F}$ is a field, we say that $\mathrm{F}$ is an order field if there is an ordering $"<"$ satisfying
	
	\begin{enumerate}
		\item[(1)] if $x<y$, then $x+z < y+z ,~\forall z \in F$
		\item[(2)] if $x>y$ and $y>0$, then $xy >0$ 
	\end{enumerate}
\end{defn}

\begin{ex*}
	$\mathrm{Q}$ and $\mathrm{R}$ are order field under the usual ordering
	
	Some basic properties of ordered field, let $\mathrm{F}$ be an ordered field with ordering $"<"$
	
	\begin{enumerate}
		\item[(a)] $x > 0 \implies -x <0$
		
		$\because x > 0 \implies x + (-x) > 0 + (-x) \implies 0 > -x$
		
		\item[(b)] $x>y \Leftrightarrow x-y >0$
		
		$\because x > y \implies x + (-y)>y = (-y) \implies x-y >0$
		
		$x-y > 0 \implies x-y+y > y \implies x + 0 > y \implies x > y$
		\item[(c)] $x >0 $ and $y<z \implies xy < xz$
		
		$\because x > 0 $ and $y < z \implies x>0$ and $z-y <0 \implies x(z-y)>0 \implies xz+x(-y) >0$
		
		$\implies xz - xy > 0 \implies xz > xy$
		
		\item[(d)] $x < 0 $ and $y < z \implies xy > xz$
		
		$\because x<0$ and $y<z \implies -x > 0 $ and $z-y > 0 \implies (-x)(z - y)> 0\implies -xz+xy > 0$
		
		$\implies xy > xz$ 
		
		\item[(e)] $\forall x \neq 0$ in $\mathrm{F}, x^2 > 0$
		
		$\because x > 0 \implies x \cdot x > x0$ by (c) or
		
		$x<0 \implies -x >0$ by (a) $\implies -x > 0$ by (a) $\implies (-x)^2 > 0 \implies x^2 > 0$ 
		\item[(f)] $1 > 0,~-1<0$
		
		$\because 1 \neq 0 \implies 1^2 > 0$ by (e) $\implies 1 >0$
		\item[(g)] $0<x<y \implies 0 < \frac{1}{y} < \frac{1}{x}$
		
		$\because$ Note that $\forall u \in \mathrm{F},~u>0 \implies \frac{1}{u} = u^{-1} > 0$
		
		$\because$ if $\frac{1}{u}<0$, then $u\cdot \frac{1}{u} < 0$ by (e) $\implies 1 < 0 (\rightarrow\leftarrow)~\therefore~\frac{1}{u}>0$
		
		Now, $\frac{1}{x},\frac{1}{y}>0$ from $x<y$ we get $(\frac{1}{x}\cdot\frac{1}{y})x < (\frac{1}{x}\cdot \frac{1}{y})y \implies 0 < \frac{1}{y} <\frac{1}{x}$
	\end{enumerate}  
\end{ex*}

\begin{rmk*}
	By (e)(f), we conclude that $\mathrm{C}$ is not an ordered field
	
	$\because \mathrm{C}$ were an ordered field, then by (e), $i^2>0 \implies -1 > 0(\rightarrow\leftarrow)$
	
	$\therefore \mathrm{C}$ is not an order field
\end{rmk*}

\subsection{The Real Number Field $\mathrm{R}$}

\begin{thm*}
	There exists an ordered field $\mathrm{R}$ containing $\mathrm{Q}$ which has the l.u.b. property. Moreover, such $\mathrm{R}$ is unique up to order-isomorphism
	
	i.e. if $"<"$ and $"<'"$ are two orders on $\mathrm{R}$, them $\exists f_i(\mathrm{R},<) \rightarrow (\mathrm{R},<') \implies$
	
	\begin{enumerate}
		\item[(i)] $f$ is a field isomorphism, i.e. $\forall a,b \in \mathrm{R},~f(a+b)=f(a)+f(b),~f(ab) = f(a)f(b),~f(1)=1$
		\item[(ii)] $f$ preserves ordering, $a<b \implies f(a) < f(b)$ 
	\end{enumerate}
	
	Such $\mathrm{R}$ is called the real number field or real number system or real line
\end{thm*}

\begin{thm*}$ $
	\begin{enumerate}
		\item[(a)] The Archimedean  property of $\mathrm{R}:$ Given $x,y \in \mathrm{R}$ with $x>0,~\exists~n \in \mathrm{N} \implies nx>y$
		\item[(b)] $\mathrm{Q}$ is dense in $\mathrm{R}: \forall x,y\in \mathrm{R}$ with $x \leq y,~\exists ~r \in \mathrm{Q} \implies x < r < y$
	\end{enumerate}
\end{thm*}

\begin{proof} $ $
	\begin{enumerate}
		\item[(a)] Let $A = \sett{nx ~|~n\in \mathrm{N}} \subseteq \mathrm{R}$
		
		if (a) were false, them $A$ is bdd above by $y$, since $\mathrm{R}$ has the l.u.b property
		
		$\alpha = \sup A$ exists in $\mathrm{R}$, since $x>0,~\alpha - x < \alpha \implies \alpha - x$ is not an upper bdd of $A$
		
		$\implies \exists m \in \mathrm{N} \ni mx > \alpha - x \implies (m+1)x > \alpha (\rightarrow\leftarrow)$
		\item[(b)] Since $x<y,~y-x>0$, by (a),$\exists n \in \mathrm{N} \implies n(y-x)>1$
		
		By (a) again, $\exists m_1,m_2 \in \mathrm{N} \implies m_1=m_11>n_x ~\&~ m_2=m_2\cdot 1 > -nx$
		
		we have $-m_2 < nx < m_1$, choose $m \in \mathrm{Z} \implies -m_2 \leq m \leq m_1 ~\&~ m-1 \leq nx < m$
		
		(in fact, $m = [nx]+1$,where $[z]$ in the greatest integer of $z$)
		
		we have $nx < m < 1+nx < ny(\because n(y-x)>1) \implies x < \frac{m}{n} < y$
		
		Let $r = \frac{m}{n} \in \mathrm{Q}$, then $x<r<y$ 
	\end{enumerate}
\end{proof}

\newpage

An application of the density property of $\mathrm{Q}$ in $\mathrm{R}:$

Given $x \in \mathrm{R} -\mathrm{Q}$ i.e. $x$ is an irrational numbers, i.e. $\forall~ \epsilon > 0, \exists r \in \mathrm{Q} \implies |x-r|<\epsilon$

equivalently, $\exists$ a sequence $\sett{r_n}$ in $\mathrm{Q} \implies r_n \rightarrow x$

In fact, one may choose $\sett{r_n}$ to $\uparrow$ or $\downarrow$

$\because \forall n \geq 1,~\exists~r_n \in \mathrm{Q} \implies x < r_n < \frac{1}{n} + x$ by Thm.1.3(b) By squeezing lemma, $r_n \rightarrow x$ on $n \rightarrow \infty$



\begin{thm*}[existence of $n$th root]
	Given $x \in \mathrm{T},x>0 ~\&~ n \in \mathrm{N},\exists~"1" y > 0 \implies y^n = x$
	
	Such $y$ is called the $n$th root of $x$ $~\&~$ denoted by $y = \sqrt[n]{x} = x^{\frac{1}{n}}$
\end{thm*}

\begin{proof}
	$"1"$. Suppose $y_1,y_2 > 0 \implies y_1^n = x ~\&~ y_2^n = x$
	
	Bt trichotomy,  we have 
	
	\begin{enumerate}
		\item[(i)]  $0<y_1<y_2 \implies y_1^n < y_2^n (\rightarrow\leftarrow)$
		\item[(ii)] $0 < y_2 < y_1 \implies y_2^n < y_1^n (\rightarrow\leftarrow)$
		\item[(iii)] $y_1 = y_2$
	\end{enumerate}
	
	$"\exists"$. Let $E = \sett{t \in \mathrm{R}~|~t^n < x}$
	
	Claim:
	
	\begin{enumerate}
		\item[$\bullet$] $E \neq \emptyset,$ Let $t = \dfrac{x}{1+x}$, then $0<t<1$, hence $t^n < t < x,~ \therefore t \in E ~\&~ E \neq \emptyset$
		\item[$\bullet$] $E$ is bdd above, in fact $E$ is bdd above by $1+x$ if $t>1+x>1,$ then $t^n > t>x$, so $E$ is bdd above by $1+1$
		
		Therefore $y = \sup E$ exists \& is finite
		\item[$\bullet$] Claim $y > 0$ \& $y^n = x$, clearly, $y>0(\because \frac{x}{1+x} \in E ~\&~\frac{x}{1+x}>0 )$
		
		by trichotomy, we have $y^n<x ,~y^n>x,~y^n=x$
	\end{enumerate}
\end{proof}














\end{document}