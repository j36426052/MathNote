

\section{Some preliminary}

\subsection{Set Theory}$ $

Set is a collection which has two presentation

\begin{enumerate}
	\item List $\{a,b,c,\cdots\}$
	\item $\{x ~|~x \in \text{alphabet}\}$
\end{enumerate}

We will assume that you are familiar with some basic set theory e.g. union, intersection, difference

\subsection{The Number System} $ $ 

$\mathbb{N}=\sett{1,2,3,\cdots}$ the set of all positive integers $n$ natural numbers

$\mathbb{Z}=\sett{\cdots,-2,-1,0,-1,-2,\cdots}$ the set of all integers called the ring of intepus

$\mathbb{Q} = \sett{\dfrac{m}{n}~:~n,m \in \mathbb{Z},n\neq 0}$ the set of all rational numbers %of the national number field on real line

$\mathbb{R}$ the set all of real numbers on the real number field on real line

$\mathbb{C} = \sett{z=a+ib~|~a,b\in\mathbb{R}}$ the set of all complex numbers or the complex number filed on complex plane, where $i = \sqrt{-1}$

and $\N \subseteq \Z \subseteq \Q \subseteq \R \subseteq \C$

\newpage

 \begin{rmk*}$ $
 	\begin{tcolorbox}
 		\begin{enumerate}
 		\item $x + 2 = 0$ no root in $\mathbb{N}$\\$3x - 5 = 0$ no root in $\mathbb{Z}$\\$x^2 + 1 = 0$ no root in $\mathbb{R}$
 		\item One can construct $\mathbb{Q}$ from $\mathbb{Z}$ in algebraic way, called the fraction field of $\mathbb{Z}$
 		\item One can construct $\mathbb{R}$ from $\mathbb{Q}$ in two ways:
 		
 		$\cdot$ Using Dedekind cut which is given in the appendix of Rudin p17-21
 		
 		$\cdot$ Using completion of matrix space
 		\item One can construct $\mathbb{C}$ from in complex analysis
 	\end{enumerate}
 	\end{tcolorbox}
 	
 \end{rmk*}

\begin{ex*} $ $
	\begin{enumerate}
		\item Between any two rational numbers, there is another one
		
		\begin{proof}
			Let $r,s \in \mathbb{Q}$ with $r<s$, then $\dfrac{r+s}{2} \in \mathbb{Q}$ and $r < \dfrac{r+s}{2} < s$
			
			\begin{tcolorbox}
				$r = \dfrac{m_1}{n_1}, s = \dfrac{m_2}{n_2}, \dfrac{r+s}{2} = \dfrac{\frac{m_1}{n_1}+\frac{m_2}{n_2}}{{2}} = \dfrac{m_1n_2 + n_1m_2}{2n_1m_1} 
				\in Q$
				
				$s = \dfrac{s+s}{2} > \dfrac{r + s}{2} > \dfrac{r+r}{2} = r$
			\end{tcolorbox}
		\end{proof}
		
		\item $x^2 = \dfrac{4}{9}$ has exactly two rational solutions, namely, $\pm \frac{2}{3}$
		\item $x^2 = 2$ has exactly two real root, namely, $\pm \sqrt{2}$
		\item Is there any rational roots of $x^2 = 2$? i.e., is $\sqrt{2}$ rational?
		\begin{tcolorbox}
			Suppose $r = \dfrac{m}{n} \in \mathrm{Q}$, is a root of $x^2 = 2$, where $(m,n) = 1$
			
			Then $\dfrac{m^2}{n^2} = 2 \implies m^2 = 2n^2 \implies 2~|~m^2 \implies 2~|~m \implies 4~|~m^2 \implies 4~|~2n^2$
			
			$\implies 2 ~|~ n^2 \implies 2 ~|~ n \implies (n,m) \neq 1$
		\end{tcolorbox}
		\item Let $A = \sett{r \in \mathrm{Q}~|~r > 0 ~\& ~r^2 < 2}, B = \sett{r \in \mathrm{Q}~|~r>0 ~\& ~r^2 > 2}$
		
		Then $A$ contains no largest numbers, i.e. max element $\& ~B$ contains no smallest numbers, i.e. min element
		\begin{tcolorbox}
			\begin{proof}
				$A$ contains no largest numbers $\Leftrightarrow$ given $r \in A$, $\exists s \in A \ni s>r$
				
				Now, given $r \in A$, Let $s = r - \dfrac{r^2 - 2}{r+2} = \dfrac{2r + 2}{r+2} ~(\star_1)$
				
				$\implies s^2 - 2 = \dfrac{2(r^2 - 2)}{(r+2)^2} ~(\star_2)$
				
				Now, $r \in A, r^2 < 2 \implies r^2-2 <0 \therefore $
				
				$(\star_1)\&(\star_2) \implies s > r ~\& ~s^2 < 2 \implies s \in A$
			\end{proof}
		\end{tcolorbox}
		\item As you know, in calculus, the sequence $\sett{1,1.4,1.41,1.414,1.4142,\cdots}$ does not converge in $\mathrm{Q}$, but it converges to $\sqrt{2}$ in $\mathrm{R}$
	\end{enumerate}
\end{ex*}

\subsection{Order Sets}

\begin{defn}[Relation]$ $


	Let $X$ be a nonempty set $A$, relation on $X$ is a subset $\mathrm{R}$ of $X \times X = \sett{(x,y)~|~x,y \in X}$
	
	Let $\mathrm{R}$ be a relation on $X$, if $(x,y) \in \mathrm{R}$, then we say that $x$ is retaliated to $y$, and is written as $xRy(x\sim y)$
\end{defn}

\begin{defn}[Order Set]
	An ordered set on $S$, is a relation denoted by $"<"$ on $S$, satisfy:
	
	\begin{enumerate}
		\item[(i)] The low of trichonomy
		
		Given $x,y \in S$, one and only one of the following holds: $x<y,x=y,y<x$
		
		\item[(ii)] Transitivity: if $x < y \& y < z,$ than $x<z$
	\end{enumerate} 
	
	\begin{tcolorbox}
		\textbf{Notation}
		\begin{enumerate}
			\item[(1)] $x<y$ means "$x$ is less than $y$" or "$x$ is smaller than $y$"
			\item[(2)] $y > x$ means $x<y$
			\item[(3)] $x \leq y$ means $x < y$ or $x = y$, i.e. the negative of $x > y$
		\end{enumerate}
	\end{tcolorbox}
	
	
\end{defn}


%\begin{defn}
%	Let "$<$" lie an order on a set $S$, then pass $(S,<)$ on simply $S$ is called an ordered set
%\end{defn}

\begin{defn}[\color{red}bdd\color{black}]
	Let $S$ is an ordered set $\&~ E \subseteq S(E \neq \emptyset)$
	
	\begin{enumerate}
		\item[$\bullet$] $E$ is bounded above if $\exists~ \alpha \in S \implies x \leq \alpha~\forall ~x \in E$
		
		such $\alpha$ is called an upper bound of $E$
		 
		\item[$\bullet$] $E$ is bounded below if $\exists~ \beta \in S \ni \beta \leq x, \forall~x\in E$, such $\beta$ is called a lower bdd of $E$
		\item[$\bullet$] $E$ is bdd is $E$ is both bdd above and below. 
	\end{enumerate}
\end{defn}

\newpage

\begin{defn}[\color{red}least upper bound\color{black}]
	Let $S$ be an ordered set and $E \subseteq S (E \neq \emptyset)$ bdd above. An element $\alpha \in S$ is called the last upper bound or supremum of $E$ if
	
	\begin{enumerate}
		\item[(i)] $\alpha$ is an upper bound of $E$
		\item[(ii)] $\alpha$ is the smallest such one. 
	\end{enumerate}
	
	Equivalently, 
	
	\begin{enumerate}
		\item[(i')] $x \leq \alpha, \forall x \in E$
		\item[(ii')] if $\beta < \alpha$, then $\beta$ is not an upper bdd of $E$, i.e. $\exists~ x \in E \ni x > \beta$
	\end{enumerate}
	
	Such $\alpha$(if exists) is denoted by
	
	$$\alpha = \text{sup}(E)$$
	
	similarly, one can defined the greatest lower bdd of infimum of $E$
\end{defn}

\begin{rmk*}
	if $\sup (E)$ exists then it is unique
	
	suppose $\alpha \neq \alpha'$ both lub of $E$
	
	$\because$ by trichotomy ,$\alpha > \alpha'$ or $\alpha = \alpha'$ or $\alpha < \alpha' (\rightarrow\leftarrow)$
\end{rmk*}

\begin{defn}[least upper bdd property]
	A ordered set $S$ is said to have the least upper bdd property if $E \subseteq S,~E \neq \emptyset$ and $E$ is bdd above, then $\sup (E)$ exists in $S$
\end{defn}

\begin{ex*}$ $
	\begin{enumerate}
		\item In $\mathrm{Q}$ with the normal ordining
		 
		$A = \sett{r \in \mathrm{Q}~|~r>0,~r^2<2} \& B = \sett{r\in \mathrm{Q}~|~r>0,~r^2>2}$
		
		Then $A$ is bdd above, in fact, bdd by every element in $B$, but $\sup (A)$ does not exist in $\mathrm{Q}$($\because$ by Ex1.5)
		\item $B$ is bdd below by every element of $A$ and $\inf B$ does not exists
		\item Note that $\sup (E) \& \inf (E)$ may not in $E$ even if exist
	\end{enumerate}
\end{ex*}

\begin{rmk*}$ $
	\begin{enumerate}
		\item By the Example above, $\mathrm{Q}$ with the usual ordering has no l.u.b property
		\item In $1.5$ we will explain that $\mathrm{R}$ with usual ordering has the l.u.b. property.
		However, we usually adopt the follwing
	\end{enumerate}
\end{rmk*}



\textbf{The Axiom of Completence or Least upper bdd property}:

Every nonempty subset $E$ of $\mathrm{R}$ which is bdd above has l.u.b

\begin{thm*}[l.u.b.p. $\rightarrow$ g.l.b.p.]
	Let $S$ is an ordered set if $S$ has the l.u.b. property, then $S$ has the g.l.b. property, i.e. if $\emptyset \neq B \subseteq S$ is bdd below, then $\inf (B)$ exists in $S$ 
\end{thm*}

\begin{proof}$\color{red}(\star)$

	Given $B(\neq \emptyset) \subseteq S$ which is bdd below
	
	Let $L = \sett{a \in S ~|~ a \text{ is a lower bdd of }B}$
	
	\begin{enumerate}
		\item[$\bullet$] $L \neq \emptyset(\because B\text{ is bdd below})$
		\item[$\bullet$] $L$ is bdd above (in fact, every element in $B$ is on upper bound of $L$)
		
		$\implies \forall a \in L \implies a \leq x ,~\forall x \in B \implies x$ is an upper bound of $L$ 
		\item[$\bullet$] $\sup (L) = \alpha$ exists by assumption
		
	\end{enumerate}
	
	\textbf{Claim} $\alpha = \inf B$
	
	\begin{enumerate}
		\item[(i)] $\alpha$ is a lower bdd of $B$, i.e. $\alpha \leq x,~\forall x \in B$
		
		By $\alpha = \sup L$, if $r < \alpha$, them $r$ is not an upper bdd of $L$($\because \alpha$ is the smallest one).Hence,$r \notin B$($\because$ every element of $B$ is an upper bdd of $L$), so $\alpha \leq x, \forall x \in B$ 
		
		We have proved ($r<\alpha \implies r \notin B$) $\implies$ ($r\in B \implies r \geq \alpha$)
		\item[(ii)] $\alpha$ is the greated one
		
		if $\alpha < \beta$ and $\beta$ is a lower bdd of $B$, then $\beta \notin L$, i.e. $\beta$ is not a lower bdd of $B$, so $\alpha$ is the greatest one. Therefore, $\alpha = \inf (B)$
	\end{enumerate}
	
\end{proof}

\begin{rmk*}
	Let $E(\neq \emptyset) \subseteq \mathrm{R}$ be bdd below, then $\inf (E)$ exists and $\inf(E) = -\sup(-E)$, where $-E = \sett{-x ~|~x \in E}$
\end{rmk*}

\subsection{Field} $ $
 
Recall the addition $\&$ multiplication in $\mathrm{R}$

$+ : \mathrm{R} \times \mathrm{R} \rightarrow \mathrm{R}((a,b)\mapsto a+b)$

$\times : \mathrm{R} \times \mathrm{R} \rightarrow \mathrm{R}((a,b)\mapsto a \cdot b = ab)$

\begin{defn}
	Let $X$ is a nonempty set $A$, binary operation on $X$ is a function, $o:X\times X \rightarrow X$
\end{defn}


\begin{defn}
	Let $\mathrm{F}$ be a nonempty set, we say that $\mathrm{F}$ is a field ($(F,+,\cdot)$ is a field) if there are two binary operator called addition $"+"$ and multiplication $"\cdot"$ on $\mathrm{F}$ property
	
	\begin{tcolorbox}
		\textbf{Axioms for $"+"$}
		\begin{enumerate}
			\item[(A1)] Commutative: $\forall x,y \in \mathrm{F},~x+y=y+x$
			\item[(A2)] Associative: $\forall x,y,z \in F, (x+y)+z = x+(y+z)$
			\item[(A3)] Additive identity or zero element: $\exists~ 0 \in F \implies x + 0 = 0 + x = x ,~\forall x \in \mathrm{F}$
			\item[(A4)] Additive inverse on negative: For each $x \in X,~ \exists~-x \in F \implies x + (-x) = (-x) + x = 0$
		\end{enumerate}
		i.e. $(\mathrm{F},+)$ is an abelian group
		\textbf{Axioms for multiplication}
		\begin{enumerate}
			\item[(M1)] Commutative: $\forall x,y \in \mathrm{F},~xy=yx$
			\item[(M2)] Associative: $\forall x,y,z \in \mathrm{F},~(xy)z = x(yz)$
			\item[(M3)] Muti identity: $\exists 1 \neq 0$ in $\mathrm{F} \ni x1 = 1x = x$
			\item[(M4)] Multiplicative inverse: For each $x \neq 0, \exists x^{-1} \in \mathrm{F} \implies xx^{-1} = x^{-1}x = 1$
		\end{enumerate}
		i.e. $(F = F\cdot \sett{0},\cdot)$ is an abelian group
	
		\textbf{Distributive Law}
		\begin{enumerate}
			\item[(D1)] $\forall x,y,z \in \mathrm{F},~(x,y)z = xz+yz ~\&~ x(y+z) = xy+xz$
		\end{enumerate}
	\end{tcolorbox}
	
	
	
	
\end{defn}

\textbf{Induction from Axioms}

let $(\mathrm{F},+,\cdot)$ be a field, we list a series of basic identity as you learn in high school in the real number system

\newpage

\begin{enumerate}
	\item[(a)] Cancellation law for $"+":x+y = x+z \implies y = z$
	
	$\because x+y = x+z \implies (-x)+(x+y)=(-x)+(x+z) \implies ((-x)+x)+y = ((-x)+x)+z$
	
	$\implies 0+y = 0+z \implies y=z$
	
	\item[(b)] $0$ is $"1"$
	
	suppose $0' \in \mathrm{F}$ is another element satisfy $A_3$, then $0 = 0+0'=0'$
	\item[(c)] $x+y = x \implies y = 0$ by (a) $\because x+y = x+0 \implies y=0$
	
	\item[(d)] negative $-x$ of $x$ is $"1"$
	
	if $x' \in F$, is another negative of $x$, them $x+x' = x'+x = 0$
	
	From $x + x' = 0 \implies (-x)+(x + x')=-x+0=-x$
	
	\item[(e)] $x+y = 0 \implies y = -x$
	
	$x +y=0 \implies (-x)+(x+y) = (-x)+0 \implies ((-x)+x)+y = -x $
	
	$\implies 0+y = -x \implies y = -x$
	
	\item[(f)] $-(-x) = x $
	
	$-(-x)+(-x) = 0,$ By (d) $x = -(x)$
	
	\item[(a')]cancellation law
	
	if $x \neq 0$, then $xy = xz \implies y = z,~\because (x^{-1})(xy) = (x^{-1})(xz)$
	
	$\implies (x^{-1})(xy) = (x^{-1}x)z \implies 1y = 1z \implies y = z$
	\item[(b')]$1$ is $"1"$
	
	if $1'$ is another identity, then $1 = 11' = 1'$
	
	\item[(c')] $x \neq 0 ~\&~ xy = x \implies y = 1$
	
	$xy = x1 \implies y = 1$
	
	\item[(d')]For $x \neq 0$ in $\mathrm{F}, x^{-1}$ is $"1"$
	
	if $x$ is another one, i.e. $x'x = xx' = 1 \implies (x^{-1})(xx') = (x^{-1})1 = x^{-1}$
	
	\item[(f')]$x \neq 0 \implies (x^{-1})^{-1} = x$
	
	$(x^{-1})^{-1}(x^{-1}) = 1 \implies x = (x^{-1})^{-1}$
	
	\item[(g')]$0x = x0 = 0$
	
	$(0+0)x = 0x+0x \implies 0x = 0$
	
	\item[(h')] $x \neq 0 ~\&~ y\neq 0 \implies xy \neq 0,$ equivalently $xy =0 \implies x=0$ or $y = 0$
	
	$\because xy = 0$ then $(x^{-1})(xy) = ((x^{-1})x)y=1y=y (\rightarrow\leftarrow)$
	
	\item[(i')] $(-x)y = -(xy) = x(-y)$
	
	$\because [(-x)+x]y = 0y = 0 = (-x)y = -(xy) \implies (-x)y = -(xy)$
	
	\item[(j')] $(-x)(-y) = xy$
	
	$\because(-x)(-y) = -(x(-y))$ by (i)
	
	$= -(-(xy)) = xy$\\
	\item[(k)] $-x = (-1)x$
	
	$\because (1-1)x = 0x = 0 = 1x + (-1)x = x + (-1)x \implies (-1)x = -x$
\end{enumerate}


\begin{defn}[\color{red}Order Field\color{black}]
	Let $\mathrm{F}$ is a field, we say that $\mathrm{F}$ is an order field if there is an ordering $"<"$ satisfying
	
	\begin{enumerate}
		\item[(1)] if $x<y$, then $x+z < y+z ,~\forall z \in F$
		\item[(2)] if $x>y$ and $y>0$, then $xy >0$ 
	\end{enumerate}
\end{defn}

\begin{ex*}
	$\mathrm{Q}$ and $\mathrm{R}$ are order field under the usual ordering
	
	Some basic properties of ordered field, let $\mathrm{F}$ be an ordered field with ordering $"<"$
	
	\begin{enumerate}
		\item[(a)] $x > 0 \implies -x <0$
		
		$\because x > 0 \implies x + (-x) > 0 + (-x) \implies 0 > -x$
		
		\item[(b)] $x>y \Leftrightarrow x-y >0$
		
		$\because x > y \implies x + (-y)>y = (-y) \implies x-y >0$
		
		$x-y > 0 \implies x-y+y > y \implies x + 0 > y \implies x > y$
		\item[(c)] $x >0 $ and $y<z \implies xy < xz$
		
		$\because x > 0 $ and $y < z \implies x>0$ and $z-y <0 \implies x(z-y)>0 \implies xz+x(-y) >0$
		
		$\implies xz - xy > 0 \implies xz > xy$
		
		\item[(d)] $x < 0 $ and $y < z \implies xy > xz$
		
		$\because x<0$ and $y<z \implies -x > 0 $ and $z-y > 0 \implies (-x)(z - y)> 0\implies -xz+xy > 0$
		
		$\implies xy > xz$ 
		
		\item[(e)] $\forall x \neq 0$ in $\mathrm{F}, x^2 > 0$
		
		$\because x > 0 \implies x \cdot x > x0$ by (c) or
		
		$x<0 \implies -x >0$ by (a) $\implies -x > 0$ by (a) $\implies (-x)^2 > 0 \implies x^2 > 0$ 
		\item[(f)] $1 > 0,~-1<0$
		
		$\because 1 \neq 0 \implies 1^2 > 0$ by (e) $\implies 1 >0$
		\item[(g)] $0<x<y \implies 0 < \frac{1}{y} < \frac{1}{x}$
		
		$\because$ Note that $\forall u \in \mathrm{F},~u>0 \implies \frac{1}{u} = u^{-1} > 0$
		
		$\because$ if $\frac{1}{u}<0$, then $u\cdot \frac{1}{u} < 0$ by (e) $\implies 1 < 0 (\rightarrow\leftarrow)~\therefore~\frac{1}{u}>0$
		
		Now, $\frac{1}{x},\frac{1}{y}>0$ from $x<y$ we get $(\frac{1}{x}\cdot\frac{1}{y})x < (\frac{1}{x}\cdot \frac{1}{y})y \implies 0 < \frac{1}{y} <\frac{1}{x}$
	\end{enumerate}  
\end{ex*}

\newpage

\begin{rmk*}
	By (e)(f), we conclude that $\mathrm{C}$ is not an ordered field
	
	$\because \mathrm{C}$ were an ordered field, then by (e), $i^2>0 \implies -1 > 0(\rightarrow\leftarrow)$
	
	$\therefore \mathrm{C}$ is not an order field
\end{rmk*}


\subsection{The Real Number Field $\mathrm{R}$}

\begin{thm*}
	There exists an ordered field $\mathrm{R}$ containing $\mathrm{Q}$ which has the l.u.b. property. Moreover, such $\mathrm{R}$ is unique up to order-isomorphism
	
	i.e. if $"<"$ and $"<'"$ are two orders on $\mathrm{R}$, them $\exists f_i(\mathrm{R},<) \rightarrow (\mathrm{R},<') \implies$
	
	\begin{enumerate}
		\item[(i)] $f$ is a field isomorphism,
		
		 i.e. $\forall a,b \in \mathrm{R},~f(a+b)=f(a)+f(b),~f(ab) = f(a)f(b),~f(1)=1$
		\item[(ii)] $f$ preserves ordering, $a<b \implies f(a) < f(b)$ 
	\end{enumerate}
	
	Such $\mathrm{R}$ is called the real number field or real number system or real line
\end{thm*}

\begin{thm*}$ $
	\begin{enumerate}
		\item[\color{red}(a)] The Archimedean  property of $\mathrm{R}:$ Given $x,y \in \mathrm{R}$ with $x>0,~\exists~n \in \mathrm{N} \implies nx>y$
		\item[(b)] $\mathrm{Q}$ is dense in $\mathrm{R}: \forall x,y\in \mathrm{R}$ with $x \leq y,~\exists ~r \in \mathrm{Q} \implies x < r < y$
	\end{enumerate}
\end{thm*}

\begin{proof} $ $
	\begin{enumerate}
		\item[\color{red}(a)] Let $A = \sett{nx ~|~n\in \mathrm{N}} \subseteq \mathrm{R}$
		
		if (a) were false, them $A$ is bdd above by $y$, since $\mathrm{R}$ has the l.u.b property
		
		$\alpha = \sup A$ exists in $\mathrm{R}$, since $x>0,~\alpha - x < \alpha \implies \alpha - x$ is not an upper bdd of $A$
		
		$\implies \exists m \in \mathrm{N} \ni mx > \alpha - x \implies (m+1)x > \alpha (\rightarrow\leftarrow)$
		\item[\color{blue}(b)] Since $x<y,~y-x>0$, by (a),$\exists n \in \mathrm{N} \implies n(y-x)>1$
		
		By (a) again, $\exists m_1,m_2 \in \mathrm{N} \implies m_1=m_11>n_x ~\&~ m_2=m_2\cdot 1 > -nx$
		
		we have $-m_2 < nx < m_1$, choose $m \in \mathrm{Z} \implies -m_2 \leq m \leq m_1 ~\&~ m-1 \leq nx < m$
		
		(in fact, $m = [nx]+1$,where $[z]$ in the greatest integer of $z$)
		
		we have $nx < m < 1+nx < ny(\because n(y-x)>1) \implies x < \frac{m}{n} < y$
		
		Let $r = \frac{m}{n} \in \mathrm{Q}$, then $x<r<y$ 
	\end{enumerate}
\end{proof}

\newpage

An application of the density property of $\mathrm{Q}$ in $\mathrm{R}:$

Given $x \in \mathrm{R} -\mathrm{Q}$ i.e. $x$ is an irrational numbers, i.e. $\forall~ \epsilon > 0, \exists r \in \mathrm{Q} \implies |x-r|<\epsilon$

equivalently, $\exists$ a sequence $\sett{r_n}$ in $\mathrm{Q} \implies r_n \rightarrow x$

In fact, one may choose $\sett{r_n}$ to $\uparrow$ or $\downarrow$

$\because \forall n \geq 1,~\exists~r_n \in \mathrm{Q} \implies x < r_n < \frac{1}{n} + x$ by Thm.1.3(b) By squeezing lemma, $r_n \rightarrow x$ on $n \rightarrow \infty$



\begin{thm*}[existence of $n$th root]
	Given $x \in \mathrm{T},x>0 ~\&~ n \in \mathrm{N},\exists~"1" y > 0 \implies y^n = x$
	
	Such $y$ is called the $n$th root of $x$ $~\&~$ denoted by $y = \sqrt[n]{x} = x^{\frac{1}{n}}$
\end{thm*}

\begin{proof}
	\textbf{\color{blue} not important}
	
	$"1"$. Suppose $y_1,y_2 > 0 \implies y_1^n = x ~\&~ y_2^n = x$
	
	Bt trichotomy,  we have 
	
	\begin{enumerate}
		\item[(i)]  $0<y_1<y_2 \implies y_1^n < y_2^n (\rightarrow\leftarrow)$
		\item[(ii)] $0 < y_2 < y_1 \implies y_2^n < y_1^n (\rightarrow\leftarrow)$
		\item[(iii)] $y_1 = y_2$
	\end{enumerate}
	
	$"\exists"$. Let $E = \sett{t \in \mathrm{R}~|~t^n < x}$
	
	Claim:
	
	\begin{enumerate}
		\item[$\bullet$] $E \neq \emptyset,$ Let $t = \dfrac{x}{1+x}$, then $0<t<1$, hence $t^n < t < x,~ \therefore t \in E ~\&~ E \neq \emptyset$
		\item[$\bullet$] $E$ is bdd above, in fact $E$ is bdd above by $1+x$ if $t>1+x>1,$ then $t^n > t>x$, so $E$ is bdd above by $1+1$
		
		Therefore $y = \sup E$ exists \& is finite
		\item[$\bullet$] Claim $y > 0$ \& $y^n = x$, clearly, $y>0(\because \frac{x}{1+x} \in E ~\&~\frac{x}{1+x}>0 )$
		
		by trichotomy, we have $y^n<x ,~y^n>x,~y^n=x$
	\end{enumerate}
	
	Now, to show that (i) \& (ii) are impossible, do (iii) holds $y^n = x$
	
	By the identity, $b^n - a^n = (b-a)(b^{n-1}+b^{n-2}a+\cdots+a^{n-1})$
	
	(i)$y^n < x$ choose $0 < h < 1 = \alpha ~\&  ~0 < \dfrac{x - y^n}{n(y+1)^{n-1}}$, $0<h<\min \sett{\alpha,\beta}$
	
	put $a = y,~ b = y+h$ in $(\star)$, we obtain
	
	$$(y+h)^n - y^n < hn(y-h)^{n-1} < hn(y+1)^{n-1} < x-y^n$$
	
	$\implies (y+h)^n < x \implies y+h \in E ~\& y+h > y (\rightarrow\leftarrow) \therefore$ (i) fails
	\newpage
	
	(ii) $y^n > x$, Let $k = \dfrac{y^n - x}{ny^{n-1}}$, Then $0<k<y,~ k = \dfrac{y^n - x}{ny^{n-1}} < \dfrac{y^n}{ny^{n-1}} = \dfrac{y}{n} < y$
	
	if $t > y-k > 0,$ then $y^n - t^n \leq y^n - (y-k)^n < kny^{n-1}$ by $(\star) = y^n - x$
	
	$\implies t^n > x \implies t \in E \implies E$ is bdd above by $y-k \implies \sup E \leq y-k (\rightarrow\leftarrow)$
	
	$\therefore$ (ii) fails
\end{proof}


\begin{cor*}
	Let $a,b \in \mathbb{R}$ with $a,b > 0,~ n \in \mathbb{N}$ Then$(ab)^{\frac{1}{n}} = a^{\frac{1}{n}} b^{\frac{1}{n}}$
	
	$\because a^{\frac{1}{n}},b^{\frac{1}{n}} > 0 ~\& ~ (a^{\frac{1}{n}} \cdot b^{\frac{1}{n}}) = ab$, By (1) in Thm 1.4 $(a,b)^{\frac{1}{n}} = a^{\frac{1}{n}}b^{\frac{1}{n}}$
\end{cor*}

\textbf{infinite in $\mathbb R$}



After discuss the real number $\mathbb{R}$, sometimes, we have to work with the extended real number system $\mathbb{R}^* = [-\infty,\infty] = \mathbb{R} \cup \sett{+\infty,-\infty}$ with observe, $x \in \mathbb{R}$

\begin{tcolorbox}
	$$\lim_{n\rightarrow \infty}(-n) = -\infty,~\lim_{n \rightarrow \infty}n=\infty,~\lim_{n \rightarrow \infty}(\frac{1}{n}+n) = \infty,~\lim_{n \rightarrow \infty}(n^2-n) = \infty$$
	
	\begin{center}
		$x \pm \infty = \pm \infty,~ 0\cdot(\pm \infty) = 0,~\infty - \infty$ is not define
	\end{center}
\end{tcolorbox}


Element in $\mathbb R \subseteq \mathbb R^{*}$ are called finite. Now, given any nonempty subset $E \subseteq \mathbb R,$

$$\sup E = \begin{cases}
	+\infty \text{ if } E \text{ is not bdd above}\\\text{finite if $E$ is bdd above}
\end{cases} ~\&~ \inf E =\begin{cases}
	-\infty \text{ if $E$ is not bdd below} \\ \text{finite if $E$ is bdd below}
\end{cases}$$



Note that if $A \subseteq B$, then $\sup A \leq \sub B$ \& $\inf A \geq \inf B$

$\therefore \emptyset \subseteq B,~ \forall B \subseteq \mathbb R$, One may define $\sup \emptyset = - \infty, \inf \emptyset = + \infty$


\subsection{The Complex Number Field $\mathbb C$} $ $

Consider the contention product $\mathbb R^2 = \mathbb R \times \mathbb R = \sett{(a,b)~|~a,b\in \mathbb R}$

Note that $(a,b) = (c,d) \Leftrightarrow a=c ~\&~ b = d$, From now, we can write $\mathbb C = \mathbb R^2 $

\begin{tcolorbox}
	\textbf{Operation on $\mathbb C$}
	Given $(a,b),(c,d) \in \mathbb C$

	\begin{enumerate}
		\item $(a,b)+(c,d) = (a+c,b+d)$
		\item $(a,b)(c,d) = (ac-bd,ad+bc)$
	\end{enumerate}
	
	It is easy to see that, with these operations, $\mathbb C$ is a field.
\end{tcolorbox}

\newpage

\textbf{Note that}
\begin{enumerate}
	\item[$\cdot$] the zero element is $(0,0)$
	\item[$\cdot$] the negative of $(a,b)$ is $-(a,b) = (-a,-b)$
	\item[$\cdot$] the identity is $(1,0)$
	\item[$\cdot$] if $(a,b) \neq (0,0)$, then $(a,b)^{-1} = \left(\dfrac{1}{a^2+b^2},\dfrac{-b}{a^2+b^2}\right)$
\end{enumerate}


\begin{tcolorbox}
	R is a subset of C \color{blue}(not vary important)\color{black}
	
	consider that map
	
	$$f:\mathbb R \rightarrow \mathbb C \text{ define by } f(a) = (a,0),~a \in \mathbb R$$
	
	we have (1)$f$ is injective (2)$f(1) = (1,0)~ \because \forall~a,b \in \mathbb R$
	
	$f(a+b) = (a+b,0) = (a,0)+(b,0) = f(a) + f(b),~f(a\cdot b) = (ab,0) = (a,0) \cdot (b,0)$
	
	$f$ is a field homomorphism
	
	$\therefore f:\mathbb R \rightarrow \mathbb C$ is an injective and isomorphism
	
	Therefore, we identify $\mathbb R$ with $f(\mathbb R)$ through the injective $f$ 
	
	i.e. $a \in \mathbb R$ is identified with $f(a,0)$ in $\mathbb C$
	
	$ab = (a,0)\cdot(b,0),~a+b = (a,0)+(b,0) ~\forall~a,b\in \mathbb R$
	
\end{tcolorbox}

\textbf{Change $(a,b)$ to $a+bi$}

\begin{tcolorbox}
	Now, we can transform an element $(a,b) \in \mathbb C$ into the normal form:

(a,b) = (a,0)+(0,b) = (a,0)(1,0)+(b,0)(0,1) = a1+bi = a+ib,

\text{ where } i = (0,1)

Therefore, from new on, we write $\mathbb C = \sett{a+ib~|~a,b \in \mathbb C}$

An element $z = a+ib \in \mathbb C$ is called a complex number

Hence, under this notification, $z = a+ib,w = c+id \in \mathbb C$

\begin{enumerate}
	\item $z+w = (a+c) + i(b+d)$
	\item $zw = (ac-bd) + i(ad + bc)$
\end{enumerate}

and the $a$ is called the real part of $z$, $a = \text{Re}(z)$, $b$ is called imaginary part of $z$, $b = \text{Im}z$

\end{tcolorbox}

\newpage

\textbf{Some basic properties of complex numbers whose proofs are easy}

$\forall z,w \in \mathbb C$

\begin{tasks}(3)
	\task[$\cdot$] $\overline{z+w} = \overline{z} + \overline{w}$
	\task[$\cdot$] $\overline{zw} = \overline{z} \cdot \overline{w}$
	\task[$\cdot$] Re$z = \dfrac{z + \overline{z}}{2}$
	\task[$\cdot$] Im$z = \dfrac{z - \overline{z}}{2i}$
	\task[$\cdot$] $|z| = 0 \Leftrightarrow z = 0$
	\task[$\cdot$] Triangle inequality\\$|z+w| \leq |z| + |w|$
	\task[$\cdot$] $||z| - |w|| \leq |z - w|$
	\task[$\cdot$] $\mathbb C$ is not an ordered field
	\task[$\cdot$] $|z|^2 = z\overline{z}$
	\task[$\cdot$] $|\overline{z}| = |z|$
	\task[$\cdot$] $|\text{Re}z| \leq |z|,|\text{Im}z| \leq |z|$
	\task[$\cdot$] $|zw| = |z||w|$  
\end{tasks}

\begin{proof}
	$|z+w| \leq |z|+|w|$
	
	$|z+w|^2 = (z+w)(\overline{z+w}) = (z+w)(\overline{z}+\overline{w}) = z\overline{z}+z\overline{w}+w\overline{z}+w\overline{w}$
	
	$=|z|^2 + 2\text{Re}(z\overline{w})+|w|^2 \leq |z|^2 + 2|z\overline{w}|+|w|^2 = |z|^2 + 2|z||w|+|w|^2 = (|z|+|w|)^2$
	
	$\therefore |z+w| \leq |z| + |w|$
\end{proof}


\begin{thm*}[basic algebraic theorem]  $ $\\
		(a) $x^2 + 1$ has no root in $\mathbb R$\\
		(b) $x^2 + 1$ has two distinct roots in $\mathbb C$
\end{thm*}

\begin{proof}$ $
	\begin{enumerate}
		\item[(a)] $1>0,~x^2>0,~\forall x \in \mathbb R - \sett{0} \implies x^2 + 1 > 0 ~\forall x \neq 0$ 
		
		$0^2 + 1 = 1 > 0$, $~\therefore x^2 + 1 > 0 ,~\forall~ x \in \mathbb R$. Hence, $x^2 + 1 = 0$ has no root in $\mathbb R$
		\item[(b)] $i^2 = (0,1)(0,1) = (0-1,0) = (-1,0) = -1$
		
		$(-i)^2 = (-(0,1))^2 = (0,-1)^2 = (0,-1)(0,-1) = -1$, $\therefore \pm i$ are root of $\mathbb C$
	\end{enumerate}
\end{proof}

\textbf{Conclusion:} Every non const polynomial $f(x) \in \mathbb R[x]$ has $n$ roots where $n = \deg f(x)$


\textbf{The complex root is even}
\begin{tcolorbox}
	\textbf{\color{blue} no important proof} 
	
	$f(x) = a_nx^n + a_{n-1}x^{n-1}+\cdots+a_1x+a_0 \in \mathbb R[x],~a_n \neq 0,~n\geq 1$
	
	if $\alpha = a+ib \in \mathbb C$ is a root of $f(x)$,then
	
	$0 = f(\alpha) = a_n\alpha^n + a_{n-1}\alpha^{n-1}+\cdots + a_1\alpha + a_0$
	
	$0 = f(\overline{\alpha}) = a_n\overline{\alpha}^n + a_{n-1}\overline{\alpha}^{n-1}+\cdots+a_1\overline{\alpha} + a_0$
	
	$\therefore (x - \alpha)|f(x),~(x-\overline{\alpha})|f(x) \implies (x-\alpha)(x-\overline{\alpha})|f(x) \implies (x^2 - (\alpha - \overline{\alpha})x + |\overline{\alpha}|^2)~|~f(x)$
	
	$\implies (x^2 - 2ax +(a^2+b^2))~|~f(x)$
	
	$\therefore$ quadratic function must have two roots in $\mathbb C$
\end{tcolorbox}

\textbf{The fundamental Theorem of Algebra}

Every non zero polynomial $f(x) \in \mathbb C[x]$ has at least one root in $\mathbb C$

Therefore, if $\deg f(x) = n$, then $f(x)$ has $n$ roots in $\mathbb C (C,M)$
\begin{tcolorbox}
	$\because f(x) = (x-\lambda_1)^{e_1}\cdots(x-\lambda_t)^{e_t}(a_1x^2+b_1+c_1)^{l_1}\cdots(a_sx^2+b_sx+c_s)^{ls},$ where $\lambda_1,\cdots,\lambda_t \in \mathbb R,~a_i,b_i,c_i \in \mathbb R \& e_1+\cdots+e_t+2l_1+\cdots+2l_s = \deg f(x)$ which shows that all roots of $f(x)$ are in $\mathbb C$
	
	In fact, we have the famous theorem: The fundamental theorem of algebra
	
	Every non zero polynomial $f(x) \in \mathbb C[x]$ has at least one root in $\mathbb C$
	
	$\therefore$ if $\deg f(x) = n$, then $f(x)$ has $n$ roots in $\mathbb C (C,M)$
\end{tcolorbox}

\begin{thm*}[Cauchy-Scheming Inequal]
	Given $z_1\cdots,z_n,w_1,\cdots,w_n \in \mathbb C$, we have
	
	$$\left|\sum^n_{j=1}z_j\overline{w}_j\right| \leq \left(\sum^n_{j=1}|z_j|^2\right)^{\frac{1}{2}}\left(\sum^n_{j=1}|w_j|^2\right)^{\frac{1}{2}}$$
	
	and $"="$ holds $\Leftrightarrow \exists~\lambda \in \mathbb C \ni~ w_j = \lambda z_j,~1\leq j \leq n,$
	
	In patricial, if $x_1,\cdots,x_n,y_1,\cdots,y_n \in \mathbb R$, then
	
	$$\left|\sum^n_{j=1}x_jy_j\right| \leq \left( \sum^n_{j=1}x_j^2\right)^\frac{1}{2}\left(\sum^n_{j=1}y_j^2\right)^{\frac{1}{2}}$$
	
	and $"="$ holds $\Leftrightarrow \exists~t \in \mathbb R \ni~ y_j = tx_j,~1\leq j \leq n$
\end{thm*}

\textbf{\color{red}The proof is too long, I am lazy}

\newpage

\subsection{Euclidean Spaces $\mathbb R^n$}

\begin{defn}
	the n-dimensional Euclidean space $\mathbb R^n$
	
	$$\mathbb = \sett{x = (x_1,\cdots,x_n)~|~x_i \in \mathbb R, 1\leq i \leq n} = \mathbb R \times \cdots \times \mathbb R$$
	
	Note that
	
	$$(x_1,\cdots,x_n) = (y_1,\cdots,y_n) \Leftrightarrow x_i=y_i ~\forall~1\leq i \leq n$$
\end{defn}

We are going to introduce the structure of $\mathbb R^n$

\begin{tasks}(2)
	\task[$\cdot$] vector space
	\task[$\cdot$] inner product space
	\task[$\cdot$] normed linear space
	\task[$\cdot$] matrix space
\end{tasks}

\begin{defn}
	Two operation on $\mathbb R^n$ as follows:
	
	\begin{enumerate}
		\item[$\cdot$] Addition $+:\mathbb R \times \mathbb R^n \rightarrow \mathbb R^n ,~(x,y) \mapsto x+y = (x_1+y_1,\cdots,x_n+y+n)$
		\item[$\cdot$] Scalar multiplication $\cdot:\mathbb R^n \times \mathbb R^n \rightarrow \mathbb R^n,~(a,x) \mapsto ax=(ax_1,\cdots,ax_n)$ 
	\end{enumerate}
\end{defn}

\textbf{\color{red} we skip space example here.}

\subsection{Countability of Sets}$ $

Given two nonempty set $A,B$ and a function $f:A \rightarrow B$,$f(A) = \sett{f(a)~|~a \in A}$ is called the image of $A$ under $f$

\textbf{Some basic things}
\begin{tcolorbox}
	$E \subseteq A,~f(E) = \sett{f(a)~|~a \in E}$ the image of $E$ under $f$
	
	$f$ is infective(one-to-one) $x_1 \neq x_2 \implies f(x_1) \neq f(x_2) \Leftrightarrow f(x_1) = f(x_2) \Leftrightarrow x_1 = x_2$
	
	$f$ is surjective(onto) if $f(A) = B$, $f$ is bijective if $f$ is one-to-one and onto
\end{tcolorbox}

Given $F \subseteq B$, $f^{-1}(F) = \sett{x \in X~|~f(x) \in \mathrm F}$ called the inverse image of $f$ under $\mathrm F$

\textbf{Example}

\begin{tcolorbox}
	$f:\mathbb R \rightarrow \mathbb R,~f(x) = x^2,~x\in \mathbb R$
	
	$f^{-1}([0,1]) = \sett{x\in \mathbb R ~|~ f(x) \in [0,1]} = \sett{x \in \mathbb R ~|~ x^2 \in [0,1]} = [-1,1]$
	
	$f^{-1}([-1,1]) = [-1,1]$
\end{tcolorbox}

\newpage

\textbf{Properties of inverse image}

\begin{enumerate}
	\item[$\bullet$] $F_1 \subseteq F_2 \subseteq B \implies f^{-1}(F_1) \subseteq f^{-1}(F_2)$
	\item[$\bullet$] Inverse image presences set operation
		
		$\forall F_{\alpha} \subseteq B,~\alpha \in I,~F\subseteq B$
		\begin{enumerate}
			\item[(i)] $f^{-1}(\cup_{\alpha \in I}F_{\alpha}) = \cup_{\alpha \in I}f^{-1}(F_{\alpha})$
			\item[(ii)] $f^{-1}(\cap_{\alpha \in I}F_{\alpha}) = \cap_{\alpha \in I}f^{-1}(F_{\alpha})$
			\item[(iii)] $f^{-1}(B - F) = f^{-1}(B) - f^{-1}(F) $
		\end{enumerate}
	\item[$\bullet$] Given $S \subseteq A,~f'(f'(S)) \supseteq S, "=" \Leftrightarrow$ one-to-one, \textbf{example:}
	
	$f:\mathbb R \rightarrow \mathbb R,~f(x) = x^2,~S=[0,1],~f(S) = [0,1],~f^{-1}(f(S)) = f^{-1}([0,1]) = [-1,1]$
	
	\item[$\bullet$] Given $F \subseteq B,~f(f^{-1}(F)) \subseteq F, "=" \Leftrightarrow$ "onto", \textbf{example}
	
	$f(x) = x^2,~x \in \mathbb R,~F = [-1,1],~f(f^{-1}([-1,1])) = f([-1,1]) = [0,1]$
	
	\item[$\bullet$] For $y \in B,~f^{-1}(\sett{y}) = f^{-1}(y) = \sett{x \in A ~|~ f(x) = y}$ the inverse image of $y$, \textbf{example}
	
	$f:\mathbb R \rightarrow \mathbb R,~f(x) = x^2,~f^{-1}(1) = \sett{1,-1},~f^{-1}(2) = \emptyset$
\end{enumerate}

\begin{defn}[cardinality]
	Let $A,B$ are two set ew say that $A$ and $B$ have the same cardinality if $\exists$ a bijective map $f:A \rightarrow B$, which is denoted by $A \sim B$
	
	From now on, we write $|A|$ as the cardinality of $A$
\end{defn}

\textbf{Claim} $"\sim"$ is an $\equiv$ relation among all sets

\begin{enumerate}
	\item[(i)] Reflexion: $\forall~$ set $A,~A\sim^{1A} A,~$ which $1_A$ is identity mapping
	\item[(ii)] Symmetry: $A \sim^{f}B \implies B\sim^{f^{-1}} A$
	\item[(iii)] Transitive: $A \sim^{f} B ~\&~ B\sim^{g}C \implies A \sim^{g o f}C$
\end{enumerate}

So we gave some property:

\begin{enumerate}
	\item[$\bullet$] Any two $"\equiv"$ are either disjoint or identical 
	\item[$\bullet$] $\overline{\underline{X}}$ is a disjoint union of $"\equiv "$ classes
	
	$[A] = \sett{B \in \overline{\underline{X}}~|~B \sim A}$ the $"\equiv "$ class set by $A$
\end{enumerate}

Ant two element in an $"\equiv "$ class have the same cardinality

Notation For $n \in \mathbb N,~\mathbb N_{m} = \sett{1,2,\cdots,n}$

\begin{defn}
	Let $A$ be a set
	
	\begin{enumerate}
		\item[(a)] $A$ is a finite set if $A = \emptyset$ or $A \sim \mathbb N_n$ for some $n \in \mathbb N$
		\item[(b)] $A$ is a infinite set if $A$ is not a finite set 
		\item[(c)] $A$ is countable if $A \sim \mathbb N$
		\item[(d)] $A$ is uncountable if $A$ is not countable.
		\item[(e)] $A$ is at most countable if $A$ is finite or countable
	\end{enumerate}
\end{defn}

\newpage

\begin{rmk*} $ $
	\begin{enumerate}
		\item when $A,B$ are finite sets, $A \sim B \Leftrightarrow |A| = |B|$, i.e. $A,B$ have same number.
		\item where $A,B$ are infinite and $A \sim B$, i.e. $|A| = |B|$, the concept is abstract.
		\item $\sett{a,b,c} \cup \mathbb N \sim \mathbb N,~f:\mathbb N \rightarrow \sett{a,b,c} \cup \mathbb N,~f(1) = a,~f(2) = b,~f(3) = c,\cdots$
		\item Any finite set can not equivalent to a proper subset, i.e. $A$ is finite, $B \subseteq A$
		
		Then $A \sim B$, In fact $|B| < |A|$, but infinite different
		\item Any finite set $A$ can be listed an $A = \sett{a_1,\cdots,a_n}$ where $n = |A|$
	\end{enumerate}
\end{rmk*}

Now, we consider the case of countable set

\begin{tcolorbox}
	Recall, in calculus, a real sequence $\sett{a_n}$, e.g.
	
	$$a_n = \frac{1}{n}\sett{\frac{1}{n}},~a_n = 1 - \frac{1}{n}~\sett{1-\frac{1}{n}},~a_n=\begin{cases}
		0 \text{ if } n \text{ is odd}\\1 \text{ if } n \text{ is even}
	\end{cases}$$
\end{tcolorbox}

\begin{defn}
	Let $X$ be a nonempty set, a sequence in $X$ is a function $a:\mathbb N \rightarrow X$

	Given a sequence $=^{a}$ in $X, a$ is $"1"$ determine by $a(n),~ \in \mathbb N$
	
	We write
	
	$$a = \sett{a(1),a(2),\cdots,a(n),\cdots} = \sett{a_1,a_2,\cdots,a_n,\cdots} = \sett{a_n} = \sett{a_n}^{\infty}_{n = 1}$$ 

\end{defn}

\begin{rmk*} $ $
	\begin{enumerate}
		\item For a sequence $\sett{a_n}$ in $X$, $a_n$ may not be distinct.
		
		If all $a_n$ are distinct, then we say that $\sett{a_n}$ is a distinct sequence in $X$.
		
		\item We usually use $\sett{a_n},\sett{b_n}$ to denote sequence
		\item A sequence $\sett{a_n}$ in $X$ in fact is a function from $\mathbb N \rightarrow X,$ So $\sett{a_n~|~n \in \mathbb N}$ is the image of the sequence.
		\item $\sett{a_n}$ is a sequence, $a_n$ is called the $n^{\text{th}}$ term of the sequence.
		\item A sequence in $X$ may begin at $0$, i.e. $\sett{a_n}_{n=0}$
		
		By a changing index, we can make it from $\sett{b_n}^{\infty}_{n=1},~b_n = a_{n+1},~n = 1,2,\cdots$ 
	
	\end{enumerate}
\end{rmk*}

\begin{defn}[increasing] $ $\\
	A function $a:\mathbb N \rightarrow \mathbb N$ is increasing, $a$ is $\uparrow$, if $a(n) \leq a(n+1) ~\forall~n \geq 1$
	
	$a$ is strictly increasing, $a$ is st. $\uparrow$, if $a(n) < a(n+1) ~\forall~n \geq 1$
\end{defn}

Now, given a st. $\uparrow$ function $n:\mathbb N \rightarrow \mathbb N$, i.e. $n(k) < n(k+1),~k \geq 1$

i.e. $n_k < n_{k+1},~k \geq 1$, i.e. $n_1<n_2<\cdots<n_k<\cdots$, i.e. $\sett{n_k}^{\infty}_{k=1}$ is a st. sequence in $\mathbb N$

\begin{defn}
	Let $\sett{a_n}$ be a sequence in $X$ and $\sett{n_k}$ be a st. $\uparrow$ sequence in $\mathbb N$, then the sequence $\sett{a_{n_k}}$ is called a subsequence of $\sett{a_n}$
	
	In fact
	
	$$\mathbb N \rightarrow^{n}_{st.} \mathbb N \rightarrow^{a}_{seq} X  \Rightarrow a \circ n:\mathbb N \rightarrow X \text{ is a function,}$$
	
	hence, it also a sequence in $X$
	
	$$a \circ n = \sett{a \circ n(k)} = \sett{a(n(k))} = \sett{a_{n(k)}} = \sett{a_{n_k}}$$
\end{defn}

\begin{rmk*}
	if $\sett{a_{n_k}}$ is st. $\uparrow$ in $\mathbb N$, then $k \leq n_k~\forall~k\geq 1$
	
	$\because$ By mathematical Induction
	
	\begin{enumerate}
		\item[$\cdot$] $1 \leq n_1$
		\item[$\cdot$] Assume it's true for $k \geq 2,$ i.e. $k \leq n_k$
		\item[$\cdot$] Consider $k+1$, $~k+1 \leq n_k + 1 \leq n_{k+1}$ 
	\end{enumerate}
	
	\textbf{Example}
	
	Let $\sett{a_n}$ be a sequence in $X$, then $\sett{a_{2k}}$ and $\sett{a_{2k-1}}$ are subsequence of $\sett{a_n}$
\end{rmk*}

Finally, we will assume that you are familiar with the following property of the countability of sets:

\begin{enumerate}
	\item Every subset of a countable set is at most countable. The proof needs the well ordering of $\mathbb N$: Every nonempty subset of $\mathbb N$ has the smallest element
	\item Countable union of countable sets is countable
	\item If $A_1,A_2,\cdots,A_n$ are countable, then so is $A_1 \times \cdots \times A_n$
	\item If $A$ is countable, then so is $A^n \equiv A \times \cdots \times A~ \forall	n \geq 1$
	\item $\mathbb N,~\mathbb Z,~\mathbb Q,~\mathbb Q^n,~\forall n \geq 1$ are countable
	\item The set $\sett{a_n~|~a_n=0\text{ or }1}$ is uncountable
	
	This can be proved by Canton diagonal process
	
	$\because$ if it is countable, then we can list it,$a_0 A = \sett{a^{(1)}_1,a^{(2)}_2,\cdots}$ where
	
	$a^{(1)} = \sett{a_n^{(1)}} = a^{(1)}_{1},a^{(1)}_2,\cdots~;~a^{(2)} = \sett{a_n^{(2)}} = a^{(2)}_{1},a^{(2)}_2,\cdots$
	
	Now, construct a sequence $\sett{a_n}$ in $A \ni \sett{a_n} \neq a^{(k)} ~\forall~k\geq 1 (\rightarrow\leftarrow)$
\end{enumerate} 

\newpage

Recall, intervals in $\mathbb R,~-\infty < a \leq b < \infty$, following are finite bdd interval

\begin{tcolorbox}
	$(a,b) = \sett{x \in \mathbb R~|~a<x<b}$  open interval
	
	$\left[a,b\right] = \sett{x \in \mathbb R~|~a\leq x \leq b}$ closed interval
	
	$\left(a,b\right] = \sett{x\in \mathbb R~|~a < x \leq b}$ open-closed
	
	$\left[a,b \right) = \sett{x \in \mathbb R~|~a\leq x < b}$ closed-open
\end{tcolorbox}

An interval $I$ in $\mathbb R$ is said to be non-degenerate if the endpoint of $I$ are distinct i.e. length $> 0$. Otherwise, it is degenerate.

\textbf{Note.}

\begin{tcolorbox}
	$(0,1)$ is uncountable, $\because (0,1) = \sett{\sum^{\infty}_{n=1} \dfrac{a_n}{2^n}~|~a_n = 0 \text{ or } 1,n\in \mathbb N}$
	
	$x \in (0,1)$ has a unique binary representation, so $(0,1) \sim A$, 
	
	where $A$ is $\sett{\sett{a_n}~|~a_n = 0 \text{ or } 1}$ which is uncountable
\end{tcolorbox}

All non-degenerate intervals in $\mathbb R$ are uncountable.

$\because$ It sufficient to consider bdd non-degenerate interval in $\mathbb R$, given $\infty < a < b < \infty$

$(a,b)$ is uncountable($\because(0,1) \sim (a,b)$)

Note that $(0,1)\sim \mathbb R$($\because (0,1) \rightarrow (\dfrac{\pi}{-2},\dfrac{\pi}{2}) \rightarrow \mathbb R$)


