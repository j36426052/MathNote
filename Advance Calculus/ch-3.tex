\section{Infinite Sequence \& Series}

\begin{enumerate}[wide,label = $\bullet$]
	\item We will assume you are familiar with all operations of real(complex) sequence
	\item We have defined sequence in a set $X$
	\begin{tcolorbox}
		Recall : let $\{a_n\}$ be a real or complex sequence, $\{a_n\}$ converges if $\exists a \in \R (\C)$ satisfying $\forall \epsilon > 0, \exists N \in \N \ni \forall n \geq N , |a_n = a| < \epsilon$
	\end{tcolorbox}
	\item Now, we study the properties of a sequence in a metric space(topological space)
\end{enumerate}

\subsection{Convergent Sequence}

Let $X$ be a metric space $\& \{ x_n\}$ be a sequence in $X,~ x:\N \rightarrow X$

\begin{defn}
	We say that $\{x_n\}$ convergences (in $X$) if $\exists p \in X$ satisfying $\forall \epsilon > 0 ,\exists N \in \N \ni \forall n \geq N, d(x_n,p) < \epsilon,$ Otherwise, $\{x_n\}$ divergences.
\end{defn}

\begin{rmk*}$ $
	\begin{enumerate}
		\item If $\{x_n\}$ convergences as in definition, them $p$ is called the limit of the sequence $\{x_n\}$ and is denoted by $\lim_{n \rightarrow \infty}x_n = p$ or $x_n \rightarrow p$ as $n \rightarrow \infty$
		\item $x_n \rightarrow p$ as $n \rightarrow \infty \Leftrightarrow$ the real sequence $\{d(x_n,p)\}$ convergences to $0$, i.e. $\lim_{n \rightarrow \infty} d(x_n,p) = 0$
		\item if $\{x_n\}$ convergences, them its limit is !
		\item The convergence of a sequence depends not only the sequence but also on the space.
		
		e.g. $\lim_{n\rightarrow \infty}\frac{1}{n} = 0$ in $\R$, but $\{\frac{1}{n}\}$ divergences in $(0,1)$
	\end{enumerate}
\end{rmk*}

\textbf{Recall} Let $\{x_n\}$ are a sequence in a set $X$ with is a function $x:\N \rightarrow X$. The image of the sequence $=$ the image of the function $X$  \\  $=\{x_n~|~ n = 1,2,\cdots\}$

\begin{rmk*}
	The range of a sequence may be finite. e.g. $\{(-1)^n\}$ in $\R$, whose range $\{-1,1\}$ is finite, but $\{\frac{1}{n}\} $ has range $\{\dfrac{1}{n}~|~ n = 1,2,\cdots\}$
\end{rmk*}

\begin{defn}
	A sequence $\{x_n\}$ in $X$ is said to be bounded if its range is a bounded subset of $X$
\end{defn}

\begin{rmk*}
	A sequence $\{x_n\}$ in $X$ is said to be bounded if its range is a bounded subset of $X$
\end{rmk*}

\newpage

\textbf{Example}

\begin{enumerate}[wide]
	\item Every const sequence $\{p\}$ in a metric space convergence, i.e. $\lim_{n \rightarrow \infty} p = p$
	\item $\lim_{n \rightarrow \infty}\frac{1}{n} = 0$ in $\R$ and $\{\dfrac{1}{n}\}$ is bounded (but the range is finite)
	\item $\{(-1)^n\}$ divergences, but $\{(-1)^n\}$ is bounded. (range is finite).
	\item $\{n^2\}$ divergences in $\R$ and is unbounded. In fact, $\lim_{n \rightarrow \infty}n^2 = + \infty$( which range is infinite)
	\item $\lim_{n \rightarrow \infty}(1 + \dfrac{(-1)^n}{n}) = 1$ and $\{1 + \dfrac{(-1)^n}{n}\}$ is bounded. (range is infinite)
	\item $\{i^n\}$ divergence and it's bounded (range is finite)
	\item Identify all convergence sequence in a discrete metric space $X$. $\{x_n\}$ convergence to $p \Leftrightarrow \forall \epsilon > 0 \exists N \in \N \ni \forall n \geq N, d(x_n,p) < \epsilon \Leftrightarrow \{x)n\}$ is almost constant.
\end{enumerate}

In metric space, we can use sequences to characterise adherent and accumulation point

\begin{thm}
	Let $\{x_n\}$ be a sequence in a metric space $X$ and $E \subseteq X$
	
	\begin{enumerate}[wide,label = $(\alph*)$]
		\item $x_n \rightarrow p$ as $n \rightarrow \infty \Leftrightarrow \forall $ neighborhood $U$ of $p, \exists N \in \N \ni \forall n \geq N,x_n \in U$
		\item If $\{x_n\}$ convergences, them its limit is !
		\item If $\{x_n\}$ convergences, them its range is bounded, but not converse
		\item $p \in \overline{E} \Leftrightarrow \exists$ a sequence $\{a_n\}$ in $E \ni a_n \rightarrow p$
		\item $p \in E' \Leftrightarrow \exists$ a distinct sequence($a_n \neq a_m \forall n \neq m$) $\{a_n\}$ in $E \ni a_n \rightarrow p$
	\end{enumerate}
\end{thm}

\begin{proof} $ $
	\begin{enumerate}[wide,label = $(\alph*)$]
		\item \begin{eqnarray*}
			x_n \rightarrow p &\Leftrightarrow & \forall \epsilon > 0, \exists N \in \N \ni d(x_n,p) < \epsilon (\forall n \geq N)\\
			&\Leftrightarrow & \forall \epsilon > 0, \exists N \in \N \ni x_n \in B(p,\epsilon) (\forall n \geq N)\\
			&\Leftrightarrow & \forall \text{ neighborhood } U \text{ of } p, \exists N \in \N \ni \forall n \geq N, x_n \in U
		\end{eqnarray*}
		\item Suppose $x_n \rightarrow p,~x_n \rightarrow q$ and $p \neq q$, let $\epsilon = \dfrac{1}{2}d(p,q)$. By definition, $\exists N_1 \ni \forall n \geq N_1, d(x_n, p < \epsilon)$ and $\exists N_2 \ni \forall n \geq N_2, d(x_n,q) < \epsilon$
		
		Let $N = \max \{N_1,N_2\}$ them $\forall n \geq N, $ above holds. Hence $d(p,q) \leq d(p,x_N) + d(x_N,q) < \epsilon + \epsilon = 2 \epsilon = d(p,q) (\rightarrow \leftarrow) \therefore p = q$
		
		\item We have seen bounded sequence may not converges. If $x_n \rightarrow p$, them for $\epsilon = 1, \exists N \in \N \ni \forall n \geq N, d(x_N,p) < 1$, i.e. $\forall n \geq N,~x_n \in B(p,1)$. Let $R = \max \{d(p,x_1),\cdots,d(p,x_{n-1})\} + 1$, Then $x_n \in B(p,R) \forall n \geq 1 \therefore \{x_n\} $ is bounded
		\newpage
		\item ($\Rightarrow$) Suppose $p \in \overline{E}$. Them $\forall n \geq 1, B(p, \dfrac{1}{n}) \cap E \neq \emptyset$. Choose $a_n \in B(p,\dfrac{1}{n}) \cap E, n \geq 1$. We get a sequence $\{a_n\}$ in $E$ and $0 \leq d(x_n , p) < \dfrac{1}{n}, \forall n \geq 1$. 
		 
		By squeezing lemma, $\lim_{n \rightarrow \infty}d(a_n , p) =0$, i.e. $a_n \rightarrow p $ as $n \rightarrow \infty$
		
		($\Leftarrow$) Suppose the conditions holds. $\forall r > 0,~\exists N \in \N \ni \forall n \geq N, d(a_n, p) < r \implies \forall n \geq N , a_n \in B(p,r) \implies B(p,r) \cap E \neq \emptyset \therefore p \in E$
		
		\item It's similar to above.
		
	\end{enumerate}

\end{proof}


\begin{thm}
	For real or complex sequences $\{x_n\}$ and $\{y_n\}$, $\lim_{n \rightarrow \infty} x_n = x,~\lim_{n \rightarrow \infty} y_n = y$, $a,b \in \R$ or $\C$, $c \in \R$ or $\C$
	
	\begin{enumerate}
		\item $\lim_{n \rightarrow \infty} c = c$
		\item $\lim_{n \rightarrow \infty} (ax_n + by_n) = ax+by = a\lim_{n \rightarrow \infty} x_n + b\lim_{n \rightarrow \infty}y_n$
		\item $\lim_{n \rightarrow \infty} x_ny_n = xy = \lim_{n \rightarrow \infty}x_n \lim_{n \rightarrow \infty}y_n$
		\item If $y \neq 0, \lim_{n \rightarrow \infty} \dfrac{x_n}{y_n} = \dfrac{x}{y} = \dfrac{\lim_{n \rightarrow \infty}x_n}{\lim_{n \rightarrow \infty} y_n}$
		\item If $\{z_n\}$ is a complex sequence, them $z_n \rightarrow z$ as $n \rightarrow \infty \Leftrightarrow \\ \text{Re}z_n \rightarrow z ~\&~ \text{Im}z_n \rightarrow z$(using $|\text{Re} w|, |\text{Im} w| \leq |w| \leq |\text{Re}w|+|\text{Im}w| \forall w \in \C$)
		\item (Squeezing Lemma) If $\{x_n\}\{y_n\}$ and $\{t_n\}$ are real sequence $\ni$
		 $$x_n \leq t_n \leq y_n \text{ for } n >> 0$$
		 and $\lim x_n = lim y_n = l,~$ them $\lim_{n \rightarrow \infty} t_n = l$
		 \item $\lim_{n \rightarrow \infty} x_n = x \implies \lim_{n \rightarrow \infty}|x_n| = |x|$(using $||x_n| - |x|| \leq |x_n - x|$)
	\end{enumerate}
\end{thm}

\textbf{Examples.} 

\begin{enumerate}[wide,label = $(\roman*)$]
	\item $\lim_{n \rightarrow \infty}(1 - \frac{i}{n}) = 1$ (Re ($1 - \dfrac{i}{n}) = 1$, Im($1 - \dfrac{i}{n}) = \frac{1}{n}$)
	\item $\lim_{n \rightarrow \infty} \frac{1}{n} \sin \dfrac{1}{n} = 0$
	\begin{eqnarray*}
		0 \leq |\frac{1}{n}\sin \frac{1}{n} | \leq \frac{1}{n} &\implies& \lim_{n \rightarrow \infty}|\frac{1}{n} \sin \frac{1}{n}| = 0\\
		&\implies & |\lim_{n \rightarrow p} \frac{1}{n} \sin \frac{1}{n} | = 0 \implies \lim_{n \rightarrow \infty} \frac{1}{n} \sin \frac{1}{n}
	\end{eqnarray*}
	\item $\{(-1)^n\}$ divergence, but $|(-1)^n| = 1 \rightarrow 1$
\end{enumerate}

For sequences in $\R^k$ including in $\C \approx \R^2 ,$ we have.

\begin{thm}
	Let $\{x_n\}$ be a sequence in $\R^k$, where
	$$x_n = (x_{1,n},x_{2,n},\cdots,x_{k,n}), n = 1,2, \cdots$$
	
	\begin{enumerate}[wide,label = $(\alph*)$]
		\item $x_n \rightarrow p (p_1,\cdots , p_k)$ in $\R^k \Leftrightarrow x_{i,n} \rightarrow p_i \forall 1 \leq i \leq k$, i.e. $\lim_{n \rightarrow \infty}(x_{1,n},\cdots,x_{k,n}) = (\lim_{n \rightarrow \infty}x_{1,n},\cdots,\lim_{n \rightarrow \infty}x_{k,n})$ if exists
		\item Let $\{x_n\} \{y_n\}$ be sequences in $\R^k$ and $\{d_n\}$ be a sequence in $\C$ and $a,b \in \R$. If $x_n \rightarrow x,y_n \rightarrow y $ and $d_n \rightarrow d$, then
		
		\begin{enumerate}[label = $\bullet$]
			\item $ax_n + by_n \rightarrow ax + by$
			\item $<x_n,y_n> \rightarrow <x,y>$
			\item $d_n x_n \rightarrow d_x$
			\item $||x_n|| \rightarrow ||x||$
		\end{enumerate}
		
		If $k = 3$, them $x_n \times y_n \rightarrow x \times y$
	\end{enumerate}
\end{thm}

\begin{proof}$ $

	(a)It follows from the inequation $\forall y \in \R^k |y_i| \leq ||y|| \leq \sum^{k}_{i = 1}|y_i|$
	
	$\because \forall 1 \leq i \leq k ~ |x_{i,n} - p_i| \leq ||x_n - p|| \leq \sum^k_{i=1}|x_{j,n} - p_j| \forall n \geq 1$
	
	($\Rightarrow$) \begin{eqnarray*}
		\text{Suppose } x_n \rightarrow p &\implies & ||x_n - p|| \rightarrow 0\\
		&\implies& \forall 1 \leq i \leq k, |x_{i,n} - p_i| \rightarrow 0 \forall 1 \leq k \leq n\\
		&\implies& \forall 1 \leq i \leq k, x_{i,n} \rightarrow p_i \text{ as } n \rightarrow \infty 
	\end{eqnarray*}
	($\Leftarrow$) 
	\begin{eqnarray*}
		\text{Suppose } x_n \rightarrow p_i, 1 \leq i \leq k &\implies & ||x_{i,n} - p_i|| \rightarrow 0 ~\forall 1 \leq k \leq n\\
		&\implies& \sum^k_{i = 1}|x_{i,n} - p_i| \rightarrow 0 \\
		&\implies& ||x_n - p|| \rightarrow 0 \implies x_n \rightarrow p
	\end{eqnarray*}
	
	(b) By (a)
	
	\begin{eqnarray*}
		ax_n + by_n &=& (ax_{1,n},ax_{2,n},\cdots,ax_{k,n}) + (by_{1,n},\cdots , by_{k,n})\\
		&=& (ax_{1,n} + by_{1,n},\cdots,ax_{k,n} + by_{k,n})\\
		&\rightarrow & (ax1 + by_1,\cdots , ax_k + by_k) = ax + by 
	\end{eqnarray*}
	
	\begin{enumerate}[label = $\bullet$]
		\item $<x_n,y_n> = \sum^k_{i = 1}x_{i,n}y_{i,n} \rightarrow \sum^n_{i = 1}x_iy_i = <x,y>$
		\item $d_nx_n = (d_nx_{1,n},\cdots , d_nx_{k,n}) \rightarrow (dx_1,\cdots,dx_k) = dx$
		
		$||x_n|| = (\sum^k_{i=1} x_{i,n}^2)^{\frac{1}{2}} \implies (\sum^k_{i=1}x_i^2)^{\frac{1}{2}} = ||x||$
		\item $x_n \times y_n = (x_{2,n}y_{3,n} - x_{3,n}y_{2,n},\cdots) \rightarrow (x_2y_3 - x_3y_2 , \cdots) = x \times y$
	\end{enumerate}
\end{proof}

\subsection{Subsequences}

\begin{thm} $ $
	\begin{enumerate}[wide, label = $(\alph*)$]
		\item If $\{x_n\}$ convergences to $p$, i.e. $\lim_{n \rightarrow \infty}x_n = p$, them so is every subsequence of $\{x_n\}$
		\item If $X$ is compact and $\{x_n\}$ is a sequence in $X$, them $\{x_n\}$ has a convergent subsequence.
		\item Every bounded sequence $\{x_n\}$ in $\R^k$ has a converge subsequence.
	\end{enumerate}
\end{thm}


\begin{proof}
	\begin{enumerate}[wide, label = $(\alph*)$]
		\item Given a subsequence $\{x_{n_k}\}$ of $\{x_n\}$( Note that $\{n_k\}$ is strictly increasing , i.e. $n_1 < n_2 < \cdots$, hence, $k \leq n_k ~ \forall k \geq 1$), $d(x_{n_k},p) < \epsilon$. This proves $x_{n_k} \rightarrow p$ as $k \rightarrow \infty$
		\item Let $T = \{x_n ~|~ n \geq 1\}$ be the range of $\{x_n\}$
		
		Case 1: $T$ is a finite set. In this case, some $x_{n_0}$ must appear infinitely many times in the sequence $\{x_n\}$. Choose $n_1 = n_0 \ni x_{n_1} = x_{n_0}$, and $n_2 > n_1 \ni x_{n_1} = x_{n_2},\cdots$. In this way, we get a const subsequence $\{x_{n_k}\}$ which convergence to $x_{n_0}$
		
		Case 2: $T$ is an infinite set. In this case, $T$ is an infinite subset of the compact metric space $X$. By Thm 2.17 (ii), $T$ has an accumulation point $p$ in $X$. By Thm 3.1 (e), $\exists$ a sequence in $T$ which converges to $p$. We may arrange such sequence to be a subsequence of $\{x_n\}$. We are done
		
		\begin{tcolorbox}
			$y_1 = x_n$, choose $n_2 \rightarrow n_1 \ni x_{n_2}$ appears in $\{y_j\}.$ Then $\{x_{n_k}\}$ is a subsequence of $\{x_n\}$ and $\{y_j\}$, $\therefore x_{n_k} \rightarrow p$
		\end{tcolorbox}
		\item $\because$ Since $\{x_n\}$ is bounded, we may choose a closed ball $\overline{B}(0,R)$ or a closed $n$-dimensional interval in $\R^k \ni \{x_n\}$ is a sequence in $K$, By (b), $\{x_n\}$ has a convergence subsequence .
	\end{enumerate}
\end{proof}


\begin{rmk*}
		Thm 3.4(a) can be used to detect the divergence of a sequence, e.g. $\{(-1)^n\}$ in $\R$ which divergences, $\because$ It has two subsequences $\begin{cases}
			x_{2n} \rightarrow 1 \\ x_{2n - 1} \rightarrow -1
		\end{cases}$ which is different.
\end{rmk*}

\begin{defn}
	Let $\{x_n\}$ be a sequence in $X$. A point $p \in X$ is called a subsequential limit of $\{x_n\}$ if $\exists$ a subsequence $\{x_{n_k}\}$ of $\{x_n\} \ni x_{n_k} \rightarrow p$ as $k \rightarrow \infty$
\end{defn}

\textbf{Examples} 

\begin{enumerate}
	\item If $\{x_n\}$ convergences to $p$, then $\{x_n\}$ has only on subsequence limit $p$ ($E = \{p\}$)
	\item $\{(-1)^n\}$ has two subsequence limit $1$ and $-1,~ E = \{1,-1\}$
	\item $\{n\}$ has no subsequence limit $(E = \emptyset)$ 
\end{enumerate}

Let $\{x_n\}$ be a sequence in $X$ and $E$ be the set of all subsequence limits of $\{x_n\}$

\begin{thm}
	As above, $E$ is a closed subset of $X$
\end{thm}

\begin{proof}
	If $E$ is a finite set, them $E$ is closed.
	
	Now, assume that $E$ is an infinite set, to show that $E$ is closed, we must prove $E' \subseteq E$, i.e. $E$ contains all its accumulation point. Given $q \in E'$, to prove $q \in E$, i.e. $\exists$ a subsequence $\{x_{n)k}\} \ni x_{n_k} \rightarrow q$. Since $E$ is infinite, $\{x_n\}$ is not a constant sequence , so we can choose  $x_{n_1} \neq q$. Let $\delta = d(x_n , q),$ We construct subsequence $\{x_{n_k}\}$ of $\{x_n\}$ satisfying : $d(x_{n_k} , q) \leq \frac{\delta}{2^{k-1}} \forall k \geq 1$. If it is done, them by squeezing lemma, $d(x,q) \rightarrow 0$ as $k \implies \infty$. 
	
	Now, to construct such subsequence $\{x_{n_k}\}$. By induction, suppose $k = 1$ we are done, we have found $n_1 < n_2 < \cdots < n_{k-1} , k \geq 2$. To find $x_{n_k}$. Since $q \in E', B(q , \frac{\delta}{2^k}) \cap E - \{q\} \neq \emptyset$. Choose $x \in B(q,\dfrac{\delta}{2^k}) \cap E - \{q\}$. Now, $x \in E, \exists$ a subsequence of $\{x_n\}$ which convergence to $x$. Hence $\exists n_k > n_{k-1} \ni d(x_{n_k},x) < \frac{\delta}{2^k}$. Finally, $d(x_{n_k},q) \leq d(x_{n_k},x) + d(x,q) < \frac{\delta}{2^k} + \frac{\delta}{2^k} = \frac{\delta}{2^{k-1}} \therefore$ By induction, such subsequence $\{x_{n_k}\}$ can be found.
\end{proof}


\subsection{Cauchy Sequences} $ $

\textbf{Recall} $x_n \rightarrow p \implies \forall \epsilon > 0 ~\exists N \in \N \ni \forall n \geq N ,~d(x_n,p) < \frac{\epsilon}{2}$.

$\therefore \forall m,n \geq N, ~d(x_m,x_n) \leq d(x_m,p) + d(p,x_n) < \frac{\epsilon}{2} + \frac{\epsilon}{2} = \epsilon$

\begin{defn}
	A sequence $\{x_n\}$ in $X$ is called a Cauchy sequence if it satisfies the Cauchy condition: $\forall \epsilon > 0,~\exists N \in \N \ni \forall n,m \geq N, d(x_n,x_m) < \epsilon$
\end{defn}

\newpage

\begin{rmk*}$ $
	\begin{enumerate}[label = (\roman*), wide]
		\item Convergence sequence is Cauchy
		\item Cauchy sequence may not convergence, e.g. in $(0,1)~\{\dfrac{1}{n}\}$ is Cauchy, but not convergence in $(0,1)$
		\begin{tcolorbox}
			$\forall n,m \in \N ,~ m \geq n,~ |\frac{1}{n} - \frac{1}{m}| \leq \dfrac{1}{n} + \dfrac{1}{m} \leq \dfrac{2}{n}$
			
			$\forall \epsilon > 0$, Choose $N \in \N \ni \frac{2}{N} < \epsilon$. 
			
			Them $\forall n,m \geq N,~|\dfrac{1}{n} - \dfrac{1}{m} | \leq \dfrac{2}{N} < \epsilon ~ \therefore$ $\{\frac{1}{n}\}$ is Cauchy.
		\end{tcolorbox}
		\item $\{x_n\}$ is Cauchy $\Leftrightarrow \lim_{n,m \rightarrow \infty} d(x_n,x_m) = 0$
		
		\item Let $E_n = \{x_n , x_{n+1} , \cdots\} n \geq 1$. Them $\{x_n\}$ is Cauchy $\Leftrightarrow \lim_{n \rightarrow \infty} $ dia $(E_n) = 0$
		
		
		
		\begin{tcolorbox}
			Recall the definition of diameter:

			Let $S \subseteq V, S \neq \emptyset.$ The diameter of $S$ is dia($S$) = sup$\{d(x,y)~|~ x,y \in S\}$

			Let  
		\end{tcolorbox}
		
		
	\end{enumerate}
\end{rmk*}


\begin{proof}
	($\Rightarrow$) Suppose $\{x_n\}$ is Cauchy. Then $\forall \epsilon > 0,~\exists N \in \N \ni \forall n,m \geq N, d(x_n,x_m) < \frac{\epsilon}{2}$. Then $\forall n \geq N,$ dia ($E_n$) $\leq \frac{\epsilon}{2} < \epsilon \implies \lim_{n \rightarrow \infty}$ dia ($E_n$) $= 0$
	
	($\Leftarrow$) Suppose $\lim_{n \rightarrow \infty}$ dia ($E_n$) $= 0$, $\forall \epsilon > 0 ,\exists N \in \N \ni \forall n \geq N$, dia($E_n$) $< \epsilon \implies \forall n,m \geq N, (x_n,x_m \in E_n), d(x_n,x_m \leq )$ dia ($E_N$) $< \epsilon$ 
	
	$\therefore \{x_n\}$ is Cauchy.
\end{proof}

\begin{rmk*}
	Every Cauchy seq $\{x_n\}$ in a metric space is bounded. For $\epsilon = 1, \exists N \in \N \ni \forall m,n \geq N, d(x_n , x_m) < 1$, In particular, $\forall n \geq N, d(x_n,x_N) < 1$. Let $R = \text{max} \{d(x_i,x_N)~|~ 1 \leq i \leq N-1\} + 1$. Then $x_n \in B(x_N,R) \forall n \geq 1$, i.e. $\{x_n ~|~ n \geq 1\} \subseteq B(x_N,R)$. Hence $\{x_n\}$ is bounded.	
\end{rmk*}


\begin{thm}
	\begin{enumerate}[wide,label = $(\alph*)$]
		\item Every Cauchy sequence in a compact metric space converges.
		\item Every Cauchy sequence in $\R^k$ convergences.
	\end{enumerate}
	
\end{thm}

\begin{proof}
	\begin{enumerate}[wide,label = $(\alph*)$]
		\item Let $\{x_n\}$ be a Cauchy sequence in compact metric space $X$. Since $X$ is compact, $X$ is sequencially compact, so $\{x_n\}$ has a subsequence $\{x_{n_k}\} \ni x_{n_k} \rightarrow p$ as $k \rightarrow \infty$ for some $p \in X$.
		
		Now to prove $x_n \rightarrow p$. Given $\epsilon > 0, \exists N \in \N \ni \forall n,m > N , d(x_n,x_m) < \dfrac{\epsilon}{2}(\because \{x_n\} \text{ is Cauchy}),~ \exists k_0 \in \N \ni \forall k \geq k_0, d(x_{n_k},p) < \dfrac{\epsilon}{2}(\because x_{n_k} \rightarrow p)$. Hence, $\forall n \geq N, d(x_n,p) \leq d(x_n,x_{n_k}) + d(x_{n_k},p) < \dfrac{\epsilon}{2} + \dfrac{\epsilon}{2} < \epsilon$, where $k >> 0 $
		
		Another proof of (a)
		
		By $(2) \& (4)$ above,
		
		$$\lim_{n \rightarrow \infty} \text{tia}(\overline{E}_n) = \lim{m \rightarrow \infty} \text{dia}(E_n) = 0$$,
		
		where $E = \{x_n , x_{n+1},\cdots\}, n \geq 1$
		
		Now, each $\overline{E_n}$ is compact($\because$ closed subset of compact set $X$) and nonempty $\forall n \geq 1$, and $\{\overline{E}_n\}$ is decreasing($\because E_n \supseteq E_{n+1} \implies \overline{E}_n \supseteq \overline{E}_{n+1}$) and dia($\overline{E}_n \rightarrow 0$). By Cantor's Intersection Theorem, $\bigcap^{\infty}_{n = 1}\overline{E_n} = \{p\}$
		Claim: $x_n \rightarrow p$ as $n \rightarrow \infty$, $\forall \epsilon > 0, \exists N \in \N \rightarrow \text{ dia}(\overline{E}_n) < \epsilon$. Since $p \in \overline{E}_n \forall n \geq 1, \forall n \geq N \& d(x_n,p) < \text{dia}(\overline{E}_n) < \epsilon,~\therefore x_n \rightarrow p$
		
		\item Given a Cauchy sequence $\{x_n\}$ in $\R^k$. Then $\{x_n\}$ is bounded, choose a large $k$-dimensional closed interval
		
		$$I = [a_1,b_1] \times \cdots \times [a_n,b_n]$$
		
		$\ni x_n \in I \forall n \geq 1$. Now, H.B. Theorem says that $I$ is compact. Therefore, $\{x_n\}$ becomes a Cauchy sequence in the compact metric space $I$. By (a) $x_n \rightarrow p$ for some $p \in I$. This proves (b).
	\end{enumerate}
	
\end{proof}

\begin{defn}
	A metric space $X$	is said to be complete if every Cauchy sequence in $X$ convergences. 
\end{defn}

\textbf{Rmks and Examples}

\begin{enumerate}[wide]
	\item In a complete metric space $X$, a sequence $\{x_n\}$ is Cauchy $\Leftrightarrow$ it convergences.
	\item By Thm. 2.6, we have two classes of complete metric spaces
	\begin{enumerate}[label = $\bullet$]
		\item Compact metric space
		\item Euclidean space $\R^k$
	\end{enumerate}
	
	In fact, $\R^k$ is a Banach Space(complete normed linear space) and Hilbert space
	\item A closed subset $S$ of a complete metric space $X$ is complete.
	
	$\because$ Let $\{x_n\}$ be a Cauchy sequence in $S$. Then $\{x_n\}$ is a Cauchy sequence in $X$, hence, $x_n \rightarrow p$ for some $p \in X$. So $p \in \overline{S} = S$. Hence, $S$ is complete.
	\item Every closed subset of $\R^k$ is complete.
	
	$\because$(3), In particular, every closed interval and closed ball in $\R^k$,
	
	\item $(0,1)$ and $\Q$ are not complete

\end{enumerate}


\begin{defn}
	Let $\{x_n\}$ be a real sequence
	
	\begin{enumerate}[wide, label = $(\roman*)$]
		\item We say that $\{x_n\}$ is increasing, if $x_n \leq x_{n+1} \forall n \geq 1$
		\item We say that $\{x_n\}$ is strictly increasing, if $x_n < x_{n+1} \forall n \geq 1$
		\item We say that $\{x_n\}$ is decreasing, if $x_n \geq x_{n+1} \forall n \geq 1$
		\item We say that $\{x_n\}$ is strictly decreasing, if $x_n > x_{n+1} \forall n \geq 1$
		\item We say that $\{x_n\}$ is monotonic if either $\{x_n\}$ is increasing or decreasing.
		\item We say that $\{x_n\}$ is strictly monotonic if either $\{x_n\}$ is strictly increasing or strictly decreasing
	\end{enumerate}
	
\end{defn}

\textbf{Examples}

\begin{enumerate}[wide,label = $\bullet$]
	\item $\{2n + 1\}$ is increasing, $2n+1 \rightarrow + \infty$
	\item $\{-n\}$ is decreasing, $-n \rightarrow - \infty$
	\item $\{\dfrac{1}{n}\}$ is decreasing, $\dfrac{1}{n} \rightarrow 0$
	\item $\{\dfrac{1}{-n}\}$ is increasing, $\dfrac{1}{-n} \rightarrow 0$
\end{enumerate}

We will show that every monotonic sequence convergences in $\R^* = [-\infty,\infty]$


\begin{thm}
	Let $\{a_n\}$ be a sequence	
	\begin{enumerate}[wide,label = ($\alph*$)]
			\item Let $\{a_n\}$ be increasing
			\begin{enumerate}[label = ($\roman*$)]
				\item If $\{a_n\}$ is bounded above, then $\{a_n\}$ convergence, in fact, $a_n \rightarrow \text{sup}a_n = \text{sup}\{a_n~|~ n \geq 1\}$
				\item If $\{a_n\}$ is not bonded above, then $a_n \rightarrow \infty$
			\end{enumerate}
			\item Let $\{a_n\}$ be decreasing
			\begin{enumerate}
				\item If $\{a_n\}$ is bounded below, then $\{a_n\}$ convergences, in fact $a_n \rightarrow \text{inf} a_n = \text{inf}\{a_n~|~ n \geq 1\}$
				\item If $\{a_n\}$ is not bounded below, them $\{a_n\} \rightarrow - \infty$
			\end{enumerate}
	\end{enumerate}

\end{thm}

\begin{rmk*}
	$\{a_n\}$ is increasing $\Leftrightarrow \{-a_n\}$ is decreasing. So, to study monotonic sequence, it suffices to consider the case of increasing sequence.	
\end{rmk*}

\begin{proof}

	It suffices to prove (a) By same argument or considering $\{-a_n\}$, one can prove
	
	\begin{enumerate}[label = $(\alph*)$]
		\item \begin{enumerate}[label = $(\roman* )$]
				\item $\{a_n\}$ is bounded above $\implies \{a_n ~|~ n \geq 1\}$ is bounded $\implies \alpha = \text{sup}a_n$ exists and is finite
			
				Claim: $a_n \rightarrow \alpha$ as $n \rightarrow \infty$.
			
				Given $\epsilon > 0 \exists n_0 \in \N \rightarrow \alpha - \epsilon < a_{n_0}$. Then $\forall n \geq n_0$, we have $\alpha - \epsilon < a_{n_0} \leq a_n \leq \alpha < \alpha + \epsilon$, i.e. $\forall n \geq n_0$, $|a_n - \alpha| < \epsilon$. This proves $a_n \rightarrow \alpha$
			\item $\forall M > 0$, since $\{a_n\}$ is not bounded above, $\exists n_0 \in \N \rightarrow a_{n_0} \geq M \implies \forall n \geq M, a_n \geq a_{n_0} \geq M$. This proves $a_n \rightarrow + \infty$
		\end{enumerate}
		
		\item is similar
		
		\begin{enumerate}[wide,label = $(\roman* ')$]
			\item $\{a_n\}$ is bounded below $\implies \{-a_n\}$ is bounded above $\implies \lim_{n \rightarrow \infty}(-a_n) = \text{sup}(-a_n) \implies -\lim a_n = - \inf a_n$
			\item Similar
			
		\end{enumerate}
		
		\end{enumerate}

	
\end{proof}

\begin{rmk*}
	Let $\{a_n\}$ be a monotonic sequence, then $\{a_n\}$ convergences $\Leftrightarrow \{a_n\}$ is bounded.	
\end{rmk*}


%Here is the start for 11/16
	%thm 3.6
	\begin{thm}$ $
 		
	\begin{enumerate}
			
		\item Every Cauchy sequence $\{x_n\}$ in compact metric metric space $X$ is convergent.
		\item In $\R^k$, every Cauchy sequence is convergent.  
	
	\end{enumerate}
	\end{thm}
	\begin{proof}$ $

		\begin{enumerate}
		\item Let $E_n = \{x_n,x_{n+1},\cdots\}$ $n\geq 1$, by (2) and (4) $\lim_{n\to \infty} diam(\bar{E_n}) = \lim_{n\to\infty} diam(E_n) = 0\ldots (\star)$. Clearly, $\{E_n\}$ is decreasing so is $\{\bar{E_n}\}$. $\bar{E_n} = \{x_n.x_{n+1}\} \supseteq \{x_{n+1},\cdots\} = \bar{E_{n+1}}$. In fact, $\{\bar{E_n}\}$ is a decreasing sequence of nonempty compact sets in $X$. Hence, by Cantor intersection theorem, $\bigcap_{n=1}^{\infty} E_n \neq \varnothing$. In fact, $\bigcap_{n=1}^{\infty}\overline{E_n} = \{p\}$ by cor 2.16.
			
			\textbf{Claim :} $x_n \to p$. $\forall \varepsilon > 0$, by ($\star$), $\exists\ N\in \N $ s.t. $diam(\overline{E_n}) - 0 < \varepsilon$ $\therefore x_n \to p $ as $n\to \infty$.
		\item Since Cauchy sequence is bounded, $\exists\ R >>0$ s.t. $x_n\in\bar{B}(0,R)$, i.e. $\{x_n\}$ is a Cauchy sequencein the compact set $\bar{B}(0,R)$. Hence $\{x_n\}$ converges in $\bar{B}(0,R)$. 
		\end{enumerate}
	\end{proof}

	\begin{defn}
		A metric space $X$ is said to be complete if every Cauchy sequence in $X$ is convergent.
	\end{defn}
	
	\textbf{Remarks and Examples : }
	\begin{enumerate}
	\item I a complete metric space, to show that a sequence is convergent, it is enough to show that it is Cauchy.
\item Thm 3.8 says compact metric space is complete. $\R^k$ is complete. Discrete metric space is complete. $X$ be a discrete metric space, $\{x_n\}$ is convergent $\Leftrightarrow$ $\{x_n\}$ is almost constant, i.e. $\exists n_0\in\N$ s.t. $\forall n\geq n_0$, $x_n = x_{n_0}$. $\{x_n\}$ is Cauchy $\Leftrightarrow$ $\forall\ 0 < \varepsilon < 1$, $\exists N \in \N$ s.t.$\forall\ m,n \geq N$  $d(x_m,x_n) < \varepsilon < 1 \Leftrightarrow \forall\ m,n \geq N, x_m = x_{n} = \cdots = x_N$.
\item Complete normed linear space over $(\mathcal{F})$ is called a Banach space, so $\R^k$ is a Banach space.
\item A Hilbert space is a complete inner product space over $(\mathcal{F})$.
\item Closed subset $F$ is a complete metric space $X$ is complete. ($\because $ Given a Cauchy sequence $\{x_n\}$ in $F$, so it is also a Cauchy sequence in $X$. $X$ is complete, $x_n\to p$ as $n\to \infty$ for some $p\in X$. By thm, $p\in\bar{F} = F$($\because F$ is closed) $\therefore F$ is complete ).
\item The closeness in (5) is necessary, e.g. $(0,1) \subseteq \R$ and $\R$ is complete, $\{1/n\}$ is a Cauchy sequence in $(0,1)$ but it does not converge in $(0,1)$.
\item Every closed ball and closed bounded interval in $\R^k$ are complete.
\item $(0,1)$ , $\Q$, $\Q^k$ are not complete. Now we prove a class of real sequence which has limit, namely, all nonconstant sequence has limits in $\R^k = [-\infty ,\infty]$.
	\end{enumerate}

\begin{defn}
Let $\{x_n\}$ be a real sequence.
	\begin{enumerate}
	\item $\{x_n\}$ is incerasing, $\{x_n\}$ is $\nearrow$, if $x_n \leq x_{n+1}$, $\forall\ n\geq 1$
\item  $\{x_n\}$ is strictly incerasing, $\{x_n\}$ is st $\nearrow$, if $x_n < x_{n+1}$, $\forall\ n\geq 1$
\item  $\{x_n\}$ is decerasing, $\{x_n\}$ is $\searrow$ , if $x_n \geq x_{n+1}$, $\forall\ n\geq 1$
\item  $\{x_n\}$ is strictly decerasing, $\{x_n\}$ is st $\searrow$, if $x_n > x_{n+1}$, $\forall\ n\geq 1$
\item We say that $\{x_n\}$ is monotonic if either $\{x_n\}$ is increasing or decreasing. (Similar description in strictly case).
	\end{enumerate}
\end{defn}
Note that $\{x_n\}$ is $\nearrow\ \Leftrightarrow \{-x_n\}$ is $\searrow$ , so we only discuss increasing sequence. 

\textbf{Examples :}
\begin{enumerate}
\item $\{2n+1\}$ is $\nearrow$ but it diverges. In fact, $\lim_{n\to\infty}(2n+1) = +\infty = \sup \{2n+1\ |\ n\geq 1\}$. (not bounded above)
\item $\{1/n\}$ is $\searrow$ and it converges, $\lim_{n\to\infty} 1/n = 0 = \inf \{1/n\ |\ n\geq 1\}$.
\end{enumerate} 
%thm3.7
\begin{thm}Let $\{a_n\}$ be a real sequence, we have
 	\begin{enumerate}
 	\item If $a_n$ is $\nearrow$ , then $\lim_{n\to\infty} a_n = \sup_{n\geq 1} a_n  = \sup\{a_n\ |\ n \geq 1\}$
	\item If $a_n$ is $\searrow$ , then $\lim_{n\to\infty} a_n = \inf_{n\geq 1} a_n  = \inf\{a_n\ |\ n \geq 1\}$
	\item If $a_n$ is monotonic , then $\{a_n\}$ is convergent $\Leftrightarrow$ it is bounded. 
 	\end{enumerate}
\end{thm}
	
	\begin{proof}$ $
		\begin{enumerate}
		\item We have 2 cases, (i) $\{a_n\}$ is not bounded above, i.e. $\sup_{n \geq 1} a_n = +\infty$. Given $M > 0$, $\because \{a_n\}$ is not bounded above, $\exists N \in \N$ s.t. $a_N \geq M \implies \forall\ n \geq N$, $a_n\geq a_N \geq M \therefore \lim_{n\to\infty } = +\infty$. (ii) $\{a_n\}$ is bounded above. Then $\lim_{n\geq 1}a_n = \alpha\in \R$. Now, we claim that $\lim_{n\to\infty}a_n = +\infty$. $\forall\ \varepsilon > 0,  \exists\ N\in\N$ s.t. $\alpha - \varepsilon < a_N \implies \forall\ n\geq N$, $\alpha +\varepsilon > \alpha \geq a_n\geq a_N > \alpha-\varepsilon$ i.e. $|a_n-\alpha| < \varepsilon$, i.e. $\lim_{n\to\infty} a_n = \alpha$.
		\item First, we use (1) to prove it. $\{a_n\}$ is $\searrow$ $\implies\{-a_n\}$ is $\nearrow$, by (1) $\implies \lim_{n\to\infty}(-a_n) = \sup_{n \geq 1}(-a_n) = -\inf_{n \geq 1} a_n$. Next, using the argument of (1). (i) $\{a_n\}$ is not bounded below i.e. $\inf_{n \geq 1}a_n = -\infty$. Given $M > 0, \exists\ N\in\N $ s.t. $a_N \leq -M$, $\forall n\geq N$ $a_n \leq a_N \leq -M \implies \lim_{n\to\infty}a_n = - \infty =\inf_{n\geq 1}a_n$ (ii) $\{a_n\}$ is bounded below , i.e. $\inf_{n\geq 1}a_n = \beta$. Now, we claim $\lim_{n\to\infty}a_n = \beta$. $\forall\ \varepsilon> 0\exists\ N\in\N$ s.t. $a_N < \beta + \varepsilon\implies\forall\ n\geq N , \beta-\varepsilon<\beta\leq a_n\leq a_N\leq \beta+\varepsilon\implies\forall\ n\geq N, |a_n-\beta|<\varepsilon$ i.e. $\lim_{n\to\infty}a_n = \beta$.
		\item Follows from (1) and (2). $\because \{a_n\}$ is monotonic, ($\Rightarrow$) Suppose $\{a_n\}$ converges, then $\{a_n\}$ is bounded. ($\Leftarrow$) Suppose $\{a_n\}$ is bounded, so $\{a_n\}$ is bounded above and below. If $\{a_n\}$ is $\nearrow$, then by (1) $\{a_n\}$ converges. If $\{a_n\}$ is $\searrow$, then by (2) $\{a_n\}$ converges.    
	\end{enumerate}
	\end{proof}

	\begin{rmk*}
	In the extended real line or system $\R^{\star} = [-\infty,\infty]$, all monotonic sequence converges in $\R^{\star}$.
	\end{rmk*}

\subsection{Limit Superior and Limit inferior}

	\begin{defn}%this definition should be checked
		Given a real sequence $\{x_n\}$. Let 
   \begin{flalign*}
   E & = \{x\in\R^{\star}\ |\ x\ \text{is a subsequence of}\ \{x_n\}\}&\\
	 & = \{x\in\R\ |\ x\ \text{is a subsequence of}\ \{x_n\}\}\cup \{\infty\}\ \text{if $\infty$ is a subsequence limit of $\{x_n\}$}&\\
     & = \{x\in\R\ |\ x\ \text{is a subsequence of}\ \{x_n\}\}\cup \{-\infty\}\ \text{or} & \\
	 & = \{x\in\R\ |\ x\ \text{is a subsequence of}\ \{x_n\}\}\cup \{-\infty,\infty\} &		
	\end{flalign*}	
	\end{defn}
\textbf{Claim : }$E\neq\varnothing$.
\begin{itemize}
\item $\{x_n\}$ is not bounded above $\implies \exists$ a subsequence $\{x_{n_k}\}$ of $\{x_n\}$ s.t. $x_{n_k}\to\infty$ as $k\to\infty$. $\forall\ k=1, \exists\ n_1$ s.t. $x_{n_1} > 1$, $\exists\ n_2 > n_1$ s.t. $x_{n_2} > \max\{x_{n_1},2\}\cdots$ $\implies x_{n_k} > k,\ \forall\ k\geq 1 \implies \lim_{n\to\infty}x_{n_k} = \infty\,,\,\therefore \infty \in E$.
\item $\{x_n\}$ is bounded above and below $\implies x_n \in [ -a,a]\ \forall\ n\geq 1$ for some $a > 0\implies \{x_n\}$ has a convergent subsequence say $x_n\to x\in[a,-a]\,,\,\therefore x\in E$.
\item Therefore in any case, $E\subseteq \R^{\star}$ is nonempty, so $\sup E$ and $\inf E$ exists in $\R^{\star}$. 
\end{itemize}
\begin{defn}
Define the limit superior or say upper limit of $\{x_n\}$ is denoted by $$\limsup_{n\to\infty}x_n = \overline{\lim_{n\to\infty}}x_n = x^{\star} = \sup E $$ The limit inferior or lower limit of $\{x_n\}$ is denoted by $$ \liminf_{n\to\infty}x_n = \underline{\lim_{n\to\infty}} = x_{\star} = \inf E $$  The following are equivalent : 
	$$ \limsup_{n\to\infty}a_n = \inf_{n\geq 1}\left(\sup_{k\geq 1}a_k\right) = \lim_{n\to\infty}\left(\sup_{k\geq 1}a_k\right) $$ and $$ \liminf_{n\to\infty}a_n = \sup_{n\geq 1}\left(\inf_{k\geq 1}a_k\right) = \lim_{n\to\infty}\left(\inf_{k\geq 1}a_k\right) $$
\end{defn}
Let $X$ be a set, $\{A_n\}$ a sequence of subset of $X$. Define $$\limsup_{n\to\infty}A_n = \overline{\lim_{n\to\infty}}A_n = \bigcap_{n=1}^{\infty}\left(\bigcup_{k=n}^{\infty}A_k\right) $$ and $$\liminf_{n\to\infty}A_n = \underline{\lim_{n\to\infty}}A_n = \bigcup_{n=1}^{\infty}\left(\bigcap_{n=k}^{\infty}A_k\right) $$
We say that the limit exists if $\limsup_{n\to\infty} A_n = \liminf_{n\to\infty}A_n$. In this case, $\lim_{n\to\infty}A_n = \limsup_{n\to\infty} A_n = \liminf_{n\to\infty} A_n$

\textbf{Example :} Let $X = \{0,1\}$ then $A_{2n} = \{0\}$ and $A_{2n-1} = \{1\}$, $\limsup_{n\to\infty}A_n = \{0,1\}$ and $\liminf_{n\to\infty}A_n = \varnothing$.

\textbf{Facts : }$x\in\limsup_{n\to\infty}A_n \Leftrightarrow x\in \cap_{k=n}^{\infty}A_k\ \forall\ n\geq 1 \Leftrightarrow \{x\in X\ |\ x\in A_n\ \text{infinitely open}\}$ and $x\in \liminf_{n\to\infty} A_n \Leftrightarrow x\in \cup_{n=1}^{\infty}(\cap_{n=k}^{\infty}A_n)\Leftrightarrow \exists\ n_0\in\N $ s.t. $x\in\cap_{k=n_0}^{\infty}A_k\Leftrightarrow \{x\in X\ |\ x\in A_n\ \text{almost always}\}$.
\textbf{Examples : } If $\{A_n\}$ is $\nearrow (\searrow)$, then $\lim_{n\to\infty}A_n = \cup_{n=1}^{\infty} A_n(\cap_{n=1}^{\infty} A_n)$ also $\limsup_{n\to\infty}A_n = \lim_{n\to\infty}A_n = \liminf_{n\to\infty} A_n$.
%Here is the end of 11/16





















