\section{Basic Point Set Topology}

To know the "closeness", "limit" and "continue"

\textbf{Notation}. Let $X$ be a nonempty set. The power set of $X$ is denoted by $p(X)$ or $2^X$, i.e. $\mathscr{P}(X) = 2^X$ which is the collect of all subset, if $|X| = n$, then $|\mathscr{P}(X)| = 2^n$

\subsection{Topological Spaces}

\begin{defn}
	Let $X$ be a nonempty set and $\mathscr{T} \subseteq \mathscr{P}(X)$, we say that $\mathscr{T}$ is a topology on $X$ if it satisfies
	
	\begin{enumerate}
		\item $\emptyset,X \in \mathscr{T}$
		\item $\mathscr{T}$ is closed under arbitrary union, \\i.e. $U_{\alpha} \in \mathscr{T},~\alpha \in I \implies \bigcup_{\alpha \in I} U_{\alpha} \in \mathscr{T}$
		\item $\mathscr{T}$ is closed under finite intersection \\i.e. $U_1,\cdots,U_n \in \mathscr{T} \implies U_1 \cap \cdots \cap U_n \in \mathscr{T}$
	\end{enumerate}
	
	In this chapter, the pair $(X,J)$ or simply $J$ is called a topological space and members in $T$ are called open set in $X$ or open subsets of $X$
\end{defn}

\begin{rmk*}$ $
	\begin{enumerate}
		\item $X$: a nonempty set, there is at least two trivial topology on $X$
			\begin{enumerate}
				\item[$\bullet$] $\mathscr{P}$(x) is the largest topology on $X$ w.r.t inclusion $X$ with this topology is called a discrete topological space
				\item[$\bullet$] $\mathscr{T}_0 = \{\emptyset, X\}$ is the smallest topology on $X$ w.r.t inclusion $X$ with this topology is called an indiscrete topological space
			\end{enumerate}
		\item How many topology can be define on $\{a\},~\{a,b\}$?
	\end{enumerate}
\end{rmk*}

In the following $X$ is a topology space

\begin{defn}[neighborhood]
	Let $p \in X$, a neighborhood of $P$ is an open set $U$ containing $p$  
\end{defn}

\begin{defn}[Hausdorff space]
	$X$ is a Hausdorff space if any two distinct points can be separated by open set, i.e. $\forall ~p \neq q$ in $X,~\exists$ neighborhood $U$ of $p$ and $V$ of $q \in U \cap V = \emptyset$
\end{defn}

\begin{defn}[closed set]
	A subset $F \subseteq X$ is said to be closed if $F^C = X - F$ is open in $X$
\end{defn}

\begin{thm}
	The collection of all closed subsets of $X$ satisfied
	
	\begin{enumerate}
		\item[(a)] $\emptyset, X$ are closed
		\item[(b)] Arbitrary intersection of closed set if closed
		\item[(c)] Finite union of closed sets is closed
	\end{enumerate}
\end{thm} 
\newpage

\begin{proof}$ $\\
	\begin{enumerate}
		\item[(a)] $X - \emptyset = X $ is open $~\therefore \emptyset$ is closed\\
	$X - X = \emptyset$ is open $~\therefore X$ is closed
		\item[(b)] Given closed sets $F_{\alpha},\alpha \in I$, $X - \bigcap_{\alpha \in I}F_{\alpha} = \bigcup_{\alpha - I}(X - F_{\alpha})$ is open,$\therefore \bigcap_{\alpha - I}F_{\alpha}$ is closed.
		\item[(c)] Given closed set $F_1,\cdots,F_n$, $X - \bigcup_{i=1}^{n}F_i = \bigcap^n_{i=1}(X-F_i)$ is open, $\therefore \bigcup^n_{i=1}F_i$ is closed.
	\end{enumerate}
	
\end{proof}

\begin{defn}
	Let $Y \subseteq X$ and
	
	$$\T_y = \{U \cap Y ~|~U \text{ is open in }X\}$$
\end{defn}

\begin{thm}
	$\mathscr T_Y$ is also a topology space
\end{thm}

\begin{proof}
	To proof $\mathscr T_Y$ is a topology space, we take the topology's definition
	\begin{enumerate}
		\item[(a)] $\emptyset,Y \in \T_Y$ ($\because \emptyset = \emptyset \cap Y,~Y = X \cap Y$)
		\item[(b)] Given $U_{\alpha} \cap Y \in \T_Y,~\alpha \in I,$ where $U_{\alpha}$ is open in $X$
		
		$\bigcup_{\alpha \in I}(U_{\alpha \cap Y}) = (\bigcup_{\alpha \in I}U_{\alpha})\cap Y \implies \bigcup_{\alpha \in I}(U_{\alpha} \cap Y) \in \T_Y$
		\item[(c)] Given $U_1 \cap Y,\cdots,U_n \cap Y$, where $U_i$ is open in $X,~1 \leq i \leq n$
		
		$\cap^n_{i=1}(U_i \cap Y) = (\cap^n_{i=1}U_i) \cap Y \implies \bigcap^n_{i=1}(U_i \cap Y) \in \T_Y$
		
		$\therefore \T_Y$ is a topology on $Y$
	\end{enumerate}
\end{proof}

\begin{defn}
	In Theorem 2.2, with the topology $\T_Y$ on $Y$, is called a topological subspace of $X$ and $\T_Y$ is called the relative topology of $Y$ in $X$. Members in $\T_Y$ are called open set in $Y$ or relative open sets in $Y$.
\end{defn}

\newpage

\subsection{Metric Spaces \& Subspace}$ $

In this chapter, we will introduce a class of topology space whose topology in induced by a metric.
\begin{defn}
	Let $X$ be a nonempty set. A metric or distance function in a function
	
	$$d:X\times X \rightarrow \mathbb R,~(a,b) \mapsto d(a,b)$$
	
	satisfying:
	
	\begin{enumerate}
		\item[(a)] $\forall a,b \in X,d(a,b) \geq 0$ and $d(a,b) = 0 \Leftrightarrow a = b$
		\item[(b)] $\forall a,b \in X,d(a,b) = d(b,a)$ \textbf{symmetry}
		\item[(c)] $\forall a,b,c \in X,d(a,b) \leq d(a,c)+d(a,b)$ \textbf{triangle inequaity}
	\end{enumerate}
	
	if $d$ is a metric on $X$, then the pair $(X,d)$ or simply $X$ is called a metric space and $\forall a,b \in X,~d(a,b)$ is called the distance between $a ~\&~b$
\end{defn}

\textbf{Examples}

\begin{enumerate}
	\item Let $X$ be a nonempty set define by
	
	$$d(a,b) = \begin{cases}
		0 \text{ if } a = b\\ 1 \text{ if } a \neq b
	\end{cases}$$
	
	Then $d$ is a metric on $X$, called the discrete metric and with this metric $X$ is called a discrete metric space. In particular, any set admits a metric.
	\item The most important metric spaces are the Euclidean space $\mathbb R^k$, the metric $d$ is called the Euclidean or standard or usual metric on $\mathbb R^k$.There are other metrics on $\mathbb R^k$ induced the same metric topology on $\mathbb R^k$, in fact, they are all equivalent, e.g $\forall~ 1 \leq p \leq \infty$, We can define a metric $d_p$ on $\mathbb R^k$ as follows
		\begin{enumerate}
			\item[$\bullet$] $1 \leq p < \infty,~d_p(x,y) = ||x-y||_p = \left( \sum^k_{i=1}|x_i-y_i|^p\right)^{\frac{1}{p}}$
			\item[$\bullet$] $p = \infty,~d_{\infty}(x,y) = \max_{1 \leq i \leq k}|x_i - y_i|$
		\end{enumerate}
		
		Note that $d_2 = d$ is the Euclidean metric on $\mathbb R^k$
		
		\begin{rmk*}
			In fact, every normed linear space$(V,||\cdot||)$ is a metric space whose metric is induced by its norm
		\end{rmk*}
	\item Let $(X,d)$ be a metric space and $Y \subseteq X,~Y \neq \emptyset$. Then the restriction of $d$ to $Y \times Y$ is also a metric on $Y$, with this metric, $Y$ is called a metric subspace of $X$
	
\end{enumerate}

\newpage

\begin{defn}[ball]
	Given $p \in X ~\&~ r > 0$
	
	$B(p,r) = \{x \in X ~|~ d(x,p) < r\}:$ open ball with center $p$ and radius $r$
	
	$\overline{B}(p,r) = \{x \in X ~|~ d(x,p) \leq r\}:$ closed ball with center $p$ and radius $r$ 
\end{defn}

\textbf{Example}$ $

\begin{enumerate}
	\item The discrete metric space $X:~ p \in X,~r>0$
	
	$B(p,r) = \begin{cases}
		\{p\} \text{ if } 0 \leq r \leq 1\\ X \text{ if } r > 1
	\end{cases}$
	
	$\overline{B}(p,r) = \begin{cases}
		\{p\} \text{ if } 0 \leq r < 1\\ X \text{ if } r \geq 1
	\end{cases}$
	
	\item In the Euclidean space $\mathbb R^k$,~ $p \in \mathbb R^k,~r>0$
	
	$B(p,r) = \{x \in \mathbb R~|~ \norm{x-p} < r \}$ is a ""true"" open
	
	$\overline{B}(p,r) = \{ x \in \mathbb R ~|~ \norm{x - p} \leq r \} $ is a "true" closed ball
	
	In particular, for $k = 1$ in $\mathbb R$
	
	$B(p,r) = (p-r,p+r):$ a symmetric opne interval
	
	$\overline{B}(p,r) = [p-r,p+r]:$ a symmetric close interval
	
	However, w.r.t $d_1 ~\&~ d_{\infty}$, we have, e.g. in $\mathbb R^2$
	
	$B_1(0,1) = \{(x,y)~|~|x-0|+|y-0|<1\}$
	
	$B_{\infty}(0,1) = \{(x,y) ~|~ \max\{|x|,|y|\} \leq 1\}$
	\item In $\C,~z \in \C , r>0$
	
	$$D(z,r) = \{w \in \C ~|~ |w-z| < r\}$$
	
	is called an open disk with center $z$ and radius $r$ 
	
	
	\item What is the open balls in $S = [0,1] \subseteq \mathbb R$?
	
	$B_S(0,\frac{1}{2}) = \{x \in S ~|~ |x - 0| < \frac{1}{2}\} = [0,\frac{1}{2}] = B(0,\frac{1}{2}) \cap [0,1]$
	
	$B_S(0,3) = [0,1] = B(0,3) \cap [0,1]$

\end{enumerate}

\textbf{Prop 2.2} Let $S$ be a metric subspace of a metric space $X$, then $\forall ~p \in S ~\&~ r > 0$, $B_S(p,r) = B(p,r) \cap S$

\begin{proof}
	$B_S(p,r) = \{x \in S ~|~ d(x,p) < r\} = \{x \in X ~|~ d(x,p) < r\} \cap S\\=B(p,r) \cap S$
\end{proof}

\subsection{Open Sets in Metric Spaces} $ $

We will see that every metric on a set induce a topology on $X$

\begin{defn}[interior point]
	Let $S \subseteq X$ be a set, and $p \in S$, we say that $p$ is an interior point of $S$ if $\exists~ r > 0,~\exists~B(p,r) \subseteq S$
	
	Denote by $S^o$ or int($S$) by the set of all interior point of $S$
\end{defn}

\begin{defn}[open]
	Let $S \subseteq X$, we say that $S$ is open if all points of $S$ are interior points of $S$
\end{defn}

\newpage

\begin{rmk*}$ $
	\begin{enumerate}[wide]
		\item Every open set $S$ is a union of a open balls in $X$.
		
		$\because \forall~x \in S,x$ is an interior point of $S$, $\exists r_x > 0 \ni B(x,r_x) \subseteq S$
		
		$\therefore S = \cup_{x \in S}B(x,r_x)$
		
		\item $S^o \subseteq S$ by definition
		\item $S$ is open $\Leftrightarrow S = S^o$
	\end{enumerate}
\end{rmk*}

\textbf{Prop 2.3}

\begin{enumerate}[wide,label=\textbf{(\alph*)}] 
	\item $S \subseteq T \implies S^o \subseteq T^o$ 
	
	$$\because p \in S^o \implies \exists~r>0 \ni B(p,r) \subseteq S \subseteq T \implies p \in T^o$$
	\item Every open ball $B(p,r)$ in $X$ is open
	
	$\because$ Give $q \in B(p,r)$, Let $\delta = r = d(p,q)$.Claim $B(q,\delta) \subseteq B(p,r)$ which says $q$ is an interior point of $B(p,r)$.Since $q \in B(p,r)$ is arbitrary, so $B(p,r)$ is open. Given $x \in B(q,\delta)$
	
	$$d(x,p) \leq d(x,q) + d(q,p) < \delta + d(q,p) = r - d(p,q)+d(q,p)=r$$
	\item $\forall~S \subseteq X,~S^o$ is always open
	
	$\because$ Given $p \in S^o,~\exists~r>0 \ni B(p,r) \subseteq S$
	
	$$\implies B(p,r) \subseteq S^o \implies p \text{ is a interior point of }S^o$$
	
	$\therefore S^o$ is open
	
	\item $\forall~ S \subseteq X,~ S^{oo} = (S^o)^o = S^o$
	 
	$\because$ by definition of open set and (c)\\
\end{enumerate}


Now, let $T = \{U \subseteq X ~|~ U \text{ is open in }X\}$

\textbf{Prop 2.4} $T$ is a topology on $X$. In particular, $X$ is a topology space.


\begin{proof}$ $
	\begin{enumerate}[wide,label = (\textbf{\roman*})]
		\item $\emptyset, X \in \T$, $\because \emptyset = \emptyset,~X^o = X$
		\item $U_{\alpha} \in \T, a \in I$ are open $\implies \bigcup_{\alpha \in I}U_{\alpha}$ is open
		
		Given an arbitrary point $p \in \bigcup_{\alpha \in I}U_{\alpha} \implies \exists \alpha_0 \in I \ni p \in U_{\alpha_0}$
		
		$U_{\alpha_0}$ is open, $\exists r > 0 \ni B(p,r) \subseteq U_{\alpha_0} \subseteq \bigcup_{\alpha \in I}U_{\alpha}$
		
		$\because p$ is an interior point of $\bigcup_{\alpha \in I}U_{\alpha}~\therefore \bigcup_{\alpha \in I}U_{\alpha}$ is open, i.e. $\bigcup_{\alpha \in I}U_{\alpha} \in T$
		\item $U_1,\cdots,U_n \in T \implies U_1 \cap \cdots \cap U_n \in T$
		 
		$\because$ Given $p \in U_1 \cap \cdots \cap U_n \implies p \in U_i~1\leq i\leq n$. Each $U_i$ is open, $\exists r_i > 0~B(p,r_i) \subseteq U_i,~1\leq i \leq n \implies B(p,r) \subseteq U_1 \cap \cdots \cap U_n$\\
		$\implies p$ is an interior point of $U_1 \cap \cdots \cap U_n$
		
		p is arbitrary, so $U_1 \cap \cdots \cap U_n$ is open, i.e. $U_1 \cap \cdots \cap U_n \in T$
		
		Therefore, $T$ is a topology on $X$
	\end{enumerate}
\end{proof}

\begin{defn}
	Let $X$ be a metric space with metric $d$. The topology $T$ in prop 2.4 is called the metric topology(include by $d$)
\end{defn}

Let $X$ be a metric space and $Y \subseteq X$, Then $Y$ is a metric subspace of $X$, and $\forall y \in Y,~r > 0,~B_Y(y,r) = B(y,r) \cap Y$. In fact, we have more\\

\textbf{Prop 2.5} A subset $A \subseteq Y$ is open in $Y \Leftrightarrow A = U \cap Y$ for some open set $U$ in $X$, in particular, the metric topology on $T$ is just the relation topology of $Y$ on $X$

\begin{proof}$ $
	($\Rightarrow$) suppose $A \subseteq Y$ is open in $Y$. Then
	
	$$A = \bigcup_{y \in A}B_{Y}(y,r_y) = \bigcup_{y \in A}(B(y,r_y)\cap Y) = (\bigcup_{y \in A}B(y,r_y))\cap Y$$
	
	Let $U = \bigcup_{y \in Y}B(y,r_y)$, then $U$ is open in $X$ and $A = U \cap Y$
	
	($\Leftarrow$) Suppose $A = U \cap Y$ where $U \subseteq X$ is open $\forall ~ y \in A, y\in U \cap Y \implies y \in U \implies \exists r > 0 \ni B(y,r) \subseteq U \implies B(y,r) \cap Y \subseteq U \cap Y = A \implies B_Y(y,r) \subseteq A$, $~\therefore A$ is open in $Y$.
\end{proof} 

\textbf{Prop 2.6} Every metric space $X$ is Hausdorff

\begin{proof}
	Given $p,q \in X,~p\neq q$. Choose $r = \frac{1}{2}d(p,q)>0$. Then $B(p,r)\cap B(q,r) = \emptyset$. So $X$ is Hausdorff \\($\because x \in B(p,r) \cap B(q,r) = d(x,p) < r ~\&~ d(x,q) < r \implies d(p,q) \leq d(p,x) + d(x,q) < r+r = 2r = d(p,q) (\rightarrow\leftarrow)$ )
\end{proof} 

\begin{rmk*}
	Let $S \subseteq X,$ where $X$ is a metric space. Then $S^o$ is the largest(w.r.t inclusion) open set contained in $S$. $\because \forall~$ open set $U \subseteq S,~U^o\subseteq S^o \implies U \subseteq S^o \subseteq S$. In fact, $S^\circ = \bigcup_{U \subseteq S}U$(which is the definition of intension of S in a topology space $X$)
\end{rmk*}


\newpage


\subsection{Closed Sets}

\begin{defn}[Closed set]
	$F \subseteq X$ is closed $\Leftrightarrow$ $F^C = X-F$ is open in $X$
\end{defn}

By Theorem 2.1, the collection of all close sets in $X$ has the properties

\begin{enumerate}[wide,label = \textbf{(\roman*)}]
	\item $\emptyset,X$ are closed in $X$
	\item $F_{\alpha}$ is closed in $X,~\alpha \in I \implies \bigcap_{\alpha \in I}F_{\alpha}$ is closed in $X$
	\item $F_1,\cdots,F_n$ are closed in $X \implies \bigcup_{i=1}F_i$ is closed in $X$
\end{enumerate}

\textbf{Example}

Intersection of infinitely many open set may not be open
	
in $\mathbb R$ with Euclidean topology, $(-\frac{1}{n},\frac{1}{n})$ is open in $\mathbb R ~\forall~n \geq 1 \implies \bigcap_{n=1}^{\infty}(-\frac{1}{n},\frac{1}{n}) = \{0\}$ is not open

\textbf{Prop 2.8} Let $X$ be a metric space and $Y \subseteq X$ and $B \subseteq Y$, Then $B$ is closed in $Y \Leftrightarrow B = F \cap Y$ for some closed set $F$ in $X$

\begin{proof}
	($\Rightarrow$) Suppose $B$ is close in $Y \implies Y - B$ is open in $Y \implies Y-B = U \cap Y(\text{by prop 2.5})$ for some open set $U$ in $X \implies Y - (Y-B) = Y-(U \cap Y) \implies B = (X - U) \cap Y$. where $(X - U)$ is close.
	
	($\Leftarrow$) Suppose $B = F \cap Y,$ where $F$ is closed in $X \implies Y - B = Y-(F \cap Y) = (X-F) \cap Y \implies Y - B$ is open in $Y \implies B$ is close in $Y$.
\end{proof}

In metic space, one can use sequence to detect the closeness of a set

\textbf{Example}

\begin{enumerate}[wide,label=\textbf{\arabic*.}]
	\item We know that $[a,b)$ is not closed in $\mathbb R$, however, $\exists$ a sequence $\{x_n\}$ in $[a,b) \ni x_n \rightarrow b$ on $n \rightarrow \infty$, e.g. $b - \frac{1}{n} \rightarrow b$
	\item $A = \{\frac{1}{n}~|~n \geq 1\} = \{1,\frac{1}{2},\frac{1}{3},\cdots\}$ is not close in $\R$
	
	if $\R - A$ will open, them $\exists r>0,~B(0,r) \subseteq \R - A(\rightarrow\leftarrow)$
	
	$A \cup \{0\}$ is closed in $\R$
	
	$$R \setminus (A \cup \{0\}) = (-\infty,0)\cup(1,\infty)\cup(\bigcup^{\infty}_{n=1}(\frac{1}{n+1},\frac{1}{n})) \text{ is open}$$
	
	$\therefore A \cup \{0\}$ is closed
\end{enumerate}



\begin{defn}[Adherent, clousure $\cdots$]
	Let $X$ be a metric space with metric $d$, $T \subseteq X$ be a subset.(\textbf{\color{red} important})
	
	\begin{enumerate}[wide, label = \textbf{(\arabic*.)}]
		\item A point $p \in X$ is said to be an adherent point of $T$ if $\forall r > 0,~B(p,r)\cap T \neq \emptyset,~$ equivalent, $\forall$ neighborhood $U$ of $p$, $U \cap T \neq \emptyset$
		\item Let $\overline{T}$ or $cl(T)$ denote the set of all adherent points of $T$, called the closure of $T$, i.e. $\overline{T} = \{p \in X ~|~p \text{ is an adherent point of }T\}$
		\item A point $p \in X$ is said to be a limit point or accumulation point of $T$ if $\forall~r > 0,~B(p,r) \cap T - \{p\} \neq \emptyset,$ equivalently, $\forall$ neighborhood $U$ of $p$, $U \cap T - \{p\} \neq \emptyset$
		
		Denote by $T'$ the set of all accumulation points of $T$, called the devied set of $T$.
		\item $p \in T$ and $p \notin T'$, then $p$ is called an isolated point of $T$, i.e. $\exists r >0 \ni B(p,r)\cap T = \{p\}$
		\item A subset $T \subseteq X$ is said to be perfect if $T$ is closed and every points of $T$ is an accumulated point of $T$, i.e. $T$ is closed $\&$ $T' = T$
		\item A subset $T \subseteq X$ is said to be bounded if $\exists R>0$ and $p \in X \ni T \subseteq B(p,R)$
		\item A subset $T \subseteq X$ is said to be dense if $\overline{T} = X$, e.g. $\overline{\Q} = \R$
		\item A point $p \in X$ is said to be a boundary point of $T$ if $\forall~ r>0,B(p,r) \cap T \neq \emptyset ~\&~ B(p,r) \cap (X \setminus T) \neq \emptyset $. Denote by $\partial T$ or bd($T$) the set of all boundary points of $T$
	\end{enumerate}
\end{defn}

\textbf{Prop 2.9} Let $X$ be a metric space. All sets and point below are subset of $X$

\begin{enumerate}[wide,label=\textbf{(\arabic*). }]
	\item $S \subseteq T \implies \overline{S} \subseteq \overline{T} ~\&~ S' \subseteq T'$
	 
	$\because p \in \overline{S} \implies \forall~ r>0,B(p,r) \cap S \neq \emptyset \implies B(p,r) \cap T \neq \emptyset \implies p \in \overline{T}$
	
	$p \in S' \implies \forall r > 0,B(p,r)\cap S - \{p\} \neq \emptyset \implies B(p,r) \cap T - \{p\} \neq \emptyset$
	
	\item $\overline{T}$ is always closed in $X$
	
	\begin{tcolorbox}
		We want to know $\overline{T}$ is closed on $X \rightarrow X - \overline{T}$ is open $\rightarrow \forall~p \in X - \overline{T}$ is an interior point $\implies \exists r > 0,~B(p,r) \subseteq X - \overline{T}$
		
		$\because p \notin \overline{T} \Rightarrow \exists r' > 0 \ni B(p,r') \cap T = \emptyset$
		
		But we want to get $B(p,r') \cap \overline{T}$, so we check every point in $B(p,r')$ is not in $\overline{T}$, let $q \in B(p,r') ,~\exists ~\delta > 0 ~,B(q,\delta) \subseteq B(p,r') \implies B(q,\delta) \cap T = \emptyset \implies q \notin \overline{T}$
		
		because if $q \in \overline{T},\forall r > 0 \ni B(q,r) \cap T \neq \emptyset$
		
		$\implies B(p,r) \cap \overline{T} = \emptyset$
	\end{tcolorbox}
	
	Let $p \in X - \overline{T} \implies p \notin \overline{T} \implies \exists~r>0 \ni B(p,r) \cap T = \emptyset \implies B(p,r) \cap \overline{T} = \emptyset$ ($\because \forall q \in B(p,r),~\exists~\delta > 0 \ni B(q,\delta) \subseteq B(p,r) \implies B(q,\delta) \cap T = \emptyset \implies q \notin \overline{T}$)
	
	$\therefore B(p,r) \subseteq X - \overline{T}$,~$\because p$ is an interior point of $X - \overline{T}$. Hence, $X - \overline{T}$ is open, i.e. $\overline{T}$ is closed.
	
	\item $T \subseteq \overline{T}(\because \forall~p \in T, \forall ~ r > 0,B(p,r) \cap T \neq \emptyset)$
	
	\item $p \in T' \implies \forall r > 0,~ B(p,r) \cap T - \{p\}$ is an infinite set, say $x_1,\cdots,x_n$, Let $\delta = \frac{1}{2} \min \{d(p,x_i)~|~1\leq i \leq n\}$. Then $B(p,\delta) \cap T - \{p\} = \emptyset (\rightarrow\leftarrow)$ to $p \in T'$, $x \in B(p,\delta) \cap T - \{p\} \implies d(x,p) < \delta \implies x = x_i$ for some $1 \leq i \leq n \&$ we get $d(x_i,p) < \delta \leq \frac{1}{2}d(x_i,p)$
	
	$\therefore$ no such $x$ i.e. $B(p,\delta) \cap T - \{p\} = \emptyset$
	\item Any finite subset of $X$ has no accumulation points in $X$ by (4). In particular, it is closed by (6)(c) below.
	\item TFAE
	\begin{enumerate}
		\item $S$ is closed
		\item $S$ contains all it's adherent point, i.e. $\overline{S} \subseteq S$
		\item $S$ contains all it's accumulation points, i.e. $S' \subseteq S$
		\item $S = \overline{S}$
	\end{enumerate}
	\begin{proof} of (6)

(a) $\Rightarrow$ (b) Suppose $S$ is closed $\implies X \setminus S$ is open $\implies \forall ~ p \in X - S \implies \exists r > 0 \ni B(p,r) \subseteq X \setminus S \implies B(p,r) \cap S = \emptyset \implies p \notin \overline{S}$

$\therefore \overline{S} \subseteq S$, i.e. (b) holds\\

(b) $\Rightarrow$ (c) $\because S' \subseteq \overline{S}$\\

(c) $\Rightarrow$ (d) Suppose $S' \subseteq S$. To prove $S = \overline{S}$ if not, then $S \subsetneqq \overline{S}$, i.e. $\exists~ p \in \overline{S} ~\&~ p\notin S \implies \forall r > 0,~B(p,r) \cap S \neq \emptyset~(\because p \in \overline{S})$

(d) $\Rightarrow$ (a) by (2)


\end{proof}

	\item $\overline{S}$ is the smallest closed set in $X$ containing $S$
	
	$\because$ We know that $S \subseteq \overline{S}$, if $F$ is closed in $X ~\&~ F \subseteq S$, then $\overline{F} \subseteq \overline{S}$ by (1), $F = \overline{F} \subseteq \overline{S}$ by (6), $\therefore \overline{S}$ is the smallest such one.
	\item In fact, $\overline{S} = \bigcap_{F \subseteq S}F$
	\item $p \in S$ is an isolated point $\Leftrightarrow \exists~ r > 0 \ni B(p,r) \cap S = \{p\}$
	
	($\Rightarrow$) Suppose $p \in S$ is an isolated point of $S$. Then $p \in S' \implies \exists~ r > 0 \ni B(p,r) \cap S - \{p\} = \emptyset \implies B(p,r) \cap S = \{p\}$
	
	($\Leftarrow$) Trivial
	\item $S$ is dense in $X$ $\Leftrightarrow  ~\forall~p \in X ~\&~ r>0,~B(p,r) \cap S \neq \emptyset \Leftrightarrow  \forall$ open set $U \neq \emptyset,~U \cap S \neq \emptyset$
	\begin{proof}$ $
		($\Rightarrow$) Suppose $S$ is dense in $X$, i.e. $\overline{S} = X$, So $\forall p \in X, p \in \overline{S} \implies \forall~ r > 0,~B(p,r) \cap S \neq \emptyset$
		
		($\Leftarrow$) Suppose the condition holds, $\forall p \in X ~\&~ r > 0,~B(p,r) \cap S \neq \emptyset \implies p \in \overline{S} \implies X \subseteq \overline{S} \subseteq X,~\therefore \overline{S} = X$
	\end{proof}
	
	\item $\partial S = \partial(X - S)$ In particular, $\partial S = \overline{S} \cap \overline{(X - S)}$, In particular, $\partial S$ is closed in $X$, $\because$ It suffices to prove $\partial S$  = $\overline{S} \cap \overline{(X - S)}$,
	
	$\because \partial(X - S) = \overline{X - S} \cap \overline{X - (X - S)} = \overline{X - S} \cap \overline{S} = \partial S$
	
	$\forall p \in \partial S \implies \forall r > 0, B(p,r) \cap S \neq \emptyset ~\&~ p \in \overline{X - S} \implies p \in \overline{S} \cap \overline{X - S}$
	
	$\therefore \partial S \subseteq \overline{S} \cap \overline{X - S}$, 
	
	Conversely, $p \in \overline{S} \cap \overline{X - S} \implies p \in \overline{S} ~\&~ p \in \overline{X - S} \implies \forall r> 0,~B(p,r) \cap S \neq \emptyset ~\&~ B(p,r)\cap(X-S) \neq \emptyset \implies p\in \partial S$
	
	$\therefore \overline{S} \cap \overline{(X - S)} \subseteq \partial S \therefore \partial S = \overline{S} \cap \overline{(X - S)}$
\end{enumerate}

\subsection{Examples}$ $

We give some simple examples of open sets, closed sets, adherent, accumulation, isolated and boundary points.

\begin{enumerate}[wide,label = \textbf{\arabic*.}]
	\item In a discrete metric space $X$, every subset of $X$ is both open and close, $\forall x \in X,~B(p,r) \begin{cases}
		\{x\} \text{ if } 0<r\leq 1\\ X \text{ if } r>1
	\end{cases} $
	
	$\therefore$ Every singleton is open in $X$, so every subset of $X$ is open.
	\item In $\R$. Consider the set $S = [0,1) \cup \{3\}$, $S^\circ = \emptyset,~S' =\{0\}$, \\$\overline{S} = S \cup \{0\}$
	\item In $\R$, consider the set $S = \{\frac{1}{n}~|~n = 1,2,\cdots\}$, $S^\circ = \emptyset,~\\S' = \{0\},~\overline{S} = S \cup \{0\}$
	\item In $\R^2,$ consider $S = \{(x,y) \in \R^2~|~x>0,y>0\}$, $S$ is open
	
	$\overline{S} = \{(x,y) \in \R^2~|~x \geq 0,y \geq 0\}$
	
	$\partial S = \{(x,0)~|~x \geq 0\} \cup \{(y,0)~|~y\geq 0\}$
	\item Let $B(0,1)$ be the unit open ball in $\R^k$. Then $\partial B(0,1) = S^{k-1}$ is the unit $(k-1)$-sphere. In particular, for $k =2, \partial B(0,1) = S^1$ in the unity circle in the plane $\R^2$. Similarly, for the closed unit ball $\overline{B}(0,1)$ in $\R^k$. Now, we define some special sets in $\R^n$
	\begin{enumerate}[label = $\bullet$]
		\item Internals in $\R:~- \infty < a \leq b < \infty$
		
		$[a,b]$ close interval which is closed in $\R$
		
		$(a,b)$ open interval which is closed in $\R$
		
		Infinite intervals:
		
		$(-\infty,b]:$ close in $\R$~,~$(-\infty,b)$ open in $\R$
		\item k-dimensional interval (rectangle or k-cell) $I$
		$$I = I_1 \times \cdots \times I_k$$
		
		where $I_j$ is an interval in $\R,~1\leq j \leq k$
		\begin{enumerate}
			\item $I$ is bounded $\Leftrightarrow$ each $I_j$ is bounded
			
			$I$ is unbounded $\Leftrightarrow I_j \neq \emptyset ~\&~$ some $I_j$ is unbounded
			\item $I = [a_1,b_1] \times \cdots [a_k,b_k],~ -\infty < a_j \leq b_j < \infty,~ 1 \leq j \leq k$
			
			$k$-dimensional closed(compact) interval in $\R^k$
		\end{enumerate}
		
		\item Convex sets in $\R^k$
		
		$S \subseteq \R^k$ is convex if $\forall~ x,y \in S,~\overline{xy}$ is the line segment joining $x ~\&~ y$
		
		Note that all open balls, closed balls, intervals are convex in $\R^k$
		
		\item Star-like sets in $\R^k$ with w.r.t some point $x_0$, $S \subseteq \R^k$ is star-like w.r.t. $x_0 \in S$ if $\forall~x \in S,~\overline{xx_0} \subseteq S$
		
	\end{enumerate}
	\item We know that $\Q$ is dense in $\R$, hence $\Q^k$ is dense in $\R^k$. Note that $\Q^k$ is countable, hence $\R^k$ has a countable dense subset $\Q^k$, i.e. $\R^k$ is separable.
	\item $\partial \Q = \R,~\partial \Q^k = \R^k$
	\item $\Z$ is closed in $\R$, $\because \R - \Z = \bigcup^{\infty}_{n = -\infty}(n-1,n)$ is open $\implies \Z$ is close. or $\Z' = \emptyset \subseteq \Z,~\therefore \Z$ is close.
	\item Let $S \subseteq \R$ be a nonempty set which is bounded above. Then $\alpha = \sup S$ exists. Moreover, $\alpha \in \overline{S}.~\because \forall~r>0,\exists~x_0 \in S \ni \alpha - r < x_0 \leq \alpha < \alpha - r \implies (\alpha - r,\alpha + r) \cap S \neq \emptyset \implies \alpha \in \overline{S}$
\end{enumerate}

\subsection{Compact Set in Metric Space}

\begin{enumerate}[wide, label = $\bullet$]
	\item Compact sets in metric space, which is closely related to the extreme value problem.
	\item Compact set $\R^k$ will be discussed in next section.
\end{enumerate}

\begin{defn}
	Let $X$ be a topology space and $S \subseteq X$. A collection $\mathscr{U} = \{U_{\alpha}\}_{\alpha \in I}$ of open sets in $X$ is called an open covering of $S$ if 
	
	$$S \subseteq \bigcup \mathscr{U} = \bigcup_{\alpha \in I}U_{\alpha} $$
\end{defn}

\begin{defn}
	Let $X$ be a topology space, $S \subseteq X$ and $\mathscr{U} = \{U_{\alpha}\}_{\alpha \in I}$ be an open covering of $S$. We say that $\mathscr{U}$ has a countable(finite) sub covering of $S$ if $\exists$ a countable(finite) sub collection of $\mathscr{U}$ which also covers $S$. i.e. $\mathscr{U}$ has a countable(finite) subcovering in $S$ if 
	
	$\exists$ a sequence $\{\alpha_n\}$ in $I \ni S \subseteq \bigcup_{n=1}^{\infty}U_{\alpha_n}$ (countable)
	
	$\exists$ a sequence $\{\alpha_n\}$ in $I \ni S \subseteq U_{\alpha} \cup \cdots \cup U_{\alpha_n}$(finite)
\end{defn}

\textbf{Example} 

\begin{enumerate}
	\item $X$ is discrete metric space. Then $\{\{x\}~|~x \in X\}$ is an open covering of $X$
	\item In $\R,~\{(0,1-\frac{1}{n})~|~n \in \N\}$ is an open covering of $(0,1)$. In fact, $(0,1) = \bigcup^{\infty}_{n=1}(0,1-\frac{1}{n})$
	\item $\{B(0,n)~|~n \in \N\}$ is an open covering of $\R^k$
\end{enumerate}

\begin{defn}[compact]
	Let $X$ be a topology space. A subset $K \subseteq X$ is said to be compact if \textbf{every} open covering of $K$ admit a finite subcovering
\end{defn}

\textbf{Examples}

\begin{enumerate}[wide]
	\item Let $X$ be a topology space and $K \subseteq X$ be a finite set. Then $K$ is compact.
	\item In a discrete metric space $X$, a subset $K \subseteq X$ is compact $\Leftrightarrow K$ is a finite set.
	\item $(0,1)$ is not compact in $\R$($\{0,1-\frac{1}{n}~|~n\in \N\}$), but $[0,1]$ is compact
\end{enumerate}

\setcounter{thm}{9}

\begin{thm}
	Let $X$ be a metric space and $K \subseteq Y \subseteq X$. Then $K$ is compact in $X \Leftrightarrow K$ is compact in $Y$.
\end{thm}

\begin{proof}
	($\Rightarrow$) Suppose $K$ is compact in $X$. Given an open covering $\{V_{\alpha}\}_{\alpha \in I}$ of open sets in $Y$ which covers $K$. By Prop 2.5, each $V_{\alpha} = U_{\alpha} \cap Y$, where $U_{\alpha}$ is open in $X$. Now, 
	$$K \subseteq \bigcup_{\alpha \in I}V_{\alpha} = \bigcup_{\alpha \in I}(U_{\alpha} \cap Y) = (\bigcup_{\alpha \in I}U_{\alpha}) \cap Y \implies K \subseteq \bigcup_{\alpha \in I}U_{\alpha}$$
	
	By the compactness of $K$ in $X$, $\exists \alpha_1 ,\cdots ,\alpha_n \in I \ni K \subseteq \bigcup^n_{i=1}U_{\alpha_i} \implies K \cap Y \subseteq (\bigcup_{i=1}^{n}U_{\alpha_i}) \implies K \subseteq \bigcup_{i = 1}^n(U_{\alpha_i} \cap Y) = \bigcup^n_{i=1}V_{\alpha_i}$

	$\therefore K$ is compact in $Y$
	
	($\Leftarrow$)Suppose $K$ is compact in $Y$. Given a open covering $\{U_{\alpha}\}_{\alpha \in I}$ of $K$ by open sets in $X$. 
	
	$$K \subseteq \bigcup_{\alpha \in I}U_{\alpha} \implies K \cap Y \subseteq (\bigcup_{\alpha \in I}U_{\alpha})\cap Y \implies K \cap Y\subseteq \bigcup_{\alpha \in I}(U_{\alpha}\cap Y)$$
	
	By Prop 2.5, $\{U_{\alpha} \cap Y ~|~\alpha \in I\}$ is an open covering of $K$ by open set in $Y$. By assumption, $K$ is compact in $Y, \exists~\alpha_1,\cdots,\alpha_n \in I \ni K \subseteq \bigcup^n_{i=1}(U_{\alpha_i} \cap Y) = (\bigcup^n_{i=1}U_{\alpha_i}) \cap Y \implies K \subseteq \bigcup^n_{i=1}U_{\alpha_i}$
	
	$\therefore K$ is compact in $X$
\end{proof}

\begin{defn}
	Let $X$ be a metric space and $S \subseteq X$ be a nonempty set. The diameter of $S$ is defined to be $\dia (S) = \sup \{d(x,y)~|~x,y \in S\}$ which generated the diameter of a circle in $\R^2$
\end{defn}

\begin{thm}
	Let $X$ be a metric space and $K \subseteq X$ be a compact set. Then $K$ is closed and bounded
\end{thm}

\begin{proof}
	\textbf{$K$ is bounded}
	
	Fix a point $p \in K$. Then $K \subseteq \bigcup^{\infty}_{n=1}B(p,n)$. $\because K$ is compact $\implies \exists N \in \N \ni K \subseteq B(p,1) \cup \cdots \cup B(p,N) \implies K \subseteq B(p,N) ~\therefore K$ is bounded
	
	\textbf{K is closed}, i.e. $X - K$ is open
	
	Fix $p \in X - K$. Then $p \neq x,~\forall~ x \in K$. Hence, $d(x,p) > 0,~\forall~x \in K$
	
	Let $r_x = \frac{1}{2}d(x,p) > 0,x\in K$. Them $\{B(x,r_x)~|~x \in K\}$ is an open covering of $K$. $\because K$ is compact $\implies \exists x_1,\cdots,x_n \in K \ni B(x_1,r_{x_1}) \cup \cdots \cup B(x_n,r_{x_i})$. Let $V = \bigcap_{i=1}^{n}B(p,r_{x_i}) = B(p,r)$, where $r = \min \{r_{x_1},\cdots,r_{x_n}\}$. Then as we can see that $V \subseteq X - K$, all point in $X - K$ are inner point. So $X - K$ is open, i.e. $K$ is close. 
	
	\begin{tcolorbox}
		To show that $V \subseteq X - K$, i.e. $V \cap K \neq \emptyset$, it suffices to show
		
		$$V \cap (\bigcup^n_{i=1}B(x_i,r_{x_i})) = \emptyset$$
		
		Now, \begin{eqnarray*}
			V \cap (\bigcup^n_{i=1}B(x_i,r_{x_i})) &=& \bigcup^n_{i=1}(V \cap B(r_i,r_{x_i})) \\
			&\subseteq &  \bigcup^n_{i=1}(B(p,r_{x_i} \cap B(x_i,r_{x_i}))) = \emptyset
		\end{eqnarray*}
	\end{tcolorbox}
\end{proof}

\begin{rmk*}
	The converse of Thm 2.11 is false, i.e. closed $\&$ bounded may not be compact, e.g. $X$ is an infinite set with discrete metric. Then $X$ is not compact, but $X$ is closed and bounded.
\end{rmk*}

\begin{thm}
	Let $X$ be a metric space, $K \subseteq X$ be compact $\&$ $L \subseteq K$ be a closed set in $X$. Then $L$ is compact.
\end{thm}

\begin{proof}
	Let $\{U_{\alpha}\}_{\alpha \in I}$ be an open covering of $L$. Then $\{U_{\alpha}\}_{\alpha \in I} \cup \{X - L\}$ is an open covering of $K$. By the compactness of $K$, $\exists \alpha_1,\cdots,\alpha_n \in I \ni K \subseteq U_{\alpha_1} \cup \cdots \cup U_{\alpha_n} \cup (X-L)$. By $L \subseteq K~ \therefore L$ is compact
\end{proof}

\newpage

\begin{cor}$ $
	\begin{enumerate}[wide,label=(\alph*)]
		\item Let $X$  be a metric space, $K \subseteq X$ be compact and $F$ be a closed set in $X$. Then $K \cap F$ is compact.
		\item If $X$ is a compact metric space, then every closed subset $F$ of $X$ is compact.
	\end{enumerate}
\end{cor}

\begin{proof}$ $
	\begin{enumerate}[wide,label=(\alph*)]
		\item 
		\begin{eqnarray*}
			K\text{ is compact } &\implies& K\text{ is closed (Thm 2.11)}\\
			&\implies&K \cap F \text{ is closed in }X \\ 
			&\implies&K \cap F \text{ is compact}
		\end{eqnarray*}
		\item follows (a)
	\end{enumerate}
\end{proof}

\begin{rmk*}
	Let $X$ be a metric space. If $K$ is closed in $X$ and $F$ is closed in $K$, then $F$ is closed in $X$. $\because F$ is closed in $F \implies F = L \cap F,$ where $L$ is closed in $K \implies F = L \cap K,$ where $L$ is closed in $X \implies F$ is closed in $X$. 
\end{rmk*}

\begin{thm}
	Let $X$ be a metric space, $\{K_{\alpha}\}_{\alpha \in I}$ be a collection of compact subsets of $X$ with the property:
	
	$$\forall \alpha_1,\cdots,\alpha_n \in I, K_{\alpha_1} \cap \cdots \cap K_{\alpha_n} \neq \emptyset$$
	
	Them $\bigcap_{\alpha \in I}K_{\alpha} \neq \emptyset$
\end{thm}

\begin{proof}
	Fix $\alpha_0 \in I$. Assume that $\bigcap_{\alpha \in I}K_{\alpha} = \emptyset \implies X - \bigcap_{\alpha \in I}K_{\alpha}$\\$\implies X - \emptyset = X \implies X = \bigcup_{\alpha \in I}(X - K_{\alpha})$
	
	each $K_{\alpha}$ is compact $\implies K_{\alpha}$ is closed $\implies X - K_{\alpha}$ is open
	
	so $\{X - K_{\alpha}\}$ is an open covering of $X$. Now,
	
	$$K_{\alpha_0} \subseteq X = \bigcup_{\alpha \in I}(X - K_{\alpha}) \implies K_{\alpha_0} \subseteq \bigcup_{\alpha \in I}(X - K_{\alpha})$$
	
	$K_{\alpha}$ is compact $\implies \exists \alpha_1,\cdots,\alpha_n \in I - \{\alpha_0\} \ni K_{\alpha_0} \subseteq (X - K_{\alpha_1})\cup \cdots \cup (X - K_{\alpha_n}) \implies K_{\alpha_0} \cap K_{\alpha_1}\cap \cdots \cap K_n = \emptyset (\rightarrow \leftarrow)$
\end{proof}

\begin{cor}
	Let $X$ be a metric space and $\{K_n\}^{\infty}_{n=1}$ be a decrease sequence of nonempty compact sets of $X$. Them $\bigcap^{\infty}_{n=1} \neq \emptyset$. In addition, if $\dia_{n \rightarrow \infty} \infty 0$, them $\bigcap^{\infty}_{n=1}K_n$ is a singleton.
\end{cor}

\begin{proof}
	$\forall j1,\cdots,j_k \in \N,~K_{j1}\cap \cdots \cap K_{jk} \neq \emptyset,~K_{j1}\cap\cdots \cap K_{jk} = K_t$, where $t = \max \{j_1,\cdots,j_k\}$. By Thm 2.14 $\bigcap^{\infty}_{n = 1}K_n \neq \emptyset$, if $\lim_{n \rightarrow \infty}\dia(K_n) = 0$ and $p,q \in \bigcap^{\infty}_{n=1}K$ and $p \neq q$, them $\dia(K_n) \geq d(p,q) ~\forall n \geq 1 \implies \lim_{n \rightarrow \infty}\dia(K_n) \geq d(p,q) > 0 (\rightarrow\leftarrow) \because \bigcap^{\infty}_{n = 1}K_n = \{p\}$ is a simpleton.
\end{proof}

\begin{rmk*}
	The usual form of Cor 2.15, $X$ is a metric space, $\{K_n\}$ is a decrease sequence of nonempty closed sets in $X$ with $K_i$ is compact $\implies \bigcap^{\infty}_{n=1}K_n \neq \emptyset$
\end{rmk*}

\textbf{Example} In $\R,~ \{(0,\frac{1}{n}~|~n \geq 1]\}$ is decrease and every finite subcollection of $\{(0,\frac{1}{n})~|~ n \geq 1\}$ is nonempty, but $\bigcap^{\infty}_{n = 1} (0,\frac{1}{n})= \emptyset,~\bigcap[0,\frac{1}{n}] = \emptyset$
\setcounter{thm}{16}

\begin{thm}
	Let $X$ be a metric space and $K \subseteq X$, TFAE:
	
	\begin{enumerate}[wide,label=(\roman*)]
		\item $K$ is compact
		\item Every infinite subset has an accumulation point in $K$
		\item $K$ is sequentially compact
		\item $K$ is complete and totally bounded
	\end{enumerate}
\end{thm}

\begin{defn}[Convergence]
	$\{a_n\}$ converge if $\exists ~ a \in X \ni \forall~\epsilon \geq 0 \exists N \ni \N \ni \forall n \geq N,~d(a_n,a) < \epsilon$. Such $a$ is called the limit of $\{a_n\}$, which is denoted by $\lim_{n \rightarrow \infty} a_n = a$ or $a_n \rightarrow a$ on $n \rightarrow \infty$. 
	
\end{defn}

\begin{defn}[Cauchy]
	We say that $\{a_n\}$ is Cauchy if $\forall \epsilon > 0,~\exists N \in \N \ni \forall n,m \geq \N,~d(a_n,a_m) < \epsilon$
\end{defn}

\begin{defn}
	A metric $X$ is said to be sequence compact if every sequence has a convergent subsequence
\end{defn}


\begin{defn}
	A metric space $X$ is said to be complete if every Cauchy sequence in $X$ convergence.
\end{defn}

\begin{defn}
	Let $X$ be a metric space $\&$ $K \subseteq X$. We say that $K$ is totally bounded if $\forall r > 0, \exists x_1,\cdots,x_n \in K \ni K \subseteq B(x_1,r) \cup \cdots \cup B(x_n,r)$
\end{defn}

\begin{rmk*}
	Totally bounded can implies bounded, but not converse.
	
	$K$ is totally bounded, for $r=1, \exists x_1,\cdots,x_n \in K \subseteq B(x_1,1) \cup \cdots \cup B(x_n,1) \implies K \subseteq B(x_1,R)$ for sime large $R$
	
	\begin{enumerate}[]
		\item[$\bullet$] Take an \textbf{"infinite"} set $X$ with discrete metric. Then $X$ is bounded(e.g. $X \subseteq B(x_0,2),$ where $x_0\in X$) but for $r = \frac{1}{2},~X \subsetneqq B(x_1,\frac{1}{2}) \cup \cdots \cup B(x_n,\frac{1}{2}) ~\forall x_1,\cdots,x_n$ 
	\end{enumerate}
\end{rmk*}

\begin{lmma}[To prove (ii) to (i)]
	$ $
	
	Suppose (ii) holds in Thm 2.17. Then $K$ is totally bounded
\end{lmma}

\begin{proof}
	If not, then $\exists ~ r > 0 , \ni$ no finite open balls with radius $r$ and center $K$ cover $K$. Choose $x_1 \in k \implies K \subsetneqq B(x_1,r) \implies \exists x_2 \in K - B(x_1,r),~ K \subsetneqq B(x_1,r) \cup B(x_2,r) \implies \exists x_3 \in K - (B(x_1,r) \cup B(x_2,r))$
	
	By induction, counting this process, we obtain an infinite set $T = \{x_1,x_2,\cdots,x_n,\cdots\} \subseteq K$ with $d(x_i,x_j) \geq r \forall i \neq j$. By (ii), $T$ has an accumulation poion $p \in K$. In particular $B(p,\dfrac{r}{4}) \cap T - \{p\}$ is an infinite set, hence, $\exists i \neq j \ni x_i,x_j \in B(p,\dfrac{r}{4}) \cap T - \{p\} \implies d(x_i,x_j) \leq d(x_i,p) + d(x,j) < \dfrac{r}{4} + \dfrac{r}{4} = \dfrac{r}{2} < r (\rightarrow\leftarrow) \therefore K$ is totally bounded.
\end{proof}

\begin{lmma}
	Suppose (ii) holds in T and $\{E_{\alpha} ~|~ \alpha \in I\}$ is an open covering of $K$. Then $\exists r > 0$(called a Lebegoue number w.r.t. the open covering $\{E_{\alpha}\}_{\alpha \in I}$) $\ni \forall x \in K,B(x,r) \subseteq E_{\alpha}$ for some $\alpha \in I$
\end{lmma}

\begin{proof}
	if $K$ is a finite set, let $K = \{x_1,\cdots,x_n\} K \subseteq \bigcup_{\alpha \in I}E_{\alpha} \implies x_i \in E_{\alpha_i}$ for some $\alpha_i \in I,~ 1 \leq i \leq n \implies \exists r_i > 0 \ni B(x_i,r_i) \subseteq E_{\alpha_i},1\leq i \leq n$. Let $r = \min \{r_1,\cdots,r_n\}$. Then $B(x_i,r) \subseteq B(x_i,r_i) \subseteq E_{\alpha_i},1\leq i \leq \alpha$. Now, assume that $K$ is infinite set. Assume that no such $r > 0$, i.e. $\forall r > 0, \exists x_r \in K \ni B(x_r,r) \subsetneqq E_{\alpha} \forall x \in I$. Now, for $r = \dfrac{1}{k},r=1,2,\cdots,$ we obtain a sequence $\{x_k\}$ in $K$, with $x_k = \dfrac{x_r}{k} \ni B(x_k,\dfrac{1}{k}) \subsetneqq E_{\alpha} \forall a \in I$. Let $T = \{x_1,x_2,\cdots,x_k,\cdots\}$. Then $T \subseteq K$ is an infinite set($\because$ For $k = 1,r=\dfrac{1}{1} = 1 \exists x_1 \in K \ni B(x_1,1) \subsetneq E_{\alpha} \forall \alpha \in I$). The conclusion of Lemma 2.19 failed for $K - \{x_1\}$ ($\because if \exists s> 0 \ni \forall x \in K-\{x_1\} B(x,s) \subseteq E_{\alpha}$ for some $\alpha \in I,~ x_1 \in E_{\alpha} \implies \exists t > 0 \ni B(x_1,t) \subseteq E_{\alpha}$. Let $r = \min \{s,t\}$. Then $\forall x \in K, B(x,r) \subseteq E_{\alpha}$ for some $\alpha \in I (\rightarrow \leftarrow)$)
	
	Then for $r = \dfrac{1}{2}, \exists x_2 \in K - \{ x_1\} \ni B(x_2,\dfrac{1}{2}) \subsetneqq E_{\alpha} \forall \alpha \in I$. Continue this process, we conclude that $x_i \neq x_j \forall i \neq j$, so $T$ is an infinite set. By the assumption of (ii). Then an accumulation point $p \in K$. Now $K = \bigcup_{\alpha \in I}E_{\alpha} \implies p \in E_{\alpha}$ for some $\alpha \in I \implies \exists \epsilon > 0 \ni B(p, \epsilon) \subseteq E_{\alpha_0}$. Since $p \ni T',B(p,\epsilon) \cap T - \{p\}$ is an infinite set. Choose $m >> 0 \ni \dfrac{1}{m} < \dfrac{\epsilon}{2} ~\&~ x_m \in B(p,\dfrac{\epsilon}{2}) \cap T$. Claim $B(x_m , \dfrac{1}{m}) \subseteq B(p,\epsilon) \subseteq E_{\alpha_0} (\rightarrow\leftarrow)$ to our constrain, hence Lemma 2.19 holds.
	
	$$y \in B(x_m,\dfrac{1}{m}) \Rightarrow d(y,p) \leq d(y,x_m) + d(x_m,p) < \dfrac{1}{m} + \dfrac{\epsilon}{2} < \dfrac{\epsilon}{2} + \dfrac{\epsilon}{2} = \epsilon $$
	
	$\Rightarrow y \in B(p,\epsilon)$
\end{proof}


\begin{proof}(Thm 2.17 (i)(ii))
	
	$(i) \Rightarrow (ii)$ Suppose $K$ is compact. Given an infinite set $T \subseteq K$. We must prove that $T$ has an accumulation point in $K$, if not, $\forall x \in K, x$ is not an accumulation point of $K,\exists r_x > 0 \ni B(x,r_n) \cap T - \{x\} = \emptyset \implies B(x,r_x) \cap T \subseteq \{x\}$. Clearly $\{B(x,r_x~|~x \in K)\}$ is an open covering of $K$. By (i), $K$ is compact
	
	 \begin{eqnarray*}
	 	\implies \exists x_1,\cdots,x_n \in K \ni K &\subseteq& B(x_1,r_{x_1}) \cup \cdots \cup B(x_n,r_{x_n})\\
	 	&=& (T \cap B(x_1,r_{x_1})) \cup \cdots \cup (T \cap B(x_n,r_{x_n}))\\
	 	&\subseteq& \{x_1\} \cup \{x_2\} \cup \cdots \cup \{x_n\}\\
	 	&=& \{x_1\} \cup \{x_2\} \cup \cdots \cup \{x_n\}\\
	 	&=& \{x_1,\cdots,x_n\} (\rightarrow \leftarrow)
	 \end{eqnarray*}
	 
	 to $T$ is an infinite set, $\therefore$ (ii) holds.
	 
	 $(ii) \Rightarrow (i)$ In Thm 2.17 i.e. we must prove that $K$ is compact under the assumption of (ii). Suppose $\mathscr{U} = \{E_{\alpha}\}_{\alpha \in I}$ is an open covering of $K$. By Lemma 2.19, $\exists$ a $r > 0 ~w.r.t.~ \mathscr{U}$, by Lemma 2.18, $\exists x_1,\cdots,x_k \in K \ni K \subseteq B(x_1,r) \cup \cdots \cup B(x_k,r) \subseteq E_{\alpha_1}\cup \cdots\cup E_{\alpha_k}$, where $B(x_i,r) \subseteq E_{\alpha_i}, 1 \leq i \leq k$. Therefore, $\mathscr{U}$ has a finite sub covering. Hence $K$ is compact and (i) holds.
\end{proof}

\begin{rmk*}$ $
	\begin{enumerate}[wide]
		\item (ii) $\implies$ (i) is exercise 26
		\item More or less, by Lemma 2.18 $\&$ 19, one can see that (i) and (ii) are also equal to (iii) and (iv) 
	\end{enumerate}
\end{rmk*}

\newpage

\subsection{Compact Sets in Euclidean Spaces $\R^k$}

\begin{enumerate}[wide,label = $\bullet$]
	\item We know that any compact set in a metric space is close and bounded
	\item Close and bounded subset my not be compact(infinite discrete)
	\item We will see that every closed and bounded subset of $\R^k$ is always compact which is the famous H.B. Theorem, i.e. $K \subseteq R^k$ is compact $\Leftrightarrow K$ is closed and bounded
\end{enumerate}

\begin{thm}
	Let $\{I_n = [a_n,b_n]\}^{\infty}_{n=1}$ be a sequence of closed and bounded intervals in $\R$, if $\{I_n\}$ is decreasing i.e. $I_1\subseteq \cdots \subseteq I_n \subseteq \cdots$, then $\bigcap^{\infty}_{n=1}I_n \neq \emptyset$. Moreover, if $\lim_{n \rightarrow \infty}(b_n - a_n) = 0,$ then $\bigcup^{\infty}_{n=1}I_n$ is a simpleton.
\end{thm}

\begin{proof}
	Claim $T = \{a_n~|~n \in \N\}$ is bounded above and $x = \sup T$ exists. $\because a_n \leq a_{m+n}$($\because \{a_n\}$ is increasing, i.e. $[a_1,b_2] \supseteq [a_2,b_2] ,~ a_2 \geq a_1$)
	
	$a_n \leq a_{m+n} \leq b_{m+n} \leq b_m(\because \{b_n\}$ is decreasing ) $\implies T$ is bounded above by all $b_n \implies x = \sup T$ exists and $x \leq b_n \forall n \geq 1$.
	
	Clearly, $a_n \leq x ~\forall n \geq 1$, $\because a_n \leq x \leq b_n ,\forall n \geq 1$, i.e. $x \in [a_n,b_n] \forall n \geq 1 \therefore x \in \bigcap^{\infty}_{n=1}I_n$. Hence, $\bigcap^{\infty}_{n=1}I_n \neq \emptyset \therefore$ The last statement follow the argument as in Corollary 2.16. 
\end{proof}

\begin{thm}
	Let $\{I_n = [a_{n,1},b_{n,1}]\times \cdots \times [a_{n,k},b_{n,k}] \}$ be a decreasing sequence of closed and bounded intervals in $\R^k$. Then $\bigcap^{\infty}_{n = 1} \neq \emptyset$. Moreover, if $\lim_{n \rightarrow \infty} \dia (I_n) = 0$, then $\bigcup^{\infty}_{n=1}$ is a simpleton.
\end{thm}

\begin{proof}
	$\forall 1 \leq j \leq k, \{[a_{n,j},b_{n,j}]\}$ is a decrease sequence of closed and bounded intervals in $\R$. By Thm 2.20, $\exists x_j \in \bigcap^{\infty}_{n = 1}[a_{n,j},b_{n,j}]$. Set $x = (x_1,\cdots,x_k)$. Then $x \in \bigcap^{\infty}_{n = 1}I_n$. Then last statement also follows from the argument in corollary 2.16.
\end{proof}

\begin{thm}
	Every $k$-dimensional closed and bounded interval $I = [a_1,b_1] \times \cdots \times [a_k,b_k]$ in $\R^k$ is compact.
\end{thm}

\newpage

\begin{proof}
	Put $\delta = (\sum^{k}_{i=1}(b_i-a_i)^2)^{\dfrac{1}{2}}$ which is the diametric of $I$. Then $\forall x,y \in I, \norm{x - y} \leq \delta$. If $I$ were not compact, them $\exists$ an open covering $\{E_\alpha\}_{\alpha \in J}$ admitting not finite sub covering$(\star)$. Put $c_j = \dfrac{a_j+b_j}{2},~1\leq j \leq k$. The intervals $[a_j,c_j]$ and $[c_j,b_j], 1 \leq j \leq k,$ determines $2^k$ closed and bounded subinterval of $I$ whose union is $I$. By ($\star$), at least one of them, say $I_1$ which cannot be covered by finitely many $E_{\alpha}$. Continuing this process, we get a sequence $\{I_n\}$ of closed and bounded subintervals of $I$ satisfy's
	
	\begin{enumerate}[label = \alph*)]
		\item $I \subseteq I_1 \subseteq \cdots$, i.e. $\{I_n\}^{\infty}_{n=1}$ is decreasing.
		\item Each $I_n$ cannot be covered by finitely many $E_{\alpha}$
		\item $\dia (I_n) = 2^{-n},~\delta \rightarrow 0$ on $n \rightarrow \infty$
	\end{enumerate}
	By Thm 2.21, $\bigcap^{\infty}_{n=1}I_n = \{x\}$ i.e. $x \in I_n \subseteq I ~\forall~ n \geq 1 \subseteq \bigcup^{\infty}_{\alpha=1}E_{\alpha}$
	
	$\because x \in E_{\alpha_0}$ for some $\alpha_0 \in J,~E_{\alpha_0}$ is open $\implies \exists r>0 \ni B(x,r) \subseteq E_{\alpha_0}$. Choose $n_0 >> 0 \ni \dfrac{1}{2^{n_0}} < \dfrac{r}{\delta}(\because \dfrac{1}{2^n} \rightarrow 0)$. Since $x \in I_{n_0}, \forall y \in I_{n_0},~\norm{y - x} \leq 2^{-n_0}\delta < \dfrac{r}{\delta} \cdot \delta = r \implies y \in B(x,r),~\because I_{n_0} \subseteq B(x,r) \subseteq E_{\alpha_0}(\rightarrow\leftarrow)$. Therefore $I$ is compact. 
\end{proof}

Combining Thm 2.22 and results in section 2.6, we conclude that the following sets in compact:

\begin{enumerate}[wide,label = \roman*)]
	\item $[a,b]$ is compact in $\R$(Thm 2.22 with $k = 1$)
	\item $[a,b] \times [c,d]$ is compact in $\R^2$(Thm 2.22 with $k = 2$)
	\item Every closed ball $\overline{B}(x,r)$ in $\R^k$ is compact by Thm 2.12 and  2.22
	\item $\{0\} ~\cup\{\dfrac{1}{n}~|~n=1,2,\cdots\}$ is compact in $\R$, In fact, if $a_n \rightarrow a$, them teh set $\{a\} \cup \{a_n~|~n = 1,2,\cdots\}$ is compact in $\R$
\end{enumerate}

\begin{thm}
	Every closed and bounded subset $K$ of $\R^k$ is compact.
\end{thm}

\begin{proof}
	Choose a large closed and bounded interval $I$ in $\R^k$, $K \subseteq I$. By Thm $2.22$, $I$ is compacted, so $K$ is a closed subset of $I$. By Thm 2.12, $K$ is compacted.
	
	Combining Thm 2.17 and Thm 2.23, we can characterize compacted set in $\R^k$
\end{proof}

\newpage

\begin{thm}
	Let $K \subseteq R^k$ TFAE:
	
	\begin{enumerate}[wide,label=\roman*)]
		\item $K$ is closed and bounded
		\item $K$ is compacted
		\item Every infinite subset of $K$ has an accumulation point
		\item $K$ is sequence compact
		\item $K$ is complete and totally bounded
	\end{enumerate}
\end{thm}

From these we can deduce:

\begin{thm}[Bolzano-Weiestrace]

Every bounded infinite subset $T$ in $\R^k$ has an accumulation point in $\R^k$
\end{thm}

\begin{proof}
	Since $T$ is bounded, choose a large closed and bounded interval $I$ in $\R^k \ni T \subseteq I$. Now, $T$ become an infinite subset of the compact set $I$. By Thm 2.24 (iii), $T$ has an accumulation point in $I$.
\end{proof}

\begin{thm}[Cantor intersection]
	Let $\{\Q_n\}$ be a sequence of nonempty set in $\R^k$ satisfying:
	\begin{enumerate}[label = \alph*)]
		\item $\{Q_n\}$ is decreasing
		\item $Q_n$ is closed $\forall~n \geq 1$ $\&$ $Q_1$ is compact.
	\end{enumerate}
	
	Then $\bigcap^{\infty}_{n = 1}Q_n \neq \emptyset,$ Moreover, if $\dia(Q_n) \rightarrow 0$ on $n \rightarrow \infty$, them $\bigcap^{\infty}_{n=1} Q_n$ is a simpleton.
\end{thm}

\begin{proof}
	By (b), each $Q_n$ is compact ($\because Q_1 \supseteq Q_n \forall~n \geq 1$). Therefore, it follows form Cor 2.16 and 2.15.
\end{proof}

\subsection{Countability $\&$ Separability}$ $

Motivation: In $\R^k$, we have two facts:

\begin{enumerate}[wide,label=$\bullet$]
	\item $\Q^k$ is dense in $\R^k$ i.e. $\overline{\Q^k} = \R^k$ $\&$ $\Q^k$ is countable. i.e. $\R^k$ has a countable dense subset. i.e. $\R^k$ is separable.
	\item $\left\{B(x,r) ~|~ x \in \Q^k, r \in \Q^+ \right\}$ is a countable collection of open ball in $\R^k$ satisfying: $\forall$ open set $U \subseteq \R^k$ and $y \in U ~\exists B(x,r) \in \mathscr{B} \ni y \in B(x,r) \subseteq U$. In particular, $U$ is a union of some sub collection of $\mathscr B$. $\therefore U = \bigcup_{y \in U} B_y$, i.e. $\R^k$ is of $2^{nd}$ countable.
	\item $X$ is a metric space $x \in X,~N_x = \{B(x,\dfrac{1}{n})~|~n \in \N\}$ is a countable collection of nbh of $x$. Such $N_x$ satisfies:
	$\forall$ nbh $U$ of $x,\exists n \in \N \ni B(x,\dfrac{1}{n}) \subseteq U$($\because \exists~ r > 0 \ni B(x,r) \subseteq U,$ choose $n >> 0 \ni \dfrac{1}{n} < r$). Then $x \in B(x,\dfrac{1}{n}) \subseteq B(x,r) \subseteq U$. i.e. Each point of $X$ has a countable nbh base(system) i.e. $X$ is of $1^{st}$ countable, i.e. $\R^k$ has a countable base.
\end{enumerate}

\begin{defn}
	Let $X$ be a topological space
	
	\begin{enumerate}[wide]
		\item $X$ is first if every point of $X$ has a countable nbh system(or base), i.e. $\exists$ a countable collection $\{V_n ~|~ n \in \N\}$ of nbh of $x \ni \forall$ nbh $U$ of $x,\exists~ n \in \N,~ V_n \subseteq U$
		\item $X$ is of second countable if $X$ has a countable a base, i.e. $\exists$ a countable collection $\mathscr B = \{B_n ~|~ n \in \N\}$ of open sets in $X \ni$ every open set $U$ is a union of some subcollection of $\mathscr B$ or $\forall$ open set $U$ in $X$ and $x \in U,~\exists n \in \N \ni x \in B_n \subseteq U$
		\item $X$ is separable if $X$ has a countable dense subset, i.e. $\exists$ a countable set $D \subseteq X \ni \overline{D} = X$.
	\end{enumerate}
\end{defn}


\begin{rmk*} $ $
	\begin{enumerate}[wide]
		\item Every $2^{nd}$ countable topology space $X$ is of $1^{st}$ countable, but not converse. $\because$ Let $\mathscr{B} = \{B_1,B_2,\cdots\}$ be a countable base for $X$. Given $p \in X$, let $\mathscr B_p = \{B_n ~|~ p \in B_n\}$ Then $\mathscr B$, so $\mathscr B_p$ is countable and is a collection of open set in $X$ containing of $p$.
		 
		$ $\\
		Claim $\mathscr B_p$ is a nbh system (or base) of $p$. Let $U$ be a nbh of $p$. Then $U = \bigcup_{n \in F}B_n,~F\subseteq \N$. In particular, $p \in B_n$ for some $n \in F$. Hence, $B_n \in B_p$ $\&$ $p \in B_n \subseteq U$, $\therefore \mathscr B_p$ is a countable nbh system of $p$. Hence, $X$ is of $1^{st}$ countable.
		
		$ $\\
		Consider $X$ an uncountable set with discrete metric. Hence, $X$ is $1^{st}$ countable. In fact, $\forall ~p\in X,~N_p = \{\{p\}\}$ is a countable nbh base of $p$. However, $X$ is not of $2^{nd}$ countable. Note that if $\mathscr B$ is a base for the discrete space $X$ then $\mathscr B \subseteq \{\{x\} ~|~ x \in X\}$
		
		$\because\{x\}$ is open $\implies \{x\} = \bigcup_{\alpha \in I}B_{\alpha},~B_{\alpha} = \{x\}~\forall~ \alpha \in I \implies$
		
		$B_\alpha = \{x\} \forall x \in I \implies \{x\} \in \mathscr B$.
		
		Now, $X$ is uncountable, so is $\mathscr B$. Hence, $X$ is not of $2^{nd}$ countable.
		\item We know that every metric space is of $1^{st}$ countable. In fact, $\forall p \in X,~N_p = \{B(p,\dfrac{1}{n})~|~n \in \N\}$ is a countable nbh system of $p$
		\item We know that $\R$ is separable with countable dense subset $\Q$. In general, $\R^n$ is separable with countable dense subset $\Q^n$(Exercise 22)
	\end{enumerate}
\end{rmk*}

\begin{thm}
	Every $2^{nd}$ countable topology space is separable
\end{thm}

\newpage

\begin{rmk*}
	Note that $X$ is a metric space. $D \subseteq X$.
	
	\begin{eqnarray*}
		D \text{ is dense in } X \Leftrightarrow \overline{D} = X &\Leftrightarrow& \forall \text{ nonempty open set } U \text{in} X, U \cap D \neq \emptyset\\
		&\Leftrightarrow& \forall x \in X, \exists \text{ a sequence } \{a_n\} \text{ in } D \\
		&\ni& a_n \rightarrow x \text{ on } n \rightarrow \infty
	\end{eqnarray*}
	
\end{rmk*}

\begin{proof}$ $

$(\Rightarrow)$ Suppose $\overline{D} = X$, i.e. $D$ is dense in $X$, i.e. $\forall x \in X,~x \in \overline{D}$. Now, given a nonempty open set $U$ in $X$. Choose $x \in U$ So $U$ is a nbh of $x$, hence $U \cap D \neq \emptyset$

$(\Leftarrow)$ Suppose the condition holds, them $\forall~x \in X$ and nbh $U$ of $x$, $U \cap D \neq \emptyset \implies x \in \overline{D} \implies X \subseteq \overline{D} \subseteq X \implies \overline{D} = X$, i.e. $D$ is dense in $X$.
\end{proof}


\begin{proof}(Theorem 2.27)
	Let $\mathscr B = \{B_1,B_2,\cdots,B_n,\cdots\}$ be a countable base of $X$. Choose a point $x_n \in B_n,~n \in \N$ and form the set $D = \{x_1,x_2,\cdots,x_n,\cdots\}$, then $D$ is countable.
	
	Claim: $D$ is dense in $X$, i.e. $\overline{D} = X$. Given a nonempty open set $U$ in $X$, then $\exists n \in \N \ni B_n \subseteq U \implies x_n \in U \implies U \cap D \neq \emptyset$. Hence, $\overline{D} = X$ by the remark above. So $X$ is separable.
\end{proof}

\begin{thm}
	Every separable metric space $X$ is of $2^{nd}$ countable.
\end{thm}

\begin{proof}
	Choose a countable dense subset $D$ of $X$. Form the countable collection of open ball $\mathscr{B} = \{B(x,r)~|~ x \in D,r\in \Q^+\}$(it is countable). Claim $\mathscr B$ is a base for $X$. We are done.
	
	$ $
	
	(Note that $\mathscr B$ is a base for a topology space $X$\\$\Leftrightarrow$ every open set in $X$ is a union of some subcollection of $\mathscr B$\\$\Leftrightarrow \forall~$ open set $U \subseteq X$ and $p \in U,~\exists B \in \mathscr B \ni x \in B \subseteq U$) 
	
	$ $
	
	($\Rightarrow$) Given an open set $U$ in $X$ and $p \in U$ by assumption $U = \bigcup_{\alpha \in I}B_{\alpha}$, where $B_{\alpha} \in \mathscr B \implies p \in B_{\alpha_0}$ for some $\alpha_0 \in I \implies p \in B_{\alpha_0} \subseteq U$
	
	($\Leftarrow$) Suppose the condition holds. To prove $\mathscr B$ is a base for $X$. Given a nonempty open set $U$ in $X$. $\forall~p \in U,\exists B_p \in \mathscr B \ni p in B_p \subseteq U.\\~\because U = \bigcup_{p \in U}B_p \therefore \mathscr B$ is a base for $X$. By the remark above, it is enough to show: Given a nonempty open set $U \subseteq X$ and $p \in U$, $\exists B(x,r) \in \B \ni p \in B(x,r) \subseteq U$. Now, $p \in U$ and $U$ is open $\implies \\ \exists t > 0 \ni B(p,r) \subseteq U$. Choose $r \in \Q^+ \ni \dfrac{t}{4} < r < \dfrac{t}{2}$, Since $D$ is dense in $X$, $B(p,r) \cap D \neq \emptyset$. Choose $x \in B(p,r) \cap D$. Then $B(x,r) \in \B$
	\newpage
	
	Claim $p \in B(x,r) \subseteq U$
	
	\begin{enumerate}[wide,label=$\bullet$]
		\item $d(x,p)<r \implies p \in B(x,r)$
		\item $\forall y \in B(x,r),~d(y,p) \leq d(y,x) + d(x,p) < r+r = 2r < t \implies \\ y \in B(p,t) \subseteq U~\therefore y \in U \therefore B(x,r) \subseteq U.$
	\end{enumerate}
\end{proof}



\begin{cor}
	The Euclidean space $\R^k$ is of $2^{nd}$ countable.
\end{cor}

Note that from the proof of Thm 2.28, $\R^k$ has a countable base of the form:

\begin{eqnarray*}
	\B &=& \{B(x,r)~|~x \in  \Q^k ~\&~ r \in \Q^+ \}\\
	&=& = \{A_1,A_2,\cdots\}
\end{eqnarray*}

\begin{thm}
	Every compact metric space $X$ is of $2^{nd}$ countable.
\end{thm}

\begin{proof}
	The last statement follows from Thm 2.27 To prove $X$ is of $2^{nd}$ countable. For each $n \in \N,~\{B(x,\dfrac{1}{n})~|~x \in X\}$ is an open covering of $X$, i.e. $X = \bigcup_{x \in X}B(x,\dfrac{1}{n})$. By companies of $X$, it has a finite subcovering say $X = \bigcup_{i=1}^{l_n}B(x_{n_i},\dfrac{1}{n})$. Then $\mathscr B$ is a countable collection of open balls in $X$.Claim $\B$ is a base for $X$. It suffices to show: given a nonempty open set $U$ and $p \in U,\exists B(x_{n_i},\dfrac{1}{n}) \ni \mathscr B \ni p \in B(x_{n_i},\dfrac{1}{n}) \subseteq U$.
	
	From $p \in U$ and $U$ is open, $\exists r > 0 \ni B(p,r)\subseteq U$. Choose $n >> 0 \ni \dfrac{2}{n} < r$. Since $X = \bigcup^{l_n}_{i=1}B(x_{n_i},\dfrac{1}{n}),~ p \in B(x_{n_i},\dfrac{1}{n})$ for some $1 \leq i \leq l_n$.
	
	Finally, $p \in B(x_{n_i},\dfrac{1}{n}) \subseteq U$
	
	\begin{enumerate}[label = $\bullet$]
		\item $d(p,x_{n_i}) < \dfrac{1}{n} \implies p \in B(x_{n_i},\dfrac{1}{n})$
		\item $\forall y \in B(x_{n_i},\dfrac{1}{n}),d(y,p) \leq d(y,x_{n_i}) + d(x_{n_i},p) < \dfrac{1}{n} + \dfrac{1}{n} = \dfrac{2}{n}$
	\end{enumerate}
	
	$\therefore p \in B(x_{n_i},\dfrac{1}{n}) \subseteq U$
\end{proof}

\begin{thm}[Lindelof Covering]
	Let $S \subseteq \R^k$. Them every open coverning $\mathscr U = \{U_\alpha ~|~ \alpha \in I\}$ of $S$ has a countable subcovernings. 
\end{thm}

\newpage

\begin{proof}
	Let $\{a_1,A_2,\cdots\}$ be the countable base of $\R^k$ defined as above. Note that $S \subseteq \bigcup_{\alpha \in I}~\forall x \in S,x \in U_{\alpha}$ for some $\alpha \in I$. Hence $\exists n \in \N \ni x in A_n \subseteq U_\alpha$. Of course, there may be infinitely many such $n$. We choose one of them and fix it, say $x \in A_{m(x)} \subseteq U_{\alpha}$(e.g. $m(x) = \min \{ n \in \N ~|~ x \in A_n \subseteq U_\alpha\}$). Then the collection $\{A_{m(x)}~|~x \in S\}$ is a countable open covering of $S$. Finally, for each $A_{m(x)}$, choose $U_{\alpha_{m(x)}} \ni A_{m(x)} \subseteq U_{\alpha_{m(x)}}$. Then $\{U_{\alpha_{m(x)}}~|~ x \in X\}$ is a countable subcoverning of $S$.
\end{proof}

\begin{cor}
	Let $S \subseteq \R^k$ be open, if $S = \bigcup_{\alpha \in I}U_\alpha $ is a union of open sets in $X$, then $S = \bigcup^{\infty}_{n = 1}U_{\alpha_n}$ is a countable union. $\because$ By Lindelof covering theorem.	
\end{cor}

\subsection{Perfect Sets in Metric Spaces}

\textbf{Recall}
a subset $E$ in a metric space $X$ is perfect if $E$ is closed in $X$ and and every point of $E$ is its accumulation point. i.e. $E' = E$

\textbf{Example}
\begin{enumerate}[wide,label=$\bullet$]
	\item $-\infty < a < b < \infty,~ [a,b]$ is perfect
	\item $\R$ is perfect
\end{enumerate}

\begin{thm}
	Every nonempty perfect set $E$ in $\R^k$ is uncountable
\end{thm}

\begin{proof}
	$E$ is an infinite set ($\because$ finite set has no accumulation point), Suppose $E$ were countable, write $E = \{x_1,x_2,\cdots\}$. We use induction to construct a sequence $\{V_n\}$ of open sets in $X$ as follows:
	
	Let $V_1$ be any neighborhood of $y_1 = x_1$, e.g. $V_1 = B(x_1,r)$, it's closure is $\overline{V_1} = \overline{B}(x_1,r)$, $x_1 \in E',~V_1 \cap E$ is an infinite set, so $\exists y_2 \in V_1 \cap E \ni y_2 \neq y_1$. Choose a neighborhood $V_2$ of $y_2 \ni$
	\begin{enumerate}[wide,label=$(\roman*)$]
		\item $\overline{V_2} \subseteq V_1$
		\item $x_1 \notin \overline{V_2}$
		\item $V_2 \cap E \neq \emptyset (\because y_2 \in E = E'$ and it's also an infinite set)
	\end{enumerate}
	
	Suppose that, for $n \geq 3,~V_n$ has been chosen $\ni V_n$ is a neighborhood of some $y_n \in E \ni$
	
	\begin{enumerate}
		\item $\overline{V_n} \subseteq V_{n-1}$
		\item $x_{n-1} \notin \overline{V_n}$
		\item $V_n \cap E \neq \emptyset$ is an infinite set 
	\end{enumerate} 
	
	Since $V_n \cap E$ is an infinite set, $\exists y_{n+1} \in V_n \cap E \ni y_{n+1} \neq y_n$. Again, choose a neighborhood  $V_{n+1}$ of $y_{n+1} \ni$
	
	\begin{enumerate}
		\item $\overline{V_{n+1}} \subseteq V_n$
		\item $x_n \notin \overline{V_{n+1}}$
		\item $V_{n+1} \cap E \neq \emptyset$ is an infinite set.
	\end{enumerate}
	
	By induction, we have constructed such sequence $\{V_n\}$. Put $K_n = \overline{V_n} \cap E,~n \geq 1$. Then $\{K_n\}$ is a decrease sequence of nonempty compact sets in $\R^k$.
	
	\begin{tcolorbox}
		$\overline{V_n}$ is closed, $E$ is closed $\implies$ $K_n = \overline{V_n} \cap E$ is closed, each $\overline{V_n}$ is bounded $\therefore K_n$ is closed and bounded by H.B. theorem, $K_n$ is compact.
		
		\begin{enumerate}[label = $\cdot$]
			\item $\emptyset \neq V_n \cap E \subseteq \overline{V_n} \cap E \implies E_n = \overline{V_n} \cap E \neq \emptyset$
			\item $\overline{V_n} \cap E \supseteq \overline{V_{n+1}} \cap E = K_{n+1} \therefore \{K_n\}$ is decrease.
		\end{enumerate}
	\end{tcolorbox}
	
	By Cantor's intersection theorem, $\bigcap^{\infty}_{n=1}K_n \neq \emptyset$. Pick $y \in \bigcap^{\infty}_{n=1}K_n,~ y \in E(\because K_n \subseteq E \forall n \geq 1)$. Since $x_n \notin \overline{V_{n+1}}~\forall n \geq 1,$ so $x_n \notin K_n \forall n \geq 1 \implies y \notin E (\rightarrow\leftarrow)$ to $E = \{x_1,x_2,\cdots\}$
	
	$\therefore E$ is uncountable.
\end{proof}

\begin{cor}
	Every nondegenerate intervvl is uncountable.
\end{cor}

\begin{proof}
	$\because$ Every nondegenerate interval $I$ in $\R$ must contain a closed and bounded interval $[a,b]$ with $a < b$ which is perfect, so it is uncountable by theorem 2.31. Hence $I$ is uncountable.
\end{proof}

Construction of the Cantor set $\underline{P} \subseteq [0,1]$ in $\R$ which is a perfect set

\begin{enumerate}[wide,label = $(\alph*)$]
	\item Remove the middle third open subinterval of $[0,1]$. There are two closed subintervals $[0,\frac{1}{3}]$ and $[\frac{2}{3},1]$. Let $C_1 = (\frac{1}{3},\frac{2}{3})$ and $E_1 = [0,\frac{1}{3}]\cup[\frac{2}{3},1]$
	\item Remove te middle thirds of $[0,\frac{1}{3}]$ and $[\frac{2}{3},1]$ respectively. There are $2^2 = 4$ subintervals $[0,\frac{1}{3^2}],~[\frac{2}{3^2},\frac{3}{3^2}],~[\frac{6}{3^2},\frac{7}{3^2}],~[\frac{8}{3^2},1]$
	\item Countinue this process, we get a sequence $\{C_n\}$ of open sets and a sequence $\{E_n\}$ of closed sets satisfy
	
	\begin{enumerate}[label = $(\roman*)$]
		\item $E_0 \supseteq E_1 \supseteq E_2 \subseteq \cdots$, i.e. $\{E_n\}$ is a decrease sequence of closed sets in $[0,1]$
		\item Each $E_n$ is a union of $2^n$ closed intervals, each of length $3^{-n}$
		\item Each $C_n$ is a union of $2^{n-1}$ open subintervals, each of length $3^{-n}$. total length is $\frac{2^{n-1}}{3^n}$
	\end{enumerate}
\end{enumerate}

\begin{defn}
	$\underline{P}$(or $C$) $ = \bigcap^{\infty}_{n = 1}E_n = [0,1] - \bigcup^{\infty}_{n = 1}C_n$( $= \bigcap^{\infty}_{n=1}([0,1] - C_n)$)
\end{defn}

Properties of Cantor set $\underline{P}:$

\begin{enumerate}[wide]
	\item $\underline{P} \neq \emptyset$ by Cantor's intersection theorem
	\item $\underline{P}$ is compact ($\because \underline{P}$ is closed bein $\cap$ of closed set and $\underline{P} \subseteq [0,1]$,~ $[0,1]$ is compact )
	\item $\underline{P}$ is nowhere dense, i.e. $\overline{\underline{P}}^{\circ} = \emptyset,$ i.e. $\underline{P}^{\circ} = \emptyset$
	
	$\because \underline{P}$ contains no nonempty open subintervals($\because \underline{P}^{\circ} \neq \emptyset \implies \exists x \in P^{\circ} \implies \exists \delta > 0 ,(x-\delta,x+\delta) \subseteq \underline{P}$).If $\alpha < \beta$ and $(\alpha,\beta) \subseteq \underline{P}$, them $(\alpha , \beta) \subseteq E_n \forall n \geq 1$.Choose $n >> 0 \ni \frac{1}{2^n} < \beta - \alpha$. Then for $n >> 0,~E_n$ contains subinterval of length $\geq \frac{1}{2^n} (\rightarrow\leftarrow)$. Hence, $\underline{P}$ is nowhere dense.
	
	\item $\underline{P} = \{\sum^{\infty}_{n=1}\frac{a_n}{3^n} ~|~ a_n = 0 \text{ or } 2 \forall n \geq 1\}$
	
	Recall the ternarry representation of a number $x \in [0,1],~ x = \sum^{\infty}_{n = 1} \frac{a_n}{3^n},~a_n = 0,1,2 \forall n \geq 1$
	
	\begin{tcolorbox}
		if $frac{a_n}{3^n}$ the $a_n$ can be $1$, then $\frac{1}{3} + \frac{1}{3} $ is not in $\underline{P}$
		
		if you want to represent $\frac{1}{3}$, you need use $0 + \frac{2}{3^2} + \frac{2}{3^3} + \cdots$, i.e. $0.1 = 0.0\overline{9}$ same things.
	\end{tcolorbox}
	
	This can be used to prove that $\underline{P}$ is uncountable by $\underline{P} \rightarrow [0,1],~x = \sum^{\infty}_{n=1}\frac{a_n}{3^n} \rightarrow \sum^{\infty}_{n=1}\frac{a_n / 2}{2^n}$ is bijative $\therefore \underline{P}$ is uncountable.
	
	\item $\underline{P}$ is of measure zero (i.e. the length of $\underline{P}$ is zero)
	
	$\because$ The totally length remove in the construction of $\underline{P}$ is $\frac{1}{3} + \frac{2}{3^2} + \frac{2^2}{3^3}+ \cdots = 1$. This also proves that $\underline{P}$ is nowhere dence.
	
	\item $\underline{P}$ is perfect. In particular, by theorem 2.31, $\underline{P}$ is uncountable
	
	$\because$ Obviously $\underline{P}$ is a nonempty closed set. Let $x \in \underline{P}$. Then $\forall (\alpha , \beta) \ni x \in (\alpha , \beta)$. We prove that $(\alpha , \beta) \cap \underline{P} - \{x\} \neq \emptyset$, which says that $x$ is an accumulation points of $\underline{P}$. Hence $\underline{P}$ is perfect. By $x \in \underline{P},~ x \in E_n ~\forall n \geq 1$. Then $\exists$ a closed subinterval $I_n \subseteq E_n \ni x \in I_n$. Choose $n >> 0 \ni I_n \subseteq (\alpha , \beta)$. Let $x_n$ be an end point of $I_n \ni x \neq x_n$. By construction, $x_n \in \underline{P},$ so 
	
	$$x_n \in (\alpha , \beta) \cap \underline{P} - \{x\}$$
	
	i.e. $x$ is an accumulation point of $\underline{P}$, Hence $\underline{P}$ is perfect.
	
\end{enumerate}

\begin{defn}
	Let $X$ be a metric space (or topological space) $A,B \subseteq X$. We sat that $A$ and $B$ are separated if both $A \cap \overline{B}$ and $\overline{A} \cap B$ are empty sets.
\end{defn}

\begin{defn}
	A subset $E \subseteq X$ is called connected if $E$ is not a union of two nonempty separated sets and $E$ is disconnected if $E$ is not connected
\end{defn}

\begin{rmk*}
	$X$ is connected $\Leftrightarrow ~X$ is not a union of two nonempty separated sets.
	
	$X$ is disconnected $\Leftrightarrow ~ X$ is a union of two nonempty separated sets, say, $X = A \cup B$,$A$ and $B$ are nonempty separated. i.e. $\overline{A} \cap B = \emptyset$ and $A \cap \overline{B} = \emptyset$, i.e. $A = \overline{A},~B = \overline{B}~\therefore A$ and $B$ are closed $\therefore A$ and $B$ are both open and closed.
	
\end{rmk*}


\textbf{Remarks and Examples}

\begin{enumerate}[wide]
	\item Separated sets and disjoint
	\item $[0,1]$ and $(1,2)$ are not separated
	\item $(0,1)$ and $(1,2)$ are separated
\end{enumerate}

\begin{thm}
	Let $E \subseteq \R$ be a set. Then $E$ is connected $\Leftrightarrow ~E$ is an interval
\end{thm}

\begin{proof}
	We may assume that $E \neq \emptyset$
	
	($\Rightarrow$) Assume that $E$ is connected. If $E$ were not an interval, then $\exists ~x < y $ in $E$ and $z \notin E \ni x < Z < y$. Let $A = (-\infty , z) \cap E$ and $B = (z , \infty) \cap E$. Then $A,B$ are nonempty, $\overline{A} \cap B = \emptyset , ~A \cap \overline{B} = \emptyset$ and 
	
	\begin{eqnarray*}
		E &=& E \cap (\R - \{z\})\\
		&=& E \cap [(-\infty,z) \cup (z,\infty)]\\
		&=& (E \cap (-\infty,z)) \cup (E \cap (z,\infty))\\
		&=& A \cup B
	\end{eqnarray*}
	
	$\therefore \{A,B\}$ is a nonempty separation of $E$ ($\rightarrow\leftarrow$) to connected. $\therefore E$ is an interval
	
	($\Leftarrow$) Suppose $E$ is an interval. To show that $E$ is connected. If not, then $\exists$ two nonempty separated sets $A$ and $B \ni E = A \cup B$. Pick $x \in A$ and $y \in B$. Then $x \neq y$($\because A \cap B = \emptyset$). We may assume that $x < y$. Define $z = \sup (A \cap [x,y])$, By (9) in section 2.5, $z \in \overline{A \cap [x,y]} \in \overline{A}$. Hence $z \notin B(\because \overline{A} \cap B = \emptyset)$. Also $x \leq z \leq y$. But $y \in B$ and $z \notin B \implies x \leq z < y$
	
	If $z \notin A$, then $x < z < y$ and $z \notin E (\rightarrow\leftarrow)$ to $E$ is an interval.
	
	If $z \in A$, then $z \notin \overline{B}(\because A \cap \overline{B} = \empty)$, since $z \notin B,~z \in \R - \overline{B}$ which is open $\implies \exists \delta > 0 \ni (z - \delta , z+\delta) \subseteq \R - \overline{B}$. Choose $z < z_1 < z + \delta < y$, i.e. $z < z_1 < y$. Then $z,y \in E,~z < y$ and $z_1 \notin E (\rightarrow\leftarrow)$ to $E$ is an interval
\end{proof}


\textbf{Application of Connectedness}

$X:$ connected topological space (or metric space)

$P:$ a property on $X$

$D = \{x \in X ~|~ P \text{ holds at } x\}$

If one can prove $D$ is nonempty and closed and open, them $D = X$

$\because X = D \cup (X - D),~\overline{D} \cap (X - D) = \empty$ and $D \cap \overline{(X-D)}$, i.e. $D$ and $X-D$ are separated

Since $X$ is connected and $D \neq \emptyset$, so $X-D = \emptyset$, i.e. $X = D$ 






