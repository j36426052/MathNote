\documentclass[12pt,reqno]{amsart}
%\usepackage[margin=1in]{geometry}
\usepackage{tcolorbox}
\usepackage{amssymb}
\usepackage{amsthm}
\usepackage{amsmath}
\usepackage{amssymb}
\usepackage{mathrsfs}
\usepackage{centernot}
\usepackage{lastpage}
\usepackage{fancyhdr}
\usepackage{accents}
\usepackage{tasks}
\usepackage{graphicx}
\usepackage{natbib}
\usepackage{tabularx}
\usepackage{multirow}
\usepackage{booktabs}
\usepackage{hyperref}
\usepackage{bm}
\usepackage{float}
\theoremstyle{plain}
\usepackage{multicol}
\usepackage{enumitem,kantlipsum}

% 加上浮水印
%\usepackage{wallpaper}
%\CenterWallPaper{.180}{../qsnake-logo.jpg}


\linespread{1.2}
\parindent = 0pt
\pagestyle{fancy}
\setlength{\parindent}{0pt}
%\everymath{\displaystyle}
%new area
%\usepackage[utf8]{inputenc}
%\usepackage{CJKutf8}
%\xeCJKsetup{AutoFakeBold=true, AutoFakeSlant=true}

% 設定頭部
\fancyhead[L]{Midterm} % 左邊頭部清空
\fancyhead[C]{} % 中間頭部清空
\fancyhead[R]{} % 右邊頭部顯示頁碼

% Adjust the footer as desired:
\fancyfoot[L]{} % Left footer: Empty.
\fancyfoot[C]{\thepage} % Center footer: Empty.
\fancyfoot[R]{} % Right footer: Empty.



% we will modify sections, subsections and sub subsections
\RequirePackage{titlesec}
% Modification of section 
\titleformat{\section}[block]{\normalsize\bfseries\filcenter}{\thesection.}{.3cm}{} 


% modification of subsection and sub sub section
\titleformat{\subsection}[runin]{\bfseries}{ \thesubsection.}
{1mm}{}[.\quad]
\titleformat{\subsubsection}[runin]{\bfseries\itshape}{ \thesubsubsection.}
{1mm}{}[.\quad]

\newenvironment{solution}
  {\renewcommand\qedsymbol{$\blacksquare$}
  \begin{proof}[Solution]}
  {\end{proof}}
\renewcommand\qedsymbol{$\blacksquare$}

\newcommand{\ubar}[1]{\underaccent{\bar}{#1}}

%%%%%%%%%%%%%%%%%%%%%%%%%%%%%% Textclass specific LaTeX commands.
%\theoremstyle{plain}
%\newtheorem{thm}{\protect\theoremname}[section]
\newtheorem{thm}{\textbf{Theroem}}[section]
\newtheorem{cor}[thm]{Corollary}
\newtheorem{lmma}[thm]{Lemma}
\newtheorem*{defn}{\underline{Definition}}
\newtheorem*{prop*}{Proposition}
\newtheorem*{ex*}{Example}
\newtheorem*{sol*}{Solution}
\newtheorem*{cor*}{Corollary}
\newtheorem*{thm*}{Theorem}
\newtheorem*{lmma*}{Lemma}
\newtheorem*{rmk*}{Remark}
\newtheorem*{pf*}{\underline{\textbf{Proof\ }}}

%%%%%%%%%%%%%%%%%%%%%%%%%%%%%% User specified LaTeX commands.
\renewcommand{\P}{\mathscr{P}}
\newcommand{\B}{\mathscr{B}}
\newcommand{\A}{\mathscr{A}}
\newcommand{\C}{\mathbb{C}}
\newcommand{\CC}{\mathscr{C}}
\newcommand{\R}{\mathbb{R}}
\newcommand{\Q}{\mathbb{Q}}
\newcommand{\Z}{\mathbb{Z}}
\newcommand{\N}{\mathbb{N}}
\newcommand{\X}{\mathcal{X}}
\newcommand{\T}{\mathscr{T}}
\newcommand{\arbuni}{\bigcup_{\alpha\in I}}
\newcommand{\finint}{\bigcap_{i=1}^n}
\newcommand{\Ua}{{\textsc{U}_\alpha}}
\newcommand{\Ui}{\textsc{U}_i}
\newcommand{\pair}[2]{\left( \,#1\,,\,#2\,\right) }
\newcommand{\dint}[2]{\int_{#1}^{#2}}
\newcommand{\sett}[1]{\left\{ \,#1 \,\right\}}
\newcommand{\linearcombination}[2]{#1_1#2_1+\cdots+#1_n#2_n}
\newcommand{\slinearcombination}[1]{#1_1+\cdots+#1_n}
\newcommand{\spann}[1]{\text{span($#1$)}}
\newcommand{\sub}[1]{\text{sup}}
\newcommand{\inn}[1]{\left< #1 \right>}
\newcommand{\kernal}[1]{Ker(#1)}
\newcommand{\image}[1]{Im(#1)}
\newcommand{\norm}[1]{\parallel #1 \parallel}
\newcommand{\dia}[0]{\text{dia}}
\newcommand{\marking}[1]{\text{\color{red} #1}}
%%%%%%%%%%%%%%%%%%%%%%%%%%%%%%


\begin{document}

\chapter{Vector Space}
 
\begin{defn}[Vector Space]
 A vector space (or linear space) $W$ over a $Field$ $\mathbb{F}$ consists of a set on which two operations (called addition and scalar multiplication, respectively) are defined so that for each pair of elements $x, y$, in $W$ there is a unique element $x+y$ in $W$, and for each element a in F and each element $x$ in $W$ there is a unique element $ax$ in $W$, such that the following conditions hold.
\end{defn} 

\section*{\S\ Subspace }

\begin{defn}[Subspace]
	A subset $W$ of a vector space $W$ over a field $\mathbb{F}$ is called a subspace of $W$ if $W$ is a vector space over $\mathrm{F}$ with the operations of addition and scalar multiplication defined on $W$.
	
\end{defn}



\begin{rmk*}
	Trivial subspaces of a vector space $V$, namely $V$ itself and $\{0\}$. Note that empty set $\phi$ is not a vector space, since it does not contains a zero vector.
\end{rmk*}

\begin{thm}Let V be a vector space and W a subset of V. Then W is a subspace of V if and only if the following three conditions hold for the operations defined in V.
\\
(a)\ \ $0 \in W.$ \\
(b)\ \ $x+y \in W$ whenever $x \in W$ and $y \in W.$ \\
(c)\ \ $cx \in W$ whenever $c \in \mathrm{F}$ and $x \in W.$ \\

\end{thm}

\begin{cor}
	Let $W$ be a subset of vector space $V$. $W$ is a subspace of $V$ if and only if $0\in W$ and $ax+y \in W$ whenever $a \in F$ and, $ x,y \in W$.
\end{cor}
%left proof to Jester

\begin{thm}
 Any intersection of subspaces of a vector space $W$ is a subspace of $V$.
\end{thm}

\begin{thm}
	Let $W_1$ and $W_2$ be subspaces of a vector space $V$, then $W_1 \cup W_2$ is a subspace of $V$ if and only if $W_1 \subseteq W_2$ or $W_2 \subseteq W_1$. 
\end{thm}

\begin{defn}
	If $S_1$ and $S_2$ are nonempty subsets of a vector space $V$, then the sum of $S_1$ and $S_2$, denoted $S_1 + S_2$, is the set $\{x+y:x \in S_1$ and $y\in S_2\}$.
\end{defn}

\begin{thm}
	Let $W_1$ and $W_2$ be subspaces of a vector space $V$.
	\begin{itemize}
		\item[(a)] $W_1 + W_2$ is a subspace of $V$ that contains both $W_1$ and $W_2$.
		\item[(b)] Any subspace of V that contains both $W_1$ and $W_2$ must also contain $W_1 + W_2$. 
	\end{itemize}
\end{thm}

\begin{defn}[Direct Sum] A vector space $V$ is called the direct sum of $W_1$ and $W_2$ if $W_1$ and $W_2$ are subspaces of V such that $W_1 \cap W_2 = \{0\}$ and $W_1 + W_2 = V$. We denote that $V$ is the direct sum of $W_1$ and $W_2$ by writing $V = W_1 \oplus W_2$.
\end{defn}

\begin{thm}
	Let $W_1$ and $W_2$ be subspaces of a vector space $V$. $V$ is the direct sum of $W_1$ and $W_2$ if and only if each vector in $V$ can be uniquely written as $x_1 + x_2$, where $x_1 \in W_1$ and $x_2 \in W_2$
\end{thm}

\section*{\S\ Linear Combinations and Bases}

\begin{defn}[Linearly Dependent]
	A subset $S$ of a vector space $W$ is called linearly dependent if there exist a finite number of distinct vectors $u_1, u_2, . . . , u_n$ in $S$ and scalars $a_1,a_2,...,a_n$, not all zero, such that 
	
		$$a_1u_1 + a_2u_2 +\cdots +a_nu_n = 0 .$$
	
In this case we also say that the vectors of $S$ are linearly dependent.

\end{defn}

\begin{defn}[Linearly Independent]
	A subset $S$ of a vector space that is not linearly dependent is called linearly independent. As before, we also say that the vectors of $S$ are linearly independent.
\end{defn}

\begin{rmk*}$ $
	\begin{enumerate}
		\item The empty set is linearly independent, for linearly dependent sets must be nonempty.
		\item A set consisting of a single nonzero vector is linearly independent. For if $\{u\}$ is linearly dependent, then $au = 0$ for some nonzero scalar $a$. Thus
			$$ u = a^{-1}(au) = a^{-1} 0 = 0 $$
		\item A set is linearly independent if and only if the only representations of $0$ as linear combinations of its vectors are trivial representations.
	\end{enumerate}
\end{rmk*}

\begin{thm}
	Let $V$ be a vector space, and let $S_1 \subseteq S_2 \subseteq V$. If $S_1$ is linearly dependent, then $S_2$ is linearly dependent.
\end{thm}

\begin{cor}
	Let $S$ be a linearly independent subset of a vector space $V$, and let $v$ be a vector in $V$ that is not in $S$. Then $S \cup \{v\}$ is linearly dependent if and only if $v \in span(S)$.
\end{cor}

\begin{defn}[Basis]
A basis $\beta$ for a vector space $V$ is a linearly independent subset of $V$ that generates $V$. If $\beta$ is a basis for $V$, we also say that the vectors of $\beta$ form a basis for $V$.
		
\end{defn}

\begin{thm}
	Let $V$ be a vector space and $\beta = \{u_1,u_2,...,u_n \}$ be a subset of $V$. Then $\beta$ is a basis for V if and only if each  $v\in$ V can be uniquely expressed as a linear combination of vectors of $\beta$, that is, can be expressed in the form

	$$ V = a_1u_1 + a_2u_2 + \cdots+ a_nu_n $$

for unique scalars $a_1, a_2, \cdots , a_n$.	
\end{thm}

\begin{thm}
	If a vector space $V$ is generated by a finite set $S$, then some subset of $S$ is a basis for $V$. Hence $V$ has a finite basis.	
\end{thm}

\begin{thm}[Replacement Theorem]
	Let $V$ be a vector space that is generated by a set $G$ containing exactly $n$ vectors, and let $L$ be a linearly independent subset of $V$ containing exactly $m$ vectors. Then $m \leq n$ and there exists a subset $H$ of $G$ containing exactly $n - m$ vectors such that $L \cup H$ generates $V$.
\end{thm}

\begin{defn}[Finite-Dimensional]
	A vector space is called finite-dimensional if it has a basis consisting of a finite number of vectors. The unique number of vectors
 in each basis for $V$ is called the dimension of $V$ and is denoted by $dim(V)$.
A vector space that is not finite-dimensional is called infinite-dimensional.
\end{defn}


\begin{cor}
	Let $V$ be a vector space with dimension $n$.
	\begin{enumerate}
		\item Any finite generating set for $V$ contains at least n vectors, and a generating set for $V$ that contains exactly n vectors is a basis for $V$.
		\item Any linearly independent subset of $V$ that contains exactly n vectors is a basis for $V$.
		\item Every linearly independent subset of $V$ can be extended to a basis for $V$.
	\end{enumerate}
\end{cor}

% The figure is left to Jester

\begin{thm}
Let $W$ be a subspace of a finite-dimensional vector space $V$. Then $W$ is finite-dimensional and $\dim(W) \leq \dim(V)$. Moreover, if $\dim(W) = \dim(V)$, then $V = W$.
\end{thm}

\begin{prop}
	Let $W_1$ and $W_2$ be subspaces of a finite-dimensional vector space $V$. $W_1 \subseteq W_2$ if and only if $\dim(W_1 \cap W_2) = \dim(W_1)$
\end{prop}

\begin{thm}
Let $v_1, v_2, \cdots , v_k, v$ be vectors in a vector space $V$, and define $W_1 = span(\{v_1, v_2, \cdots , v_k\})$, and $W_2 = span(\{v_1, v_2, \cdots , v_k , v \})$.Then $v \in \mathrm{span}(W_1)$ if and only if $\dim(W_1) = \dim(W_2)$.
\end{thm}

\begin{rmk*}
	We may give an example for satisfying the conditions on above but $\dim(W_1) \neq \dim(W_2)$.
\end{rmk*}

%\begin{defn}[Direct Sum (Recall)]
%	A vector space $V$ is called the direct sum of $W_1$ and $W_2$ if $W_1$ and $W_2$ are subspaces of V such that $W_1 \cap W_2 = \{0\}$ and $W_1 + W_2 = V$. We denote that $V$ is the direct sum of $W_1$ and $W_2$ by writing $V = W_1 \oplus W_2$.
% \end{defn}

\begin{thm}Let $W_1$ and $W_2$ be finite-dimensional subspaces of a vector space $V$.
\begin{enumerate} 
	\item [(a)]Then the subspace $W_1 + W_2$ is finite-dimensional, and $$\dim(W_1 + W_2) = \dim(W_1) + \dim(W_2) - \dim(W_1 \cap W_2)$$
    \item [(b)]  Let $V = W_1 + W_2$. Deduce that $V$ is the direct sum of $W_1$ and $W_2$ if and only if $$\dim(V) = \dim(W_1) + \dim(W_2)$$
\end{enumerate}	
\end{thm}

\begin{thm}Let $W_1$ and $W_2$ be subspaces of a vector space $V$ such that $V = W_1 \oplus W_2$ if and only if there exist base $\beta_1$ , $\beta_2$ of $W_1$ , $W_2$, respectively such that $\beta_1 \cup \beta_2$ is a basis for $V$.
\end{thm}

\begin{thm}
		\item If $W_1$ is any subspace of vector space of $V$, then there exists a subspace $W_2$ of $V$ such that $$ V = W_1 \oplus W_2 $$
\end{thm}





\end{document}