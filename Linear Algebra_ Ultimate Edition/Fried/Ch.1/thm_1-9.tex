\begin{proof}
	We prove it by mathematical induction on $m$. For $m=0$, $L = \emptyset$, so take $H=G$. Assume the theorem holds for some integer $m \geq 0$. For $ m+1 $, let $L = \{ v_1,...,v_{m+1} \}$ be a linear independent subset of $V$ containing $m+1$ vectors. Because $\{v_1,...,v_m\}$ is linear independent, so by the induction  hypothesis, we have $m \geq n$ and $\exists$ a subset $\{u_1,...,u_{n-m}\}$ of $G$ such that $\{v_1,...,v_m\} \cup \{u_1,...,u_{n-m}\}$  generates $v$. Thus $\exists$ scalars $a_1,...,a_m,b_1,...,b{n-m}$ such that $a_1v_1+...+a_mv_m+b_1u_1+...+b_{n-m}u{n-m}=v{m+1}$. We have $n-m>0$ and $v_{m+1}$ is a linear combination of $\{v_1,...,v_m\}$ which contradicts the assumption that $L$ is linear independent. Hence $n>m$ that is $n \geq m+1$. Moreover, say $b_1$ is nonzero, otherwise we obtain the same contradiction. We have $u_1 = (-b_1^{-1}a_1)v_1+...+(-b_1^{-1}a_m)v_m+(b_1^{-1})v_{m+1}+(-b_1^{-1}b_2)u_2+...+(-b_1^{-1}b_{n-m})u_n-m$. Let $H = \{u_2,...,u_{n-m}\}$. Then $u_1 \in span(L \cup H)$ and $v_1,...,v_m,u_2,...,u_{n-m} \in span(L \cup H)$, so $ \{v_1,...,v_m,u_1,...,u_{n-m} \} \subseteq span(L \cup H)$.  We have $span(L \cup H)=V$. Since $H$ is a subset of $G$  contains $(n-m)-1=n-(m+1)$ vectors. So by mathematical induction, we are done.
\end{proof}