\documentclass[12pt]{article}
\usepackage[margin=1in]{geometry} 
\usepackage{amsmath}
\usepackage{tcolorbox}
\usepackage{amssymb}
\usepackage{amsthm}
\usepackage{mathrsfs}
\usepackage{lastpage}
\usepackage{fancyhdr}
\usepackage{accents}
\usepackage{tasks}
\linespread{1.2}
\parindent = 0pt
%\pagestyle{fancy}
\setlength{\headheight}{40pt}
\everymath{\displaystyle}

% we will modify sections, subsections and sub subsections
\RequirePackage{titlesec}
% Modification of section 
\titleformat{\section}[block]{\normalsize\bfseries\filcenter}{\thesection.}{.3cm}{} 

% modification of subsection and sub sub section
\titleformat{\subsection}[runin]{\bfseries}{ \thesubsection.}
{1mm}{}[.\quad]
\titleformat{\subsubsection}[runin]{\bfseries\itshape}{ \thesubsubsection.}
{1mm}{}[.\quad]

\newenvironment{solution}
  {\renewcommand\qedsymbol{$\blacksquare$}
  \begin{proof}[Solution]}
  {\end{proof}}
\renewcommand\qedsymbol{$\blacksquare$}

\newcommand{\ubar}[1]{\underaccent{\bar}{#1}}

%%%%%%%%%%%%%%%%%%%%%%%%%%%%%% Textclass specific LaTeX commands.
%\theoremstyle{plain}
\newtheorem{thm}{\protect\theoremname}[section]
\newtheorem{cor}[thm]{Corollary}
\newtheorem{lmma}[thm]{Lemma}
\newtheorem*{defn}{\underline{Definition}}
\newtheorem*{ex*}{Example}
\newtheorem*{sol*}{Solution}
\newtheorem*{cor*}{Corollary}
\newtheorem*{thm*}{Theorem}
\newtheorem*{lmma*}{Lemma}
\newtheorem*{rmk*}{Remark}
\newtheorem*{pf*}{\underline{\textbf{Proof\ }}}

%%%%%%%%%%%%%%%%%%%%%%%%%%%%%% User specified LaTeX commands.
\renewcommand{\P}{\mathscr{P}}
\newcommand{\B}{\mathscr{B}}
\newcommand{\A}{\mathscr{A}}
\newcommand{\C}{\mathbb{C}}
\newcommand{\CC}{\mathscr{C}}
\newcommand{\R}{\mathbb{R}}
\newcommand{\Q}{\mathbb{Q}}
\newcommand{\Z}{\mathbb{Z}}
\newcommand{\N}{\mathbb{N}}
\newcommand{\X}{\mathcal{X}}
\newcommand{\T}{\mathscr{T}}
\newcommand{\arbuni}{\bigcup_{\alpha\in I}}
\newcommand{\finint}{\bigcap_{i=1}^n}
\newcommand{\Ua}{{\textsc{U}_\alpha}}
\newcommand{\Ui}{\textsc{U}_i}
\newcommand{\pair}[2]{\left( \,#1\,,\,#2\,\right) }
\newcommand{\dint}[2]{\int_{#1}^{#2}}
\newcommand{\sett}[1]{\left\{ \,#1 \,\right\}}
\newcommand{\linearcombination}[2]{#1_1#2_1+\cdots+#1_n#2_n}
\newcommand{\slinearcombination}[1]{#1_1+\cdots+#1_n}
\newcommand{\spann}[1]{\text{span($#1$)}}
\newcommand{\sub}[1]{\text{sup}}
\newcommand{\inn}[1]{\left< #1 \right>}
%%%%%%%%%%%%%%%%%%%%%%%%%%%%%%



\begin{document}

\title{A First Course In Stochastic Processes}

\author{QSnake Edition}

\maketitle

%\section*{Probability Lecture 1}

%\lhead{Linear Algebra} 
%\rhead{Sabrina Edition} 
%\cfoot{\thepage} %\ of \pageref{LastPage}}

\section*{1. 03/08/2024}$ $

\subsection*{Chapter 1}

\begin{enumerate}
	\item $(\Omega,\mathscr{F},P)$ a sample space
	\item random variable $X$
	\item $P(X \in A) \in \mathscr{F}$ for all $A \in mathscr{B}$
	\item moment generating function $\phi_X(\lambda) = E[e^{\lambda X}]$
\end{enumerate}


\begin{prop*}
	If $\phi_X(\lambda) = \phi_Y(\lambda) \Rightarrow X$ and $Y$ has same distribution.
\end{prop*}

\begin{ex*}
	$X = N(0,1)$
	
	\begin{eqnarray*}
		\phi_X(\lambda) = E[e^{\lambda X}] &=& \int_{\R}e^{\lambda X}\frac{1}{\sqrt{2\pi}}e^{-\frac{x^2}{2}} dx \\
		&=& \int_R \frac{1}{\sqrt{2\pi}}e^{-\frac{1}{2}[x^2 - \frac{\lambda x}{2} + (\frac{\lambda}{2})^2] + \frac{\lambda^2}{2}}dx\\
		&=& \int_{\R}\frac{1}{2\pi}e^{-\frac{1}{2}(x - \frac{\lambda}{2})^2 + \frac{\lambda^2}{2}}dx \\
		&=& e^{\frac{\lambda^2}{2}} \cdots (\star)
	\end{eqnarray*}
	
	If $X = N(\mu,\sigma^2)$
	
	\begin{eqnarray*}
		\phi_X(\lambda) &=& \int_{\R}e^{\lambda X}e^{-\frac{(X - \mu)^2}{2\sigma^2}\frac{1}{\sqrt{2\pi\sigma^2}}}dx \\
		&=& \int_{\R}e^{\lambda(\mu + \sigma y)}e^{-\frac{y^2}{2}\frac{1}{\sqrt{2\pi}}}dy  \quad\text{   (let y = $\frac{X - \mu}{\sigma}$)}\\
		&=& e^{\lambda\mu}\int_{\R}e^{\lambda \sigma y}e^{-\frac{y^2}{2}\frac{1}{\sqrt{2\pi}}}dy \\
		&=& e^{\lambda\mu}\int_{\R}e^{\lambda\sigma y}e^{-\frac{y^2}{2}\frac{1}{2\pi}}dy\\
		&=& e^{\lambda\mu + \frac{\lambda^2\sigma^2}{2}} \text{(by $(\star)$)}
	\end{eqnarray*}
	
	If we want to calculate $EX^{2n}$ when $X = N(0,1)$
	
	$$\int_\R x^{2n}\frac{1}{2\pi}e^{-\frac{x^2}{2}}$$
	
	but we have moment
	
	\begin{eqnarray*}
		E[e^{\lambda X}] &=& e^{-\frac{x^2}{2}} = \sigma^{\infty}_{n = 0}\frac{1}{n!}(\frac{\lambda^2}{2})^n \\
		&=& E[\sigma^{\infty}_{n = 0}\frac{\lambda^n}{n!}] = \sum^{\infty}_{n=0}\frac{\lambda^n}{n!}E[X^n]
	\end{eqnarray*}
	
	$\implies E[X^{2n + 1}] = 0$ and
	
	$$\frac{\lambda^{2^n}}{(2n)!} E[X^{2n}] = \frac{1}{n!}(\frac{\lambda^{2n}}{2^n}) \text{\quad for all $\lambda \in \R$}$$
	
	$\implies E[X^{2n}] = \frac{(2n)!}{n!2^n}\quad n = 1,2,\cdots$
\end{ex*}



\begin{ex*}
	If $X,Y$ are independent and standard normal (i.e. $N(0,1)$). Then $(X,Y)$, we can define a moment generating function
	
	\begin{eqnarray*}
	\phi(a_1,a_2) &=& E[e^{a_1X + a_2Y}] \\
	&=& E[e^{a_1X}]E[e^{a_2^Y}] \\
	&=& \phi(a_1)\phi(a_2) \\
	&=& e^{\frac{a_1^2 + a_2^2}{2}}
	\end{eqnarray*}
	
	On the other hand the distribution
	
	\begin{eqnarray*}
	P(X \leq c_1, Y \leq c_2) &=& P(X \leq c_1)P(Y \leq c_2) \\
	&=& \left(\int^{c_1}_{-\infty}\frac{1}{\sqrt{2\pi}}e^{-\frac{x^2}{2}}dx\right)\left(\int^{c_2}_{-\infty}\frac{1}{\sqrt{2\pi}}e^{-\frac{y^2}{2}}dy\right)\\
	&=& \int^{c_1}_{-\infty}\int^{c_2}_{-\infty}\frac{1}{2\pi}e^{-(\frac{x^2+y^2}{2})}dxdy\\
	& & \implies f(x,y) = \frac{1}{2\pi}e^{-\frac{x^2+y^2}{2}}
	\end{eqnarray*}
	
	$X_1,\cdots,X_n$ are $N(0,1)$ and independent
	
	$\implies f(x_1,\cdots,x_n) = (\frac{1}{2\pi})^{\frac{n}{2}}e^{-(\frac{x_1^2+\cdots+x_n^2}{2})},\quad n\geq 1$
	
	and the MGF $\phi(a_1,\cdots,a_n) = e^{\frac{a_1^2 + \cdots + a_n^2}{2}}$
\end{ex*}

Important inequality:

\begin{enumerate}
	\item Markov inequality: $P(X > a) \leq \frac{E|X|}{a}$
	\item Chebyshev inequality: $P(X > a) \leq \frac{EX^2}{a^2}$
\end{enumerate}

\subsection*{An n dimensional boll's volume}


\subsection*{Chapter 2}

\begin{rmk*}
	$Cov(X,Y) = 0 \Leftrightarrow X \text{ and } Y$ are not correlated $\centernot\implies$ they are independent.
\end{rmk*}

Consider a random vector $X = (X_1,\cdots,X_n):(\Omega,\mathscr{F},P) \rightarrow \R^n$

and

$$E[X] = (EX_1,\cdots,EX_n) = m = (m_1,\cdots,m_n)$$

The covariance matrix is defined as follow:

$$\mathscr{C} = (\mathscr{C})_{1 \leq 1,j \leq n}, \mathscr{C}_ij = E[(X_i - E[X_i])(X_j - E[X_j])], i,j = 1,2,\cdots,n$$

\begin{prop*}
	$\mathscr{C}$ is symmetry and $\geq 0$
	
	Consider for any $a = (a_1,\cdots,a_n)^T \in \R^n$
	
	\begin{eqnarray*}
		0 &\leq & E[a_1X_1 + \cdots + a_nX_n - E(aX_1 + \cdots + a_nX_n)]^2 \\
		&=& E[\sum^n_{k = 1}(a_kX_k - a_kE[x_k])(\sum^n_{j=1}(a_jX_j - a_j E[X_j]))]^2 \\
		&=& \sum^n_{k,j = 1}a_ja_kE[(X_k - E[X_k])(X_jE[X_j])] \\
		&=& \sum^n_{k,j=1}a_ka_j\mathscr{C}_{k,j} = a^T\mathscr{C}a \implies e \geq 0
	\end{eqnarray*}
\end{prop*}

\begin{defn} $X,Y$ are uncorrelated if $(\mathscr{C})_{ij} = 0$ for all $i \neq j$

\end{defn}

\begin{prop*}
	$E[X^n] = E[Y^n]$ for all $n \in \N \implies X,Y$ has the same distribution.
	
	$$\phi(e^{\lambda X}) = \phi_X(\lambda) = \phi_Y(\lambda) = \phi(e^{\lambda Y})$$
\end{prop*}

\begin{defn}
	A n-dimensional random variable $X = (X_1,\cdots,X_n)$ is called to be Gaussian $\Leftrightarrow \quad a_1X_1 + \cdots + a_nX_n$ is Gaussian for any $a_1,\cdots,a_n \in \R$
\end{defn}

\begin{rmk*}
\begin{enumerate}
	\item If $X$ is Gaussian $\Rightarrow X_k$ is Gaussian for $k = 1,2,\cdots,n$
	\item If $X$ and $Y$ are independent $N(0,1)$ then $a_1X + a_2Y$ is a $N(0,a_1^2 + a_2^2)$ for all $a_1,a_2 \in \R \implies (X,Y)$ is Gaussian
\end{enumerate}
\end{rmk*}

\begin{lmma*}
	If $X$ is an n'dimensional Gaussian vector and $M$ is $m*n$ matrix then $MX$ is m-dimensional Gaussian vector.
	
	$$\left( \begin{matrix}
	M_1 \\ M_2 \\ \vdots \\ M_m
	\end{matrix}\right)
	X = 
	\left(\begin{matrix}
	X_1X \\ M_2X \\ \vdots \\ M_nX
	\end{matrix}\right)$$
\end{lmma*}

\begin{prop*}
	A random variable $X = (X_1,\cdots X_n)$
	
	$X$ is Gaussian $\Leftrightarrow$ MGF of $X$ is $e^{a^Tm + \frac{1}{2}a^Tea} = E[e^{a^TX}] = \phi(a_1,\cdots,a_n) = \phi(a)$
	
	$X = N(\mu,\sigma^2) \implies \phi_X(\lambda) = e^{E(\lambda X) + \frac{E(\lambda(X - EX))^2}{2}}$
	
	\begin{eqnarray*}
	\phi(a) = E[e^{a^TX}] &=& e^{E(a^TX) + \frac{E(a^T(X-E[X^2]))}{2}} \\
	&=& e^{a^Tm + \frac{E(a^TXX^Ta)}{2}} \\
	&=& e^{a^Tm + \frac{a^Tea}{2}}
	\end{eqnarray*}
	
	note $E[a^TX - E(a^TX)]^2 = (a^TE(X-EX))^2 = a^TE(X - EX)(E(X-EX))^Ta = e$
\end{prop*}

\begin{prop*}
	The covariance matrix is diagonal $\Leftrightarrow$ the random variable are independent.
	
	\begin{proof}
	\begin{enumerate}
	\item[$(\Rightarrow)$] By $e^{a^Tm + \frac{1}{2}a^T\mathscr{C}a}$
	
	\begin{eqnarray*}
		\phi(a) &=& e^{\sum^n_{j=1}m_ja_j + \frac{1}{2}\sum^n_{j=1}d_{jj}^2a_{j}^2} \\
		&=& \prod^n_{j=1}e^{m_ja_j + \frac{d_jja_j^2}{2}} \\
		&=& \prod^n_{j=1}\phi_{X_j}(a_j) \text{ where }X_j = N(m_{jj},d_j^2) \\
		&\implies & f(x_1,\cdots,x_n) = \prod^n_{j=1}f_{X_j}(x_j)
	\end{eqnarray*}	
	\item[$(\Leftarrow)$] If $X_1,\cdots,X_n$ is independent
	
	$\Rightarrow \mathscr{C}_{ij} = E[(X_i - EX_i)(X_j - EX_j)] = 0$ for all $i \neq j$
	
	\end{enumerate}
	\end{proof}
\end{prop*}

Example: $Z_1, Z_2, Z_3 = N(0,1)$ are independent

$X = Z_1 + Z_2 + Z_3$

$Y = Z_1 + Z_2$

$Z = Z_3$

$(X,Y,Z)$ is Gaussian vector with mean $(0,0,0)$

covariance $\mathscr{C} =
\left(\begin{matrix}
EX^2 & EXY & EXZ \\
EYX & EY^2 & EYZ \\
EZX & EZY & EZ^2
\end{matrix}\right)
= 
\left(\begin{matrix}
3 & 2 & 1 \\
2 & 2 & 0\\
1 & 0 & 1
\end{matrix}\right)$

det$\mathscr{C} = 0$(degenerate Gaussan vector)

Note $X - Y - Z =0$ independent.

\begin{lmma}
	Let $X = (X_1,\cdots,X_n)$ be a Gaussian vector. Then $X$ is degerate (i.e. det$\mathscr{C} = 0$) $\Leftrightarrow$ the coordinates are linear dependent.
\end{lmma}

\begin{prop*}
	$X = (X_1,\cdots,X_n)$ be a non-degenerate Gaussian vector with mean $m$ and covariance matrix $\mathscr{C}$ then the joint distribution of $X$ is given by the joint PDF.
	
	$$f(x_1,\cdots,x_n) = \frac{e^{-\frac{1}{2}(x - m)^Te^{-1}(x-m)}}{(2\pi)^{\frac{n}{2}\text{det}(e)^{\frac{1}{2}}}}$$
\end{prop*}

Example: Consider a Gaussian random variable $X_1,X_2$ mean $0$ and covariance $\mathscr{C} =
\left(\begin{matrix}
2 & 1 \\ 1 & 2
\end{matrix}\right)$

$\implies \mathscr{C}^{-1} =
\left(\begin{matrix}
\frac{2}{3} & \frac{-1}{3} \\ -\frac{1}{3} & \frac{2}{3}
\end{matrix}\right)$
and det$(\mathscr{C}) = 3$

$\implies f(x,y) = \frac{e^{(-\frac{x^2}{3} + \frac{1}{3}xy - \frac{1}{3}y^2)}}{2\pi\sqrt{3}}$



\begin{prop*}
	Let $X = (X_1,\cdots,X_n)$ be a Gaussian vector with mean $0$. If $X$ is Non-degenerate $\exists \quad n$ i.i.d. $N(0,1)$ random variable $Z = (Z_1,\cdots,Z_n)$ and invertible $n*n$ matrix $A \ni X = AZ$.
\end{prop*}

Example:

$X_1 = W_1 + W_2$

$X_2 = W_1 - W_2$

$W_1,W_2 = N(0,1)$, independent

$X = AW$ where $A =
\left(\begin{matrix}
1 & 1 \\ 1 & -1
\end{matrix}\right)$

The covariance matrix of $X$ is 
$\left(\begin{matrix}
2 & 0 \\ 0 & 2
\end{matrix}\right)$

$\implies$ Let $Z_1 = \frac{X_1}{\sqrt{2}},~Z_2 = \frac{X_2}{\sqrt{2}}$

$\implies Z_1,Z_2$ is $N(0,1)$

\begin{proof}
	Take $Z_1 = \frac{X_1}{\sqrt{c_{11}}} \implies Z_1$ is $N(0,1)$ define $Z_2' = X_2 - E[X_2Z_1]Z_1 \implies E[Z_2X_2] = 0$
	
	Let $Z_2 = \frac{Z_2'}{\sqrt{\text{var}(Z_2')}} = N(0,1)$
	
	$\vdots$
	
	$Z = (Z_1,Z_2,\cdots,Z_n),~Z_k = N(0,1)$ is independent
	
	$A = 
	\left(\begin{matrix}
	\frac{1}{\sqrt{var}X_1} & 0 & \cdots & 0\\
	\sqrt{X_1Z_1} & c & \cdots & 0 \\
	\vdots & \cdots & \vdots
	\end{matrix}\right)
	\left(\begin{matrix}
	X_1\\
	X_2\\
	\vdots
	\end{matrix}\right)$
	
	$$f(x_1,\cdots,x_n) = \frac{e^{-\frac{1}{2}(x-m)^T\mathscr{C}^{-1}(x-m)}}{(2\pi)^{\frac{n}{2}}\text{det}(\mathscr{C})^{\frac{1}{2}}}$$\
	
	WLOG let $m = \vec{0}$ and $\mathscr{C} = AA^T$
	
	$\because$ det$\mathscr{C} =$ det$(AA^T) =$ det$(A)^2 > 0$
	
	$\therefore A$ is invertible.
	
	Let $X = AZ$ where $Z_i = N(0,1)$
	
	Consider
	
	\begin{eqnarray*}
		P(X \in B) &=& P(Z \in  A^{-1}B) \\
		&=& \int_{A^{-1}B} \frac{1}{(2\pi)^{\frac{\pi}{2}}}e^{\frac{1}{2}Z^TZ}dz_1,\cdots dz_n
	\end{eqnarray*}
	
	change variable $x = AZ \implies Z = A^{-1}x$
	
	$Z \in A^{-1}B \implies x \in B$
	
	\begin{eqnarray*}
		Z^TZ &=& (A^{-1}x)^T(A^{-1}x) \\
		&=& x^T(A^{-1})^TA^{-1}x\\
		&=& x^T(A^T)^{-1}A^{-1}x\\
		&=& x^T(AA^T)^{-1}x \\
		&=& x^T\mathscr{C}^{-1}x
	\end{eqnarray*}
	
	\begin{eqnarray*}
		dz_1,\cdots ,dz_n &=& |det A^{-1}|dx_1,\cdots,dx_n \\
		&=& (det \mathscr{C})^{\frac{-1}{2}}dx_1,\cdots , dx_n
	\end{eqnarray*}
	
	$$\therefore \int_B \frac{1}{(2\pi)^{\frac{n}{2}}}e^{-\frac{x^T-\mathscr{C}x}{2}} \frac{1}{(det \mathscr{C})^{\frac{1}{2}}}dx_1,\cdots,dx_n$$
\end{proof}



\end{document}